    \section{Συναρτηστιακοί χώροι}
    Διανυσματικός χώρος \( S \)
    \[
    \bar x,\quad \bar y\quad S
    \]

    \paragraph{Εσωτερικό γινόμενο}
    \[ \left\langle\bar x,\bar y\right\rangle\ \in \mathbb C  \]
    \begin{enumpar}
        \item \( \left\langle\bar x,\bar y\right\rangle
        = \left\langle\bar y,\bar x\right\rangle^* \)
        \item \( c\left\langle\bar x,\bar y\right\rangle
        =\left\langle c\bar x,\bar y\right\rangle \)
        \item \( \left\langle\bar x+\bar y,\bar z\right\rangle
        = \left\langle\bar x,\bar z\right\rangle+\left\langle\bar y,\bar z\right\rangle \)
        \item \( \left\langle\bar x,\bar x\right\rangle \ \geq 0 \) με
        \( \left\langle\bar x,\bar x\right\rangle = 0 \) ανν \( \bar x = \bar 0 \)
    \end{enumpar}

    \paragraph{Νόρμα}
    \[
    \bar x \in S
    \]\[
    ||\bar x|| \geq0
    \]
    \begin{enumpar}
        \item \( ||\bar x|| = 0 \) ανν \( \bar x = \bar 0 \)
        \item \( ||a\bar x|| = |a|||\bar x|| \quad x \in\mathbb C \)
        \item \( ||\bar x+\bar y|| \leq ||\bar x|| + ||\bar y|| \)
    \end{enumpar}
    \paragraph{Μέτρο:} Απόσταση μεταξύ \( \bar x,\bar y \in S \)
    \begin{enumpar}
        \item \( d(\bar x,\bar y)\geq 0 \qquad d(\bar x,\bar y)=0 \)
        ανν \( \bar x = \bar y \)
        \item \( d(\bar x,\bar y) = d(\bar y,\bar x) \)
        \item \( d(\bar x,\bar y) \leq d(\bar x,\bar z) + d(\bar y,\bar z)
        \quad \bar z\in S
         \)

    \end{enumpar}

    \paragraph{Συναρτησιακός χώρος}
    \[
    x(t),y(t) \in S =
    \left\lbrace x(t)/x(t):[t_1,t_2]\to\mathbb R  \right\rbrace
    \]
    \begin{gather*}
    \left\langle
    x(t),y(t)
    \right\rangle  = \int_{t_1}^{t_2}x(t)y(t)\dif t\\
    \left\vert\middle\vert x(t)\middle\vert\right\vert =
    \left[ \int_{t_1}^{t_2} x^2(t)\dif t \right]^{\sfrac{1}{2}}\\
    d\left(
    x(t),y(t)
    \right)=\left[
    \int_{t_1}^{t_2} \left[ x(t)-y(t) \right]^2\dif t
    \right]^{\sfrac{1}{2}}
    \end{gather*}

    \begin{align*}
    \text{Αν } & \left\langle \phi_1(t),\phi_2(t) \right\rangle
    = 0 \quad \phi_1(t) \perp \phi_2(t) \\
    & \left\langle \phi_1(t),\phi_1(t)\right\rangle = 1 \quad
    \phi_1(t) \text{ κανονική}
    \end{align*}

    \paragraph{Τερατοχώρος}

    \begin{tikzpicture}[scale=1,baseline]
        \draw[->] (-0.5,0) -- (2,0) node[below right] {$\hat x$};
        \draw[->] (0,-0.5) -- (0,2) node[above left] {$\hat y$};

        \draw[thick,->] (0,0) -- (1.5,1.5) node[right] {$\vec a$};
        \draw[thick,->] (0,0) -- (1.5,0) node[below] {$\tilde a$};

        \draw (0.2,0) -- (0.2,0.2) -- (0,0.2);
    \end{tikzpicture}

    \( \hat x,\hat y \) όχι εξαρτημένα (συνευθειακά)

    Ποια είναι η καλύτερη προσέγγιση για το \(\vec a\) εφ' όσον δεν υπάρχει
    το \( \vec y \);

    \( \tilde a \) best γιατί \(\mathrm d(\vec a,\tilde a)\) min.

    Άρα:
    \begin{gather*}
        \tilde a = k\hat x \\
        \vec a = a_x\hat x+a_y\hat y\\
        \vec a -\tilde a = (a_x-k)\hat x - a_y\hat y\\
        d(\vec a,\tilde a) = \sqrt{(a_x-k)^2+a_y^2} \\
        \od{}{k}\left( d(\vec a,\tilde a) \right) = \frac{a_x-k}{\cdots}
        = 0 \implies k=a_x = \tilde a \cdot \hat x\\
        \boxed{\vec a\cdot\hat x = a_x}
    \end{gather*}

    Η βέλτιστη έκφραση του \( \vec a \) στο δισδιάστατο χώρο είναι το ίδιο το \( \vec a \).

    \paragraph{Μη κάθετα διανύσματα} \hspace{0pt}

    \begin{tikzpicture}[scale=1]
    \draw[->,dotted] (4,0) -- (6,0);
    \draw[dotted] (6,0) -- (6,2);
    \draw[->] (0,0) -- (4,0) node[below right] {$\hat x$};
    \draw[->] (0,0) -- (6,2) node[right] {$\vec a$};

    \draw[->] (0,0) -- (50:2) node[above right] {$\hat y$};
    \draw[->,blue,thick] (0.5,0) node[below] {$\mathsmaller{<\frac{\pi}{2}}$}
    arc (0:50:0.5);
    \end{tikzpicture}

    \begin{tikzpicture}[scale=1]
    \draw[very thick,draw=blue,->] (0,0) -- (29.8:4.88)
    node[above right] {$\vec a = a_x\hat x + a_y\hat y$};

    \draw[->] (0,0) -- (3,0) node[below right] {$\hat x$};

    \draw[very thick,blue!50!green,yshift=-5pt,xshift=2pt,mark position=1(X),->]
    (0,0) -- (2,0) node[midway,below] {$a_x$};
    \draw[very thick,blue!50!green,yshift=4pt,xshift=-4pt,mark position=1(Y),->]
    (0,0) -- (50:3) node[midway,above,sloped] {$a_y$};

    \draw[->] (0,0) -- (50:2) node[above right] {$\hat y$};

    \draw[dashed] (X) -- ++(50:3.4);
    \draw[dashed] (Y) -- ++(2.5,0);

    \draw (5,0.5) node {$\hat x \not\perp \hat y$};
    \end{tikzpicture}

    \begin{gather*}
    \vec a = a_x\hat x+a_y\hat y\\
    \vec a - \tilde a = (a_x-k)\hat x+a_y\hat y\\
    \mathrm d(\vec a,\tilde a)= ||\vec a - \tilde a|| = \sqrt{(\vec a-\tilde a)(\vec a-\tilde a)}
    =\left(
    \left[ (a_x-k)\hat x+a_y\hat y \right]\cdot
    \left[ (a_x-k)\hat x+a_y\hat y \right]
    \right)^{\sfrac{1}{2}}\\
    \left[
    (a_x-k)^2+a_y^2+2(a_x-k)a_y\hat x\cdot\hat y
    \right]^{\sfrac{1}{2}} \\
    \vec a_{\mathrm{best}} = (\vec a \cdot \hat x)\hat x \neq a_x \\
    \boxed{\vec a \cdot \hat x = a_x+a_y\cos\phi\neq a_x}
    \quad \leftarrow
    \text{Η βέλτιστη περιγραφή του $\vec a$ στον διδιάστατο στον άλλο χώρο}
    \end{gather*}

    \paragraph{Συναρτησιακός κόσμος}
    \( \phi_n(t) \) παράγουν χώρο με το μηχανισμό:
    \[
    f(t)=\sum_{n=0}^\infty a_n\phi_n(t)\quad t \in \Delta
    \]

    \( \phi_n(t) \) ανεξάρτητες μεταξύ τους (βάση απειροδιάστατου χώρου)

    \[
    \hat f(t)=\sum_{n=0}^M \underbrace{\hat a_n}_{%
   	\mathclap{\raisebox{-2ex}{\ensuremath{ \scriptstyle
   				\neq a_n\text{, επειδή η βάση δεν είναι ορθοκανονική}
   			}}}
   		}\phi_n(t) \text{ βέλτιστη, ώστε η απόσταση με την $f$
   		να είναι ελάχιστη}
   	\]

    \begin{align*}
        \overbrace{I^2}^{\mathclap{\text{σφάλμα}}} &=
        \int_\Delta\left[ f(t)-\hat f(t) \right]^2\dif t
        \\ &=
        \int_\Delta \left[
            \sum_{n=0}^{+\infty}a_n\phi_n(t)-\sum_{n=0}^M\hat a_n\phi(t)
        \right]^2\dif t
        \\ &= \int_\Delta f^2(t)\dif t + \int_\Delta \left(
            \sum_{n=0}^M \hat a_n\phi_n(t)
        \right)^2\dif t -2\int_{\Delta}\left[
            f(t)\sum_{n=0}^M \hat a_n\phi_n(t)
        \right] \dif t
    \end{align*}

    Άρα:
    {
    \setlength{\mathindent}{0cm}
    \begin{align*}
        I^2 &= \int_{\Delta} f^2(t)\dif t + \int_\Delta \sum_{n=0}^M \left[
            \hat a_n\phi_n(t)
        \right]^2\dif t \\ &\hphantom{=}+ 2 \int_\Delta \left[
            \sum_{n=0}^M\sum_{m=n+1}^M \hat a_n\cdot\hat a_m \phi_n(t)\phi_m(t)
        \right] \dif t 
        \\ &\hphantom{=}
        - 2 \int_\Delta \sum_{n=0}^M \hat a_n f(t)\phi_n(t)\dif t
        \\ &= \int_\Delta f^2(t)\dif t + \sum_{n=0}^M \hat a_n^2\int_\Delta
        \phi_n^2(t)\dif t + 2 \sum_{n=0}^M \sum_{n=m+1}^M \hat a_n\hat a_m
        \int_\Delta \phi_n(t)\phi_m(t)\dif t - 2\sum_{n=0}^M \hat a_n
        \int_\Delta f(t)\phi_n(t)\dif t\\
        \od{(I^2)}{\underbrace{\hat a_i}_{\mathclap{\text{από $0$ έως $n$}}}}
        &= 2\hat a_i\int_\Delta \phi_i^2(t)\dif t +2\sum_{m\neq i}
        \hat a_m\int_\Delta \phi_i(t)\phi_m(t)\dif t - 2\int_\Delta f(t)\phi(t)\dif t = 0
    \end{align*}
    }

    \paragraph{Σύστημα εξισώσεων}
    Αν \( \phi_i^{(t)} \) μοναδιαία, τότε: \( \int_\Delta \phi_i^2(t)\dif t = 1 \)

    Αν \( \phi_i(t) \) είναι ορθογώνια, τότε:
    \( \int_{\Delta}  \phi_i(t) \phi_j(t)\dif t = 0, \quad i \neq j \)

    Αν \( \left\lbrace \phi_i(t) \right\rbrace \) είναι ορθοκανονική βάση, τότε:
    \begin{gather*}
    2\vec{a_i}-2\int_{\Delta} f(t)\phi(t)\dif t = 0 \implies
    \vec{a_i} = \overbrace{\int_\Delta
    \underbrace{f(t)}_{\mathclap{\text{προβολή του διανύσματος στο μοναδιαίο}}}
    \phi_i(t)\dif t}^{\mathclap{\text{όπως το } k=\vec a \cdot \vec x}} = a_i
    \end{gather*}

    Με άλλη γραφή:
    \[
    2\vec{a_i}\left\langle\phi_i,\phi_i \right\rangle
    +2\sum_{m+i}\vec{a_m}\left\langle \phi_i,\phi_m \right\rangle
    -2\left\langle f,\phi_i \right\rangle=0
    \]

    Είναι:
    \[
    \left\langle f,\phi_i \right\rangle = \left\langle
    \sum_{n=0}^{+\infty}a_n\phi_n,\phi_i\right\rangle
    = \sum_{n=0}^{+\infty} a_n\left\langle \phi_nm\phi_i \right\rangle
    = a_i \quad \text{όπως στα διανύσματα}
    \]

    Ηθικό δίδαγμα: Αν η βάση του χώρου είναι ορθοκανονική και μας ζητηθεί να υπολογίσουμε
    μία προσέγγιση της συνάρτησης σε έναν υποχώρο, μπορούμε άμεσα να υπολογίσουμε την
    προβολή της συνάρτησης πάνω στη βάση.

    \paragraph{Ex.}
    \( f(t)=e^{-3t}\mathrm u(t)
    \qquad \phi_1(t)=e^{-t}\mathrm u(t) \quad \& \quad
    \phi_2(t) = e^{-2t}\mathrm u(t)
     \)

    \begin{gather*}
    \overbrace{\hat f(t)}^{\mathclap{\text{βέλτιστη}}} = a_1e^{-t}\mathrm u(t) +
     a_2e^{-2t}\mathrm u(t) \\
    \int \left[ a_1\phi_1+a_2\phi_2 -f \right]\phi_1\dif t = 0\\
    \int_0^\infty \left[ a_1e^{-t}+a_2e^{-2t}-e^{-3t}\right]e^{-t}\dif t = 0\implies
    \\
    a_1\int_0^\infty e^{-2t}\dif t+a_2\int_0^\infty e^{-3t}\dif t -\int_0^\infty
    e^{-4t}\dif t = 0
    \implies \boxed{ \frac{a_1}{2}+\frac{a_2}{3}-\frac{1}{4} = 0 } \\
    \int\left[ a_1e^{-t}+a_2e^{-2t}-e^{-3t} \right]e^{-2t}\dif t = 0 \implies
    \boxed{\frac{a_1}{3}+\frac{a_2}{4}-\frac{1}{5}=0}\\
    a_1 = -\sfrac{3}{10},\ a_2 = \sfrac{6}{5}
    \end{gather*}

    \paragraph{}
    \[
    \underset{ \text{γενικότερη μορφή}
    	}{\boxed{E \overset{\triangle}{=} \int_\Delta f^2(t)\dif t
    = \sum_{n=0}^{\infty} a_n^2 \quad
    \begin{array}{l} \text{Parseval's} \\ \text{Theorem}\end{array}}}
    \]
