\documentclass[11pt,a4paper,notitlepage,fleqn]{article}

\usepackage{amsmath}
\usepackage{amsfonts}
\usepackage{amssymb}
\usepackage{libs/commath2}
\usepackage[table]{xcolor}
\usepackage[hidelinks,draft=false]{hyperref}
\usepackage[skins,theorems]{tcolorbox}
\usepackage{titlesec}
\usepackage{tikz}
\usepackage{libs/circuitikz} % use our own recent version to make sure some bugs are fixed
\usepackage{pgfplots}
\usepackage{mathtools}
\usepackage[makeroom]{cancel}
\usepackage{mathrsfs}
\usepackage{wrapfig}
%\usepackage{subcaption}
%\usepackage{floatrow}
\usepackage{esint}
\usepackage{enumitem}
%\usepackage{bm}
\usepackage{relsize}
\usepackage{xfrac}
\usepackage{comment}
%\usepackage{siunitx}
%\usepackage{MnSymbol}
\usepackage[obeyDraft,disable]{todonotes}
%\usepackage[linesnumbered,lined]{algorithm2e}


\pgfplotsset{compat=1.13}
\usetikzlibrary{arrows.meta}
\usetikzlibrary{patterns}
\usetikzlibrary{decorations.pathmorphing,patterns}
\usetikzlibrary{decorations.markings}
\usetikzlibrary{backgrounds}
\usetikzlibrary{shapes.misc}
\usetikzlibrary{shapes.multipart}
\usetikzlibrary{shadows.blur}
\usetikzlibrary{fadings}
\usetikzlibrary{intersections}
\usetikzlibrary{arrows.meta}
\usetikzlibrary{calc}
\usetikzlibrary{matrix}
\usetikzlibrary{positioning}
\usetikzlibrary{shapes}
\usetikzlibrary{shadings}

\tcbuselibrary{breakable}

\tikzset{cross/.style={cross out, draw,
        minimum size=2*(#1-\pgflinewidth),
        inner sep=0pt, outer sep=0pt}}
\tikzset{
    mark position/.style args={#1(#2)}{
        postaction={
            decorate,
            decoration={
            	post length=1mm, % ??? Magic to fix "Dimension
            	pre length=1mm, % ???  too large" errors.
                markings,
                mark=at position #1 with \coordinate (#2);
            }
        }
    }
}
\makeatletter
\tikzset{
  use path for main/.code={%
    \tikz@addmode{%
      \expandafter\pgfsyssoftpath@setcurrentpath\csname tikz@intersect@path@name@#1\endcsname
    }%
  },
  use path for actions/.code={%
    \expandafter\def\expandafter\tikz@preactions\expandafter{\tikz@preactions\expandafter\let\expandafter\tikz@actions@path\csname tikz@intersect@path@name@#1\endcsname}%
  },
  use path/.style={%
    use path for main=#1,
    use path for actions=#1,
  }
}
\makeatother

\pgfmathdeclarefunction{sinc}{1}{%
	\pgfmathparse{abs(#1)<0.01 ? int(1) : int(0)}%
	\ifnum\pgfmathresult>0 \pgfmathparse{1}\else\pgfmathparse{sin(#1 r)/#1}\fi%
}
\pgfmathdeclarefunction{gauss}{2}{%
	\pgfmathparse{1/(#2*sqrt(2*pi))*exp(-((x-#1)^2)/(2*#2^2))}%
}

\usepackage[left=2cm,right=2cm,top=2cm,bottom=2cm]{geometry}

%\usepackage[no-math]{fontspec}
%\usepackage{fontspec}
\usepackage{mathspec}
%\usepackage{newtxtext,newtxmath}
%\usepackage{unicode-math}
%\setmainfont{texgyretermes-regular.otf}
%\setsansfont{texgyreheros-regular.otf}
%\newfontfamily\greekfont[Script=Greek]{Linux Libertine O}
%\newfontfamily\greekfontsf[Script=Greek]{Linux Libertine O}
\usepackage{polyglossia}
%\newfontfamily\greekfont[Script=Greek]{texgyretermes-regular.otf}
\newfontfamily\greekfontsf[Script=Greek]{texgyreheros-regular.otf}
\newfontfamily\greekfonttt[Script=Greek]{Latin Modern Mono}
%\usepackage[greek]{babel}
\setdefaultlanguage{greek}
\setotherlanguage{english}

%\usepackage[utf8]{inputenc}
%\usepackage[greek]{babel}


%\usepackage{tkz-euclide} % loads  TikZ and tkz-base
%\usetkzobj{angles} % important you want to use angles

\newlist{enumparen}{enumerate}{1}
\setlist[enumparen]{label=(\arabic*)}
\newlist{enumpar}{enumerate}{1}
\setlist[enumpar]{label=\arabic*)}

\newlist{enumgreek}{enumerate}{1}
\setlist[enumgreek]{label=\alph*.}
\newlist{enumgreekparen}{enumerate}{1}
\setlist[enumgreekparen]{label=(\alph*)}
\newlist{enumgreekpar}{enumerate}{1}
\setlist[enumgreekpar]{label=\alph*)}


\newlist{enumroman}{enumerate}{1}
\setlist[enumroman]{label=(\roman*)}

\newlist{enumlatin}{enumerate}{1}
\setlist[enumlatin]{label=(\alph*)}

\newlist{invitemize}{itemize}{1}
\setlist[invitemize]{noitemsep,label=}

\usepackage{letltxmacro}

\LetLtxMacro\OriginalLongrightarrow\Longrightarrow
\LetLtxMacro\OriginalLongleftarrow\Longleftarrow

% Implement new macros
% --------------------
\usepackage{trimclip}
\DeclareRobustCommand\Longrightarrow{\NewRelbar\joinrel\Rightarrow}
\DeclareRobustCommand\Longleftarrow{\Leftarrow\joinrel\NewRelbar}

\makeatletter
\DeclareRobustCommand\NewRelbar{%
  \mathrel{%
    \mathpalette\@NewRelbar{}%
  }%
}
\newcommand*\@NewRelbar[2]{%
  % #1: math style
  % #2: unused
  \sbox0{$#1=$}%
  \sbox2{$#1\Rightarrow\m@th$}%
  \sbox4{$#1\Leftarrow\m@th$}%
  \clipbox{0pt 0pt \dimexpr(\wd2-.6\wd0) 0pt}{\copy2}%
  \kern-.2\wd0 %
  \clipbox{\dimexpr(\wd4-.6\wd0) 0pt 0pt 0pt}{\copy4}%
}
\makeatother


\makeatletter
\pgfdeclareradialshading[tikz@ball]{ball}{\pgfqpoint{0bp}{0bp}}{%
	color(0bp)=(tikz@ball!50!white);
	color(10bp)=(tikz@ball!50!white);
	color(15bp)=(tikz@ball!70!black);
	color(20bp)=(black!70);
	color(30bp)=(black!70)}%
\makeatother


\makeatletter
\let\anw@true\anw@false

%\newcommand{\attnboxed}[1]{\textcolor{red}{\fbox{\normalcolor\m@th$\displaystyle#1$}}}
\makeatother
\tcbset{highlight math style={enhanced,colframe=red,colback=white,%
        arc=0pt,boxrule=1pt,shrink tight,boxsep=1.5mm,extrude by=0.5mm}}
\newcommand{\attnboxed}[1]{\tcbhighmath[colback=red!5!white,drop fuzzy shadow,arc=0mm]{#1}}
\newcommand{\infoboxed}[1]{%
	\tcbhighmath[colframe=blue!50!white,colback=blue!5!white,arc=0mm]{#1}}
\titleformat{\section}{\bf\Large}{Κεφάλαιο \thesection}{1em}{}
\newtcolorbox{attnbox}[1]{colback=red!5!white,%
    colframe=red!75!black,fonttitle=\bfseries,title=#1}
\newtcbox{quickattnbox}[1]{colback=red!5!white,%
	colframe=red!75!black,fonttitle=\bfseries,title=#1}
\newtcolorbox{infobox}[1]{colback=blue!5!white,%
    colframe=blue!75!black,fonttitle=\bfseries,title=#1}

\AtBeginDocument{%
\let\arg\relax
\let\Re\relax
\let\Im\relax
\DeclareMathOperator{\arg}{Arg}
\DeclareMathOperator{\Re}{Re}
\DeclareMathOperator{\Im}{Im}
}
\DeclareMathOperator{\sinc}{sinc}
\DeclareMathOperator{\sgn}{sgn}
\DeclareMathOperator{\erf}{erf}
\DeclareMathOperator{\cov}{cov}

\newif\ifhidetikz
\hidetikzfalse
%\hidetikztrue   % <---- comment/uncomment that line

\ifhidetikz

\let\oldtikzpicture\tikzpicture
\let\oldendtikzpicture\endtikzpicture

\renewenvironment{tikzpicture}{
    \tiny
    \tt
    \color{blue}
    \newcommand{\draw}{\textit{draw}}
    \newcommand{\filldraw}{\textit{filldraw}}
    %\newcommand{\x}{\textit{x}}
    %\newcommand{\p}{\textit{x}}
    \newcommand{\x1}{\textit{x1}}
    \newcommand{\y1}{\textit{y1}}
    \newcommand{\p1}{\textit{p1}}
}{
}
\newenvironment{axis}{
    \newcommand{\addplot}{\textit{addplot}}
}{
}
\fi

% Global amount of samples
% Set to a higher value (e.g. 200) for nicer graphs
% Set to a low value (e.g. 10) for performance
\newcommand*{\gsamples}{70}

% Equals command as a workaround for CircuiTikZ bug
% not allowing the = sign in labels
\newcommand*{\equals}{=}

\newcommand{\nesearrow}{%
	\,%
	\smash{\raisebox{-1.1ex}
		{$%
			\stackrel{\displaystyle\nearrow}{\displaystyle\searrow}%
			$}}%
}
\newcommand{\degree}{^{\circ}} % not great
\newcommand\numberthis{\addtocounter{equation}{1}\tag{\theequation}} % add an equation number to a number-less math environment

\newtcbtheorem[number within=section]{theorem}{Θεώρημα}%
{colback=green!5,colframe=green!35!black,colbacktitle=green!35!black,fonttitle=\bfseries,enhanced,attach boxed title to top left={yshift=-2mm,xshift=-7mm},width=.9\textwidth,arc=.7mm}{th}
\newtcbtheorem[number within=section]{defn}{Ορισμός}%
{colback=blue!5,colframe=cyan!35!black,colbacktitle=blue!35!black,fonttitle=\bfseries,enhanced,attach boxed title to top left={yshift=-2mm,xshift=-2mm}}{def}
\newtcbtheorem[number within=section]{exercise}{Άσκηση}%
{colback=gray!3,colframe=gray!35!black,colbacktitle=gray!35!black,fonttitle=\bfseries,enhanced,attach boxed title to top left={yshift=-2mm,xshift=-2mm}}{exc}




\title{Αναλογικές Τηλεπικοινωνίες
	\\
	{ 
		\normalsize Σημειώσεις από τις παραδόσεις
	}}
\date{Οκτώβριος-Ιανουάριος 2017
	\\
	{ 
		\small Τελευταία ενημέρωση: \today
	}
}
\author{
	Για τον κώδικα σε \LaTeX, ενημερώσεις και προτάσεις:
	\\
	\url{https://github.com/kongr45gpen/ece-notes}}

\setallmainfonts(Digits,Latin,Greek){Asana Math}
\setmainfont{Noto Serif}
\setsansfont{Ubuntu}
%\usepackage{unicode-math}
\usepackage{polyglossia}
\newfontfamily\greekfont[Script=Greek,Scale=0.95]{Noto Serif}
\setmathfont{XITS Math}

\hypersetup{pdftitle = {Αναλογικές Τηλεπικοινωνίες}}


\begin{document}
\maketitle

\hrule
\vspace{50pt}
	
Αναλογικές Τηλεπικοινωνίες

\begin{attnbox}{Εγγραφή στη λίστα}
	Μήνυμα στο \href{mailto:dimakis@auth.gr}{\texttt{dimakis@auth.gr}} με θέμα
	\textit{\textbf{Αναλογικές Τηλεπικοινωνίες}}.
	
	\tcblower
	
	Στη λίστα θα στέλνονται ασκήσεις χρήσιμες για τις εξετάσεις και λοιπές ανακοινώσεις.
\end{attnbox}

Κάθε Τρίτη 18:00 - 20:00 θα γίνονται ασκήσεις από τον Κ. Αρκουδογιάννη σε \textit{ένα τμήμα},
και η θεωρία θα γίνεται κανονικά σε τμήματα τις υπόλοιπες ημέρες από τον κ. Δημάκη.

Εξετάσεις: Όλα ανοιχτά.

Το μάθημα θα γίνει με βάση το βιβλίο του Haykin.

\section{Εισαγωγή}
Επικοινωνία είναι η μεταφορά μηνυμάτων, που μπορεί να έχουν τη μορφή απλών συμβόλων / φθόγγων
ή πιο περίπλοκων μηνυμάτων.

Για τον μηχανικό, επικοινωνία είναι η μετάδοση ή μεταφορά πληροφορίας από ένα σημείο \( A \) σε ένα σημείο \( B \) του χώρου.

\begin{center}
	\begin{tikzpicture}
	\draw[thin,gray,dashed] (0,0) -- (2,0);
	\filldraw (0,0) circle (2pt) node[left] {$A$};
	\filldraw (2,0) circle (2pt) node[right] {$B$};
	\end{tikzpicture}
\end{center}

Την πληροφορία μπορούμε να την ορίσουμε ως ένα σύνολο ταξινομημένων συμβόλων, που μαζί ίσως
σχηματίζουν μια λέξη, μια πρόταση, ή ένα νόημα. Ένας καλύτερος ορισμός έχει δοθεί από τον
Shannon στη Θεωρία Πληροφοριών.

Μία ακόμα παράμετρος είναι ο χρόνος μεταφοράς της πληροφορίας, αν και συνήθως δεν μας
ενδιαφέρει στις αναλογικές τηλεπικοινωνίες (δεδομένης της ταχύτητας του φωτός), εκτός
αν προσπαθούμε να επικοινωνήσουμε με κάτι εκτός του πλανήτη.

Τα σήματα αυτά μπορούν να μεταφέρουν αριθμούς, κείμενο, εικόνα, ήχο, βίντεο, αρχεία κ.ά, και
βρίσκονται σε σχετικά χαμηλές συχνότητες (\textbf{baseband}). Για παράδειγμα, το εύρος της ανθρώπινης φωνής που
απαιτείται για να είναι καταληπτή είναι \( 300 \ \mathrm{Hz} \)-\( 3300\ \mathrm{Hz} \), ενώ
τα τηλεοπτικά σήματα κωδικοποιούνται σε συχνότητες έως \( 6 \ \mathrm{MHz} \). Τέτοιες
συχνότητες όμως είναι δύσκολο να μεταδοθούν (αν π.χ. σκεφτούμε ότι οι συχνότητες αρκετών
ηλεκτρομαγνητικών κυμάτων ή του ορατού φωτός είναι της τάξης των \( \mathrm{GHz} \) και
\( \mathrm{THz} \)). Επομένως, για να επιτύχει η επικοινωνία απαιτείται η αύξηση της
συχνότητας του σήματος, μέσω μιας διαδικασίας που λέγεται \textbf{διαμόρφωση}.

\begin{tikzpicture}[scale=1]
\def\h{0.6}
\filldraw (-0.5,0) circle(1.5pt) node[left] {$A$};
\draw (0,-\h/2) rectangle ++(1.5,\h) node[midway] {Πηγή};
\draw (1.5,0) -- (1.95,0);
\node[cloud, cloud puffs=15.7, cloud ignores aspect,
rotate=90,minimum width=3cm, minimum height=1.2cm, align=center, draw]
(cloud) at (2.63cm, 0cm) {Διαμόρφωση};
\draw (3.3,0) -- (3.7,0);
\draw (3.7,-\h/2) rectangle ++(1.5,\h) node[midway] {Πομπός};

\draw[->] (5.2,0) -- ++ (0.2,0) -- ++(0,-1) -- ++(0.3,0);
\draw 
(5.7,-1-2*\h/2) rectangle ++(3,2*\h) node[midway,align=center,rectangle] {Κανάλι\\Μέσο διάδοσης};
\draw (8.7,-1) -- ++(0.3,0) -- ++(0,1) -- ++(0.2,0);
\draw (9.2,-\h/2) rectangle ++(1.5,\h) node[midway] {Δέκτης};
\draw (10.7,0) -- ++(0.35,0);
\node[cloud, cloud puffs=15.7, cloud ignores aspect,
rotate=90,minimum width=3cm, minimum height=1.6cm, align=center, draw]
(cloud) at (11.85cm, 0cm) {Αποδιαμόρφωση};
\draw (12.6,0) -- (13,0);
\draw (13,-\h/2) rectangle ++(1.8,\h) node[midway,scale=.9] {Αποδέκτης};
\filldraw (14.8+0.5,0) circle(1.5pt) node[right] {$B$};
\end{tikzpicture}

Η επιλογή της κατάλληλης συχνότητας του σήματος που θα στείλουμε από την κεραία, εξαρτάται
από τα χαρακτηριστικά του ηλεκτρομαγνητικού κύματος. Για παράδειγμα, πιο χαμηλές συχνότητες
(π.χ. AM) μπορούν να περάσουν μέσα από βουνά και εμπόδια, φτάνοντας σε μεγάλες αποστάσεις
στον πλανήτη, και ακολουθώντας την καμπύλη της γης. Τα βραχέα μπορούν να χτυπήσουν στην
ιονόσφαιρα και να ανακλαστούν για ακόμα μεγαλύτερη κάλυψη. Αντιθέτως, οι υψηλές συχνότητες
(π.χ. FM) επιτρέπουν υψηλότερη ποιότητα μετάδοσης.

\begin{wrapfigure}{r}{0.3\textwidth}\centering
	\begin{tikzpicture}[scale=.5,every node/.style={scale=.7}]
	\def\l{1}
	\def\h{3.5}
	\def\N{7}
	
	\begin{scope}[every node/.style={midway,above,scale=.4}]
	\draw[->] (0,1) -- (0,0) node[midway,right] {user};
	\draw (-\l,0) rectangle (\l,-\h);
	\draw(-\l,-1*\h/7) -- ++(2*\l,0) node {Application};
	\draw(-\l,-2*\h/7) -- ++(2*\l,0) node {Presentation};
	\draw(-\l,-3*\h/7) -- ++(2*\l,0) node {Session};
	\draw(-\l,-4*\h/7) -- ++(2*\l,0) node {Transport};
	\draw(-\l,-5*\h/7) -- ++(2*\l,0) node {Network};
	\draw(-\l,-6*\h/7) -- ++(2*\l,0) node {Data Link};
	\draw(-\l,-7*\h/7) -- ++(2*\l,0) node[scale=1.3,yshift=-1mm] {Physical};
	\end{scope}
	
	\begin{scope}[every node/.style={midway,above,scale=.4,baseline},xshift=5cm]
	\draw[<-] (0,1) -- (0,0) node[midway,right] {user};
	\draw (-\l,0) rectangle (\l,-\h);
	\draw(-\l,-1*\h/7) -- ++(2*\l,0) node {Application};
	\draw(-\l,-2*\h/7) -- ++(2*\l,0) node {Presentation};
	\draw(-\l,-3*\h/7) -- ++(2*\l,0) node {Session};
	\draw(-\l,-4*\h/7) -- ++(2*\l,0) node {Transport};
	\draw(-\l,-5*\h/7) -- ++(2*\l,0) node {Network};
	\draw(-\l,-6*\h/7) -- ++(2*\l,0) node {Data Link};
	\draw(-\l,-7*\h/7) -- ++(2*\l,0) node[scale=1.3,yshift=-1mm] {Physical};
	\end{scope}
	
	\draw (0,-\h) -- ++(0,-1) -- ++(1.5,0);
	\draw (1.5,-\h-1+0.4) rectangle ++(2,-0.8) node[midway] {Κανάλι};
	\draw (3.5,-\h-1) -- ++(1.5,0) -- ++(0,1);
	
	\draw[<->,thick,gray] (1.5,-1.7) to[bend left] node[midway,above] {Peer} ++(2,0);
	
	\draw (current bounding box.north) node[rectangle,align=center]
	{OSI\\7 επιπέδων};
	\end{tikzpicture}
\end{wrapfigure}

Για την κωδικοποίηση και αποκωδικοποίηση των δεδομένων, πρέπει ο πομπός και ο δέκτης να
συμφωνήσουν σε ένα κοινό πρότυπο, για παράδειγμα στο TCP/IP ή το OSI 7 επιπέδων.

Σε αυτό το μάθημα μας ενδιαφέρει το φυσικό επίπεδο μόνο.

Οι ψηφιακές επικοινωνίες αναφέρονται σε ψηφιακά δεδομένα, αλλά πρακτικά η μετάδοση
του σήματος μέσω των καναλιών (π.χ ηλεκτρομαγνητικά κύματα) είναι αναλογική, αφού δεν γίνεται
να έχουμε άμεση μετάβαση της κατάστασης από 0 ως 1:

\begin{tikzpicture}[scale=1]
\draw (0,-2) -- (0,2);
\draw (0,0) -- (6,0);

\draw[very thick,black!70!blue] plot[const plot]
coordinates {(0,1) (1,-1) (3,1) (4,-1) (5.5,1) (6,1)};

\draw[very thick,black!20!cyan!80!blue] plot[const plot,smooth,tension=0.8]
coordinates {(0,0.8) (0.8,0.7) (1.5,-1.4) (2.7,-0.8) (3.5,1) (4.9,-1.2) (5.5,1) (6,1.2)};

\filldraw[fill=black!70!blue] (7,-0.3) rectangle ++(0.2,0.2) node[midway,right,xshift=1mm] {ψηφιακό σήμα};

\filldraw[fill=black!20!cyan!80!blue] (7,-0.8) rectangle ++(0.2,0.2) node[midway,right,xshift=1mm] {πραγματικό σήμα};
\end{tikzpicture}

Πρακτικά οι αναλογικές τηλεπικοινωνίες χρησιμοποιούνται πλέον μόνο στους ραδιοφωνικούς
σταθμούς FM (που αρχίζουν και αυτοί να καταργούνται), αλλά συνεχίζουμε να τις μελετάμε για
λόγους ιστορικούς, διδακτικούς, και επειδή το σήμα όπως αναφέρθηκε παραπάνω είναι εν γένει
αναλογικό. Στο νέο πρόγραμμα σπουδών δεν υπάρχει ακριβώς αυτό το μάθημα.

\subsection{Βασικές έννοιες}
\paragraph{Σήμα βασικής συχνότητας (baseband)}
Τα σήματα βασικής συχνότητας (\textbf{baseband}) προέρχονται από το αρχικό σήμα σε
"χαμηλές" συχνότητες όπως αναφέρθηκε παραπάνω (συνήθως από 0 μέχρι π.χ. 20 \( \mathrm{kHz} \)
ή 6 \( \mathrm{MHz} \)):

\begin{tikzpicture}[scale=1]
\def\s{ (0,0) (0.5,1) (0.7,1.2+0.1*rand) (1,1.5) (1.2,1.2+0.1*rand) (1.4,1+0.1*rand)
	(1.6, 0.7+0.2*rand) (1.8,0.4+0.15*rand) (2,0)}

\pgfmathsetseed{15}
\draw[very thick, orange] plot [smooth] coordinates \s;
\pgfmathsetseed{15}
\draw (2,0) node[below] {$w$};

\draw (0,0) node[below left] {$0$};
\draw (-3,0) -- (3,0) node[below right] {$\mathrm{Hz}$};
\draw (0,-0.2) -- (0,2);
\end{tikzpicture}

Ή, επειδή χρησιμοποιούμε \textit{δίπλευρα} φάσματα:

\begin{tikzpicture}[scale=1]
\def\s{ (0,0) (0.5,1) (0.7,1.2+0.1*rand) (1,1.5) (1.2,1.2+0.1*rand) (1.4,1+0.1*rand)
	(1.6, 0.7+0.2*rand) (1.8,0.4+0.15*rand) (2,0)}

\pgfmathsetseed{15}
\draw[very thick, orange] plot [smooth] coordinates \s;
\pgfmathsetseed{15}
\draw[very thick, orange!90!brown, xscale=-1] plot [smooth] coordinates \s;
\draw (2,0) node[below] {$w$};
\draw (-2,0) node[below] {$-w$};

\draw (0,0) node[below left] {$0$};
\draw (-3,0) -- (3,0) node[below right] {$\mathrm{Hz}$};
\draw (0,-0.2) -- (0,2);
\end{tikzpicture}

Η μέγιστη θετική συχνότητα \( w \) ορίζει το \textbf{εύρος ζώνης (bandwidth)} του σήματος.

Η διαδικασία που θα χρησιμοποιήσουμε για να αυξήσουμε τη συχνότητα του σήματος ονομάζεται
\textbf{διαμόρφωση (modulation)}.

Συνήθως έχουμε μια \textbf{φέρουσα συχνότητα}:
\[
c(t) = A_c\cos(2\pi f_c t)
\]
και πρέπει να βρούμε έναν τρόπο να προσθέσουμε σε αυτήν τις πληροφορίες του αρχικού σήματος.
Στην παραπάνω εξίσωση έχουμε τρεις παραμέτρους που μπορούμε να επηρεάσουμε: το πλάτος,
τη συχνότητα και τη φάση:
\[
c(t) =
\underset{\substack{\downarrow\\\mathclap{A_c(t)}}}{A_c}
\cos(2\pi
\underset{\substack{\downarrow\\\mathclap{f_c(t)}}}{f_c}
t
+
\underset{\substack{\downarrow\\\mathclap{\phi(t)}}}{\phi}
)
\]

Έτσι έχουμε τρία είδη διαμόρφωσης:
\begin{description}
	\item[AM] Διαμόρφωση Πλάτους (Amplitude Modulation)
	\item[FM] Διαμόρφωση Συχνότητας (Frequency Modulation)
	\item[PM] Διαμόρφωση Φάσης (Phase Modulation)
\end{description}

\subsection{AM}
Έστω το φέρον:
\[
c(t) = A_c\cos(2\pi f_c t)
\]
και θέλουμε να μεταφέρουμε ένα σήμα:
\[
m(t) \qquad \text{στη βασική ζώνη}
\]

Θεωρούμε, για λόγους που θα δούμε παρακάτω, ότι το φέρον έχει συχνότητα πολύ μεγαλύτερη
από το εύρος ζώνης της πληροφορίας:
\[
f_c \gg w
\]

Το σήμα που εκπέμπουμε κατά AM είναι το εξής:
\[
s(t) = A_c\left[ 1 + k_a \cdot m(t) \right]\cos(2\pi f_c t)
\]
το οποίο μπορεί να εκφράζεται σε Volt ή Ampere και ίσως εκπέμπεται από κάποια κεραία.

Γραφικά:

\begin{center}
\begin{tikzpicture}[scale=1.2,xscale=1.3]
\def\f{1.3+0.30362*\x-6.94276*\x^2+5.23511*\x^3-0.0466465*\x^4-0.696833*\x^5+0.128634*\x^6}
%coordinates {(0,1.5) (0.7,0) (1.2,-0.5) (1.5,0) (2,1.5)  (3,1.5) (3.2,1.5)}
\draw (0,-1.5) -- (0,2);
\draw (0,0) -- (3,0);

\draw[very thick,black!70!blue,variable=\x,samples=\gsamples,domain=0:3]
plot ({\x},{\f}) (0,1.3) node[left] {$m(t)$};

\begin{scope}[yshift=-4cm]
\draw (0,-2) -- (0,2);
\draw (0,0) -- (3,0);

\draw[thick,blue!70!black,variable=\x,samples=\gsamples,domain=0:3] plot ({\x},{(1.5*sin(\x r*40))}) (0,1.5) node[above right] {$c(t)$};
\draw (0,-1.5) node[left] {$-A_c$};
\draw (0,1.5) node[left] {$A_c$};
\end{scope}

\begin{scope}[xshift=5cm]
\draw (0,-0.5) -- (0,3);
\draw (0,0) -- (3,0);

\draw[very thick,black!70!blue,variable=\x,samples=\gsamples,domain=0:3]
plot ({\x},{\f+1}) (0,1.3+1) node[left] {$\left[1+k_am(t)\right]$};
\end{scope}

\begin{scope}[xshift=5cm,yshift=-4cm]
\draw[thin,dashed,cyan!70!blue,variable=\x,samples=\gsamples,domain=0:3]
plot ({\x},{\f+1}) (0,1.3+1);
\draw[very thick,green!70!blue,variable=\x,samples=\gsamples,domain=0:3]
plot ({\x},{(\f+1)*sin(\x r*40)}) (1.3+1,0);

\draw[->,cyan!70!blue] (2.65,3) to[bend left] ++(1,1) node[above right] {Περιβάλλουσα};

\draw (0,-3) -- (0,3);
\draw (0,0) -- (3,0);
\end{scope}
\end{tikzpicture}
\end{center}

Η περιβάλλουσα του διαμορφωμένου σήματος περιέχει την πληροφορία που θέλουμε.

Η σταθερά \( k_a \) ονομάζεται \textbf{ευαισθησία πλάτους} του διαμορφωτή, και θέλουμε
να είναι τέτοια ώστε \( \left[1 + k_a m(t)\right] > 0 \), διότι σε διαφορετική περίπτωση:

\begin{tikzpicture}[scale=1,xscale=1.3]
\def\f{1.3+0.30362*\x-6.94276*\x^2+5.23511*\x^3-0.0466465*\x^4-0.696833*\x^5+0.128634*\x^6}
%coordinates {(0,1.5) (0.7,0) (1.2,-0.5) (1.5,0) (2,1.5)  (3,1.5) (3.2,1.5)}
\draw (0,-2) -- (0,2);
\draw (0,0) -- (3,0);

\draw[very thick,black!70!blue,variable=\x,samples=\gsamples,domain=0:3]
plot ({\x},{\f}) (0,1.3);

\begin{scope}[xshift=5cm]
\draw[very thick,green!70!blue,variable=\x,samples=\gsamples,domain=0:3]
plot ({\x},{(\f)*sin(\x r*40)}) (1.3+1,0);
\draw[very thick,black!70!blue,variable=\x,samples=\gsamples,domain=0:3]
plot ({\x},{abs(\f)}) (0,1.3);

\draw (0,-2) -- (0,2);
\draw (0,0) -- (3,0);
\end{scope}
\end{tikzpicture}

Επειδή ο αποδιαμορφωτής βλέπει μόνο τις θετικές κορυφές του σήματος, εδώ δεν έχει μεταφέρει
σωστά την πληροφορία στα σημεία όπου \(  \left[1 + k_a m(t)\right] < 0  \), αλλά την έχει
μεταφέρει ανεστραμμένη. Αυτό ονομάζεται \textbf{υπερδιαμόρφωση}.

Επομένως, θέλουμε:
\begin{align*}
	1+ k_a m(t) &\geq 0 \implies \\
	\Aboxed{\left\lvert k_a m(t) \right\rvert &\leq 1} \implies \\
	-1 \leq k_a m(t) &\leq 1
\end{align*}

Παρατηρούμε ότι για να μην έχουμε υπερδιαμόρφωση, το σήμα μας δεν μπορεί να αποκτά πολύ
μεγάλο πλάτος.

\begin{defn}{Ποσοστό διαμόρφωσης}{}
	Ως \textbf{ποσοστό διαμόρφωσης} ορίζουμε:
	\[
	\left\lvert
	\max k_a m(t)
	\right\rvert \cdot 100
	\]
\end{defn}

\paragraph{}
Ξαναγράφουμε το σήμα και παίρνουμε το μετασχηματισμό Fourier:
\begin{align*}
	s(t) &= A_c \cos 2\pi f_c t + Ak_a m(t) \cos 2\pi f_c (t) \\
	S(f) &=
	\frac{A_c}{2}\left[ δ(f-f_c)+δ(f+f_c) \right]
	+ \frac{k_aA_c}{2}\left[ M(f-f_c) + M(f+f_c) \right]
\end{align*}

\todo{Graph 8}
Βλέπουμε ότι το φάσμα του σήματος μετακινήθηκε στη συχνότητα.

Παρατηρούμε επίσης ότι το φάσμα είναι δίπλευρο, και θυμόμαστε από το αναλογικό σήμα ότι
η αρνητική συχνότητα δεν έχει φυσική σημασία, αλλά εκφράζει τον αρνητικό εκθέτη στην
έκφραση του συνημιτόνου \( 
\mathrm{Re}\left[
\frac{e^{jωt}+e^{-jωt}}{2}
\right]
 \).
 
Αν έχουμε \textbf{μικρή συχνότητα} \( f_c \), τότε το φάσμα του σήματος δεν περνάει το 0:

\todo{Graph 9}

και το αποτέλεσμα είναι παραμορφωμένο και μακριά από το επιθυμητό.

Το εύρος φάσματος του σήματος είναι \( 2w \) (από \( f_c-w \) μέχρι \( f_c+w \)), το οποίο
είναι περισσότερο απ' όσο χρειάζεται (αφού το αριστερό του μέρος είναι ίδιο με το δεξί),
ενώ το σήμα είναι και ενεργειοβόρο, αφού τα \( \sfrac{2}{3}  \) της ενέργειας καταναλώνονται
στον όρο \( δ \) και όχι στην πληροφορία.

\subsubsection{Για ημιτονοειδή είσοδο}
Έστω ένα αρχικά ημιτονοειδές σήμα:
\[
m(t) = A_m\cos(2\pi f_mt)
\]

Τότε το διαμορφωμένο σήμα AM γίνεται:
\begin{align*}
s(t) &= A_c \left[
1 + \overbrace{μ}^{\mathclap{μ = k_aA_m}}
\cdot \cos(2\pi \cdot f_m t)
\right]\cos(2\pi f_c t)
\boxed{\hspace{200pt}}
μ \leq 1
\\
&= A_c\cos(2\pi f_c t) + A_c μ\cos(2\pi f_c t)\cos(2\pi f_mt)
\\ &=
A_c\cos2πf_ct + \frac{1}{2}μA_c \cos\left[2π(f_c+f_m)t\right]
+\frac{1}{2} μA_c\cos\left[
2π(f_c-f_m)t
\right]
\\
\updownarrow &\quad \text{Μ. F}
\\
S(f) &= \mathsmaller{\mathsmaller{\frac{1}{2} A_c \left[
δ(f-f_c)+δ(f+f_c) \right]
+ \frac{1}{4}μA_c \left[
δ(f-f_c-f_m)+δ(f-f_c+f_m)
\right]
+ \frac{1}{4} μA_c\left[
δ(f-f_c+f_m)+δ(f+f_c-f_m)
\right]}}
\end{align*}

%Γραφικά:

\todo{Graph 11}

\[
\underset{\substack{\downarrow\\\mathclap{\text{Bandwidth}}}}{\mathrm{B_T}}
=2f_m = 2w
\]

Επίσης προκύπτει ότι:
\begin{align*}
\frac{A_{\max}}{A_{\min}} &= \frac{A_c(1+μ)}{A_c(1-μ)} \implies \\
μ &= \frac{A_{\max} -A_{\min} }{A_{\max} + A_{\min}}
\end{align*}

\subsubsection{Ισχύς}
Αν μας ζητούνταν η ισχύς του σήματος, θα απαντούσαμε \( \frac{1}{2}A_c^2R \), αν θεωρήσουμε
ότι το σήμα είναι μια ένταση ρεύματος που διαρρέει κάποια αντίσταση \( R \). Στα σήματα
όμως θεωρούμε ότι η αντίσταση αυτή είναι 1, άρα παίρνουμε ίδιο αποτέλεσμα, είτε θεωρούμε ότι
το σήμα αναπαριστά ρεύμα, είτε τάση.

Επομένως η ισχύς π.χ. του φέροντος είναι:
\[
\boxed{
\frac{1}{2}A_c^2
	}
\]

Το πλευρικό σήμα για ημιτονοειδή είσοδο έχει ενέργεια:
\[
2\times \frac{1}{8}μ^2A_c^2
\]

και ο λόγος του με τη συνολική ενέργεια είναι:
\[
\frac{2\cdot \frac{1}{8} μ^2A_c^2}{\frac{1}{2}A_c^2+2\cdot\frac{1}{8}μ^2A_c^2}
= \frac{μ^2}{2+μ^2}
\]

Γραφικά:

\begin{tikzpicture}
\def\c{gray!50!black}
\draw (0,0) -- (4,0) node[below] {$μ$};
\draw (0,-0.5) -- (0,4) node[left] {Ισχύς \%};

\draw[\c,thin] (3,-0.5) -- (3,4);

\draw (0,0) node [below left] {$0$};
\draw (3,0) node [below left] {$1$};

\draw[\c,thin] (0,1) node[left] {$\sfrac{1}{3}$} -- (4,1);
\draw[\c,thin] (0,2) node[left] {$\sfrac{2}{3}$} -- (4,2);
\draw[\c,thin] (0,3) node[left] {$\sfrac{3}{3}$} -- (4,3);

\draw[gray,dashed] (0.2*3,-0.4) node[right,scale=.8] {$0.2=20\%$} -- (0.2*3,4);
\draw[gray,dashed] (0.2*3,{3*0.2^2/(2+0.2^2)}) -- ++(-0.2*3,0)
node[scale=0.6,left] {$2\%$};

\draw[->,black!80!brown] (1.51,2.67) to[bend left=20] ++(0.5,1) node[above right] {Ισχύς φέροντος};
\draw[->,black!80!brown] (2.61,0.82) to[bend right=20] ++(0.8,-0.5) node[right] {
	Ισχύς πλευρικών συνιστωσών};

\draw[very thick,black!80!brown,variable=\m,domain=0:3,samples=\gsamples]
plot ({\m}, {3*(\m/3)^2/(2+(\m/3)^2)});
\draw[very thick,black!80!brown,variable=\m,domain=0:3,samples=\gsamples]
plot ({\m}, {3-3*(\m/3)^2/(2+(\m/3)^2)});
\end{tikzpicture}

Όσο αυξάνουμε το \( μ \), αυξάνεται το ποσοστό της ισχύος που καταναλώνεται για τη μετάδοση
του σήματος και όχι του φέροντος, αλλά η ισχύς του φέροντος συνεχίζει να είναι μεγάλη.

\subsubsection{Διαμορφωτής AM}
Ένα ερώτημα που προκύπτει είναι ποιό κύκλωμα θα πραγματοποιήσει τον πολλαπλασιασμό του
σήματος με το φέρον. Για αυτό παρουσιάζεται ο \textbf{διαμορφωτής AM (διακοπτικός - switching modulator)}:

\begin{circuitikz}[american,scale=1.3]
	\draw (0,0) to[esource=$m(t)$] (0,2)
	to[sV=$c(t)$] (2,2)
	to[Do] (4,2)
	to[R=$R_L$] (4,0)
	-- (0,0);
	
	%\draw (2,2) to[V=$v_1(t)$,*-*] (2,0);
	%\draw (4,2) to[V=$v_2(t)$,*-*] (4,0);
\end{circuitikz}

Επίσης απαιτούμε το σήμα \( m(t) \) να έχει \textbf{αρκετά μικρότερο πλάτος} από το φέρον:
\[
\left\lvert  m(t) \right\rvert \ll A_c
\]

Τότε, η τάση \( u_1 \) γίνεται:
\[
u_1(t) = m(t)+A_c \cos(2πf_ct)
\]
\todo{Graph 13}

Και, αφού θυμηθούμε την καμπύλη λειτουργίας της διόδου, \todo{Graph 14}, ισχύει:
\[
u_2(t) \simeq \begin{cases}
u_1(t), &\quad \text{όταν } c(t) > 0 \\
0,&\quad \text{όταν } c(t) < 0
\end{cases}
\]

Εναλλακτικά, μπορούμε να εκφράσουμε την \( u_2(t) \) ως γινόμενο της
εισόδου \( u_1(t) \) και μιας συνάρτησης \( g_{T_0} \) που μηδενίζεται για \( c(t) <0 \) και
είναι μονάδα για \( c(t) > 0 \), δηλαδή μιας παλμοσειράς:

\todo{Graph 15}

\begin{align*}
	u_2(t) &\simeq
	\left[
	A_c\cos 2π f_c t + m(t)
	\right] \cdot g_{T_0}(t) \\
	g_{T_0(t)} &= \frac{1}{2} + \frac{2}{\pi}
	\sum_{n=1}^\infty \frac{(-1)^{n-1}}{2n+1}\cos[2πf_ct(2n-1)]
	\intertext{Άρα}
	u_2(t) &\simeq
	\left(
	A_c\cos 2πf_ct +m(t
	\right) \cdot \left[
	\frac{1}{2} + \frac{2}{π} \cos\left(
	2πf_ct
	\right)-\frac{2}{π}\frac{1}{3}\cos(2π3f_ct)
	+ \frac{2}{π}\frac{1}{5}\cos(5f_ct)+\dots
	\right]
	\\
	&=
	\frac{A_c}{2}\cos 2 π f_ct
	+ \frac{1}{2}m(t)
	\\ &\hphantom{=}
	+ \frac{2}{π}A_c\cos 2πf_ct + \infoboxed{\frac{2}{π}m(t)\cos 2πf_c t}
	\\ &\hphantom{=}
	-\frac{2}{3π}A_c\cos 2π (3f_c) \cos 2πf_ct - \frac{2}{3π}
	m(t)\cos 2π(3f_ct)+\dots
\end{align*}

Αν σχεδιάσουμε τις συχνότητες που δίνει ο τύπος σε ένα διάγραμμα φάσματος:

\todo{Graph 16}

Παρατηρούμε ότι στο φάσμα υπάρχει το επιθυμητό διαμορφωμένο AM σήμα, όπως και ο
πολλαπλασιασμός του \( m(t) \) με το φέρον. Επομένως, με ένα ζωνοπερατό φίλτρο, μπορούμε
να πάρουμε από τις άπειρες συχνότητες μόνο το τελικό φάσμα:
\begin{align*}
u(t) = &\frac{A_c}{2}\left[ 1+\frac{4}{πA_c}m(t) \right]\cos 2π f_c t
\intertext{που αντιστοιχεί στον τύπο:}
A_c\left[ 1+k_am(t) \right] \cos 2πf_c
\end{align*}

δηλαδή το κύκλωμα γίνεται:
\todo{Circuit with BPF}

όπου το Band Pass Filter πρέπει να έχει κέντρο τη συχνότητα \( f_c \) και εύρος ζώνης
από \( f_c - w \) μέχρι \( f_c + w \).

Αυτό ήταν ένα παράδειγμα χρήσης \textit{μη γραμμικών στοιχείων} (δίοδος) για spectral
spread.

\end{document}
