\documentclass[11pt,a4paper,notitlepage,fleqn,final]{article}

\usepackage{amsmath}
\usepackage{amsfonts}
\usepackage{amssymb}
\usepackage{libs/commath2}
\usepackage[table]{xcolor}
\usepackage[hidelinks,draft=false]{hyperref}
\usepackage[skins,theorems]{tcolorbox}
\usepackage{titlesec}
\usepackage{tikz}
\usepackage{libs/circuitikz} % use our own recent version to make sure some bugs are fixed
\usepackage{pgfplots}
\usepackage{mathtools}
\usepackage[makeroom]{cancel}
\usepackage{mathrsfs}
\usepackage{wrapfig}
%\usepackage{subcaption}
%\usepackage{floatrow}
\usepackage{esint}
\usepackage{enumitem}
%\usepackage{bm}
\usepackage{relsize}
\usepackage{xfrac}
\usepackage{comment}
%\usepackage{siunitx}
%\usepackage{MnSymbol}
\usepackage[obeyDraft,disable]{todonotes}
%\usepackage[linesnumbered,lined]{algorithm2e}


\pgfplotsset{compat=1.13}
\usetikzlibrary{arrows.meta}
\usetikzlibrary{patterns}
\usetikzlibrary{decorations.pathmorphing,patterns}
\usetikzlibrary{decorations.markings}
\usetikzlibrary{backgrounds}
\usetikzlibrary{shapes.misc}
\usetikzlibrary{shapes.multipart}
\usetikzlibrary{shadows.blur}
\usetikzlibrary{fadings}
\usetikzlibrary{intersections}
\usetikzlibrary{arrows.meta}
\usetikzlibrary{calc}
\usetikzlibrary{matrix}
\usetikzlibrary{positioning}
\usetikzlibrary{shapes}
\usetikzlibrary{shadings}

\tcbuselibrary{breakable}

\tikzset{cross/.style={cross out, draw,
        minimum size=2*(#1-\pgflinewidth),
        inner sep=0pt, outer sep=0pt}}
\tikzset{
    mark position/.style args={#1(#2)}{
        postaction={
            decorate,
            decoration={
            	post length=1mm, % ??? Magic to fix "Dimension
            	pre length=1mm, % ???  too large" errors.
                markings,
                mark=at position #1 with \coordinate (#2);
            }
        }
    }
}
\makeatletter
\tikzset{
  use path for main/.code={%
    \tikz@addmode{%
      \expandafter\pgfsyssoftpath@setcurrentpath\csname tikz@intersect@path@name@#1\endcsname
    }%
  },
  use path for actions/.code={%
    \expandafter\def\expandafter\tikz@preactions\expandafter{\tikz@preactions\expandafter\let\expandafter\tikz@actions@path\csname tikz@intersect@path@name@#1\endcsname}%
  },
  use path/.style={%
    use path for main=#1,
    use path for actions=#1,
  }
}
\makeatother

\pgfmathdeclarefunction{sinc}{1}{%
	\pgfmathparse{abs(#1)<0.01 ? int(1) : int(0)}%
	\ifnum\pgfmathresult>0 \pgfmathparse{1}\else\pgfmathparse{sin(#1 r)/#1}\fi%
}
\pgfmathdeclarefunction{gauss}{2}{%
	\pgfmathparse{1/(#2*sqrt(2*pi))*exp(-((x-#1)^2)/(2*#2^2))}%
}

\usepackage[left=2cm,right=2cm,top=2cm,bottom=2cm]{geometry}

%\usepackage[no-math]{fontspec}
%\usepackage{fontspec}
\usepackage{mathspec}
%\usepackage{newtxtext,newtxmath}
%\usepackage{unicode-math}
%\setmainfont{texgyretermes-regular.otf}
%\setsansfont{texgyreheros-regular.otf}
%\newfontfamily\greekfont[Script=Greek]{Linux Libertine O}
%\newfontfamily\greekfontsf[Script=Greek]{Linux Libertine O}
\usepackage{polyglossia}
%\newfontfamily\greekfont[Script=Greek]{texgyretermes-regular.otf}
\newfontfamily\greekfontsf[Script=Greek]{texgyreheros-regular.otf}
\newfontfamily\greekfonttt[Script=Greek]{Latin Modern Mono}
%\usepackage[greek]{babel}
\setdefaultlanguage{greek}
\setotherlanguage{english}

%\usepackage[utf8]{inputenc}
%\usepackage[greek]{babel}


%\usepackage{tkz-euclide} % loads  TikZ and tkz-base
%\usetkzobj{angles} % important you want to use angles

\newlist{enumparen}{enumerate}{1}
\setlist[enumparen]{label=(\arabic*)}
\newlist{enumpar}{enumerate}{1}
\setlist[enumpar]{label=\arabic*)}

\newlist{enumgreek}{enumerate}{1}
\setlist[enumgreek]{label=\alph*.}
\newlist{enumgreekparen}{enumerate}{1}
\setlist[enumgreekparen]{label=(\alph*)}
\newlist{enumgreekpar}{enumerate}{1}
\setlist[enumgreekpar]{label=\alph*)}


\newlist{enumroman}{enumerate}{1}
\setlist[enumroman]{label=(\roman*)}

\newlist{enumlatin}{enumerate}{1}
\setlist[enumlatin]{label=(\alph*)}

\newlist{invitemize}{itemize}{1}
\setlist[invitemize]{noitemsep,label=}

\usepackage{letltxmacro}

\LetLtxMacro\OriginalLongrightarrow\Longrightarrow
\LetLtxMacro\OriginalLongleftarrow\Longleftarrow

% Implement new macros
% --------------------
\usepackage{trimclip}
\DeclareRobustCommand\Longrightarrow{\NewRelbar\joinrel\Rightarrow}
\DeclareRobustCommand\Longleftarrow{\Leftarrow\joinrel\NewRelbar}

\makeatletter
\DeclareRobustCommand\NewRelbar{%
  \mathrel{%
    \mathpalette\@NewRelbar{}%
  }%
}
\newcommand*\@NewRelbar[2]{%
  % #1: math style
  % #2: unused
  \sbox0{$#1=$}%
  \sbox2{$#1\Rightarrow\m@th$}%
  \sbox4{$#1\Leftarrow\m@th$}%
  \clipbox{0pt 0pt \dimexpr(\wd2-.6\wd0) 0pt}{\copy2}%
  \kern-.2\wd0 %
  \clipbox{\dimexpr(\wd4-.6\wd0) 0pt 0pt 0pt}{\copy4}%
}
\makeatother


\makeatletter
\pgfdeclareradialshading[tikz@ball]{ball}{\pgfqpoint{0bp}{0bp}}{%
	color(0bp)=(tikz@ball!50!white);
	color(10bp)=(tikz@ball!50!white);
	color(15bp)=(tikz@ball!70!black);
	color(20bp)=(black!70);
	color(30bp)=(black!70)}%
\makeatother


\makeatletter
\let\anw@true\anw@false

%\newcommand{\attnboxed}[1]{\textcolor{red}{\fbox{\normalcolor\m@th$\displaystyle#1$}}}
\makeatother
\tcbset{highlight math style={enhanced,colframe=red,colback=white,%
        arc=0pt,boxrule=1pt,shrink tight,boxsep=1.5mm,extrude by=0.5mm}}
\newcommand{\attnboxed}[1]{\tcbhighmath[colback=red!5!white,drop fuzzy shadow,arc=0mm]{#1}}
\newcommand{\infoboxed}[1]{%
	\tcbhighmath[colframe=blue!50!white,colback=blue!5!white,arc=0mm]{#1}}
\titleformat{\section}{\bf\Large}{Κεφάλαιο \thesection}{1em}{}
\newtcolorbox{attnbox}[1]{colback=red!5!white,%
    colframe=red!75!black,fonttitle=\bfseries,title=#1}
\newtcbox{quickattnbox}[1]{colback=red!5!white,%
	colframe=red!75!black,fonttitle=\bfseries,title=#1}
\newtcolorbox{infobox}[1]{colback=blue!5!white,%
    colframe=blue!75!black,fonttitle=\bfseries,title=#1}

\AtBeginDocument{%
\let\arg\relax
\let\Re\relax
\let\Im\relax
\DeclareMathOperator{\arg}{Arg}
\DeclareMathOperator{\Re}{Re}
\DeclareMathOperator{\Im}{Im}
}
\DeclareMathOperator{\sinc}{sinc}
\DeclareMathOperator{\sgn}{sgn}
\DeclareMathOperator{\erf}{erf}
\DeclareMathOperator{\cov}{cov}

\newif\ifhidetikz
\hidetikzfalse
%\hidetikztrue   % <---- comment/uncomment that line

\ifhidetikz

\let\oldtikzpicture\tikzpicture
\let\oldendtikzpicture\endtikzpicture

\renewenvironment{tikzpicture}{
    \tiny
    \tt
    \color{blue}
    \newcommand{\draw}{\textit{draw}}
    \newcommand{\filldraw}{\textit{filldraw}}
    %\newcommand{\x}{\textit{x}}
    %\newcommand{\p}{\textit{x}}
    \newcommand{\x1}{\textit{x1}}
    \newcommand{\y1}{\textit{y1}}
    \newcommand{\p1}{\textit{p1}}
}{
}
\newenvironment{axis}{
    \newcommand{\addplot}{\textit{addplot}}
}{
}
\fi

% Global amount of samples
% Set to a higher value (e.g. 200) for nicer graphs
% Set to a low value (e.g. 10) for performance
\newcommand*{\gsamples}{70}

% Equals command as a workaround for CircuiTikZ bug
% not allowing the = sign in labels
\newcommand*{\equals}{=}

\newcommand{\nesearrow}{%
	\,%
	\smash{\raisebox{-1.1ex}
		{$%
			\stackrel{\displaystyle\nearrow}{\displaystyle\searrow}%
			$}}%
}
\newcommand{\degree}{^{\circ}} % not great
\newcommand\numberthis{\addtocounter{equation}{1}\tag{\theequation}} % add an equation number to a number-less math environment

\newtcbtheorem[number within=section]{theorem}{Θεώρημα}%
{colback=green!5,colframe=green!35!black,colbacktitle=green!35!black,fonttitle=\bfseries,enhanced,attach boxed title to top left={yshift=-2mm,xshift=-7mm},width=.9\textwidth,arc=.7mm}{th}
\newtcbtheorem[number within=section]{defn}{Ορισμός}%
{colback=blue!5,colframe=cyan!35!black,colbacktitle=blue!35!black,fonttitle=\bfseries,enhanced,attach boxed title to top left={yshift=-2mm,xshift=-2mm}}{def}
\newtcbtheorem[number within=section]{exercise}{Άσκηση}%
{colback=gray!3,colframe=gray!35!black,colbacktitle=gray!35!black,fonttitle=\bfseries,enhanced,attach boxed title to top left={yshift=-2mm,xshift=-2mm}}{exc}




\title{Στοχαστικό Σήμα
	\\
	{
	\normalsize Σημειώσεις από τις παραδόσεις
	}}
\date{2017
	\\
	{
	\small Τελευταία ενημέρωση: \today
	}
	}
\author{
	Για τον κώδικα σε \LaTeX, ενημερώσεις και προτάσεις:
\\
 \url{https://github.com/kongr45gpen/ece-notes}}

\setmainfont{Linux Libertine O}
\setsansfont{Ubuntu}
%\newfontfamily\greekfont[Script=Greek]{Linux Libertine O}
%\newfontfamily\greekfontsf[Script=Greek]{Linux Libertine O}
\usepackage{polyglossia}
\newfontfamily\greekfont[Script=Greek,Scale=0.95]{GFS Artemisia}


\begin{document}
	\maketitle

	\tableofcontents

	\vspace{50pt}

    \href{mailto:dimakis@auth.gr}{dimakis@auth.gr}
    
    Ένα email με το όνομά μας και κάπου να γράφει στοχαστικό
    σήμα μέχρι το τέλος της εβδομάδας.
    
    \textbf{Στοχαστικό σήμα} - δεν σημαίνει στοχάζομαι
    (σκέφτομαι), αλλά
    τυχαίο σήμα
    
    \paragraph{Ντετερμινιστικό σήμα}
    \( A_c\cos(2\pi f t + \sfrac{\pi}{2}) \)
    
    \begin{tikzpicture}[scale=1.2]
    
    
    \draw (0,-1) -- (0,1);
    \draw[->] (-0.5,0) -- (2.6,0);
    
    \draw[very thick,xscale=0.5,yscale=0.7,samples=50,domain=0:5,smooth,variable=\x,blue]
    plot ({\x},{sin(pi*\x r)});
    \end{tikzpicture}
    
    Για κάθε χρόνο \(t\) μπορούμε να βρούμε την τιμή του
    σήματος.
    
    Αν γνωρίζουμε τα πάντα για
    την αρχική κατάσταση ενός συστήματος
    (π.χ. του σύμπαντος), μπορούμε να προβλέψουμε
    (ίσως με δυσκολία) κάθε επόμενη κατάστασή του.
    
    Οι περισσότερες διαδικασίες στη φύση είναι τυχαίες.
    
    \paragraph{Διαδικαστικά}
    Το μάθημα γίνεται σε δύο τμήματα μόνο από τον κ. Δημάκη. Το
    ένα τμήμα της Τρίτης θα μεταφερθεί στο μεσημέρι της Πέμπτης.
    
    Ασκήσεις και λύσεις τους θα αποστέλλονται στο e-mail.
    
    \paragraph{}
    \begin{itemize}
    	\item Ένα σκυλί δάγκωσε έναν άνθρωπο στο δρόμο
    	\( \leftarrow \) δεν είναι πληροφορία, επειδή έχει
    	μεγάλη πιθανότητα να συμβεί.
    	\item Ένας άνθρωπος δάγκωσε έναν σκύλο στο δρόμο
    	\( \leftarrow \) είναι πληροφορορία, επειδή έχει μικρή
    	πιθανότητα.
    \end{itemize}
    
    \section{Θεωρία Πιθανοτήτων}
    Ασχολείται με τυχαία γεγονότα, δηλαδή γεγονότα το αποτέλεσμα
    των οποίων:
    \begin{itemize}
    	\item Δεν μπορούμε να προβλέψουμε
    	\item Δεν μπορούμε να καταλάβουμε ή
    	\item Έχει πάρα πολλά στοιχεία που το επηρεάζουν
    \end{itemize}
    και πειράματα τα οποία όταν επαναλαμβάνουμε βγάζουν
    διαφορετικό αποτέλεσμα στις ίδιες συνθήκες (π.χ. ρίψη
    ζαριού, σε αντίθεση με το μήλο του Νεύτωνα).
    
    Επομένως στα πειράματα τύχης \textbf{δεν μπορεί να
    προβλεφθεί ακριβώς η έξοδος}.

	\paragraph{Ορισμός Πιθανότητας από Laplace}
	\hspace{0pt}

	    \begin{tikzpicture}[scale=1.2]
	    
	    \filldraw[fill=green!20,draw=black!70!green,very thick] plot [smooth cycle]
	    coordinates {(0.8,1) (1.5,1.7) (2,1.5) (2.6,1.5) (2.5,0.8) (1.9, 0.6) (1.6,0.3) (1.2,0.2) (0.6, 0.7)};
	    
	    \draw (0,0) rectangle (2.7,2.2) node[below left] {$N$};
	    \filldraw (1,1) circle(1pt) node[above right] {$1$};
	    \filldraw (1.5,1.2) circle(1pt) node[above right] {$2$};
	    \filldraw (0.2,0.4) circle(1pt) node[above right] {$3$};
	    \filldraw (2,1.1) circle(1pt) node[above right] {$4$};
	    \filldraw (1.3,0.3) circle(1pt) node[above right] {$5$};
	    \filldraw (2.1,0.4) circle(1pt) node[below right] {$6$};
	    \filldraw (0.3,1.7) circle(1pt) node[above right] {$7$};
	    \filldraw (0.5,1.1) circle(1pt) node[above right] {$8$};
	    
	    \draw[->,blue!30!black] (0.2,1.7) to[bend left=45] (-0.5, 2.2) node[above] {ισοπίθανα ενδεχόμενα};
	    \draw[->,green!30!black] (0.8,0.35) to[bend right=85] (0.5, -1) node[right] {γεγονός};
	    
	    \end{tikzpicture}
	\[
	P(A) = \frac{N_A}{N}
	\]
	Ο ορισμός έχει τα προβλήματα:
	\begin{itemize}
		\item Δεν αντιστοιχεί σε πραγματικό πείραμα τύχης
		\item Είναι κυκλικός (ορίζει την πιθανότητα με βάση
		ισο\underline{πίθανα} ενδεχόμενα)
	\end{itemize}
	\paragraph{Ορισμός Πιθανότητας von Mises}
	Εκτελούμε ένα πείραμα τύχης με \(N\) επαναλήψεις. Αν ένα
	γεγονός συμβεί \(N_A\) φορές, τότε
	\[
	P(A) = \lim_{N\to  \infty } \frac{N_A}{N}
	\]
	Όμως είναι δύσκολο να χρησιμοποιηθεί επειδή αντιστοιχεί σε
	πραγματικό πείραμα.
	\paragraph{Ιδιότητες ιδιότητας που προκύπτουν από
		τους παραπάνω ορισμούς}
	\begin{enumroman}
		\item \( 0 \leq P(A) \leq 1 \)
		\item \( S \text{ δειγματικός χώρος } \quad 
		P(S) = 1 \)
		\item \( P(A\cup B) = P(A+B) = P(A \text{ ή } B)
		= P(A) + P(B) \qquad (A,B \text{ ασυμβίβαστα})
		 \)
	\end{enumroman}
	
	\subsubsection{Αξιωματικός ορισμός της πιθανότητας
		Wiener - Kolmogorov}
	
	\textbf{Πείραμα} είναι κάθε νοητική ή φυσική διεργασία,
	με την οποία
	συνδέουμε μία πράξη με κάποιο αποτέλεσμα.
	
	Κάθε εκτέλεση του πειράματος ονομάζεται \textbf{δοκιμή}
	
	Ένα πείραμα έχει έναν δειγματικό χώρο με όλα τα δυνατά
	αποτελέσματά του:
	\[
	S = \left\lbrace J_1,J_2,\dots,J_n \right\rbrace
	\]
	
	Ένα \textbf{γεγονός} είναι ένα από τα \( 2^n \) δυνατά
	υποσύνολα του \(S\): \[ A \subseteq S \]
	\paragraph{Ορισμός}
	\( A \to P(A) \)	
	\begin{enumroman}
		\item \( P(A) \geq 0 \)
		\item \( P(S) = 1 \)
		\item αν \( A \cap B = 0 \), τότε \( P(A \cup B)
		= 0 \)
	\end{enumroman}
	
	Ο ορισμός αυτός δεν μας δίνει συγκεκριμένη τεχνική υπολογισμού
	πιθανοτήτων, επομένως μπορούμε να χρησιμοποιήσουμε οποιαδήποτε
	τεχνική (π.χ. Laplace) θέλουμε.
	
	
	\def\boundingbox{(-2cm,-2cm) rectangle (4cm,2cm)}
	\colorlet{circle edge}{blue!50}
	\colorlet{circle area}{blue!20}
	\tikzset{filled/.style={fill=circle area, draw=circle edge, thick},
		outline/.style={draw=circle edge, thick}}
	
	\paragraph{Ιδιότητες}
	\begin{enumpar}
		\item \( A = B \implies P(A)=P(B) \) \\
		\item \( P(\emptyset) = 0 \)
		\item Αν \( A\cdot\bar A = \emptyset \quad
		A+\bar A = S \), τότε:
		\begin{align*}
			P(A+\bar A) = P(S) = 1 & \quad P(\bar A) = 1 - P(A) \\
			P(A) + P(\bar A) = 1
		\end{align*}
		\item Αν \( A \cdot B \neq \emptyset \), τότε
		\[
		P(A+B) = P(A) + P(B) - P(A\cdot B) \leq 1
		\]
		\begin{tikzpicture}[scale=0.65]
		\def\firstcircle{(0,0) circle (1.5cm)}
		\def\secondcircle{(0:2cm) circle (1.5cm)}
		\begin{scope}
		\clip \firstcircle;
		\fill[filled] \secondcircle;
		\end{scope}
		\draw[outline] \firstcircle node {$A$};
		\draw[outline] \secondcircle node {$B$};
		\draw \boundingbox;
		\draw (1,0) node[scale=0.6] {$A\cap B$};
		\end{tikzpicture}
		
		
		\subparagraph{Απόδειξη}
		\begin{align*}
			A+B &= A+ \bar A \cdot B \\
			B &= A\cdot B + \bar A \cdot B \\
			P(A+B) &= P(A) + P(\bar A \cdot B) \\
			P(B) &= P(A\cdot B) + P(\bar A \cdot B)\\
			P(A+B) &= P(B)+P(A) -P(AB)
		\end{align*}
		\item \( A \subseteq B \implies P(A) \leq P(B) \)

		\begin{tikzpicture}[scale=.7]
		\draw[outline] (1,0) circle (1.5cm) node {$A$};
		\draw[outline] (0.5,-0.75) circle (0.4cm) node {$B$};
		\draw \boundingbox;
		\end{tikzpicture}
		
		\begin{align*}
			B &= A + \bar A \cdot B \\
			P(B) &= P(A) + P(\bar A\cdot B) \geq P(A)
		\end{align*}
		
		\item Υπό συνθήκη πιθανότητα
		\[
		P(A/B) = \frac{P(AB)}{P(B)}
		\]
		
		\begin{tikzpicture}
		\filldraw (-0.1,0) circle (1.5pt);
		\draw (0,0) node[below right,yshift=2.5mm,rectangle,align=left]
		{$A\cdot B=\emptyset$\\$P(A/B)=0$};
		\begin{scope}[xshift=5cm,yshift=3mm]
		\draw (0,0) rectangle (-1.5,-1);
		\draw[fill=blue,fill opacity=0.2] (-0.375,-0.5) circle (2.5mm);
		\draw[fill=blue,fill opacity=0.2] (-1.125,-0.5) circle (2.5mm);
		\draw (-0.8,-0.2) node[scale=.8] {$A$};
		\draw (-0.13,-0.2) node[scale=.8] {$B$};
		\draw (-1.3,0) arc(360:270:0.2) node[xshift=0.8mm,yshift=1.1mm,scale=.5] {$S$};
		\end{scope}
		
		\begin{scope}[yshift=-1.5cm]
		\filldraw (-0.1,0) circle (1.5pt);
		\draw (0,0) node[below right,yshift=2.5mm,rectangle,align=left]
		{$A\subset B$\\$A\cdot B = A$\\$P(A/B)=\frac{P(A)}{P(B)}$};
		\begin{scope}[xshift=5cm,yshift=3mm]
		\draw (0,0) rectangle (-1.5,-1);
		\draw[fill=blue,fill opacity=0.2] (-0.75,-0.5) circle (4mm);
		\draw[fill=blue,fill opacity=0.2] (-0.75,-0.5) circle (1.5mm);
		\draw (-0.75,-0.5) node[scale=.7] {$A$};
		\draw (-0.25,-0.2) node[scale=.8] {$B$};
		\draw (-1.3,0) arc(360:270:0.2) node[xshift=0.8mm,yshift=1.1mm,scale=.5] {$S$};
		\end{scope}
		\end{scope}
		
		\begin{scope}[yshift=-3.5cm]
		\filldraw (-0.1,0) circle (1.5pt);
		\draw (0,0) node[below right,yshift=2.5mm,rectangle,align=left]
		{$A\supset B$\\$A\cdot B = B$\\$P(A/B)=\frac{P(B)}{P(B)}=1$};
		\begin{scope}[xshift=5cm,yshift=3mm]
		\draw (0,0) rectangle (-1.5,-1);
		\draw[fill=blue,fill opacity=0.2] (-0.75,-0.5) circle (4mm);
		\draw[fill=blue,fill opacity=0.2] (-0.75,-0.5) circle (1.5mm);
		\draw (-0.75,-0.5) node[scale=.7] {$B$};
		\draw (-0.25,-0.2) node[scale=.8] {$A$};
		\draw (-1.3,0) arc(360:270:0.2) node[xshift=0.8mm,yshift=1.1mm,scale=.5] {$S$};
		\end{scope}
		\end{scope}
		
		\begin{scope}[xshift=5cm,yshift=2cm]
		\draw (0,0) rectangle (-1.5,-1);
		\draw[fill=blue,fill opacity=0.2] (-0.6,-0.5) circle (3mm);
		\draw[fill=blue,fill opacity=0.2] (-0.9,-0.5) circle (3mm);
		\draw (-1.33,-0.5) node[scale=.8] {$A$};
		\draw (-0.17,-0.5) node[scale=.8] {$B$};
		\draw (-1.3,0) arc(360:270:0.2) node[xshift=0.8mm,yshift=1.1mm,scale=.5] {$S$};
		\end{scope}
		\end{tikzpicture}
		
		\item \begin{tikzpicture}[baseline=(current bounding box.north)]
		\draw (0,0) rectangle (4,2);
		
		\draw[orange,fill=orange!50!white] (2,1) circle (0.8cm);
		
		\draw plot [smooth,tension=1] coordinates {(0.6,2) (1.5,1.6) (2,1.3) (2.5,1.6) (3.4,2)};
		\draw plot[smooth,tension=1.5] coordinates {(0.4,0) (2,0.9) (3.6,0)};
		\draw plot[smooth,tension=2] coordinates {(2,1.3) (2.1,1.1) (2,0.9)};
		
		\draw[brown!50!black] (0,1.7) arc (270:360:0.3);
		\draw[brown!50!black] (0.12,1.87) node {$\mathsmaller{S}$};
		
		\draw (2,1.7) node {$A_1$};
		\draw (3.4,1.1) node {$A_2$};
		\draw (2,0.4) node {$A_3$};
		\draw (0.8,1.1) node {$A_4$};
		
		\draw[orange!70!black] (2.85,0.35) node {$B$};
		
		\end{tikzpicture}

		Έστω ένας διαμερισμός του δειγματικού χώρου και ένα
		ενδεχόμενο \( B \) αυτού, δηλαδή:
		\begin{gather*}
			A_1+A_2+A_3+A_4 = S \\
			A_i A_j = \emptyset \\[.3ex]
			(B \cdot A_i) \cdot (B \cdot A_j)
		\end{gather*}
		
		Τότε:
		\begin{align*}
			B &= B\cdot S = B(A_1+A_2+A_3+A_4) \\
			B &= B\cdot A_1 + B\cdot A_2 + B\cdot A_3 + B\cdot A_4 \\
			P(B) &= P(BA_1) + P(BA_2) + P(BA_3) + P(BA_4)
			\intertext{\( P(B/A_i) = \frac{P(B)\cdot A_i}{P(A_i)} \)}
			\intertext{\( P(BA_i) = P(B/A_i)\cdot P(A_i) \)}
			P(B) &= P(B/A_1)P(A_1)+P(B/A_2)P(A_2)+P(B/A_3)P(A_3)
			+P(B/A_4)P(A_4)
		\end{align*}
		
		Το παραπάνω (\textbf{θεώρημα ολικής πιθανότητας})
		εκφράζεται γενικά ως εξής:
		\[
		P(B) = \sum_{i=1}^{m} P(A_i)P(B/A_i)
		\]
		\item
		Από την παραπάνω ιδιότητα έχουμε:
		\begin{align*}
			P(A_iB) &= P(B)P(A_i/B) = P(A_i)P(B/A_i) \\
			P(A_i/B) &= \frac{P(B/A)P(A_i)}{P(B)} \\
			\Aboxed{
     			P(A_i/B) &=
     			\frac{P(B/A_i)P(A_i)}{\sum_{i=1}^{n}P(A_i)P(B/A_i)}
     		} \qquad \text{\textbf{Θεώρημα Bayes}}
		\end{align*}
	\end{enumpar}
	
	\paragraph{Παράδειγμα}
	\hspace{0pt}
	
	\begin{tikzpicture}[every node/.style={scale=0.8}]
	\draw (0,0) rectangle node[midway] {Πομπός} (2,1);
	\draw (2.5,0.2) rectangle node[midway] {Κανάλι} ++(1.2,0.6);
	\draw (4.2,0) rectangle node[midway] {Δέκτης} ++(2,1);
	
	\draw[->] (2,0.5) --++ (0.5,0);
	\draw[->] (3.7,0.5) --++ (0.5,0);
	
	\draw (0,-0.5) node[right] {$A=\lbrace A_1,A_2,A_3,A_4\rbrace$};
	\end{tikzpicture}
	
	Η είσοδος μπορεί να είναι ένα αλφάβητο \( A \):
	\begin{align*}
		A &= \left\lbrace A_1,A_2,A_3,A_4 \right\rbrace
		\intertext{με πιθανότητες για το καθένα:}
		\Pi_A &= \left\lbrace 
		\Pi_{A_1},\Pi_{A_2},\Pi_{A_3},\Pi_{A_4}
		 \right\rbrace
	\end{align*}
	
	Το κανάλι ποτέ δεν θα μεταφέρει ακριβώς την πληροφορία, αλλά θα
	την μεταβάλλει με κάποιον τρόπο (π.χ. θόρυβος). Αν \( B \) είναι το
	αλφάβητο εξόδου, έχουμε:
	\begin{align*}
	B &= \left\lbrace B_1,B_2,B_3,B_4 \right\rbrace \\
	\Pi_B &= \left\lbrace 
	\Pi_{B_1},\Pi_{B_2},\Pi_{B_3},\Pi_{B_4}
	\right\rbrace
	\end{align*}
	
	Για να μελετήσουμε το κανάλι, μπορούμε να βάλουμε κάποιον να μετράει
	τις πιθανότητες εμφάνισης κάποιας εξόδου με δεδομένη είσοδο (για
	παράδειγμα, να στείλουμε 100\,000 φορές την είσοδο \( A_1 \), και
	να μετρήσουμε πόσες φορές εμφανίζεται η κάθε έξοδος). Έτσι έχουμε
	τις υπό συνθήκη (\textbf{a posteriori}) πιθανότητες:
	\[
	\Pi_{B/A} = \begin{cases}
	\Pi(B_1/A_1),\ \Pi(B_1/A_2),\ \Pi(B_1/A_3), \dots \\
	\Pi(B_2/A_1),\ \Pi(B_2/A_2),\ \dots \\
	\vdots
	\end{cases}
	\]
	
	Από τις παραπάνω ιδιότητες έχουμε:
	\[
	P(B_1) = P(A_1)P(B_1/A_1)
	+P(A_2)P(B_1/A_2)
	+P(A_3)P(B_1/A_3)
	+P(A_4)P(B_1/A_4)
	\]
	Άρα:
	\[
%	\mathlarger{
		P(A_i/B_j) = \frac{P(B_j/A_i)P(A_i)}{
			\sum_{i=1}^n P(A_i)P(B_j/A_i)
			}
%		}
	\]
	
	Έτσι, για κάθε έξοδο, μπορούμε να υπολογίσουμε την πιθανότητα
	να προέρχεται από μια είσοδο.
	
	Αν π.χ. έχουμε: \begin{align*}
		P(A_1/B_4) &= 70\% \\
		P(A_2/B_4) &= 15\% \\
		P(A_3/B_4) &= 8\% \\
		P(A_4/B_4) &= 7\%
	\end{align*}
	μπορούμε να φανταστούμε ότι η έξοδος \( B_4 \) αντιστοιχεί με
	μεγαλύτερη πιθανότητα σε είσοδο \( A_1 \), σύμφωνα με το κριτήριο
	\textbf{MAP} (\textit{maximum a-posteriori probability}).
	
	Είναι συχνό βέβαια να μη γνωρίζουμε τις \textbf{a-posteriori}
	πιθανότητες \( P(A_i) \), οπότε μπορούμε να καταφύγουμε σε κόλπα,
	όπως να θεωρήσουμε ότι τα ενδεχόμενα της εισόδου είναι ισοδύναμα.
	
	\paragraph{Σήματα}
	\hspace{0pt}
	
	\begin{tikzpicture}[scale=1.3]
	\fill[cyan!40,path fading=north] (-1,1.125) rectangle (8,2);
	\fill[cyan!40,path fading=south] (-1,-1.125) rectangle (8,-2);
	\fill[green,opacity=.15] (-1,1.125) rectangle (8,0);
	\fill[yellow,opacity=.2] (-1,-1.125) rectangle (8,0);
	
	\draw[dashed] (-1,1.125) -- (8,1.125);
	\draw[dashed] (-1,-1.125) -- (8,-1.125);
	
	\draw (-1,0) -- (7.2,0);
	\draw (0,-2) -- (0,2);
	\def\1{1.5};
	\def\2{0.75}
	\def\3{-0.75}
	\def\4{-1.5}
	
	\draw (-0.1,\1) node[left,scale=.7] {$1$} -- (0.1,\1);
	\draw (-0.1,\2) node[left,scale=.7] {$2$} -- (0.1,\2);
	\draw (-0.1,\3) node[left,scale=.7] {$3$} -- (0.1,\3);
	\draw (-0.1,\4) node[left,scale=.7] {$4$} -- (0.1,\4);
	
	\draw (-1.5,\1) node {$a$};
	\draw (-1.5,\2) node {$b$};
	\draw (-1.5,\3) node {$c$};
	\draw (-1.5,\4) node {$d$};
	\draw (-0.8,\1) node[scale=.6] {$\mathtt{11}$};
	\draw (-0.8,\2) node[scale=.6] {$\mathtt{10}$};
	\draw (-0.8,\3) node[scale=.6] {$\mathtt{01}$};
	\draw (-0.8,\4) node[scale=.6] {$\mathtt{00}$};
	
	\draw[very thick,black] plot [const plot] coordinates
	{(0,\2) (1,\4) (2,\1) (3,\2) (4,\3) (5,\2) (6,\4) (7,\1) (8.1,\1)}
	node[above right] {αρχικό σήμα};
	
	\foreach \x in {1,2,3,...,5} {
		\draw[draw=gray] (\x,-0.1) -- (\x,0.1) node[midway,below left,scale=.6] {$\x T$};
	}
	\draw (6,0) node[below left,scale=.6] {$\cdots$};
	
	\draw[very thick,orange!70!brown] plot [smooth,tension=.2] coordinates
	{(0,0) (0.15,0.4) (0.3,1.05) (0.4,1.1) (0.5,1) (0.6,1.10)
		(0.7,1+rand/5) (1,rand/10) (1.2,-1.5+rand/3) (1.4,-1.5+rand/4) (1.5,-1.5+rand/5)
		(1.6,-1.5+rand/4) (1.8,-1.5+rand/2) (1.9,-0.75+rand/2) (2,rand/5) (2.1,0.7+rand/3)
		(2.2,1.2+rand/3) (2.4,1.5+rand/5) (2.6,1.5+rand/5) (2.7,1.2+rand/6) (2.9,1+rand/5)
		(3.1, 0.7+rand/4) (3.3, 0.8+rand/5) (3.4, 0.8+rand/4) (3.6,0.5+rand/3) (3.8,0.9+rand/4)
		(4,1+rand/4) (4.1,rand/4) (4.2, 0.4+rand/4) (4.3, -0.3+rand/3) (4.4, -0.5+rand/4)
		(4.5,-0.5+rand/5) (4.6,-0.5+rand/6) (4.8,-0.3+rand/4) (5,0+rand/4) (5.2,0.3+rand/4)
		(5.4,0.6+rand/4) (5.6,1+rand/4) (5.8,0.8+rand/4) (5.9,rand/4) (6,rand/7)
		(6.2,-0.2+rand/4) (6.4,-1.3+rand/4) (6.6, -0.5+rand/5) (6.8,0.5+rand/3)
		(7,1+rand/3) (7.2,2+rand/4) (7.4,1.6+rand/4) (7.6,1.3+rand/6) (7.8,1.3+rand/8)
		(8,1.3)} node[below right] {σήμα με θόρυβο};
	
	\draw[very thick,green!50!black] plot [smooth] coordinates
	{(0,0) (0.3,1) (0.7,1) (1.5,-1.5) (2.5,1.5) (3.6,0.6) (3.9,0.8)
		(4.6,-0.5) (5.6,0.85) (6.4,-1.2) (7.1,1.7) (7.7,1.5) (8,1.4)}
	node[right] {σήμα χωρίς υψηλές συχνότητες};
	\end{tikzpicture}
	
	Ένα πραγματικό σήμα έχει θόρυβο, όπως φαίνεται στο διάγραμμα. Για
	να εξάγουμε την αρχική τιμή του σήματος από την είσοδο με θόρυβο,
	μπορούμε να χωρίσουμε τη ζώνη του πλάτους σε διαφορετικές περιοχές,
	και κάθε μία από αυτές να την αντιστοιχούμε σε μία τιμή εξόδου,
	\( \alpha, \beta, \gamma \) ή \( \delta \). Όπως και πριν, κάθε
	είσοδος αντιστοιχεί με διαφορετική πιθανότητα σε κάθε έξοδο:
	\begin{center}
	\begin{tikzpicture}[xscale=1.3]
	\coordinate (a) at (2,6);
	\coordinate (b) at (2,4);
	\coordinate (c) at (2,2);
	\coordinate (d) at (2,0);
	\coordinate (1) at (0,6);
	\coordinate (2) at (0,4);
	\coordinate (3) at (0,2);
	\coordinate (4) at (0,0);
	
	\draw[green!50!blue!50!white,->] (1.7,6.02) to[bend left] (3.5,7)
	node[right] {$P(1/a)$};
	
	\draw[green!50!blue!50!white,->] (1.58,5.65) to[bend right] (3.2,6)
	node[right] {$P(2/a)$};
	
	\draw[green!50!blue!50!white,->] (1.20,4.56) to[bend left=20] (3,5.2)
	node[right] {$P(3/a)$};
	
	\draw[green!50!blue!50!white,->] (1.36,4.37) to[bend right=20] (3,4.4)
	node[right] {$P(4/a)$};
	
	\foreach \x in {2,3,4}
	\foreach \y in {a,b,c,d}
	\draw (\x)  to[bend left={5-\x}] (\y);
	\draw (1) to[bend left=4] node[midway,above] {$P(a/1)$} (a);
	\draw (1) to[bend left=4] node[midway,above,sloped] {$P(\beta/1)$}(b);
	\draw (1) to[bend left=4] node[midway,above,sloped] {$\cdots$} (c);
	\draw (1) to[bend left=4] (d);
	
	\foreach \x in {1,2,3,4}
	\fill (\x) circle (2pt) node[left] {$P_\x \qquad \x$};
	\fill (a) circle (2pt) node[right] {$a$};
	\fill (b) circle (2pt) node[right] {$\beta$};
	\fill (c) circle (2pt) node[right] {$\gamma$};
	\fill (d) circle (2pt) node[right] {$\delta$};
	\end{tikzpicture}
		\end{center}

	\paragraph{Ανεξαρτησία}
	Δύο γεγονότα λέγονται ανεξάρτητα ανν:
	\[
	P(A\cdot B) = P(A)P(B)
	\]
	
	Ομοίως ορίζεται η ανεξαρτησία για περισσότερα ενδεχόμενα, ανν:
	\[
	P(A_1A_2\cdots A_n) = P(A_1)P(A_2)\cdots P(A_m)
	\]
	
	\subsubsection{Πιθανοτικό Μοντέλο}
	Ένας δειγματικός χώρος \( S \) έχει μεγάλο αριθμό δυνατών υποσυνόλων,
	και στο καθένα από αυτά αντιστοιχεί μία πιθανότητα. Για εμάς μπορεί
	να είναι εύκολο/δυνατό να βρούμε την πιθανότητα μόνο για μερικά
	από αυτά τα υποσύνολα, επομένως ονομάζουμε αυτά γεγονότα, και τα
	τοποθετούμε σε μια τάξη συνόλων \( \mathcal F \), που θα ικανοποιούν:
	
	\begin{enumroman}
		\item \( \emptyset \in \mathcal F,\
		S \in \mathcal{F}
		 \)
		\item \( A \in \mathcal F \) και το \( \bar A \in \mathcal F \)
		\item \( A \in \mathcal F \) και \( B \in \mathcal F \)
		\( \implies A + B \in \mathcal F \) \quad
		(τότε θα έχουμε και \( A\cdot B,\ A-B \in \mathcal F \))
	\end{enumroman}
	
	Τότε προκύπτει, αν \( A_1,A_2,\dots,A_n \in \mathcal F \):
	\begin{gather*}
		A_1+A_2+\dots +A_n \in \mathcal F \\
		A_1\cdot A_2 \cdots A_n \in \mathcal F
	\end{gather*}
	
	Το πεδίο \( \mathcal F \) ονομάζεται \textbf{πεδίο Borel}.
	
	Αν έχουμε τα \( \mathlarger{\mathlarger{(S,\mathcal F,P)}} \),
	δηλαδή το
	δειγματικό χώρο, ένα πεδίο Borel, και τις αντίστοιχες πιθανότητές
	του, τότε έχουμε ένα πλήρες πιθανοτικό μοντέλο.
	
	\paragraph{Σύνολο με πεπερασμένο αριθμό γεγονότων}
	\begin{align*}
		S &= \left\lbrace J_1,J_2,\dots,J_n \right\rbrace \\
		P_i \to \quad A_1 &= 
		\left\lbrace J_i \right\rbrace \in \mathcal F \\
		& P_1+P_2+\dots+P_N = 1 \\
		A &= \left\lbrace J_{k_1},J_{k_2},\dots,J_{k_i} \right\rbrace
		\quad i \leq n \qquad \text{(επιλέγουμε μερικά
			απλά ενδεχόμενα από τον χώρο ) S} \\
		P(A) &= P\left\lbrace J_{k_1} \right\rbrace +
		P\left\lbrace J_{k_2} \right\rbrace + \dots +
		P\left\lbrace J_{k_i} \right\rbrace \\ &=
		P_{k_1} + P_{k_2} + \dots + P_{k_n}
	\end{align*}
	
	\paragraph{Σύνολο με πραγματικούς αριθμούς}
	Όταν, για παράδειγμα, περιμένουμε το λεωφορείο στη στάση, ο χρόνος
	\( t \) που πρέπει να περιμένουμε μέχρι να έρθει είναι πραγματικός
	αριθμός:
	\begin{gather*}
		t_1 \leq t \leq t_2 \\ t_1,t_2 \in [0,+\infty)
	\end{gather*}
	
	Θα προσπαθήσω να δώσω μια πιθανότητα της μορφής 
	\( P(0 \leq t \leq t_1) \)
	\\σε κάθε διάστημα.
	
	Έστω λοιπόν μια \underline{\( f(t) \)} με \( f(t) \leq 0 \) και
	\( \displaystyle \int_0^\infty f(t) = 1 \). Τότε μπορώ να ορίσω:
	\[
	P\left\lbrace 0 \leq t \leq t_1 \right\rbrace =
	\int_0^{t_1} f(t)\dif t.
	\]
	
	Με αυτόν τον τρόπο μπορώ να ορίσω και πιθανότητες για \( t \) που
	δεν ξεκινούν από το 0, παρατηρώντας ότι:
	\begin{align*}
	0 \leq t \leq t_2 &= 0 \leq t \leq t_1 \ + \ t_1\leq t \leq t_2 \\
	P\left\lbrace 0\leq t \leq t_2 \right\rbrace &=
	P\left\lbrace 0 \leq t \leq t_1 \right\rbrace + P\left\lbrace 
	t_1 \leq t \leq t_2 \right\rbrace \\
	\int_0^{t_2} f(t)\dif t - \int_0^{t_1} f(t)|dif t &=
	P\left\lbrace t_1 \leq t \leq t_2 \right\rbrace \\
	P\left\lbrace t_1<t\leq t_2 \right\rbrace =
	\int_{t_1}^{t_2} f(t)\dif t
	\end{align*}

	\begin{tikzpicture}
	\draw[->] (-0.2,0) -- (3,0);
	
	\filldraw (0,0) circle (1pt) node[below] {$0$};
	\filldraw (1,0) circle (1pt) node[below] {$t_1$};
	\filldraw (2,0) circle (1pt) node[below] {$t_2$};
	
	\draw[thick,black!70!blue] (0,0) -- (0.25,0.3) -- (0.75,0.3) -- (1,0);
	\draw[thick,black!70!blue] (1,0) -- (1.25,0.3) -- (1.75,0.3) -- (2,0);
	\draw[thick,black!80!blue] (0,0) -- (0,0.5) -- (2,0.5) -- (2,0);
	\end{tikzpicture}

	Η \( f(t) \) επομένως είναι μια συνάρτηση γραμμικής πυκνότητας.
	
	\subparagraph{Πιθανότητα στιγμής}
	Πόση είναι η πιθανότητα για μια στιγμή, "μηδενικής" διάρκειας;
	
	\begin{align*}
		\lim_{\epsilon\to 0} P(t_2-\epsilon < t \leq t_2) &=
		\lim_{\epsilon\to 0} \int_{t_2-\epsilon}^{t_2} f(t)\dif t
		\\ &= \int_{t_2}^{t_2} f(t)\dif t = 0.
	\end{align*}
	
	\paragraph{Πιθανότητα πολλαπλών πειραμάτων}
	Έχουμε δύο πειράματα με διαφορετικούς δειγματικούς χώρους:
	\begin{align*}
		S_A &= \left\lbrace a_1,a_2,\dots,a_n \right\rbrace \\
		S_B &= \left\lbrace \beta_1,\beta_2,\dots,\beta_n \right\rbrace
	\end{align*}
	
	Αν πραγματοποιήσουμε και τα δύο πειράματα, παίρνουμε έναν δειγματικό
	χώρο που είναι το καρτεσινό γινόμενο του καθενός, το οποίο
	αποτελείται από διατεταγμένα ζεύγη γεγονότων:
	\[
	S = S_A \times S_B = \left\lbrace 
	(a_1,\beta_1),(a_1,\beta_2),\dots,(a_2,\beta_1),(a_2,\beta_2),\dots,
	(a_n,\beta_n)
	 \right\rbrace
	\]
	με παρόμοιο αποτέλεσμα αν τα \( S_A,S_B \) είναι σύνολα πραγματικών
	αριθμών.
	
	Για παράδειγμα, έχουμε:
	\begin{align*}
		A_i &= \left\lbrace a_1,a_2 \right\rbrace \in S_A \\
		B_j &= \left\lbrace \beta_3 \right\rbrace \in S_B \\
		C &= A_i \times B_j = \left\lbrace
		(a_1,\beta_3),(a_2,\beta_3) \right\rbrace \in S = S_A \times S_B
	\end{align*}
	
	Το ερώτημα που προκύπτει είναι, αν γνωρίζουμε τις πιθανότητες
	\( P(A_i), P(B_i) \), πώς μπορούμε να υπολογίσουμε την πιθανότητα
	των γεγονότων του \( C \).
	
	Αν τα δύο πειράματα είναι \textit{ανεξάρτητα}, τότε έχουμε:
	\[
	P(A_i,B_j) = P(A_i, B_j)
	\]
	
	Στην περίπτωση που δεν υπάρχει ανεξαρτησία, θα πρέπει να μελετηθεί
	εκ νεου ο δειγματικός χώρος \( S \) των δύο πειραμάτων. Αν όμως
	γνωρίζουμε τις πιθανότητες του \( S \), μπορούμε να υπολογίσουμε
	τις αρχικές (marginal) πιθανότητες:
	\begin{align*}
		P(A_i) &= \sum_j P(A_i, B_j) \\
		P(B_j) &= \sum_i P(A_i, B_j) \\
		P(A_i/B_j) &= \frac{P(A_iB_j)}{P(B_j)}
	\end{align*}
	
	\section{Τυχαίες Μεταβλητές}
	\subsection{Ορισμός}
	Ας κάνουμε ένα πείραμα. Στο πείραμα έχουμε τρία κουτιά. Σε κάθε
	κουτί υπάρχει μια άσπρη και μια μαύρη μπάλα, και τραβάω μία από
	κάθε κουτί.
	%TODO Graph 9
	\begin{align*}
		S_A &= \left\lbrace \varLambda, M \right\rbrace
		= \left\lbrace J_{A_1},J_{A_2} \right\rbrace \\
		S_B &= \left\lbrace \varLambda, M \right\rbrace
		= \left\lbrace J_{B_1},J_{B_2} \right\rbrace \\
		S_\varGamma &= \left\lbrace \varLambda, M \right\rbrace
		= \left\lbrace J_{\varGamma_1},J_{\varGamma_2} \right\rbrace
	\end{align*}
	\begin{align*}
		S &= S_A \times S_B \times S_\varGamma \\
		&= \left\lbrace \varLambda\varLambda\varLambda,
		\varLambda\varLambda M,\dots,MMM \right\rbrace
		\\ &= \left\lbrace J_1,J_2,\dots,J_8 \right\rbrace
	\end{align*}
	
	\[
	P(J_i) = P(J_{A_i}) P(J_{B_i}) P(J_{\varGamma_i})
	\]
	
	Έστω ότι αυτό είναι ένα παιγνίδι με πόντους, όπου οι λευκές μπάλες
	δίνουν 1 πόντο, ενώ οι μαύρες αφαιρούν 1.
	
	\[
	\begin{array}{c|c|c|r}
		i &              J_i               &  P(J_i)  & 3 \\ \hline
		1 & \varLambda\varLambda\varLambda &   p^3    & 1 \\ \hline
		2 &     \varLambda\varLambda M     & p^2(1-p) & 1 \\ \hline
		3 &    \varLambda M \varLambda     & p^2(1-p) & 1 \\ \hline
		4 &     M\varLambda\varLambda      & p^2(1-p) & -1 \\ \hline
		5 &         \varLambda M M         & p(1-p)^2 & -1 \\ \hline
		6 &         M \varLambda M         & p(1-p)^2 & -1 \\ \hline
		7 &    \varLambda M \varLambda     & p(1-p)^2 & -1 \\ \hline
		8 &             M M M              & (1-p)^3  & -3 \\ \hline
	\end{array}
	\]
	
	Αν σε ένα πείραμα τύχης αποκαταστήσουμε μια σχέση που αντιστοιχεί
	το αποτέλεσμα σε έναν αριθμό (ακέραιο, πραγματικό, \dots), τότε
	έχουμε μία τυχαία μεταβλητή.
	
	\paragraph{2\textsuperscript{ο} Παράδειγμα}
	Σε ένα κουτί υπάρχουν αντιστάσεις \( R= 100\, \Omega \)
	με ανοχή \( 0.1\% \).
	
	Αυτό είναι ένα πείραμα τύχης με δειγματικό χώρο
	\( S = \left\lbrace J_1,\dots,J_n \right\rbrace \), και το εύρος
	της τιμής για κάθε αντίσταση είναι \( x=x(J_i), \quad
	99,9 \leq x(J_i) \leq 100.1 \), δηλαδή \( x \in [99.9,\ 100.1] \).
	
	\paragraph{Τυχαία μεταβλητή} ονομάζεται μία συνάρτηση \( X \)
	που αντιστοιχεί κάθε ένα αποτέλεσμα \( J \in S \) του πειράματός μας
	σε έναν αριθμό \( X=X(J) \).
	
	\paragraph{}
	Για παράδειγμα, στο αρχικό παιχνίδι μπορώ να γράψω:
	\begin{align*}
		C &= \left\lbrace \varLambda\varLambda\varLambda,
		\varLambda\varLambda M, \varLambda M \varLambda,
		M\varLambda\varLambda \right\rbrace
		\\ &= \left\lbrace 1 \leq x \leq 3 \right\rbrace
	\end{align*}
	
	\paragraph{}
	
	Η συνάρτηση \( X \) επομένως αντιστοιχεί κάθε ενδεχόμενο του \( S \)
	σε έναν αριθμό (φυσικό αν έχουμε αριθμήσιμα αποτελέσματα, πραγματικό
	αν είναι μη αριθμήσιμα):
	\todo{Graph 10}
	
	\subsection[Συνάρτηση Κατανομής Πιθανότητας]{
		Συνάρτηση Κατανομής Πιθανότητας \\ (Probability Distribution
		Function - PDF)}
	\( \mathlarger{F(x)} \)
	
	\begin{defn}{Συνάρτηση κατανομής πιθανότητας}{}
	Ως συνάρτηση κατανομής πιθανότητας για μία τυχαία μεταβλητή \( X \)
	ορίζεται η συνάρτηση που ικανοποιεί:
	\[
	P\left\lbrace X \leq x \right\rbrace = F_X(x)
	\]
	\end{defn}
	
	Δηλαδή μας δίνει την πιθανότητα η τυχαία μεταβλητή να είναι 
	μικρότερη από έναν αριθμό:
	\todo{Graph 11}
	
	\paragraph{Ιδιότητες}
	Η συνάρτηση αυτή ικανοποιεί μερικές ιδιότητες:
	\begin{enumroman}
		\item \( F_X(-\infty) = 0 \qquad
		\mathsmaller{P\left\lbrace X\leq-\infty \right\rbrace = 0} \)
		\item \( F_X(+\infty) = 1 \qquad 
		\mathsmaller{P\left\lbrace X\leq+\infty \right\rbrace = P(S)}\)
		\item \( F_X(x_1) \leq F_X(x_2) \text{ ανν } x_1 \leq x_2 \)
		(δηλαδή είναι \textbf{αύξουσα}, όχι όμως απαραίτητα γνησίως αύξουσα)
		\item Απαιτούμε να είναι \textbf{συνεχής από τα δεξιά}, δηλαδή
		\( F_X(x^+) = F_X(x) \)
		\todo{Graph 12}
	\end{enumroman}
	
	\paragraph{Παράδειγμα για την ιδιότητα (iv)}
	\begin{align*}
		F_X(x^+) &= \underset{\epsilon\to 0}{P}
		\left\lbrace X \leq x+\epsilon \right\rbrace
		\\ &= P\left\lbrace X \leq x \right\rbrace
		+ \underset{\epsilon\to 0}{P}\left\lbrace x<X\leq x+\epsilon
		 \right\rbrace
	\end{align*}
	\todo{?}
	
	\paragraph{}
	Από τη στιγμή που έχουμε την συνάρτηση αυτή, μπορούμε να υπολογίσουμε
	τις πιθανότητες οποιουδήποτε διαστήματος. Το πιο συχνό διάστημα είναι
	το:
	\[
	\left\lbrace x < X \leq x_2 \right\rbrace,
	\]
	του οποίου συχνά καλούμαστε να υπολογίσουμε την πιθανότητα, ενώ θα
	έχουμε την συνάρτηση κατανομής \( F_X(x) \).
	
	Για να την υπολογίσουμε έχουμε:
	\todo{Graph 14}
	\begin{align*}
		\left\lbrace X \leq x_2 \right\rbrace &=
		\left\lbrace X \leq x_1 \right\rbrace +
		\left\lbrace x_1 < X \leq x_2 \right\rbrace \\
		P\left\lbrace X \leq x_2 \right\rbrace &=
		P\left\lbrace X \leq x_1 \right\rbrace +
		P \left\lbrace x_1 < X \leq x_2 \right\rbrace \\
		\Aboxed{
		F_X(x_2) - F_X(x_1) &= P \left\lbrace x_1<X\leq x_2 \right\rbrace
	}
	\end{align*}
	
	Ένα αποτέλεσμα του παραπάνω είναι:
	\begin{align*}
	\underset{\epsilon\to 0}{P} \left\lbrace x-\epsilon < X \leq x
	\right\rbrace = F_X(x) - \underset{\epsilon\to 0}{F_X}(x-\epsilon)
	&= F_X(x^+)-F_X(x^-) \\ &= \begin{cases}
	0 & \quad \text{αν είναι συνεχής} \\
	\text{η διαφορά των ορίων} & \quad \text{αν δεν είναι συνεχής}
	\end{cases} \\ &= \mathlarger{P\left\lbrace X=x \right\rbrace}
	\end{align*}
	
	Παρατηρούμε ότι:
	\begin{align*}
		\text{συνεχής} &\implies P\left\lbrace X=x \right\rbrace = 0\\
		\text{ασυνεχής} &\implies P\left\lbrace X=x \right\rbrace\neq 0
	\end{align*}
	δηλαδή με "πηδήματα" της συνάρτησης κατανομής πιθανότητας, μπορούμε
	να προσδώσουμε πιθανότητα σε σημείο.
	
	\subsection{Συνάρτηση Πυκνότητας Πιθανότητας}
	\begin{defn}{Συνάρτηση Πυκνότητας Πιθανότητας}{}
		Ως συνάρτηση πυκνότητας πιθανότητας \( f_X(x) \) ορίζουμε:
		\[
		f_X(x) = \od{F_X(x)}{x}
		\]
	\end{defn}
	\paragraph{Ιδιότητες}
	\begin{enumroman}
		\item \( f_X(x) \geq 0 \)
		\item \( \displaystyle
		\int_{-\infty}^{\infty} f_X(x) \dif x = 1\)
		\item \( \displaystyle
		F_X(x) = \int_{-\infty}^{\infty} f_X(x)\dif x \)
		\item \( \displaystyle
		P\left\lbrace x_1 < X \leq x_2 \right\rbrace 
		= \int_{x_1^+}^{x_2^+} f(x)\dif x \quad
		\mathsmaller{=\ F_X(x_2) - F_X(x_1)}
		\)
	\end{enumroman}
	
	\begin{center}
	\todo{Graph 15}
	\end{center}
	
	Παρατηρούμε ότι η συνάρτηση πυκνότητας πιθανότητας μοιάζει π.χ. με
	συνάρτηση πυκνότητας γραμμικής μάζας, και για να λάβουμε την
	πιθανότητα μεταξύ δύο τιμών, απλώς παίρνουμε το εμβαδόν της
	συναρτησης.
	
	Μεταξύ των \( f_X(x) \) και \( F_X(x) \) επιλέγουμε αυτήν που μας
	βολεύει περισσότερο, με βάση την ευκολία υπολογισμών. Για
	παράδειγμα, η κανονική κατανομή \( f_X(x) =
	\frac{1}{\sqrt{2\sigma^2\pi} } \; e^{ -\frac{(x-\mu)^2}{2\sigma^2} } 
	\) έχει αναλυτική έκφραση, αλλά όχι η αντίστοιχη \( F_X(x) \), για
	την οποία χρειαζόμαστε πίνακες ή υπολογιστή.
	
	\subsection{Πολλαπλές τυχαίες μεταβλητές}
	Ηχογραφούμε την φωνή μας να λέει την ίδια λέξη κάθε ένα λεπτό. Σε
	κάθε ηχογράφηση παίρνουμε ένα διαφορετικό σήμα:
	
	\begin{tikzpicture}
	\draw (0,-1) -- (0,2);
	\draw (-1,0) -- (7,0);
	
	\draw [blue, very thick, name path=a]
	plot [smooth, tension=0.4, domain=0:7, samples=25] (\x,{0.7+rand/1.2});
	\draw [green!50!orange, very thick, name path=b]
	plot [smooth, tension=0.4, domain=0:7, samples=25] (\x,{0.4+rand/1.2});
	
	\foreach \i in {1,2,...,4} {
		\path[name path=l1] (\i*1.4,-1) -- ++(0,2.5);
		\path[name intersections={of=a and l1,by=m}];
		\path[name intersections={of=b and l1,by=n}];
		\draw[gray!50!black,thick] (m) -- (m |- 0,0)
		node[below,rectangle,align=center,yshift=-4mm] {$t_\i$\\$x(t_\i)$};
		\draw[red!50!black,thick] (n) -- (n |- 0,0);
	}
	\end{tikzpicture}
	
	Παίρνουμε αυθαίρετα 4 χρονικές στιγμές \( t_1,t_2,t_3,t_4 \). Σε κάθε
	μία από αυτές, η τιμή της έντασης του ήχου για την κάθε ηχογράφηση
	είναι διαφορετική. Για την πρώτη στιγμή, για παράδειγμα, έχουμε μια
	τυχαία μεταβλητή \( X(t_1) \) που εκφράζει την ένταση του ήχου εκείνη
	τη στιγμή, και άλλες 3 αντίστοιχες τυχαίες μεταβλητές για τις \( 
	t_2,t_3,t_4 \).
	
	Τις 4 στιγμές τις επιλέξαμε αυθαίρετα, αλλά το πραγματικό σήμα έχει
	άπειρες τυχαίες μεταβλητές, μία για κάθε στιγμή. Είναι αυτές μεταξύ
	τους ανεξάρτητες;
	
	\paragraph{Παράδειγμα \#2}
	Έχουμε ένα σύστημα που δέχεται μία μόνο είσοδο \( X \), και παράγει
	μια έξοδο \( Υ \) (που εξαρτάται μόνο από την \( X \)):
	\begin{align*}
		X &= X(J) \\
		Y &= Y(J)
	\end{align*}
	
	\begin{tikzpicture}[scale=1.3]
	\draw (0,0) rectangle (2,1);
	\draw (0,0.5) -- (-1,0.5) node[below,midway] {$X(J)$};
	\draw (2,0.5) -- (3,0.5) node[below,midway] {$Y(J)$};
	\end{tikzpicture}
	
	Έστω το ενδεχόμενο \( A \) για το οποίο:
	\[
	A = \left\lbrace X\leq x, Y \leq y \right\rbrace
	= \left\lbrace X \leq x \right\rbrace \cdot
	\left\lbrace Y \leq y \right\rbrace
	\]
	
	Τότε ορίζουμε μία κοινή πιθανότητα μεταξύ τους (joint probability distribution function):
	\[
	F_X(x,y) = P\left\lbrace X \leq x,\ Y \leq y \right\rbrace
	\]
	
	\paragraph{Ιδιότητες}
	\begin{enumroman}
		\item \( F(-\infty,y) = F(x,-\infty) = F(-\infty,-\infty) 
		= 0 \)
		επειδή:
		\begin{gather*}
			\left\lbrace X=-\infty, Y \leq y \right\rbrace
			\subset \left\lbrace X=-\infty \right\rbrace, \text{ αλλά }
			P\left\lbrace X=-\infty \right\rbrace = 0.
		\end{gather*}
		
		\item \( F(+\infty,+\infty) = 1 \)
		\item \( F(x,y) \, \uparrow \) αύξουσα ως προς \( x,\ y,\ x \)
		και \( y \) (όχι απαραίτητα γνησίως)
		
		Για \( x_1<x_2 \), έχουμε:
		\begin{align*}
			P\left\lbrace X\leq x_2,Y\leq y \right\rbrace
			&= P\left\lbrace X\leq x_1, Y \leq y \right\rbrace
			+ P\left\lbrace x_1<X\leq x_2,Y \leq y \right\rbrace\\
			\Aboxed{P\left\lbrace x_1<X\leq x_2,Y\leq y \right\rbrace
			&= F(x_2,y)  - F(x_1,y) \leq 0 \implies F(x_2,y)\geq 
			F(x_1,y)}
		\end{align*}
		
		Με τον παραπάνω τύπο μπορούμε να υπολογίσουμε την πιθανότητα
		μιας κομμένης λωρίδας, όπως φαίνεται στο διάγραμμα:
		\todo{Graph 18}
		
		\item \( 0 \leq F(x,y) \leq 1 \)
		\item \( F_{XY}(+\infty,y) = F_Y(y) \) \\
		\( F_{XY}(x,+\infty) = F_X(x) \) επειδή
		\begin{align*}
			\left\lbrace X\leq x, Y \leq +\infty \right\rbrace &=
			\left\lbrace X\leq x \right\rbrace\cdot
			\cancelto{S}{\left\lbrace Y \leq +\infty \right\rbrace}
			\\ &\rightarrow P\left\lbrace X\leq x \right\rbrace
		\end{align*}
	\end{enumroman}
	
	\paragraph{Υπολογισμός πιθανότητας}
	Ποιά είναι η πιθανότητα
	\( \left\lbrace x_1<X\leq x_2,\
	y_1<Y\leq y_2 \right\rbrace \)
	να βρεθούμε μέσα στο ορθογώνιο του σχήματος;
	\todo{Graph 19}
	
	Έχουμε:
	\begin{align*}
	\left\lbrace x_1<X\leq x_2,Y\leq y_2 \right\rbrace
	&= \left\lbrace x_1<X\leq x_2, Y\leq y_1 \right\rbrace
	+ \left\lbrace x_1<X\leq x_2,y_1<Y\leq y_2 \right\rbrace
	\\
	P\left\lbrace x_1<X\leq x_2,Y\leq y_2 \right\rbrace
	&= P\left\lbrace x_1<X\leq x_2, Y\leq y_1 \right\rbrace
	+ P \left\lbrace x_1<X\leq x_2,y_1<Y\leq y_2 \right\rbrace
	\end{align*}
	Άρα:
	\[
	\mathlarger{
		P \left\lbrace x_1<X\leq x_1,\ y_1<Y\leq y_2 \right\rbrace
		= F(x_2,y_2) - F(x_1,y_1) - F(x_2,y_1) + F(x_1,y_1)
	}
	\]
	Δηλαδή μπορούμε να βρούμε την πιθανότητα να "πέσουμε" μέσα στο
	ορθογώνιο (άρα και σε κάθε χωρίο), χρησιμοποιώντας τις 
	πιθανότητες από τις άκρες.
	
	\subsubsection{Συνάρτηση πυκνότητας πιθανότητας (από κοινού)}
	\[
	\mathlarger{f_{XY}(x,y)
			= \frac{\partial^2 F_{XY}(x,y)}{\partial x\,\partial y}
			}
	\]
	
	\paragraph{Ιδιότητες}
	\begin{enumroman}
		\item \( f(x,y) \leq 0 \)
		\item \( \displaystyle
		\int_{-\infty}^{\infty} \int_{-\infty}^{\infty} f(x,y)\dif x
		\dif y = 1
		 \)
		\item \( \displaystyle
		F(x,y) = \int_{-\infty}^{x}\int_{-\infty}^{y} f(x,y)\dif x\dif y
		 \)
		\item \( \displaystyle
		f_X(x) = \int_{-\infty}^{\infty} f(x,y)\dif y \) \\
		\( \displaystyle
		f_Y(y) = \int_{-\infty}^{\infty} f(x,y)\dif x \)
		\item \( \displaystyle
		P\left\lbrace x_1<X\leq x_2,\ y_1<Y\leq y_2 \right\rbrace
		= \int_{x_1^+}^{x_2^+} \int_{y_1^+}^{y_2^+} f(x,y)\dif x\dif y
		 \)
	\end{enumroman}
	
	Επομένως για να βρούμε την πιθανότητα ενός χωρίου, αρκεί να
	ολοκληρώσουμε τη συνάρτηση πυκνότητας πιθανότητας σε αυτό.
	
	\textbf{Παράδειγμα:} Η κανονική κατανομή (καμπάνα)
	για δύο μεταβλητές
	
	\begin{tikzpicture}
	\begin{axis}[
	no markers, domain=0:8, samples=50,
	axis lines*=left, xlabel=$x$, ylabel=$y$,
	every axis y label/.style={at=(current axis.above origin),anchor=south},
	every axis x label/.style={at=(current axis.right of origin),anchor=west},
	height=6cm, width=7cm,
	xtick=\empty, ytick=\empty,
	enlargelimits=false, clip=false, axis on top,
	grid = major
	]
	\addplot [very thick,cyan!50!black] {gauss(4,1.5)};
	\end{axis}
	
	\begin{axis}[xshift=8cm]
	\addplot3[surf,domain=-4:4,domain y=-4:4] 
	{exp(-( (x-0)^2 + (y-0)^2)/3 )};
	\node[circle,inner sep=1pt,fill=blue,pin=90:$\mu$] 
	at (axis cs:0,0,1) {};
	\end{axis}
	\end{tikzpicture}
	
	Η κανονική κατανομή δεν μπορεί να ολοκληρωθεί με αναλυτική έκφραση,
	επομένως χρησιμοποιούμε πίνακες ολοκληρωμάτων ή υπολογιστές.
	
	\paragraph{}
	Τελικά, η συνάρτηση πυκνότητας πιθανότητας είναι μία επιφάνεια,
	η οποία αν ολοκληρωθεί σε κάποιο χωρίο για να λάβουμε τον όγκο που
	καλύπτει, θα πάρουμε την πιθανότητα του χωρίου (μοιάζει με συνάρτηση
	πυκνότητας μάζας).
	
	\subsection{Συνάρτηση κατανομής υπό συνθήκη πιθανότητας}
	Έστω:
	\[
	A = \left\lbrace X \leq x \right\rbrace
	\qquad
	B = \left\lbrace X \leq x_2 \right\rbrace
	\]
	
	Τότε:
	\[
	P(A/B) = \frac{P(A\cdot B)}{P(B)}
	\]
	Επομένως μπορούμε να ορίσουμε μια συνάρτηση:
	\[
	\boxed{
		F_X(x/B) = P\left\lbrace X\leq x/B \right\rbrace
		= \frac{P\left\lbrace X\leq x, B \right\rbrace}{P(B)}
		}
	\]
	
	για την οποία ισχύουν οι ιδιότητες, αντίστοιχα με προηγουμένως:
	\todo{Itemize}
	\begin{gather*}
		F_X(+\infty/B) = 1, \quad F(-\infty/B) = 0 \\[3ex]
		P\left\lbrace x_1<X\leq x_2\ /B \right\rbrace
		=\frac{P\left\lbrace x_1 < X\leq x_2,\ B \right\rbrace}{P(B)}
		=F(x_2/B) - F(x_1/B)
		\\[3ex]
		f_X(x/B) = \frac{\dif F_X(x/B)}{\dif x} \\
		f_X(x/B) \leq 0 \ \forall x \\
		\int_{-\infty}^{\infty} f_x(x/B) \dif x = F_X(\infty/B)
		- F_X(-\infty/B) = 1 \\
		F_X(x_1/B) = \int_{-\infty}^{x_1} f_x(x/B)\dif x
		\\[3ex]
		P \left\lbrace x_1<X\leq x_2\ /B \right\rbrace
	    = \int_{x_1^+}^{x_2^+} f_X(x/B)\dif x
	\end{gather*}
	
	\todo{Graph 20}
	Επίσης έχουμε:
	\begin{align*}
		F_X(x/B) &= \begin{cases}
		\frac{F_X(x)}{F_X(x_2)} &\quad x<x_2 \\
		1 &\quad x\leq x_2
		\end{cases} \\
		f_X(x/B) &= \begin{cases}
		\frac{f_X(x)}{\int_{-\infty}^{x^2} f_X(x) \dif x}
		&\quad x<x_2 \\ 0 &\quad x \geq x_2
		\end{cases}
	\end{align*}
	
	\subsubsection{Υπό συνθήκη πιθανότητα με δύο τυχαίες μεταβλητές}
	
	\paragraph{Έστω} ότι \( A=\left\lbrace X \leq x \right\rbrace 
	\qquad B = \left\lbrace Y \leq y \right\rbrace
	\)
	\[
	\boxed{F_X(x/B)
		= \frac{F_{XY}(x,y)}{F_Y(y)}} 
	\]
	
	Για τη συνάρτηση πυκνότητας πιθανότητας:
	\begin{align*}
		\pd{F_{XY}(x,y)}{x} &= \pd{}{x} \left[
		\int_{-\infty}^{y} \int_{-\infty}^{x} f_{XY} (x,y)\dif x\dif y
		\right] \\ &=
		\int_{-\infty}^{y} \left[\pd{}{x}\int_{-\infty}^{x}f(x,y)
		\dif x\right]\dif y = \int_{-\infty}^{y} f_{XY}(x,y)\dif y
	\end{align*}
	
	Άρα:
	\begin{align*}
		f_X (x,\ Y \leq y) &= \frac{1}{F_Y(y)}\pd{F_{XY}(x,y)}{x} \\
		&= \frac{\int_{-\infty}^{y}f_{XY}(x,y)\dif y}{
			\int_{-\infty}^{\infty} \int_{-\infty}^{y}
			f(x,y)\dif y\dif x}
	\end{align*}

    \paragraph{Έστω όμως} ότι
    \( B = \left\lbrace y_1 < Y \leq y_2 \right\rbrace \)
    
    \begin{align*}
    	F_X(x/\ y_1 < Y \leq y_2) &=
    	\frac{P\left\lbrace x\leq x,\ y_1 < Y \leq y_2\right\rbrace}{
    		P\left\lbrace y_1 < Y \leq y_2 \right\rbrace}
    	\\ &= \frac{F_{XY}(x,y_2) - F_{XY}(x,y_2)}{
    		F_Y(y_2)-F_Y(y_1)}
    	\\ &= \frac{\int_{y_1}^{y_2} f_{XY}(x,y)\dif y}{
    		\int_{-\infty}^{\infty} \int_{y_1}^{y_2} f_{XY}(x,y)
    		\dif y\dif x}
    \end{align*}
    
    \paragraph{Έστω ότι \( B = \left\lbrace Y=y \right\rbrace \)}
    \begin{align*}
    	F_X(Y=y) &= \frac{
    		\sfrac{\partial F_{XY}(x,y)}{\partial\. y} }{
    		\sfrac{\dif }{den} }
    \end{align*}
    
    \subsection{Ανεξαρτησία Τυχαίων Μεταβλητών}
    Ο γνωστός ορισμός της ανεξαρτησίας είναι:
    \[
    P(A\cdot B) = P(A) \cdot P(B)
    \]
    
    Επομένως, για δύο τυχαίες μεταβλητές \( X,Y \) και τα αντίστοιχα
    ενδεχόμενα \( A=\left\lbrace X\leq x \right\rbrace \) και
    \( B = \left\lbrace Y \leq y \right\rbrace \), έχουμε:
    \begin{align*}
    	P\left\lbrace X\leq x,\ Y\leq y \right\rbrace &=
    	P\left\lbrace X \leq x \right\rbrace \cdot
    	P\left\lbrace Y \leq y \right\rbrace \\
    	F_{XY}(x,y) &= F_X(x) \cdot F_Y(y) \\
    	\xRightarrow{\frac{\partial^2 F(x,y)}{\partial x\,\partial y}}
    	f_{XY}(x,y) &= f_X(x)\cdot f_Y(y)
    \end{align*}
    
    Επομένως, για δύο ανεξάρτητες τυχαίες μεταβλητές έχουμε:
    \begin{align*}
    	F_X(x/\ Y\leq y) = \frac{P\left\lbrace X\leq x,\ Y\leq y
    		 \right\rbrace}{P\left\lbrace Y\leq y \right\rbrace}
        = P\left\lbrace X\leq x \right\rbrace = F_X(x)
    \end{align*}
    
    \subsection{Ντετερμινιστική σχέση μεταξύ τυχαίων ματαβλητών}
    \[ \mathlarger{\left(X,f_X(x)\right)
    \xrightarrow{\qquad} \left(Y,f_Y(y)\right)}
    \]
    
    Έστω ότι υπάρχει μία \textit{ντετερμινιστική} σχέση \( g(\cdot) \)
    που συνδέει τις
    \textit{τυχαίες} μεταβλητές \( X,Y \):
    \[
    \mathlarger{Y=g(X)}
    \]
    
    Ομοίως μπορεί να έχουμε:
    \( 
    (X,Y,f_{XY}(x,y)) \to (Z,W,f_{ZW}(z,w))
     \)
     
    \( Z =g(X,Y)\)
    
    \( W=h(X,Y) \)
    
	\todo{Graph 21}
	\todo{Graph 22}
	
	\[
	f_Y(y)\dif y = P\left\lbrace y<Y\leq y+\dif y \right\rbrace
	= P\left\lbrace x_1<X\leq x_1+\dif x_1 \right\rbrace
	+ P\left\lbrace x_2<X\leq x_2+\dif x_2 \right\rbrace
	+ P\left\lbrace x_3<X\leq x_3+\dif x_3 \right\rbrace
	\]
	
	Άρα:
	\begin{align*}
		f_Y(y) &=  f_X(x_1) \frac{|\dif x_1|}{\dif y}
		+ f_X(x_2)\frac{|\dif x_2|}{\dif y}
		+ f_X(x_3)\frac{|\dif x_3|}{\dif y} \\
		\Aboxed{ f_Y(y) &= \frac{f_X(x_1)}{\left|g'(x_1)\right|}
		+ \frac{f_X(x_2)}{\left|g'(x_2)\right|}
		+ \frac{f_X(x_3)}{\left|g'(x_3)\right|}}
	\intertext{και γενικότερα}
	    \Aboxed{
	    	f_Y(y) &= \sum_{i=1}^{m} \left.
	    	 \frac{f_X(x)}{\left|g'(x)\right|}
	    	\right|_{x=x_i}
	    	}
	\end{align*}
	
	\paragraph{Για δύο ζευγάρια τυχαίων μεταβλητών}
	προκύπτει ότι, για \( z=g(x_i,y_i) \) και \( w=h(x_i,y_i) \):
	\[
	\boxed{
		f_{ZW}(z,w) = \sum_{i=1}^{n} \left.
		\frac{f_{XY}(x,y)}{\left|\mathrm J(x,y)\right|}
		\right|_{\scriptsize \begin{matrix}
			x=x_i \\ y=y_i
			\end{matrix}}
		}
	\]
	
	όπου \( \mathrm J \) η ιακωβιανή ορίζουσα \( 
	\mathrm J = \left|\begin{matrix}
	\pd{g(x,y)}{x} & \pd{g(x,y)}{y} \\
	\pd{h(x,y)}{x} & \pd{h(x,y)}{y}
	\end{matrix}\right|.
	 \)
	 
	\paragraph{Παράδειγμα}
	Έστω ότι έχουμε ένα μοντέλο \( \left(X,Y,f_{XY}\right) \), και τις
	τυχαίες μεταβλητές:
	\begin{align*}
		Z &= g(X,Y) \leftrightarrow f_Z(z) \\
		W &= X = h(X,Y)
	\end{align*}
	
	Τότε έχουμε:
	\begin{align*}
	    f_Z(z) &= \int_{-\infty}^{\infty} f_{ZW}(z,w)\dif w\\
		f_{ZW}(z,w) &= \sum_{i=1}^{n} \left.
		\frac{f_{XY}(x,y)}{\left|J(x,y)\right|} \right|_{x=x_i,\ y=y_i}
	\end{align*}
	
	\paragraph{Παράδειγμα}
	Έστω μοντέλο \( \left(X,Y,f_{XY}\right) \) και οι:
	\begin{align*}
		Z &= X+Y \\
		W &= X
	\end{align*}
	
	Αν λύσουμε το παραπάνω σύστημα, έχουμε:
	\begin{align*}
		y_1 &= z-w \\
		x_1 &= w
	\end{align*}
	
	Επομένως:
	\begin{align*}
		J(x) &= \left|\begin{matrix}
		1 & 1 \\ 1 & 0
		\end{matrix}\right| = -1\\
		f_{ZW}(z,w) &= \left. \frac{f_{XY}(x,y)}{|-1|}
		\right|_{x_1=w,\ y_1=z-w} \\ &= f_{XY}(w,\ z-w)
	\end{align*}
	
	Έχει ενδιαφέρον όταν οι \( X \) και \( Y \) είναι ανεξάρτητες,
	οπότε:
	\begin{align*}
		f_{XY}(x,y) &= f_X(x) \cdot f_Y(y) \\
		f_{ZW}(z,w) &= \left. \frac{f_X(x)f_Y(y)}{|-1|}\right|_{
			x_1=w,\ y_1=z-w} \\ 
		f_Z(z) &= \int_{-\infty}^{\infty} f_X(w)f_Y(z-w)
		= f_X(z) * f_Y(z)
	\end{align*}
	\todo{Note convolution}
\end{document}
