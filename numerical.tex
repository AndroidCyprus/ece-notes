\ifcsname ispublic\endcsname
    \documentclass[11pt,a4paper,notitlepage,fleqn]{article}
\else
    \documentclass[11pt,a4paper,notitlepage,fleqn,draft]{article}
\fi

\usepackage{amsmath}
\usepackage{amsfonts}
\usepackage{amssymb}
\usepackage{libs/commath2}
\usepackage[table]{xcolor}
\usepackage[hidelinks,draft=false]{hyperref}
\usepackage[skins,theorems]{tcolorbox}
\usepackage{titlesec}
\usepackage{tikz}
\usepackage{libs/circuitikz} % use our own recent version to make sure some bugs are fixed
\usepackage{pgfplots}
\usepackage{mathtools}
\usepackage[makeroom]{cancel}
\usepackage{mathrsfs}
\usepackage{wrapfig}
%\usepackage{subcaption}
%\usepackage{floatrow}
\usepackage{esint}
\usepackage{enumitem}
%\usepackage{bm}
\usepackage{relsize}
\usepackage{xfrac}
\usepackage{comment}
%\usepackage{siunitx}
%\usepackage{MnSymbol}
\usepackage[obeyDraft,disable]{todonotes}
%\usepackage[linesnumbered,lined]{algorithm2e}


\pgfplotsset{compat=1.13}
\usetikzlibrary{arrows.meta}
\usetikzlibrary{patterns}
\usetikzlibrary{decorations.pathmorphing,patterns}
\usetikzlibrary{decorations.markings}
\usetikzlibrary{backgrounds}
\usetikzlibrary{shapes.misc}
\usetikzlibrary{shapes.multipart}
\usetikzlibrary{shadows.blur}
\usetikzlibrary{fadings}
\usetikzlibrary{intersections}
\usetikzlibrary{arrows.meta}
\usetikzlibrary{calc}
\usetikzlibrary{matrix}
\usetikzlibrary{positioning}
\usetikzlibrary{shapes}
\usetikzlibrary{shadings}

\tcbuselibrary{breakable}

\tikzset{cross/.style={cross out, draw,
        minimum size=2*(#1-\pgflinewidth),
        inner sep=0pt, outer sep=0pt}}
\tikzset{
    mark position/.style args={#1(#2)}{
        postaction={
            decorate,
            decoration={
            	post length=1mm, % ??? Magic to fix "Dimension
            	pre length=1mm, % ???  too large" errors.
                markings,
                mark=at position #1 with \coordinate (#2);
            }
        }
    }
}
\makeatletter
\tikzset{
  use path for main/.code={%
    \tikz@addmode{%
      \expandafter\pgfsyssoftpath@setcurrentpath\csname tikz@intersect@path@name@#1\endcsname
    }%
  },
  use path for actions/.code={%
    \expandafter\def\expandafter\tikz@preactions\expandafter{\tikz@preactions\expandafter\let\expandafter\tikz@actions@path\csname tikz@intersect@path@name@#1\endcsname}%
  },
  use path/.style={%
    use path for main=#1,
    use path for actions=#1,
  }
}
\makeatother

\pgfmathdeclarefunction{sinc}{1}{%
	\pgfmathparse{abs(#1)<0.01 ? int(1) : int(0)}%
	\ifnum\pgfmathresult>0 \pgfmathparse{1}\else\pgfmathparse{sin(#1 r)/#1}\fi%
}
\pgfmathdeclarefunction{gauss}{2}{%
	\pgfmathparse{1/(#2*sqrt(2*pi))*exp(-((x-#1)^2)/(2*#2^2))}%
}

\usepackage[left=2cm,right=2cm,top=2cm,bottom=2cm]{geometry}

%\usepackage[no-math]{fontspec}
%\usepackage{fontspec}
\usepackage{mathspec}
%\usepackage{newtxtext,newtxmath}
%\usepackage{unicode-math}
%\setmainfont{texgyretermes-regular.otf}
%\setsansfont{texgyreheros-regular.otf}
%\newfontfamily\greekfont[Script=Greek]{Linux Libertine O}
%\newfontfamily\greekfontsf[Script=Greek]{Linux Libertine O}
\usepackage{polyglossia}
%\newfontfamily\greekfont[Script=Greek]{texgyretermes-regular.otf}
\newfontfamily\greekfontsf[Script=Greek]{texgyreheros-regular.otf}
\newfontfamily\greekfonttt[Script=Greek]{Latin Modern Mono}
%\usepackage[greek]{babel}
\setdefaultlanguage{greek}
\setotherlanguage{english}

%\usepackage[utf8]{inputenc}
%\usepackage[greek]{babel}


%\usepackage{tkz-euclide} % loads  TikZ and tkz-base
%\usetkzobj{angles} % important you want to use angles

\newlist{enumparen}{enumerate}{1}
\setlist[enumparen]{label=(\arabic*)}
\newlist{enumpar}{enumerate}{1}
\setlist[enumpar]{label=\arabic*)}

\newlist{enumgreek}{enumerate}{1}
\setlist[enumgreek]{label=\alph*.}
\newlist{enumgreekparen}{enumerate}{1}
\setlist[enumgreekparen]{label=(\alph*)}
\newlist{enumgreekpar}{enumerate}{1}
\setlist[enumgreekpar]{label=\alph*)}


\newlist{enumroman}{enumerate}{1}
\setlist[enumroman]{label=(\roman*)}

\newlist{enumlatin}{enumerate}{1}
\setlist[enumlatin]{label=(\alph*)}

\newlist{invitemize}{itemize}{1}
\setlist[invitemize]{noitemsep,label=}

\usepackage{letltxmacro}

\LetLtxMacro\OriginalLongrightarrow\Longrightarrow
\LetLtxMacro\OriginalLongleftarrow\Longleftarrow

% Implement new macros
% --------------------
\usepackage{trimclip}
\DeclareRobustCommand\Longrightarrow{\NewRelbar\joinrel\Rightarrow}
\DeclareRobustCommand\Longleftarrow{\Leftarrow\joinrel\NewRelbar}

\makeatletter
\DeclareRobustCommand\NewRelbar{%
  \mathrel{%
    \mathpalette\@NewRelbar{}%
  }%
}
\newcommand*\@NewRelbar[2]{%
  % #1: math style
  % #2: unused
  \sbox0{$#1=$}%
  \sbox2{$#1\Rightarrow\m@th$}%
  \sbox4{$#1\Leftarrow\m@th$}%
  \clipbox{0pt 0pt \dimexpr(\wd2-.6\wd0) 0pt}{\copy2}%
  \kern-.2\wd0 %
  \clipbox{\dimexpr(\wd4-.6\wd0) 0pt 0pt 0pt}{\copy4}%
}
\makeatother


\makeatletter
\pgfdeclareradialshading[tikz@ball]{ball}{\pgfqpoint{0bp}{0bp}}{%
	color(0bp)=(tikz@ball!50!white);
	color(10bp)=(tikz@ball!50!white);
	color(15bp)=(tikz@ball!70!black);
	color(20bp)=(black!70);
	color(30bp)=(black!70)}%
\makeatother


\makeatletter
\let\anw@true\anw@false

%\newcommand{\attnboxed}[1]{\textcolor{red}{\fbox{\normalcolor\m@th$\displaystyle#1$}}}
\makeatother
\tcbset{highlight math style={enhanced,colframe=red,colback=white,%
        arc=0pt,boxrule=1pt,shrink tight,boxsep=1.5mm,extrude by=0.5mm}}
\newcommand{\attnboxed}[1]{\tcbhighmath[colback=red!5!white,drop fuzzy shadow,arc=0mm]{#1}}
\newcommand{\infoboxed}[1]{%
	\tcbhighmath[colframe=blue!50!white,colback=blue!5!white,arc=0mm]{#1}}
\titleformat{\section}{\bf\Large}{Κεφάλαιο \thesection}{1em}{}
\newtcolorbox{attnbox}[1]{colback=red!5!white,%
    colframe=red!75!black,fonttitle=\bfseries,title=#1}
\newtcbox{quickattnbox}[1]{colback=red!5!white,%
	colframe=red!75!black,fonttitle=\bfseries,title=#1}
\newtcolorbox{infobox}[1]{colback=blue!5!white,%
    colframe=blue!75!black,fonttitle=\bfseries,title=#1}

\AtBeginDocument{%
\let\arg\relax
\let\Re\relax
\let\Im\relax
\DeclareMathOperator{\arg}{Arg}
\DeclareMathOperator{\Re}{Re}
\DeclareMathOperator{\Im}{Im}
}
\DeclareMathOperator{\sinc}{sinc}
\DeclareMathOperator{\sgn}{sgn}
\DeclareMathOperator{\erf}{erf}
\DeclareMathOperator{\cov}{cov}

\newif\ifhidetikz
\hidetikzfalse
%\hidetikztrue   % <---- comment/uncomment that line

\ifhidetikz

\let\oldtikzpicture\tikzpicture
\let\oldendtikzpicture\endtikzpicture

\renewenvironment{tikzpicture}{
    \tiny
    \tt
    \color{blue}
    \newcommand{\draw}{\textit{draw}}
    \newcommand{\filldraw}{\textit{filldraw}}
    %\newcommand{\x}{\textit{x}}
    %\newcommand{\p}{\textit{x}}
    \newcommand{\x1}{\textit{x1}}
    \newcommand{\y1}{\textit{y1}}
    \newcommand{\p1}{\textit{p1}}
}{
}
\newenvironment{axis}{
    \newcommand{\addplot}{\textit{addplot}}
}{
}
\fi

% Global amount of samples
% Set to a higher value (e.g. 200) for nicer graphs
% Set to a low value (e.g. 10) for performance
\newcommand*{\gsamples}{70}

% Equals command as a workaround for CircuiTikZ bug
% not allowing the = sign in labels
\newcommand*{\equals}{=}

\newcommand{\nesearrow}{%
	\,%
	\smash{\raisebox{-1.1ex}
		{$%
			\stackrel{\displaystyle\nearrow}{\displaystyle\searrow}%
			$}}%
}
\newcommand{\degree}{^{\circ}} % not great
\newcommand\numberthis{\addtocounter{equation}{1}\tag{\theequation}} % add an equation number to a number-less math environment

\newtcbtheorem[number within=section]{theorem}{Θεώρημα}%
{colback=green!5,colframe=green!35!black,colbacktitle=green!35!black,fonttitle=\bfseries,enhanced,attach boxed title to top left={yshift=-2mm,xshift=-7mm},width=.9\textwidth,arc=.7mm}{th}
\newtcbtheorem[number within=section]{defn}{Ορισμός}%
{colback=blue!5,colframe=cyan!35!black,colbacktitle=blue!35!black,fonttitle=\bfseries,enhanced,attach boxed title to top left={yshift=-2mm,xshift=-2mm}}{def}
\newtcbtheorem[number within=section]{exercise}{Άσκηση}%
{colback=gray!3,colframe=gray!35!black,colbacktitle=gray!35!black,fonttitle=\bfseries,enhanced,attach boxed title to top left={yshift=-2mm,xshift=-2mm}}{exc}




\title{Αριθμητική Ανάλυση
	\\
	{
	\normalsize Σημειώσεις από τις παραδόσεις
	}}
\date{2017
	\\
	{
	\small Τελευταία ενημέρωση: \today
	}
	}
\author{
	Για τον κώδικα σε \LaTeX, ενημερώσεις και προτάσεις:
\\
 \url{https://github.com/kongr45gpen/ece-notes}}

\setmainfont{Linux Libertine O}
\setsansfont{Ubuntu}
%\newfontfamily\greekfont[Script=Greek]{Linux Libertine O}
%\newfontfamily\greekfontsf[Script=Greek]{Linux Libertine O}
\usepackage{polyglossia}
\newfontfamily\greekfont[Script=Greek,Scale=0.95]{GFS Artemisia}

\hypersetup{
	pdftitle = {Αριθμητική Ανάλυση}
}

\begin{document}
	\maketitle

	\tableofcontents

	\vspace{50pt}

	\textbf{Αριθμητική ανάλυση - Numerical Analysis}
	
	Μάθημα 4 ώρες την εβδομάδα - δεν υπάρχει διάκριση μεταξύ θεωρίας και ασκήσεων.
	
	Και τα δύο βιβλία προτείνονται, το μάθημα γίνεται περισσότερο με βάση το βιβλίο του κ.
	Πιτσούλη, του κ. Δούγαλη είναι περισσότερο μαθηματικό.
	
	Στις εξετάσεις δεν θα υπάρχει τυπολόγιο/βιβλίο, αλλά θα δίνονται τύποι που χρειάζονται
	στα θέματα. Απαραίτητο το κομπιουτεράκι.
	
	\section{Εισαγωγή}
	Η αριθμητική ανάλυση μάς δίνει \textit{προσεγγιστικές} λύσεις σε μοντέλα και μαθηματικά
	προβλήματα.
	
	Σε δύσκολα προβλήματα, ζητάμε:
	\begin{itemize}
		\item \textbf{Ακρίβεια} αποτελέσματος
		\item \textbf{Ταχύτητα} υπολογισμού
	\end{itemize}
	
	Θα δούμε τα εξής προβλήματα:
	\begin{itemize}
		\item \textbf{Επίλυση εξισώσεων}
		\paragraph{Παράδειγμα}
		Ένας πελάτης θέλει να καταθέσει \( P \) € για \( N \) χρόνια στην τράπεζα, και εγώ
		του λέω ότι θα του επιστρέψω \( A \) € από την κατάθεση. Ο πελάτης όμως ενδιαφέρεται
		για το ετήσιο επιτόκιο \( R \).
		
		Προκύπτει μια εξίσωση της μορφής:
		\begin{gather*}
		A = P + P \left( 1 + \frac{R}{12} \right) + \dots \\
		f(R) = \frac{P}{\sfrac{R}{12}} \left[
		    \left(1+\frac{R}{12}\right)^N -1
		\right] = 0, \quad R = ?
		\end{gather*}
		
		Θυμόμαστε ότι μπορούμε να χρησιμοποιήσουμε το θεώρημα Bolzano για να βρούμε ότι
		υπάρχει μια τουλάχιστον λύση μέσα σε ένα διάστημα, οπότε μπορούμε να "φανταστούμε"
		έναν αριθμό κοντά στη λύση, και να κλείνουμε συνεχώς ένα διάστημα γύρω από αυτήν (το
		διάστημα στις εξετάσεις θα δίνεται, π.χ. \textit{βρείτε μία λύση στο διάστημα
			\( [2.5,\ 3.5] \) με ακρίβεια \( 10^{-5} \)}), αν και αυτό δεν θα γίνεται στον πραγματικό κόσμο.
		\item \textbf{Παρεμβολή}
		
		Σε έναν σταθμό διοδίων μετράω πόσα αυτοκίνητα περνάν το κάθε λεπτό (π.χ. το
		11\textsuperscript{ο} λεπτό περνάν 4, το 12\textsuperscript{ο} περνάν 7, κλπ.)

		\begin{center}
		\begin{tikzpicture}[scale=1]
		\matrix (values) [matrix of nodes,
		nodes={align=center,text width=0.7cm}
		] {
			11 & 12 & 13 & 14 & 15 \\
			4 & 7 & 11 & 1 & 22 \\
		};
		\draw (values-1-1.south west)--(values-1-5.south east);
		\draw (values-1-1.north west) -- (values-2-1.south west);
		\foreach \x in {1,2,...,5}
		\draw (values-1-\x.north east) -- (values-2-\x.south east);
		
		\draw[thick,->] (4,0) -- ++(1.7,0);
		
		
		\begin{scope}[xshift=7cm,yshift=-1cm]
		\draw (-0.5,0) -- (2,0);
		\draw (0,-0.5) -- (0,2);
		
		\begin{scope}[yshift=2mm,xshift=3mm,yscale=.5,xscale=.5]
		
		\coordinate (A) at (0,1);
		\coordinate (B) at (1,2);
		\coordinate (C) at (2,3);
		\coordinate (D) at (3,0.3);
		\coordinate (E) at (4,4);
		
		\draw[thick,blue]
		plot [smooth] coordinates { (A) (B) (C) (D) (E) };
		
		\foreach \p in {(A),(B),(C),(D),(E)}
		\filldraw \p circle (2pt);
		
		\end{scope}
		\end{scope}
		
		\end{tikzpicture}
		\end{center}
		
		Θέλω να βρω ένα πολυώνυμο που να συνδέει όλα τα σημεία μεταξύ τους (θα αποδείξουμε
		ότι τέτοιο πολυώνυμο πάντα υπάρχει), ή ένα πολυώνυμο (αρκετά χαμηλού βαθμού, ώστε
		να μην γίνονται πολλές πράξεις) με αρκετά καλή προσέγγιση.
		
		Με αυτόν τον τρόπο θα μπορώ να προσεγγίσω τιμές που δεν γνωρίζω, π.χ. αν πήγα για
		καφέ στο 13\textsuperscript{ο} λεπτό
		\item \textbf{Προσέγγιση}
		\item \textbf{Αριθμητική Γραμμική Άλγεβρα}
		
		Exact και προσεγγιστικές λύσεις συστημάτων πολλών αγνώστων.
		\item \textbf{Ολοκλήρωση}
		\item \textbf{Υπολογισμός ιδιοτιμών \& ιδιοδιανυσμάτων}
		\item \textbf{Παραγοντοποίηση πινάκων σε γινόμενο πινάκων}
		
		Βολεύει κυρίως για την επίλυση συστημάτων.
		\item \textbf{Επίλυση κανονικών διαφορικών εξισώσεων}
		\item \textbf{Βελτιστοποίηση}
		
		Για παράδειγμα, να πρέπει να ελαχιστοποιήσω μια συνάρτηση τη στιγμή που πρέπει να
		τηρούνται κάποιες συνθήκες.
	\end{itemize}
	
	\subsection{Ακρίβεια vs Ταχύτητα}
	
	Απόλυτο Σφάλμα: \( \displaystyle
	\left| X_t - X_c \right|
	 \) \quad (απόσταση της λύσης που βρήκα από την πραγματική)
	 
	\vspace{5pt}
	
	Σχετικό Σφάλμα: \( \displaystyle
	\frac{\left|X_t - X_c\right|}{\left|X_t\right|}
	 \)
	
	Επειδή δεν θα γνωρίζουμε την πραγματική λύση \( X_t \), θα βρίσκουμε το μέγιστο σφάλμα.
	
	Για παράδειγμα, σε υπολογιστές έχουμε σφάλματα στρογγύλευσης και αποκοπής:
	\[
	\begin{array}{ll}
		0.66666 & \\
		0.66 & \leftarrow \text{αποκοπή} \\
		0.67 & \leftarrow \text{στρογγύλευση}
	\end{array}
	\]
	
	\section{Επίλυση Εξισώσεων}
	\begin{center}
		\( \displaystyle \mathlarger{
		f(x): \qquad \text{βρείτε $\bar x$ έτσι ώστε \( f(\bar x) = 0 \).}}
		\)
	\end{center}
	
	Η επίλυση είναι εύκολη όταν η \( f \) είναι πολυώνυμο μέχρι 2\textsuperscript{ου} βαθμού,
	όχι όμως όταν είναι μεγαλύτερου, ή όταν έχει κι άλλους όρους (π.χ. εκθετικούς) μέσα.
	
	\begin{itemize}
		\item Δημιουργούμε μια ακολουθία \( x_1, \dots, x_n \) προσέγγισης της \( \bar x \).
		\item Σε κάθε βήμα κάνουμε έλεγχο σύγκλισης για να δούμε πόσο κοντά είμαστε.
	\end{itemize}
	
	\subsection{Μέθοδος Διχοτόμησης}
	Θα χρησιμοποιήσω θεώρημα Bolzano (\( \mathsmaller{f(a) \cdot f(b) < 0} \))
	
	\begin{center}
		\begin{tikzpicture}[scale=1.9]
		\draw (0,-0.5) -- (0,2);
		\draw (-1,0) -- (3,0);
		
		\draw[thick, orange,
		mark position=0.5(c),
		mark position=0.3(a),
		mark position=0.8(b),
		mark position={0.55}(c1),
		mark position={(0.55/2+0.8/2)}(c2),
		mark position={(0.55/4+0.8/4+0.55/2)}(c3)
		] plot[smooth,tension=0.7]
		coordinates {(-0.5,0.7) (0.5,1.5)  (2.2,-0.7) (3.2,-0.9)};
		\draw[orange!90!black] (c) node[above] {$f$};
		
		\draw[dashed] (a) -- (a |- 0,0) node[below] {$a$};
		\draw[dashed] (b) -- (b |- 0,0) node[above] {$b$};
		\draw[thick,gray!50!blue] (c1) -- (c1 |- 0,0) node[below] {$c_1$};
		\draw[thick,gray!35!blue] (c2) -- (c2 |- 0,0) node[above] {$c_2$};
		\draw[thick,gray!20!blue!60!green] (c3) -- (c3 |- 0,0) node[below] {$\mathsmaller{c_3}$};
		
		\draw (1,-1.5) node {$\displaystyle c_1 = \frac{a+b}{2}$};
		\draw (1,-2) node {$\displaystyle c_2 = \frac{c_1+b}{2}$};
		\end{tikzpicture}
	\end{center}
	
	Ξεκινάω από μία αρχική προσέγγιση/διάστημα (η μέθοδος διχοτόμησης δεν δίνει λύση
	συγκεκριμένη, αλλά διάστημα - τη λύση την παίρνω σαν το μέσο του διαστήματος).
	
	Συνέχεια κόβω το διάστημα στη μέση, έτσι ώστε \( f(\text{των άκρων}) \) να είναι
	ετερόσημα, και παίρνω συνεχώς ένα μικρότερο διάστημα.
	
	Σταματάω όταν είναι επιτυχής ο έλεγχος σύγκλισης, π.χ
	\( \mathlarger{\left|c_{k+1}-c_k\right| < \epsilon \rightarrow 10^{-5}} \), δηλαδή
	το διάστημα στο οποίο βρίσκεται η ρίζα είναι αρκετά μικρό.
	
	Θα διαπιστώσουμε ότι, αν και αυτή η μέθοδος λειτουργεί πάντα, είναι αρκετά αργή.
	
	\paragraph{Παράδειγμα}
	Να βρεθεί ρίζα της εξίσωσης
	\( \mathlarger{
	\mathlarger{f(x) = x^3+x-1} }
	 \) στο διάστημα \( [0,1] \) με ακρίβεια \( 10^{-3} \).
	
	\begin{gather*}
		f(0) \cdot f(1) < 0 \\
		k=1,\ a_1=0,\ b_1=1,\ m=\frac{a_1+b_1}{2} = 0.5 \\
		\quad f(m) = -0.375 < 0 \\
		\quad f(0.5) \cdot f(1) < 0
	\end{gather*}
	Άρα η λύση βρίσκεται μεταξύ \( 0.5 \) και \( 1 \).
	
	Για το επόμενο βήμα:
	\begin{gather*}
		k = 2,\ m_2 = \frac{0.5+1}{2} = 0.75 \\
		\quad f(m_2) = 0.172 > 0 \\
		\quad f(m_1)f(m_2) < 0 \\[3ex]
		k=3,\ m_3 = \frac{0.5+0.75}{2} = 0.625 \\
		\quad f(m_3) = -0.131 \\
		\quad f(m_2)f(m_3) < 0 \\[3ex]
		\mathlarger{\mathlarger{\vdots}}
	\end{gather*}
	
	\subsubsection{Σύγκλιση}
	Όπου \( r \) και \( m_n \) η πραγματική και προσεγγιστική λύση αντίστοιχα:
	\begin{align*}
		\left|r - m_n\right| \leq \left(\frac{1}{2}\right)^n (b-a)
	\end{align*}
	επειδή στη \( n- \)οστή επανάληψη έχω κόψει το διάστημα στα 2 σε \( n \) φορές.
	
	Για σφάλμα \( 10^{-5} \), θέλουμε \( |r-m_n| = 10^{-5} \):
	\begin{align*}
		&10^{-5} \leq \left(\frac{1}{2}\right)^n (1-0) \implies
		\\ \implies & n \simeq 16.667 \implies \mathlarger{n = 17}
	\end{align*}
	
	\subsection{Μέθοδος Χορδής ή Τέμνουσας}
	Σαν τη μέθοδο Bolzano, αλλά δεν παίρνουμε το μέσο του διαστήματος, αλλά κάποια άλλη τιμή.
	
	Παρατηρείται ότι αυτή η μέθοδος συγκλίνει γρηγορότερα.
	
	\begin{tikzpicture}[scale=2]
	\draw (0,-0.2) -- (0,2);
	\draw (-0.5,0) -- (3,0);
	
	\draw[dashed] (2.7,1.7) node [right] {$\left(b_1,f(b_1)\right)$}
	-- (2.7,0);
	
	\draw[very thick, blue!50!magenta] plot[smooth,tension=1]
	coordinates {(0.5,-0.4) (2,0.3) (2.7,1.7)};
	\draw[blue!50!magenta] (2.11,0.2) node {$f$};
	
	\draw[dashed] (0.5,-0.4) node[below left] {$\left(a_1,f(a_1)\right)$} -- (0.5,0) node[above] {$a_1$};
	\draw[cyan,thick] (0.5,-0.4) -- (2.7,1.7);
	\draw[dashed] (0.93,-0.3) node[below,yshift=-2mm] {$\left(a_2,f(a_2)\right)$} -- (0.93,0) node[above] {$a_2$};
	\draw [cyan,thick] (0.93,-0.3) -- (2.7,1.7);
	\draw[dashed] (1.2,0) -- (1.2,-0.2);
	\draw [cyan,thick] (1.2,-0.2) -- (2.7,1.7);
	\end{tikzpicture}
	
	Πρέπει να βρούμε τη ρίζα εντός του διαστήματος \( (a_1,b_1) \).
	\begin{itemize}
		\item Παίρνουμε τη χορδή που ενώνει τα \( a_1, b_1 \).
		\item Η χορδή αυτή τέμνει τον άξονα των \( x \) στο σημείο \( a_ 2\),
		επομένως η ρίζα βρίσκεται μεταξύ των \( a_2, b_1 \).
		\item Παίρνουμε τη χορδή που ενώνει τα \( a_2, b_2 \)
		\item Η χορδή αυτή τέμνει τον άξονα των \( x \) στο σημείο \( a_3 \),
		επομένως η ρίζα βρίσκεται μεταξύ των \( a_3,b_1 \)
		\item \( \cdots \)
	\end{itemize}
	
	Παρατηρούμε ότι η μέθοδος χορδής λειτουργεί για
	κυρτές συναρτήσεις.
	
	\begin{infobox}{Κυρτές συναρτήσεις}
		Όταν λέμε ότι μια συνάρτηση είναι κυρτή,
		εννοούμε ότι η καμπύλη της είναι κυρτή,
		δηλαδή για δύο σημεία της, η χορδή που τα ενώνει
		δεν τέμνει κάποιο σημείο της \( f \). Στην αντίθετη περίπτωση,
		η συνάρτηση λέγεται μη κυρτή.
	\end{infobox}
	
	\paragraph{}
	\( a_1,b_1 \) όπου \( f(a_1)f(b_1) < 0 \) for
	\( k=1,2,\dots \)
	\begin{align*}
		& \left(a, f(a)\right) \qquad \left(b,f(b)\right)
		\\
		y-f(b) &= \frac{f(b)-f(a)}{(b-a)}(x-b) \\
		\text{Σημείο τομής με άξονα } x:
		\Aboxed{x &= \frac{af(b)-bf(a)}{f(b)-f(a)}}
		\\
		c &= \frac{a_kf(b_k)=b_kf(a_k)}{f(b_k)-f(a_k)}
		\\ f: \quad f(a_k)f(c) < 0 &\implies
		a_{k+1} = a_k, \quad b_{k+1}=c \\
		\text{else} &\implies 
		a_{k+1} = c, \quad b_{k+1} = b_k
	\end{align*}
	
	Αν και η μέθοδος χορδής είναι αργή, συνεχίζει να
	είναι γρηγορότερη από τη μέθοδο διχοτόμησης.
	
	\subsection{Μέθοδος Μεταβαλλόμενης Χορδής}
	\begin{tikzpicture}[scale=2]
	\draw (0,-0.2) -- (0,2);
	\draw (-0.5,0) -- (3,0);
	
	\draw[dashed] (2.7,1.7) node [right] {$\left(b_1,f(b_1)\right)$}-- (2.7,0);
	
	\draw[very thick, blue!50!magenta] plot[smooth,tension=1]
	coordinates {(0.5,-0.4) (2,0.3) (2.7,1.7)};
	
	\draw[dashed] (0.5,-0.4) -- (0.5,0) node[above] {$a_1$};
	\draw[draw=cyan,thick] (0.5,-0.4) -- (2.7,1.7);
	\draw[dashed] (0.93,-0.3) -- (0.93,0) node[above,yshift=1mm] {$a_2$};
	\draw [draw=cyan,thick] (0.93,-0.3) -- (2.7,0.85)
	node[right]  {$ \left(b_1,\frac{f(b_1)}{2}\right)$};
	\draw (1.4,0) node[above] {$a_3$} -- (1.4,-0.12);
	\end{tikzpicture}
	
	Όπως η μέθοδος χορδής, αλλά μεταβάλλουμε την κλίση
	της χορδής γρηγορότερα, ώστε να φτάσει πιο κοντά
	στη ρίζα.
	
	Αυτό μπορούμε να το πετύχουμε π.χ. θεωρώντας
	ως 2\textsuperscript{ο} σημείο το \( \left(b_1,\frac{f(b_1)}{2}\right) \)
	αντί για το \( \left( b_1,f(b_1) \right) \), όπως φαίνεται στο
	σχήμα. Αντί για να διαιρέσουμε με 2, μπορούμε να επιλέξουμε έναν
	άλλον αριθμό, π.χ. \( 4 \) για συναρτήσεις που έχουν πιο κάθετες
	χορδές.
	
	\subsection{Μέθοδος Newton}
	Αν και η μέθοδος Newton δεν συγκλίνει πάντα και απαιτεί παραγωγισιμότητα,
	είναι πολύ πιο γρήγορη από τις προηγούμενες μεθόδους, και χρησιμοποιείται
	πολύ πιο συχνά.
	
	Παίρνουμε το ανάπτυγμα Taylor της \( f \):
	\begin{gather*}
		f(x) = f(x_0) + (x-x_0)f'(x_0) \\
		f(x_0) + (x-x_0) f'(x_0) = 0 \\
		\boxed{x = x_0 - \frac{f(x_0)}{f'(x_0)}}
	\end{gather*}
	
	\paragraph{}
	for \( \kappa = 1,2,\dots \)
	\[
	x_{\kappa+1} = x_\kappa - \frac{f(x_1)}{f'(x_1)} \ 
	\rightarrow \text{ΕΠΑΝΑΛΗΠΤΙΚΗ}
	\]
	\begin{gather*}
	\mathlarger{\text{if }} \left|x_{\kappa+1}-x_\kappa\right|<\epsilon  \
	\rightarrow \text{ΚΡΙΤΗΡΙΟ ΤΕΡΜΑΤΙΣΜΟΥ} \\
	\mathlarger{\text{stop}}, \text{ αλλιώς } \kappa = \kappa+1
	\end{gather*}
	
	Η μέθοδος Newton έχει αρκετά περίπλοκες συνθήκες
	σύγκλισης που δεν θα μελετήσουμε.
	
	\subsection{Μέθοδος Σταθερού Σημείου}
	\begin{defn}{}{Σταθερό Σημείο}
		\[
		\begin{array}{ll}
		f(x) & \quad \text{το $\hat{x}$ σταθερό σημείο}
		\\
		\text{ανν} & \quad f(\hat x) = \hat x
		\end{array}
		\]
	\end{defn}
	
	\( \bar x \) της \( f(x) \) στο \( (a,b) \)
	αν κατασκευάσω \( g(x) \):
	\[
	\bar x = g(\bar x) \iff f(\bar x) = 0
	\]
	
	\paragraph{Παράδειγμα}
	\[
	\mathlarger{f(x) = x^2-x-2}
	\]
	
	Μπορούμε να θέσουμε:
	\begin{gather*}
		g(x) = x^2 - 2 \\
		g(x) = \sqrt{x+2} \\
		g(x) = 1 - \frac{2}{x}
	\end{gather*}
	
	Αντί να λύσω την \( f \), βρίσκω υποψήφιες λύσεις
	(δύσκολες επαναληπτικές λύσεις θα δίνονται στις
	εξετάσεις)
	
	\paragraph{Μέθοδος}
	for \(k=1,2,\dots \qquad \left(x_0,g(x),\epsilon\right) \)
	\[
	\mathlarger{\mathlarger{x_i = g\left( x_{i-1} \right)}}
	\]
	\[
	\begin{array}{rl}
		\mathlarger{\text{if }}
		\left|x_i-g(x_i)\right| < \epsilon
		& \rightarrow \text{ΚΡΙΤΗΡΙΟ ΤΕΡΜΑΤΙΣΜΟΥ}
	\end{array}
	\]
	
	\paragraph{Προϋποθέσεις σύγκλισης}
	\begin{enumerate}
		\item Για αρχικό \( x_0 \), τα \( x_1,x_2,\dots \) να
		είναι υπολογίσιμα στην \( g(x) \)
		\subparagraph{π.χ.}
		\( g(x) = -\sqrt{x} \quad \) για \( x_0 > 0 \)
		\begin{gather*}
			x_1 = g(x_0) = -\sqrt{x_0} \\
			x_2 = g(x_1) = -\sqrt{x_1}
		\end{gather*}
		\item \( x_1,x_2,\dots  \) να συγκλίνουν σε ένα
		\( \bar x \)
		\item Το σημείο σύγκλισης \( \gamma \) να είναι σταθερό
		σημείο της \( g(x) \).
	\end{enumerate}
	
	Τα παραπάνω μετασχηματίζονται σε 3 κριτήρια σύγκλισης:
	\paragraph{Κριτήρια Σύγκλισης}
	\begin{enumerate}
		\item Υπάρχει \( [a,b] \) στο οποίο ορίζεται η
		\( g(x) \) και \( g(x) \in [a,b] \), δηλαδή:
		\[
		g: [a,b] \to [a,b]
		\]
		\item \( g(x) \) συνεχής στο \( [a,b] \)
		\item \( g(x) \) παραγωγίσιμη και να υπάρχει \(k<1\):
		\[
		\forall x \in [a,b], \
		\left|g'(x)\right| \leq k
		\]
	\end{enumerate}
	
	\paragraph{Θεώρημα}
	Αν ισχύουν οι 3 παραπάνω προϋποθέσεις, τότε στο διάστημα
	\( [a,b] \) υπάρχει ακριβώς ένα σταθερό σημείο \(\gamma\)
	της \(g\), και η μέθοδος σταθερού σημείου συγκλίνει σε αυτό
	το \( \gamma \).
	
	\subsection{Ασκήσεις}
	\paragraph{Άσκηση}
	\[
	\mathlarger{x^3+2x-1=0}
	\]
	Να αποδειχθεί ότι έχει μια μόνο ρίζα στο \( \left[0,\frac{1}{2}\right] \) και
	να προσεγγιστεί με τη μέθοδο σταθερού σημείου.
	\subparagraph{Λύση}
	\[ \left.
	\begin{array}{l}
	f(0) = -1,\ f\left(\frac{1}{2}\right) = \frac{1}{8} \implies
	f(0)f\left(\frac{1}{2}\right) < 0 \\
	f'() = 3x^2 + 2 > 0
	\end{array} \right\rbrace \text{μοναδικότητα}
	\]
	\begin{gather*}
	g(x) = x \iff f(x) = 0 \\
	g(x) = \frac{1}{2} (1-x^3) \text{ (περιμένω να δουλέψει, δηλαδή να ικανοποιούνται
		τα κριτήρια σύγκλισης)}
	\end{gather*}
	\begin{align*}
		&\mathrm Y_1 \ \checkmark \\
		&\mathrm Y_2 \ \checkmark \\
		&\mathrm Y_3:\ g'(x) = \frac{-3}{2} x^2 \\
		&\ \left|g'(x)\right| = \frac{3}{2} x^2 \leq \frac{3}{2}\left(\frac{1}{2}\right)^2
		= \frac{3}{8} < 1 \\
		\text{Άρα: } & \mathrm Y_3 \ \checkmark
	\end{align*}
	(όπου \( \mathrm Y \) τα κριτήρια σύγκλισης)
	
	Επομένως, η \( g(x) \) που επέλεξα ικανοποιεί τη σύγκλιση.
	
	Επιλέγω: \( x_0 = 0 \)
	\begin{align*}
		x_1 &= g(x_0) = g(0) = \frac{1}{2} \\
		x_2 &= g(x_1) = g\left(\frac{1}{2}\right) = \frac{1}{2}\left(
		1-\left(\frac{1}{2}\right)^3
		\right) = 0.4375 \\
		x_3 &= g(x_2) = \frac{1}{2} \left(1 - 0.4375^3\right) = 0.4581298 \\
		x_4 &= g(x_3) = \frac{1}{2} \left(1-0.4581298^3\right) = 0.451923191
	\end{align*}
	\paragraph{Άσκηση}
	Να υπολογιστή ένας τύπος με τη μέθοδο Newton που θα βρίσκει την τετραγωνική ρίζα θετικού
	αριθμού.
	\subparagraph{Λύση}
	\begin{gather*}
		x = \sqrt{a} \implies x^2=a \\[0.3ex] \\
		f(x) = x^2-a \\
		f'(x) = 2x \\
		x_{n+1} = x_n - \frac{f(x_n)}{f'(x_n)} = x_n - \frac{x_n^2-a}{2x_n}
		= \frac{1}{2} \left( x_n + \frac{a}{x_n} \right)
	\end{gather*}
	\subparagraph{Παράδειγμα}
	\[
	\mathlarger{\sqrt{3} = ?, \qquad \text{με } x_0 = 1.5}
	\]
	\begin{gather*}
		x_1 = \frac{1}{2} \left(x_0+\frac{3}{x_0}\right)
		= \frac{1}{2} \left(1.5+\frac{3}{1.5}\right) = 1.75 \\
		x_2 = \frac{1}{2} \left(x_1+\frac{3}{x_1}\right) = 1.7321428 \\
		\intertext{Υπολογισμός με ακρίβεια 3 δεκαδικών ψηφίων (δύο διαδοχικές ρίζες
			να απέχουν \( 0.5\cdot 10^{-3} \implies \))
			(σφάλμα) \(\to 0.5\cdot 10^{-3}\)  }
		x_3 = \frac{1}{2} \left( x_2 + \frac{3}{x_2} \right) = 1.7320528 \\
		\left|x_3-x_2\right| < 0.5\cdot 10^{-3} \text{ (ικανοποίηση κριτηρίου τερματισμού)}
	\end{gather*}
	
	Άρα:
	\[
	\boxed{x_3 = 1.7320528}
	\]
	\paragraph{Άσκηση}
	\[
	\mathlarger{x^3+2x^2+10x-20=0}
	\]
	Να προσεγγιστεί μια ρίζα της παραπάνω εξίσωσης (π.χ. για 3 επαναλήψεις)
	\subparagraph{Λύση}
	\begin{gather*}
		x_{n+1} = x_n - \frac{f(x_n)}{f'(x_n)} = \frac{2x_n^3+2x_n^2+20}{3x_n^2+4x_n+10} \\
		x_0 = 1 \\
		x_1 = \frac{2\cdot 1^3 + 2\cdot 1^2 + 20}{3\cdot 1^2 + 4\cdot 1 + 10}
		= \dots = 1.41764706 \\
		x_2 = \frac{2x_1^3+2x_1^2+20}{3x_1^2+4x_1+20} = 1.369336471 \\
		x_3 = \frac{2x_2^3+2x_2^2+20}{3x_2^2+4x_2+10} = 1.368908108
	\end{gather*}
	\paragraph{Άσκηση}
	Να βρεθεί μια λύση της εξίσωσης
	\[
	\mathlarger{x^3+x+1}
	\]
	στο διάστημα \( (0,1) \)
	με ακρίβεια 2 δεκαδικών ψηφίων χρησιμοποιώντας τη μέθοδο
	χορδής, και να συγκριθεί με τη μέθοδο Newton.
	\subparagraph{Λύση}
	\begin{align*}
	x_{n+1} &= x_n - f(x_n)\frac{x_n-x_{n-1}}{f(x_n)-f(x_{n-1})}
	\\
	x_2 &= x_1 - f(x_1) \frac{x_1-x_0}{f(x_1)-f(x_0)}
	= 1-1\frac{1}{2} = 0.5
	\\
	x_3 &= x_2 - f(x_2) \frac{x_2-x_1}{f(x_2)-f(x_1)} = 0.636
	\\
	x_4 &= x_3 - f(x_3) \frac{x_3-x_2}{f(x_3)-f(x_2)} = 0.606
	\\
	\vdots
	\\ x_6 &= \dots = 0.687
	\\ x_7 &= \dots = 0.682
	\end{align*}
	Άρα με ακρίβεια 2 δεκαδικών ψηφίων, \( \rho=0.68 \), και
	χρησιμοποιήσαμε 7 επαναλήψεις.
	\subparagraph{Λύση με μέθοδο Newton}
	\begin{align*}
		x_{n+1} &= x_n - \frac{f(x_n)}{f'(x_n)} \\
		x_{n+1} &= x_n - \frac{x_n^3+x_n-1}{3x_n^2+1} \\
		x_{n+1} &= \frac{2x_n^3+1}{3x_n^2+1} \\
		\intertext{\(x_0 = 0.5\)}
		x_1 &= \frac{2\cdot 0.5^2 + 1}{3\cdot 0.5^2+1}
		= 0.714 \\
		x_2 &= 0.683 \\
		x_3 &= 0.682
	\end{align*}
	
	\paragraph{Άσκηση}
	Να βρεθεί μια λύση της:
	\[
	\mathlarger{f(x) = x^{10} -1} \qquad (0,1.3)
	\]
	
	\begin{tikzpicture}[scale=.7]
		\begin{axis}[samples=30,very thick,domain=0:1.3,no marks]
		\addplot+{x^10-1};
		\end{axis}
	\end{tikzpicture}
	
	\subparagraph{Λύση με διχοτόμηση}
	Με τη μέθοδο της διχοτόμησης έχουμε:
	\[
	\begin{array}{c|c|c}
	a_k & b_k & c  \\ \hline
	0 & 1.3 & 0.65 \\
	0.65 & 1.3 & 0.975 \\
	0.975 & 1.3 & 1.1375 \\
	0.975 & 1.1375 & 1.05625 \\
	0.975 & 1.05625 & 1.015625
	\end{array}
	\]
	
	\subparagraph{Λύση με μέθοδο χορδής}
	Με τη μέθοδο της χορδής έχουμε (εφαρμόζοντας τον τύπο
	όπως και στην προηγούμενη άσκηση):
	\[
	\begin{array}{c|c|c}
	a_k & b_k & c  \\ \hline
	0 & 1.3 & 0.09430 \\
	0.09430 & 1.3 & 0.18176 \\
	0.18176 & 1.3 & 0.26287 \\
	0.26287 & 1.3 & 0.33811 \\
	0.33811 & 1.3 & 0.40708
	\end{array}
	\]
	
	Παρατηρούμε ότι σε αυτήν την περίπτωση, η μέθοδος χορδής
	είναι αρκετά πιο αργή. Για να το αποτρέψουμε αυτό, μπορούμε
	να χρησιμοποιήσουμε μεταβαλλόμενη χορδή.
	
	\subparagraph{Λύση με μέθοδο Newton}
	\[
	x_{i+1} = x_i - \frac{x_i^{10}-1}{10x_i^9}
	\]
	
	\[
	\begin{array}{c|c}
	i & x_i \\ \hline
	0 & 0.5 \\
	1 & 51.65 \\
	2 & 46.485 \\
	3 & 41.8365 \\
	4 & 37.65285 \\
	\vdots & \vdots \\
	40 & 1.002316
	\end{array}
	\]
	
	\paragraph{Άσκηση}
	Να βρεθεί μία ρίζα της εξίσωσης
	\[
	\mathlarger{f(x) = x^3-2x-5}
	\]
	στο διάστημα \( [2,3] \) χρησιμοποιώντας τη μέθοδο της
	διχοτόμησης, με ακρίβεια \( \epsilon = 10^{-6} \).
	\subparagraph{Λύση}
	\[
	\begin{array}{r|c|c|c}
	i & a_k & b_k & c \\ \hline
	1 & 2 & 3 & 2.5 \\
	2 & 2 & 2.5 & 2.25 \\
	3 & 2 & 2.25 & 2.125 \\
	\vdots & \vdots & \vdots & \vdots \\
	19 & & & 2.0945530
	\end{array}
	\]
	
	\section{Παρεμβολή}
	\subsection{Το πρόβλημα}
	
	Έστω μια άγνωστη \( f(x) \) ορισμένη στο \( [a,b] \) και οι τιμές
	\( x_i\ (i=0,\dots,n-1) \). Επίσης, έστω μια \( p(x) \) που
	παρεμβάλλει την \( f(x) \), δηλαδή:
	\[
	p(x_i) = f(x_i).
	\]
	
	Θα ψάξω να βρω την \( p(x) \) με βάση τις τιμές \( p(x_i) \).
	
	Όταν ψάχνω να βρω μια προσέγγιση για κάποια τιμή \( \bar x \
	(\bar x \in \left[x_0,x_{n-1}\right]) \) με βάση την \( p(x) \) που
	προκύπτει με δεδομένες τιμές για \( x_i \), λέγεται ότι κάνω
	\textbf{intrapolation}. \\
	Για \( \bar x \neq \left[x_0,x_{n-1}\right] \) (πιο επικίνδυνη
	περίπτωση), ονομάζεται \textbf{extrapolation}.
	
	\begin{theorem}{}{Weierstrass}
		Για οποιαδήποτε συνάρτηση, υπάρχει ένα πολυώνυμο που την
		παρεμβάλλει, δηλαδή:
		\[
		\lim_{n\to \infty} P_n(x) = f(x)
		\]
	\end{theorem}
	\begin{tikzpicture}
	
	\draw (-1,0) -- (3,0);
	\draw (0,0) -- (0,3);
	
	\draw[green!50!black,very thick] plot[smooth] coordinates {(-0.7,0.8) (-0.5,1) (1,1) (2,1.2) (2.5,1.3)}
	node[below right] {$f(x)$};
	\draw[cyan!50!black,very thick] plot[smooth] coordinates {(-0.8,0.9) (-0.5,1.2) (1,1.1) (2,1.3) (2.5,1.7)}
	node[right] {$p(x)$};
	
	\end{tikzpicture}
	
	Και μάλιστα:
	\[
	\left|P_n(x)-f(x)\right| \leq \epsilon \quad
	\text{$\epsilon$ δεδομένη ακρίβεια}
	\]
	
	Όσο το \( \epsilon \) μικραίνει, ο βαθμός του \( P_n(x) \) αυξάνεται.
	
	Για παράδειγμα, αν έχουμε μερικά σημεία της \( f \):
	\[
	\begin{array}{c|c|c|c|c}
	x_i & 1 & 2 & 3 & 4 \\ \hline
	f(x_i) & 7 & 12 & 21 & 20
	\end{array}
	\]
	ψάχνω ένα πολυώνυμο (κάποιου βαθμού) που να περνάει από αυτά τα
	σημεία.

	\begin{tikzpicture}
	\draw (-1,0) -- (3,0);
	\draw (0,-0.5) -- (0,2);

	\filldraw[cyan] (0.5,0.5) circle (2.5pt)
	(0.9,1) circle (2.5pt)
	(1.5,1.5) circle (2.5pt)
	(2.4,0.7) circle (2.5pt);
	
	\draw[green!30!black,very thick] plot[smooth,tension=0.7]
	coordinates {(0.1,0,1) (0.5,0.5) (0.9,1) (1.5,1.5) (2.4,0.7) (2.6, 0.2)}
	node[right,scale=0.6] {$p(x)$};						;
	
	\end{tikzpicture}
	
	Αν έχω δύο σημεία, το ζητούμενο πολυώνυμο είναι μια ευθεία, δηλαδή
	ένα πολυώνυμο 1\textsuperscript{ου} βαθμού. Για τρία σημεία, θα
	χρειαστώ πολυώνυμο 2\textsuperscript{ου} βαθμού, και γενικά έχω
	ένα μοναδικό
	πολυώνυμο \( n \) βαθμού για \( n+1 \) διακριτά σημεία,
	όπως θα δούμε αργότερα.

	\begin{tikzpicture}[scale=1.3]
	\draw (-1,0) -- (1,0);
	\draw (0,-0.5) -- (0,1.5);
	
	
	\filldraw[cyan!50!blue] (-0.5,1) circle (2pt)
	(0,0) circle (2pt)
	(0.5,1) circle (2pt);
	
	\draw[red!90!blue,very thick] (0,0) -- (0.5,1) --(0.7,1.4);
	
	\draw[green!30!black,very thick] plot[smooth,tension=0.7]
	coordinates {(-0.5,1) (0,0) (0.5,1)};
	
	\begin{scope}[xshift=2.5cm]
	\draw (-0.2,0) -- (2,0);
	\draw (0,-0.5) -- (0,1.5);
	
	\draw[red!90!blue,very thick] (0.1,-0.2) -- (2.2,2.2);
	
	\draw[green!70!black,thick] plot[smooth,tension=0.7]
	coordinates {(0.25,0) (0.5,0.25) (0.75,1.2)  (1.25,0.8) (1.5,1.45) (2,2)};
	\filldraw[green!50!red] (0.5,0.25) circle (1pt) (1.5,1.45) circle (1pt);
	\end{scope}
	\end{tikzpicture}
	
	Η πραγματική συνάρτηση μπορεί να έχει διαφορετική συμπεριφορά
	ανάμεσα στα σημεία της τα οποία γνωρίζουμε.
	
	Στο επόμενο κεφάλαιο θα μάθουμε την προσέγγιση, δηλαδή την εύρεση
	ενός πολυωνύμου συγκεκριμένου βαθμού που να φτάνει κοντά στην
	\( f \), έτσι ώστε να είναι πιο εύκολα υπολογίσιμο αν π.χ. έχουμε
	300 σημεία της συνάρτησης.
	
	\subsection{Μορφές Αναπαράστασης Πολυωνύμου}
	\paragraph{Εκθετική μορφή}
	\( 
	\displaystyle P_n(x) = a_0 + a_1x+a_2x^2 + \dots + a_nx^n
	 \)
	\paragraph{Μορφή κέντρων}
	\( 
	\displaystyle P_n(x) = b_0 + b_1(x-c) + b_2(x-c)^2 + \dots
	+ b_n(x-c)^n
	 \)
	\paragraph{Μορφή Newton}
	\( 
	\displaystyle P_n(x) = a_0 + a_1(x-c_1) + a_2(x-c_1)(x-c_2)
	+ \dots + a_n(x-c_1)(x-c_2)\cdots(x-c_n)
	= a_0 + \sum_{k=1}^{n} a_k \prod_{i+1}^{k} (x-c_i)
	 \)
	\paragraph{Φωλιασμένη μορφή Newton}
	\begin{align*}
	P_n(x) &= a_0 + (x-c_1)\left[a_1
	+ a_2(x-c_2) + a_3(x-c_2)(x-c_3) + \dots + a_n(x-c_2)\cdots(x-c_n)
	\right] = \dots \\ &= a_0 + (x-c_1)\left[a_0+(x-c_2)
	\left[a_2+ (x-c_3)\left[a_3+\dots + (x-c_{n-1})\left[
	a_{n-1}+(x-c_n)a_n
	\right]\right]\right]
	\right]
	 \end{align*}
	 
	 \paragraph{Μορφή Lagrange}
	 Δεδομένων \( n+1 \) σημείων \( x_0,x_1,\dots,x_n \), έχουμε 
	 τα πολυώνυμα:
	 
	 \[
	 l_j(x) = \frac{
	 	(x-x_0)(x-x_1)\cdots(x-x_{j-1})(x-x_{j+1})\cdots(x-x_n)
	 	}{
	 	(x_j-x_0)(x_j-x_1)\cdots(x_j-x_{j-1})(x_j-x_{j+1})
	 	\cdots (x_j-x_n)
	 	}
	 \]
	 για \( j=0,1,\dots,n \) ονομάζονται \textbf{πολυώνυμα Lagrange} (ο
	 παρονομαστής σταθερός).
	 
	 Παρατηρούμε ότι:
	 \[
	 l_j(x_i) = \begin{cases}
	 0 & \quad i \neq j \\
	 1 & \quad i = j
	 \end{cases} \ = \delta_{ij}
	 \]
	 
	 Άρα για σταθερές \( a_0,a_1,\dots,a_n \) και σημεία \( x_0,x_1,
	 \dots,x_n \), το πολυώνυμο \[
	 P_n(x) = a_0l_0(x) + a_1l_1(x) + \dots + a_nl_n(x)
	 \]
	 ονομάζεται \textbf{πολυώνυμο σε μορφή Lagrange}.
	 
	 \subsection{Μέθοδοι Παρεμβολής}
	 \begin{gather*}
	    P(x_i) = f(x_i) \\
	 	P(x_i) = \sum_{k=0}^{n} a_kl_k(x_i) = a_i \\[.3ex]
	 	\boxed{P(x) = \sum_{k=0}^n f(x_k)l_k(x)}
	 \end{gather*}
	 
	 \paragraph{Παράδειγμα}
	 Έχουμε τη συνάρτηση:
	 \[
	 \begin{array}{r|c|c|c}
	 i & 0 & 1 & 2 \\ \hline
	 x_i & -1 & 0 & 1 \\ \hline
	 f(x_i) & 1 & 1 & 2
	 \end{array}
	 \]
	 
	 Αναζητώ πολυώνυμο που να παρεμβάλλει την \( f \) σε
	 \begin{enumgreekpar}
	 	\item μορφή Lagrange
	 	\item εκθετική μορφή
	 \end{enumgreekpar}
	 
	 \subparagraph{Λύση}
 	\begin{align*}
 		l_0(x) &=
 		\frac{(x-0)(x-1)}{(-1-0)(-1-1)} = \frac{x(x-1)}{2} \\
 		l_1(x) &=
 		\frac{(x+1)(x-1)}{(0+1)(0-1)} = 1-x^2 \\
 		l_2(x) &= \frac{(x+1)(x-0)}{(1+1)(1-0)} = \frac{x(x+1)}{2} \\
 		\Aboxed{P(x) &= 1l_0(x)+1l_1(x)+2l_2(x)}
 		\quad \leftarrow \text{ μορφή Lagrange}
 		\\ &= \frac{1}{2}x^2 + \frac{1}{2}x + 1
 		\quad \leftarrow \text{ εκθετική μορφή}
 	\end{align*}
 	
 	Ένας άλλος τρόπος θα ήταν να θεωρήσουμε πολυώνυμο
 	2\textsuperscript{ου} βαθμού με 3 άγνωστους συντελεστές, και να λύσω
 	το σύστημα για να τους βρω, κάτι που θα χρησιμεύει αργότερα στην
 	προσέγγιση:
 	\begin{align*}
 	    P(x) &= a_0 + a_1x + a_2x^2 \\
 	    P(-1) &= a_0 - a_1 + a_2 = 1 \\
 	    P(0) &= a_0 = 1 \\
 	    P(1) &= a_0 + a_1 + a_2 = 2
 	\end{align*}
 	
 	\subsubsection{Μέθοδος διηρημένων διαφορών}
 	Η ιδέα είναι να ξεκινάμε από τον μικρό βαθμό, και να χτίζουμε το
 	ζητούμενο πολυώνυμο βαθμό-βαθμό.
 	\begin{align*}
 	P_{k+1}(x) &=
 	\underbrace{a_0+a_1(x-x_0)+\dots+a_k(x-x_0)\cdots(x-x_{k-1})}_{%
 		P_k(x)} + a_{k+1} (x-x_0)\cdots(x-x_k) \\[.4ex]
 	f(x_{k+1}) &= P_{k+1}(x_{k+1})
 	= P_k(x_{k+1}) + a_{k+1}\prod_{i=0}^{k} \left(x_{k+1}-x_i\right) \\
 	a_{k+1} &= \frac{f(x_{k+1}) - P_k(x_{k+1})}{
 		\prod_{i=0}^{k} \left(x_{k+1}-x_i\right)
 		}
 	\end{align*}
 	
 	Για να διευκολύνω τις πράξεις ορίζω:
 	\begin{defn}{Πρώτη Διηρεμένη Διαφορά}{}
 		\[
 		\mathlarger{
 		f\left[x_i,\ x_{i+1}\right] =
 		\frac{f(x_{i+1})-f(x_i)}{x_{i+1}-x_i}
 	    }
 		\]
 	\end{defn}
 	
 	Η \( k \)-οστή Διηρεμένη Διαφορά είναι:
 	\begin{defn}{k-οστή Διηρεμένη Διαφορά}{}
 		\[
 		\mathlarger{
 			f\left[x_i,\ x_{i+1}, \dots,\ x_{i+k}\right] =
 			\frac{
 				f[x_{i+1},x_{i+2},\dots,x_{i+k}]
 				- f[x_i,x_{i+1},\dots,x_{i+k-1}]
 				}{x_{i+k}-x_i}
 		}
 		\]
 	\end{defn}
 	
 	Για διευκόλυνσή μας, αντί να χρησιμοποιούμε τους τύπους, μπορούμε
 	να χρησιμοποιήσουμε τον Πίνακα Διηρημένων Διαφορών.
 	
 	\paragraph{Πίνακας ΔΔ}
 	\todo{Convert matrix to tikz}
 	\[
 	\begin{array}{llllll}
 	 & & 1^{\text{η}} & 2^\text{η} & 3^\text{η} & 4^\text{η}\\
 	x_0 & f(x_0) & f[x_0,x_1] & = \frac{f(x_1)-f(x_2)}{x_1-x_2} & \\
 	x_1 & f(x_1) & f[x_1,x_2] & f[x_0,x_1,x_2] & =
 	\frac{f[x_1,x_2,x_3]-f[x_0,x_1,x_2]}{x_3-x_0} \\
 	x_2 & f(x_2) & f[x_2,x_3] & f[x_1,x_2,x_3] & f[x_0,x_1,x_2,x_3]\\
 	x_3 & f(x_3) & f[x_3,x_4] & f[x_2,x_3,x_4] & f[x_1,x_2,x_3,x_4]
 	& f[x_0,x_1,x_2,x_3,x_4]
 	\\
 	x_4 & f(x_4) & &
 	\end{array}
 	\]
 	
 	Τότε το πολυώνυμο θα είναι:
 	\begin{align*}
 		P_k(x) &= f(x_0) + f[x_0,x_1](x-x_0) +
 		f[x_0,x_1,x_2](x-x_0)(x-x_1)
 		\\ & + \dots +
 		f[x_0,x_1,\dots,x_k](x-x_0)(x-x_1)\cdots(x-x_{k+1})
 	\end{align*}
 	
 	\paragraph{Παράδειγμα}
 	\hspace{0pt}
 	
 	\begin{tabular}{|c|c|c|c|c|}
 		\hline 
 		\(i\) & 0 & 1 & 2 & 3 \\ 
 		\hline 
 		\(x_i\) & 0 & 1 & 2 & 4 \\ 
 		\hline 
 		\(f(x_i)\) & 1 & 1 & 2 & 5 \\ 
 		\hline 
 	\end{tabular}
 	
 	Πίνακας ΔΔ:
 	\[
 	\begin{array}{ccccc}
 	0 & 1 \searrow & & &  \\
 	& & \frac{1-1}{1-0} \searrow & & \\
 	1 & 1 \nesearrow & & \frac{1}{2} \searrow & \\
 	& & 1 \nesearrow & & \frac{1}{12} \\
 	2 & 2 \nesearrow & & \frac{1}{6} \nearrow & \\
 	& & \frac{3}{2} \nearrow & & \\
 	4 & 5 \nearrow & & &
 	\end{array}
 	\]
 	
 	Άρα:
 	\begin{align*}
 	P_3(x) &= 1 + 0(x-0) + \frac{1}{2}(x-0)(x-1)-\frac{1}{12}
 	(x-0)(x-1)(x-2) \\ &= \frac{1}{12} (-x^3+9x^2-8x+12)
 	\end{align*}
 	\todo{Plot this?}
 	
 	\subsubsection{Μέθοδος πεπερασμένων διαφορών}
 	Για τη μέθοδο αυτή, θεωρούμε ότι τα σημεία της \( f \)
 	που έχουμε ισαπέχουν μεταξύ τους:
 	\begin{gather*}
 		x_0,x_1,\dots,x_n \ \in [a,b] \\
 		x_i = x_0 + ih \quad \text{όπου } h=\frac{b-a}{n}
 		\\[.4ex]
 		f\left[x_i,x_{i+1}\right] =
 		\frac{f(x_{i+1})-f(x_i)}{h}
 	\end{gather*}
 	
 	Ορίζω συμβολισμούς για ευκολία:
 	\begin{align*}
 		\Delta^0 f(x_i) &= f(x_i) \\
 		\Delta^1 f(x_i) &=
 		f(x_{i+1}) - f(x_i) = f(x_i+h) - f(x_i) \\
 		\vdots \\
 		\Delta^k f(x_i) &=
 		\Delta \left(\Delta^{k-1}\left(f(x_i)\right)\right)
 		= \Delta^{k-1} f(x_{i+1}) - \Delta^{k-1} f(x_i)
 	\end{align*}
 	
 	\begin{defn}{k-οστή πεπερασμένη διαφορά}{}
 		\[
 			f\left[x_i,x_{i+1},\dots,x_{i+k}\right]
 			= \frac{\Delta^k f(x_i)}{k! h^k}
 			\qquad \text{προκύπτει μετά από πράξεις}
 		\]
 	\end{defn}
 	
 	Άρα το πολυώνυμο γίνεται:
 	\begin{align*}
 	P_n(x) &= f(x_0) + f[x_0,x_1](x-x_0) + \dots
 	\\ & \quad \dots +
 	f[x_0,x_1,\dots,x_n](x-x_0)(x-x_1)\dots(x-x_{n-1})
 	\\ &= f(x_0) + \frac{\Delta^1 f(x_0)}{1!h^1}(x-x_0)
 	+ \frac{\Delta^2 f(x_0)}{2!h^2}(x-x_0)(x-x_1) + \dots \\
 	& \quad \dots + \frac{\Delta^n f(x_n)}{n!h^n}
 	(x-x_0)(x-x_1)\cdots
 	(x-x_{n-1})
 	\end{align*}
 	
 	Θα βρούμε μια διαδικασία ώστε να υπολογίζουμε το πολυώνυμο
 	γρηγορότερα. Αν \( x = x_0+rh \), αποδεικνύεται πως
 	(οι πράξεις υπάρχουν στο βιβλίο):
 	\[
 	\mathlarger{
 		\boxed{
 			P_n(x_0+rh) = \sum_{i=0}^{n}
 			\left(\begin{matrix}
 			r\\i
 			\end{matrix}\right)
 			\Delta^i f(x_0)
 			}
 		}
 	\]
 	όπου \( \displaystyle \binom{r}{i} =
 	\frac{r(r-1)\cdots\left(r-(i-1)\right)}{i!} \)
 	
 	\paragraph{Παράδειγμα}
 	
 	\begin{tabular}{|c|c|c|c|c|c|}
 		\(i\) & 0 & 1 & 2 & 3 & 4\\ 
 		\hline 
 		\(x_i\) & 1 & 2 & 3 & 4 & 5\\ 
 		\hline 
 		\(f(x_i)\) & 1 & -1 & 1 & -1 & 1
 	\end{tabular}
 	
 	\todo{Graph 10}
 	
 	\begin{align*}
 		P_4(1+r) &= 1-2\binom{r}{1}+4\binom{r}{2}-8\binom{r}{3}
 		+6\binom{r}{4}
 		\intertext{όπου \( x = x_0+rh \implies
 			r=1.5 \text{ για } x = 2.5 \)}
 		\\
 		P(2.5) &= 1 -2 \binom{1.5}{1} + 4\binom{1.5}{2}
 		-8\binom{1.5}{3} + 16\binom{1.5}{4}
 	\end{align*}
 	
 	\subsubsection{Μέθοδος Aitken}
 	Έστω \( p,q \) δύο πολυώνυμα παρεμβολής μικρότερου
 	βαθμού (για \( m+2 \) σημεία),
 	και \( r \) το πολυώνυμο παρεμβολής για όλα
 	τα σημεία (\( m+3 \) σημεία) \( \qquad x_0,x_1,\dots,
 	x_m,y,z \), και γνωρίζω:
 	\begin{itemize}
 		\item \( p(x_i)=q(x_i)=r(x_i)=f(x_i)
 		\text{ για } i=0,1,\dots,m \)
 		\item \( p(y) = r(y) = f(y) \)
 		\item \( q(z) = r(z) = f(z) \)
 	\end{itemize}
 	
 	Τότε η \( r(x) \) που είναι η συνάρτηση που θέλω να
 	παρεμβάλλει την \( f \) είναι:
 	\[
 	\boxed{
 	r(x) = \frac{(y-x)q(x) - (z-x)p(x)}{y-z}
    }
 	\]
 	
 	\paragraph{}
 	Έστω \( A_{k,r} \) η τιμή στο \( \bar x \) του πολυωνύμου
 	παρεμβολής της \( f(x) \) στα σημεία \( x_0,x_1,
 	\dots,x_{r-1},x_r \):
 	\begin{align*}
 	A_{k,r} &= f[x_0] + f[x_0,x_1](\bar x-x_0)
 	+ f[x_0,x_1,x_2](\bar x-x_0)(\bar x-x_1) + \dots
 	\\ &\quad \dots + f[x_0,x_1,\dots,x_k](\bar x-x_0)
 	(\bar x-x_1)\cdots(\bar x-x_{k-1}) \\
 	A_{k,r} &= \frac{(x_{k-1}-\bar x)A_{k-1,r}
 		-(x_r-\bar x)A_{k-1,k-1}}{x_{r-1}-x_r}
	\end{align*}
	\todo{Graph 12}
	\paragraph{π.χ.}
	\begin{align*}
		A_{2,3} &=
		\frac{(x_1-\bar x)A_{1,3}-(x_3-\bar x)A_{1,1}}{x_1-x_3}
	\end{align*}
	
	\subsection{Ασκήσεις}
	\paragraph{Άσκηση}
	Να βρεθεί πολυώνυμο Lagrange που να παρεμβάλλει την
	\( \mathlarger{f(x)=e^x} \) στα σημεία \( x_0=-1,\ x_1=0,\ x_2=1 \).
	\subparagraph{Λύση}
	Θυμόμαστε ότι το πολυώνυμο Lagrange δίνεται από τους τύπους:
	\begin{align*}
	P_n(x) &= l_0(x)f(x_0)+l_1(x)f(x_1)+l_2(x)f_2(x_2) \\
	l_0(x) &= \frac{(x-x_1)(x-x_2)}{(x_0-x_1)(x_0-x_2)}
	= \frac{(x-0)(x-1)}{(-1)(-2)} = \frac{1}{2}(x^2 - x) \\
	l_1(x) &= \frac{(x-x_0)(x-x_2)}{(x_1-x_0)(x_1-x_2)}
    = \frac{(x+1)(x-1)}{1\cdot (-1)} = -x^2+1 \\
    l_2(x) &= \frac{(x-x_0)(x-x_1)}{(x_2-x_0)(x_2-x_1)} = \frac{1}{2}
    (x^2+x)
    \intertext{Άρα}
    P_2(x) &= \frac{1}{2e}(x^2-x)+ (-x^2+1) + \frac{e}{2}(x^2+x)
    \\ &=
    \left(1+\frac{e}{2}+\frac{1}{2e}\right)x^2
    +\left(\frac{e}{2}-\frac{1}{2e}\right)x -1
    \qquad \text{(δεν είναι απαραίτητο να το φτάσουμε μέχρι εδώ)}
	\end{align*}
	
	\todo{Add labels and points}
	\begin{tikzpicture}[scale=.7]
	\begin{axis}[samples=30,very thick,domain=-1.5:1.5,no marks]
	\addplot+{e^x};
	\addplot+{(x^2-x)/(2*e) + (1-x^2) + (e/2)*(x^2+x) };
	\end{axis}
	\end{tikzpicture}
	
	\paragraph{Άσκηση}
	Να βρεθεί πολυώνυμο Newton που να παρεμβάλλει την παρακάτω \( f \)
	με τη μέθοδο διηρημένων διαφορών:
	\[
	f(-1) = 1,\quad f(1) = 2 \quad \text{και} \quad f(2)=-1
	\]
	\subparagraph{Λύση}
	\todo{Graph 14}
	\[
	P_2(x) = 1 + \frac{1}{2}(x+1) - \frac{7}{6}(x+1)(x-1)
	\]
	
	\paragraph{Άσκηση}
	Να βρεθεί η τιμή \( f(0.5) \) (intrapolation)
	χρησιμοποιώντας τη μέθοδο πεπερασμένων διαφορών. Δίνεται ο πίνακας
	τιμών της \( f \)
	\[
	\begin{array}{r|c|c|c|c|c}
	x_i & 0 & 1 & 2 & 3 & 4 \\ \hline
	f(x_i) & 1 & 0 & -1 & 1 & 2
	\end{array}
	\]
	\subparagraph{Λύση}
	Οι τύποι για τις \( k \)-οστές διαφορές θα δίνονται στις εξετάσεις,
	αλλά είναι πιο κομψό και εύκολο να χρησιμοποιήσουμε τον πίνακα:
	\todo{Graph 15}
	\begin{align*}
	P_4(\overset{\mathclap{\substack{x=x_0+rh\\\uparrow}}}{x}) &=
	    f(x_0) + \binom{r}{1}\Delta f(x_0) + \binom{r}{2}\Delta^2 f(x_0)
	    + \binom{r}{3}\Delta^3 f(x_0) + \binom{r}{4} \Delta^4 f(x_0)
	\intertext{\( 0.5 = 0 + r\cdot 1 \implies r=0.5 \)}
    &= 1-\binom{0.5}{1}+0\binom{0.5}{2}+3\binom{0.5}{3}-7\binom{0.5}{4}
    = 0.9609375
    \intertext{όπου
    	\( \binom{r}{k} = \frac{r(r-1)\cdots\left(r-(k-1)\right)}{k!} \)
    	}
	\end{align*}
	
	\paragraph{Άσκηση}
	Να βρεθεί η τιμή του πολυωνύμου παρεμβολής \( P(5) \) για τα σημεία
	\( x_i \) και τις αντίστοιχες τιμές της συνάρτησης \( f(x_i) \) που
	δίνονται στον πίνακα με τη μέθοδο Aitken:
	\[
	\begin{array}{r|cccc}
	x_i & 2 & 4 & 6 & 8 \\ \hline
	f(x_i) & 1 & -2 & 3 & 5
	\end{array}
	\]
	\subparagraph{Λύση}
	\begin{align*}
		x_i - \underbrace{\bar x}_{5} & \\
		2-5 &= -3 \\ 4-5 &= -1 \\ 6-5 &= 1 \\ 8-5 &= 3
	\end{align*}\
	\todo{Graph 16}
	Άρα \( P(5) = \frac{3}{16} \), λύση που φαίνεται λογική, αφού
	βρίσκεται ανάμεσα στο -2 και το 3.
	
	
	\section{Προσέγγιση}
	Η προσέγγιση μοιάζει με την παρεμβολή, με τη διαφορά ότι δεν
	επιθυμούμε το πολυώνυμο που βρούμε να περνάει από όλα τα σημεία της
	αρχικής συνάρτησης, αλλά να έχει την ελάχιστη απόσταση από αυτά. Το
	κέρδος μας με αυτόν τον τρόπο είναι ότι το πολυώνυμο που θα βρούμε
	είναι μικρότερου βαθμού.
	
	\begin{tikzpicture}[scale=0.6]
	\draw (0,2) node {ΠΑΡΕΜΒΟΛΗ};
	\draw (0,1) node {+};
	\draw (0,0) node {ΠΡΟΣΕΓΓΙΣΗ};
	
	\draw[->] (3,2) -- ++(0.8,0);
	\draw[->] (3,0) -- ++(0.8,0);
	
	\draw (4,2) node[right] {$f(x_i) = p(x_i)$};
	\draw (4,0) node[right] {$
		\underbrace{\left|f(x_i)-p(x)\right|}_{
			\mathclap{\text{πολυώνυμο μικρότερου βαθμού}}}$
		ελάχιστη};
	\end{tikzpicture}
	
	\paragraph{}
	Έστω οι τιμές μιας συνάρτησης:
	\[
	\begin{array}{r|c|c|c|c}
	x_i & x_0 & x_1 & \dots & x_{m-1} \\ \hline
	f(x_i) & f(x_0) & f(x_1) & \dots & f(x_{m-1})
	\end{array}
	\]
	\todo{Graph 17}
	
	Αν ζητάμε η προσέγγιση \( P(x) \) να είναι μια ευθεία, τότε στην
	ιδανική περίπτωση:
	\[
	P(x_i) = a_0 + a_1x_i = f(x_i) \quad \forall x_i
	\]
	τότε έχουμε ένα σύστημα:
	\[
	\left[
	\begin{matrix}
	1 & x_0 \\ 1 & x_1 \\ 1 & x_2 \\ \vdots & \vdots \\ 1 & x_{m-1}
	\end{matrix}
	\right] \left[
	\begin{matrix}
	a_0 \\ a_1
	\end{matrix}
	\right] = \left[
	\begin{matrix}
	f(x_0) \\ f(x_1) \\ f(x_2) \\ \vdots \\ f(x_{m-1})
	\end{matrix}
	\right]
	\]
	που είναι σύστημα 2 αγνώστων αλλά \( m \) εξισώσεων, κάτι που μάλλον
	θα είναι αδύνατο.
	
	Επομένως αντί να λύσω την εξίσωση:
	\[
	Ax = b
	\]
	θα λύσω την:
	\[
	A\bar x \simeq b
	\]
	με σκοπό να ελαχιστοποιήσω την απόσταση (υπόλοιπο):
	\[
	A\bar x -b.
	\]
	
	Η παραπάνω σχέση όμως δεν εκφράζει έναν αριθμό, αλλά έναν πίνακα!
	Για να "ελαχιστοποιήσουμε" αυτόν τον πίνακα θα ορίσουμε κάποια
	μεγέθη.
	
	\subsection{Στοιχεία Αριθμητικής Γραμμικής Άλγεβρας}
	\subsubsection{Εσωτερικό γινόμενο}
	\textit{Εσωτερικό γινόμενο} σε χώρο \( V \) είναι μια \textbf{πράξη}
	για κάθε \( x,y \in V \) 
	(συμβολίζεται με \( (x,y) \) ή \( \langle x,y \rangle \))
	ορίζει έναν \textbf{αριθμό} που ικανοποιεί:
	\begin{enumroman}
		\item \( (x,x) \geq 0 \), ισότητα μόνο αν \( x=0 \)
		\item \( (x,y) = (y,x) \quad \forall x,y\in V \)
		\item \( (ax+by,z) = a(x,z)+b(y,z) \)
	\end{enumroman}
	
	Πιο συγκεκριμένα, για κάποιους χώρους ορίζουμε:
	\begin{align*}
	\mathbb R^n &: \ (x,y) = x^{\mathrm T}y \quad
	\text{για } x,y \in \mathbb R^n
	\\
	\mathbb R^{m\times n} &: (A,B) =
	\sum_{i=1}^{m}\sum_{j=1}^{n} a_{ij}b_{ij}
	\end{align*}
	
	Για συναρτήσεις, έχουμε \( (f,g) = \int f(x)g(x)\dif x \)
	
	Δύο διανύσματα \( x,y \in V \) λέγονται \textbf{ορθογώνια} όταν
	\( (x,y) = 0\), και γράφουμε \( x \perp y \).
	
	\subsubsection{Μήκος - Νόρμα (norm) διανύσματος}
	Ως νόρμα μπορούμε να επιλέξουμε οποιαδήποτε συνάρτηση που καλύπτει
	κάποια αξιώματα (για παράδειγμα, νόρμα ενός διανύσματος μπορεί να
	είναι το μέγιστο στοιχείο ενός διανύσματος).
	
	Ένας χώρος \( V \) είναι εφοδιασμένος με \textbf{μήκος} ή
	\textbf{νόρμα} αν για κάθε \( x \in V \) υπάρχει αριθμός
	\( \Vert x\Vert \) που ικανοποιεί:
	
	\begin{enumroman}
		\item \( \Vert x\Vert > 0 \) με ισότητα μόνο εάν \( x=0 \)
		\item \( \Vert ax\Vert = |a|\,\Vert x\Vert \) για κάθε σταθερά \( a \)
		\item \( \Vert x+y\Vert \leq \Vert x\Vert+\Vert y\Vert \ \forall x,y\in V  \)
	\end{enumroman}
	
	\paragraph{Είδη νορμών}
	Για ένα διάνυσμα \( x=(x_1,x_2,\dots,x_n)^{\mathrm T} \)
	\begin{align*}
		l_{\infty} &: \Vert x\Vert _{\infty} = \max_{1\leq i \leq n}|x_i|
		\qquad \text{ (άπειρη νόρμα)}
		\\
		l_p &: \Vert x\Vert _{p} =
		\left(\sum_{i=1}^n |x_i|^p\right)^{\frac{1}{p}}
		\quad \text{ ($p$-οστή νόρμα)}
		\\
		l_2 &: \Vert x\Vert _2 = \sqrt{x^{\mathrm T}x} =
		\sqrt{x_1^2 + \dots + x_n^2}
		\quad \text{ (ευκλείδια νόρμα)}
	\end{align*}
	
	\paragraph{Απόσταση}
	μεταξύ διανυσμάτων \( x,y \in \mathbb R  \) είναι ο αριθμός:
	\[
    \mathlarger{\Vert x-y\Vert }
	\]
	
	\paragraph{Θεωρήματα}
	\[
	\Vert x+y\Vert ^2 = \Vert x\Vert ^2 + \Vert y\Vert ^2
	\]
	
	Cauchy-Schwartz:
	\[
	\left\Vert(x,y)\right\Vert \leq \Vert x\Vert \,\Vert y\Vert 
	\]
	
	\subsubsection{Ορθογώνιοι Υποχώροι}
	\begin{defn}{Ορθογώνιοι Υποχώροι}{}
		Δύο υποχώροι \( X,Y \) του  \( \mathbb R^n \) λέγονται
		\textbf{ορθογώνιοι}, αν \( x^{\mathrm T}y = 0 \) \textbf{για
		κάθε} \( x \in X \) και \( y \in Y \).
	
	    Γράφουμε: \( \mathlarger{X \perp Y} \)
	\end{defn}
	\begin{defn}{Ορθογώνιο Συμπλήρωμα}{}
		Το σύνολο:
		\[
		\mathlarger{
			Y^\perp
			} = \left\lbrace 
			x \in \mathbb R^n : x^{\mathrm T}y = 0 \quad
			\forall y \in Y
			 \right\rbrace
		\]
		είναι το \textbf{ορθογώνιο συμπλήρωμα} του \( Y \).
		
		Για παράδειγμα, το ορθογώνιο συμπλήρωμα του πίνακα στην αίθουσα
		Α3 είναι ο κάθετος τοίχος.
	\end{defn}
	\begin{defn}{}{}
		Για \( A \in \mathbb R^{m \times n} \):
		\begin{itemize}
			\item \textbf{Μηδενοχώρος} του \( A \):
			\( \mathrm N(A) = \left\lbrace x\in\mathbb R^n: Ax=0
			 \right\rbrace \)
			\item \textbf{Χώρος των στηλών} του \( A \):
			\( \mathrm R(A)  = \left\lbrace
			b \in \mathbb R^m: b = Ax \text{ για }
			x \in \mathbb R^n \right\rbrace
			\)
			\item \textbf{Χώρος των γραμμών} του \( A \):
			\( \mathrm R(A^{\mathrm T}) = \left\lbrace 
			y \in \mathbb R^n: y = A^{\mathrm T}x \text{ για }
			x \in \mathbb R^m
			 \right\rbrace \)
		\end{itemize}
	\end{defn}
	\begin{theorem}{Θεμελιώδες Θεώρημα Γραμμικής Άλγεβρας}{}
		\begin{align*}
			\mathlarger{N(A)} &\mathlarger{= R(A^{\mathrm T})^\perp}
			\\
			\mathlarger{N(A^{\mathrm T})} &\mathlarger{= R(A)^\perp}
		\end{align*}
	\end{theorem}

    \subsection{Πίσω στο αρχικό πρόβλημα}
    Ορίζω:
    \begin{align*}
    \text{διαφορά } r(x) &= b - Ax \\
    \left\Vert r(x)\right\Vert &= \Vert b-Ax\Vert
    \end{align*}
    
    \todo{Graph 18}
    
    Το διάνυσμα \( r(x) \) ανήκει στο ορθογώνιο συμπλήρωμα του χώρου
    των στηλών, άρα στο μηδενοχώρο του \( A^{\mathrm T} \) (από το
    Θεμελιώδες Θεώρημα Γραμμικής Άλγεβρας), επομένως:
    
    \begin{align*}
    	A^{\mathrm T} r(x) &= 0 \\
    	A^{\mathrm T} (b-A\bar x) &= 0
    \end{align*}
    
    Άρα: \[
    \boxed{
    \mathlarger{\mathlarger{\mathlarger{
    		Α^{\mathrm T}Ax = A^{\mathrm T}b
    	}}}} \quad \xleftarrow{\hspace{5pt}} \text{ κανονικές εξισώσεις}
    \]
    
    \paragraph{Παράδειγμα}
    Να προσεγγιστεί η \( f(x) \) με πολυώνυμο πρώτου βαθμού, όπου:
    \[
    \begin{array}{r|ccc}
    x_i & 0 & 3 & 6 \\ \hline
    f(x_i) & 1 & 4 &  5
    \end{array}
    \]
    
    \subparagraph{Λύση}
    \[
    A = \left[ \begin{matrix}
    1 & 0 \\ 1 & 3 \\ 1 & 6
    \end{matrix} \right], \quad x = \left[ \begin{matrix}
    a_0 \\ a_1
    \end{matrix}\right], \quad b = \left[\begin{matrix}
    1 \\ 4 \\ 5
    \end{matrix} \right]
    \]
    
    \begin{align*}
    	A^{\mathrm T} A &= \left[\begin{matrix}
    	3 & 9 \\ 9 & 45
    	\end{matrix} \right] \\
    	A^{\mathrm T} b &= \left[ \begin{matrix}
    	10 \\ 42
    	\end{matrix} \right]
    \end{align*}
    
    Άρα πρέπει να λύσουμε το σύστημα:
    \[
    \left[\begin{matrix}
    3 & 9 \\ 9 & 45
    \end{matrix}\right] \left[\begin{matrix}
    a_0 \\ a_1
    \end{matrix}\right] = \left[\begin{matrix}
    10 \\ 42
    \end{matrix}\right]
    \]
    \( a_0 = \sfrac{4}{3},\ a_1 = \sfrac{2}{3}  \)
    
    Επομένως \( \displaystyle P_1(x) = \frac{4}{3} + \frac{2}{3}x \).
    
    \paragraph{Άσκηση}
    Να προσεγγιστούν τα παρακάτω σημεία με μία παραβολή, χρησιμοποιώντας
    τη μέθοδο ελαχίστων τετραγώνων:
    \[
    \begin{array}{r|ccccc}
    x_i & 0 & 1 & 2 & 3 & 4 \\ \hline
    f(x_i) & 1 & 2 & 2 & 4 & 5
    \end{array}
    \]

	\subparagraph{Λύση}
    \begin{align*}
    	A &= \left[\begin{matrix}
   	    1 & 0 & 0 \\ 1 & 1 & 1 \\ 1 & 2 & 4 \\ 1 & 3 & 9 \\ 1 & 4 & 1
    	\end{matrix}\right] \\
    	x &= \left[\begin{matrix}
    	a_0 \\ a_1 \\ a_2
    	\end{matrix}\right] \qquad P(x) = a_0+a_1x+a_1x^2 \\
    	b &= \left[
    	\begin{matrix}
    	1 \\ 2 \\ 2 \\ 4 \\ 5
    	\end{matrix}
    	\right]
    \end{align*}
    \[
    \left[
    A^{\mathrm T}
    \right] \left[ A \right]\left[
    \begin{matrix}
    a_0 \\ a_1 \\ a_2
    \end{matrix}
    \right] = \left[A^{\mathrm T}\right] \left[b\right]
    \]

    \section{Αριθμητική Ολοκλήρωση}
    Πολλές φορές δεν μπορούμε να υπολογίσουμε αναλυτικά ένα ολοκλήρωμα,
    ή αυτός ο υπολογισμός είναι υπολογιστικά δύσκολος, επομένως σε αυτό το
    κεφάλαιο θα αναπτύξουμε τεχνικές για αριθμητική ολοκλήρωση.
    
    \begin{gather*}
    	f(x) \quad [a,b] \\
    	\mathlarger{\int_a^b f(x) \dif x}
    \end{gather*}
    \todo{Graph 19}
    
    Ο πιο απλός τρόπος να υπολογίσουμε το ολοκλήρωμα μιας συνάρτησης
    είναι να πάρουμε τον εμβαδόν της συνάρτησης μεταξύ δύο σημείων
    (κατά Riemann).
    
    Σε αυτό το κεφάλαιο η γενική ιδέα είναι να παρεμβάλλουμε την \( f \)
    με ένα πολυώνυμο \( p \):
    \[
    f \rightarrow p
    \]
    και να υπολογίσουμε το ολοκλήρωμα του \( p \). Κάτι τέτοιο έχει
    νόημα επειδή το ολοκλήρωμα είναι μια γραμμική διαδικασία:
    \begin{align*}
    	I(f) &= \int_{a}^{b} f(x)\dif x\\
    	I\left(f(x)+g(x)\right) &= I\left(f(x)\right)+I\left(g(x)\right)
    	\\ I\left(af(x)\right) &= aI\left(f(x)\right)
    \end{align*}
    
    Επίσης θα υπολογίζουμε το σφάλμα (ή πιο συγκεκριμένα, το μέγιστο
    σφάλμα) της προσέγγισής μας:
    \begin{align*}
    	f(x) &= p(x) + \overbrace{e(x)}^{\mathclap{\text{σφάλμα}}}
    	\\
    	I\left(f(x)-p(x)\right) &= I\left(e(x)\right)
    \end{align*}
    
    \paragraph{}
    Για πολυώνυμα σε μορφή Lagrange έχουμε:
    \begin{align*}
    	P_n(x) &= \sum_{k=0}^{n} f(x_k) \cdot l_k(x) \\
    	I\left(p_n(x)\right) &= I\left(\sum_{k=0}^{n}f(x_k)l_k(x)\right)
    	= \sum_{k=0}^n f(x_k) \cdot I\left(l_k(x)\right) \\
    	I(f) &= \int_{a}^{b} f(x) \dif x \simeq
    	 \sum_{k=0}^n f(x_k) \cdot I\left(l_k(x)\right)
    \end{align*}
    
    Γενικά θα χρησιμοποιούμε πολυώνυμα το πολύ 2\textsuperscript{ου}
    βαθμού:
    \begin{itemize}
    	\item \( n=0,1,2,\dots \rightarrow \) \textbf{απλός κανόνας}
    	\item \textbf{Σύνθετος κανόνας:} Εφαρμόζουμε τον απλό κανόνα σε
    	πολλά διαστήματα της \( f \).
    \end{itemize}
    
    \subsection{Κανόνας Παραλληλογράμμου}
    \( \mathsmaller{(n = 0)} \)
    
    \todo{Graph 20}
    
    \paragraph{Απλός κανόνας παραλληλογράμμου/ορθογωνίου}
    Για να εφαρμόσω τον κανόνα του παραλληλογράμμου, θα παρεμβάλλω την
    \( f \) σε ένα σημείο.
    
    Ας υποθέσω ότι παρεμβάλλω την \( f \) στο σημείο \( \left(a,f(a)
    \right) \):
    \begin{align*}
    	p_0(x) &= f(a) \\
    	I\left(p_0(x)\right) &= \int_{a}^{b} f(a)\dif x \\
    	&= (b-a)f(a) = I_{\mathrm R \ \leftarrow \text{Rectangle}}
    \end{align*}
    \begin{align*}
    	\text{Σφάλμα } E_{\mathrm R} &= I\left(f(x)\right) -
    	I_{\mathrm R}
    	\\ &= \frac{(b-a)^2}{2} f'(z) \quad \text{(με βάση πράξεις του
    		βιβλίου, για κάποιο $z: a<z<b$)}
    \end{align*}
    
    Για να φράξω το σφάλμα (να βρω την μέγιστη δηλαδή τιμή του):
    \begin{align*}
    |E_\mathrm{R}| = \frac{(b-a)^2}{2} \left|f'(z)\right| &\leq
    \frac{(b-a)^2}{2} \max_{a<z<b} \left|f'(z)\right|
    \end{align*}
    
    \paragraph{Σύνθετος κανόνας παραλληλογράμμου}
    Ο σύνθετος κανόνας λειτουργεί σαν τον απλό, με τη διαφορά ότι
    διαμερίζει το διάστημα ολοκλήρωσης \( [a,b] \) σε \( N \) ίσα
    τμήματα:
    \begin{align*}
    \overset{\substack{\mathclap{\text{απόσταση}}\\\uparrow}}{h} =
    \frac{b-a}{N}
    \hspace{70pt}
    I_{\mathrm R, N} &= hf(x_0) + hf(x_1) + \dots + hf(x_{N-1}) =
    \\ &= h \sum_{i=0}^{N-1} f(x_i)
    \end{align*}

    Για το σφάλμα απλά αθροίζουμε το σφάλμα του κάθε ορθογωνίου:
    \begin{align*}
    E_{\mathrm R, N} &= \frac{h^2}{2} f'(z_1) + \frac{h^2}{2} f'(z_2)
    + \dots + \frac{h^2}{2} f'(z_N) =
    \\ &= \frac{h^2}{2} \sum_{i=1}^{N} f'(z_i)
    \\ &= \frac{h^2}{2} N f'(z) \quad \mathsmaller{\left(
    	\text{\small όπου } f'(z) = \frac{\sum f'(z_i)}{N},
    	\text{\small δηλ. ο μέσος όρος} \right)}
    \\ &= \frac{h}{2}(b-a)f'(z)\\
    \left|E_{\mathrm R,N}\right| &\leq \frac{h}{2} (b-a)
    \max_{a<z<b} \left|f'(z)\right|
    \end{align*}
    
    \paragraph{Παράδειγμα}
    Να υπολογιστεί το ολοκλήρωμα της \( \mathlarger{f(x)=x^3} \) στο
    διάστημα \( \mathlarger{[1,2]} \) για \( N=1 \text{ και } N=8 \).
    \subparagraph{Λύση για \( N=1 \)}
    \begin{align*}
    	I_R &= f(a)\big(b-a\big) = f(1)\big(2-1\big) = 1\cdot 1 = 1 \\
    	E_R &= \frac{(z-1)^2}{2} f'(z) \quad 1<z<2 \\
    	\left|E_R\right| &\leq \frac{1}{2}\max_{1<z<2} 3z^2 =
    	\frac{1}{2} \cdot 3 \cdot 2^2 = 6
    \end{align*}
    \subparagraph{Λύση για \( N=8 \)}
    \begin{align*}
    	I_{R,8} &= hf(x_0) + h_1f(x_1) + \dots + hf(x_7) =
    	\\ &= \frac{1}{8}f(1) + \frac{1}{8}f\left(\sfrac{9}{8} \right)
    	+ \dots + \frac{1}{8}\left(\sfrac{15}{8}\right) = 3.32
    	\\ \left|E_{R,8}\right| &\leq \frac{1}{8}\cdot\frac{1}{2}\cdot 1
    	\max_{1<z<2} \left|3z^2\right| = 0.75
    \end{align*}
    
    \subsection{Κανόνας Τραπεζίου}
    \( \mathsmaller{(n = 1)} \)
    
    Παρεμβάλλω την \( f \) σε δύο σημεία για να πάρω ένα πολυώνυμο
    1\textsuperscript{ου} βαθμού:
    \todo{Graph 21}
    
    \paragraph{Απλός κανόνας τραπεζίου}
    
    \[x_0 = a \hspace{50pt} x_1=b \]
    \begin{align*}
    	p_1(x) &= f(a)l_0(x) + f(b)l_1(x) \qquad
    	\text{($l_i$ πολυώνυμα Lagrange)} \\
    	&= \frac{(x-b)}{a-b} + f(b)\frac{x-a}{b-a} \\
    	I\left(p_1(x)\right) &= \int_{a}^{b} p_1(x)
    	\\ &= \dots = \frac{b-a}{2}\left(f(a)-f(b)\right) =
    	I_{\mathrm T \ \leftarrow \text{Trapezoid}}
    \end{align*}

    Για το σφάλμα έχουμε:
    \begin{align*}
    	E_{\mathrm T} &= I\left(f(x)\right) - I\left(p_1(x)\right)
    	= \dots \\ &=
    	-\frac{(b-a)^3}{12}f''(z) \qquad a<z<b \\
    	\left|E_{\mathrm T}\right| &\leq \frac{(b-a)^3}{12}
    	\max_{a<z<b} f''(z)
    \end{align*}
    
    \paragraph{Σύνθετος κανόνας τραπεζίου}
    Όπως και προηγουμένως, διαμερίζουμε το τραπέζιο σε \( N \) τμήματα,
    και εφαρμόζουμε τον απλό κανόνα σε καθ' ένα από αυτά:
    \begin{gather*}
    	h = \frac{b-a}{N} \\
    	a = x_0,x_1,\dots,x_N=b
    \end{gather*}
    
    \begin{align*}
    	I_{\mathrm T,N} &= \frac{h}{2} \left(f(x_0)+f(x_1)\right)
    	+ \frac{h}{2} \left(f(x_1)+f(x_2)\right) + \dots + \frac{h}{2}
    	\left(f(X_{N-1})+f(x_N)\right)
    	\\ &= \frac{h}{2} \left[
    	f(x_0)+2f(x_1)+2f(x_2)+\dots+2f(x_{N-1})+f(x_N)
    	\right]
    \end{align*}
    
    Αντίστοιχα με παραπάνω, για το σφάλμα θα αθροίσω τα επιμέρους
    σφάλματα:
    \begin{align*}
    	E_{\mathrm T,N} &= \sum_{i=1}^{N} -\frac{h^3}{12} f''(z_i)
    	\qquad x_{i-1} < z_i < x_i
    	\\ &= - \frac{h^3}{12} N f''(z) \qquad a<z<b
    	\\ &= -\frac{h^2}{12}(b-a)f''(z) \\
    	|E_{\mathrm T,N}| &\leq \frac{h^2}{12}(b-a)\max_{a<z<b}
    	\left|f''(z)\right|
    \end{align*}
    
    \paragraph{Παράδειγμα}
    Να υπολογιστεί με τον κανόνα τραπεζίου για \( N=4,\ 8 \) το
    ολοκλήρωμα:
    \[
    \mathlarger{\int_{1}^{2} x^3\dif x}
    \]
    \subparagraph{Λύση για \( N=4 \)}
    \begin{align*}
    	I_{T,4} &= \frac{h}{2} \left[
    	f(x_0)+2f(x_1)+2f(x_2)+2f(x_3)+f(x_4)
    	\right] \\ &= \frac{\sfrac{1}{4} }{2}\left[
    	f(1)+2f\left(\sfrac{5}{4}\right)+2f\left(\sfrac{6}{4}\right)
    	+2f\left(\sfrac{7}{4}\right)+f\left(\sfrac{8}{4}\right)
    	\right]
    	\\ &= ??? \\[3ex]
    	E_{T,4} &= -\frac{h^2}{12}(2-1)f''(z) \qquad 1<z<2 \\
    	&= - \frac{\sfrac{1}{16} }{12} \cdot 6z = -\frac{1}{32}z
    	\\ \left|E_{T,4}\right| &\leq \frac{1}{32}\cdot 2
    	=\frac{1}{16} = 0.0625
    \end{align*}\todo{Find result}
    
    \subparagraph{Λύση για \( N=8 \)}
    \begin{align*}
        I_{T,8} &= \frac{h}{2} \left[
        f(x_0)+2f(x_1)+2f(x_2)+\dots+2f(x_7)+f(x_8)
        \right] \\ &= \frac{\sfrac{1}{8} }{2}\left[
        f(1)+2f\left(\sfrac{9}{8}\right)+2f\left(\sfrac{10}{8}\right)
        +\dots+f\left(\sfrac{15}{8}\right)
        \right]
        \\ &= 3.7617 \\[3ex]
    	E_{T,8} &= -\frac{h^2}{12}(2-1)f''(z) \qquad 1<z<2 \\
    	&= -\frac{\sfrac{1}{8^2} }{12}\cdot 6z=-\frac{1}{128}z \\
    	|E_{T,8}| &\leq \frac{1}{128}z=\frac{1}{64}=0.01562
    \end{align*}
    
    \subsection{Κανόνας Simpson}
    \( \mathsmaller{(n = 2)} \)
    \todo{Graph 22}
    
    \paragraph{Απλός κανόνας Simpson}
    \begin{align*}
    	P_2(x) &= f(a)l_0(x) +f\left(\frac{a+b}{2}\right)l_1(x)+f(b)
    	l_2(x) \\ &= f(a) \frac{(x-x_1)(x-x_2)}{(x_0-x_1)(x_0-x_2)}
    	+ f\left(\frac{a+b}{2}\right)
    	\frac{(x-x_0)(x-x_2)}{(x_1-x_0)(x_1-x_2)}
    	+ f(b) \frac{(x-x_0)(x-x_1)}{(x_2-x_0)(x_2-x_1)}
    	\\[3ex]
    	I_{\mathrm S} &= \int_a^b P_2(x)\dif x
    	\\ &= \dots = \frac{b-a}{6} \left[
    	f(a)+4f\left(\frac{a+b}{2}\right)+f(b) \right] \\[3ex]
    	E_{\mathrm S} &= I\left(f(x)\right)-I\left(P_2(x)\right) = \dots
    	\\ &= -\frac{\left(\frac{b-a}{2}\right)^5}{90} f^{(4)}(z) \\
    	\left|E_{\mathrm S}\right| &\leq
    	\frac{\left(\frac{b-a}{2}\right)^5}{90}\max_{a<z<b}f^{(4)}(z)
    \end{align*}
\end{document}
