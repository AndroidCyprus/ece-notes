\documentclass[11pt,a4paper,notitlepage,fleqn,final]{article}

\usepackage{amsmath}
\usepackage{amsfonts}
\usepackage{amssymb}
\usepackage{libs/commath2}
\usepackage[table]{xcolor}
\usepackage[hidelinks,draft=false]{hyperref}
\usepackage[skins,theorems]{tcolorbox}
\usepackage{titlesec}
\usepackage{tikz}
\usepackage{libs/circuitikz} % use our own recent version to make sure some bugs are fixed
\usepackage{pgfplots}
\usepackage{mathtools}
\usepackage[makeroom]{cancel}
\usepackage{mathrsfs}
\usepackage{wrapfig}
%\usepackage{subcaption}
%\usepackage{floatrow}
\usepackage{esint}
\usepackage{enumitem}
%\usepackage{bm}
\usepackage{relsize}
\usepackage{xfrac}
\usepackage{comment}
%\usepackage{siunitx}
%\usepackage{MnSymbol}
\usepackage[obeyDraft,disable]{todonotes}
%\usepackage[linesnumbered,lined]{algorithm2e}


\pgfplotsset{compat=1.13}
\usetikzlibrary{arrows.meta}
\usetikzlibrary{patterns}
\usetikzlibrary{decorations.pathmorphing,patterns}
\usetikzlibrary{decorations.markings}
\usetikzlibrary{backgrounds}
\usetikzlibrary{shapes.misc}
\usetikzlibrary{shapes.multipart}
\usetikzlibrary{shadows.blur}
\usetikzlibrary{fadings}
\usetikzlibrary{intersections}
\usetikzlibrary{arrows.meta}
\usetikzlibrary{calc}
\usetikzlibrary{matrix}
\usetikzlibrary{positioning}
\usetikzlibrary{shapes}
\usetikzlibrary{shadings}

\tcbuselibrary{breakable}

\tikzset{cross/.style={cross out, draw,
        minimum size=2*(#1-\pgflinewidth),
        inner sep=0pt, outer sep=0pt}}
\tikzset{
    mark position/.style args={#1(#2)}{
        postaction={
            decorate,
            decoration={
            	post length=1mm, % ??? Magic to fix "Dimension
            	pre length=1mm, % ???  too large" errors.
                markings,
                mark=at position #1 with \coordinate (#2);
            }
        }
    }
}
\makeatletter
\tikzset{
  use path for main/.code={%
    \tikz@addmode{%
      \expandafter\pgfsyssoftpath@setcurrentpath\csname tikz@intersect@path@name@#1\endcsname
    }%
  },
  use path for actions/.code={%
    \expandafter\def\expandafter\tikz@preactions\expandafter{\tikz@preactions\expandafter\let\expandafter\tikz@actions@path\csname tikz@intersect@path@name@#1\endcsname}%
  },
  use path/.style={%
    use path for main=#1,
    use path for actions=#1,
  }
}
\makeatother

\pgfmathdeclarefunction{sinc}{1}{%
	\pgfmathparse{abs(#1)<0.01 ? int(1) : int(0)}%
	\ifnum\pgfmathresult>0 \pgfmathparse{1}\else\pgfmathparse{sin(#1 r)/#1}\fi%
}
\pgfmathdeclarefunction{gauss}{2}{%
	\pgfmathparse{1/(#2*sqrt(2*pi))*exp(-((x-#1)^2)/(2*#2^2))}%
}

\usepackage[left=2cm,right=2cm,top=2cm,bottom=2cm]{geometry}

%\usepackage[no-math]{fontspec}
%\usepackage{fontspec}
\usepackage{mathspec}
%\usepackage{newtxtext,newtxmath}
%\usepackage{unicode-math}
%\setmainfont{texgyretermes-regular.otf}
%\setsansfont{texgyreheros-regular.otf}
%\newfontfamily\greekfont[Script=Greek]{Linux Libertine O}
%\newfontfamily\greekfontsf[Script=Greek]{Linux Libertine O}
\usepackage{polyglossia}
%\newfontfamily\greekfont[Script=Greek]{texgyretermes-regular.otf}
\newfontfamily\greekfontsf[Script=Greek]{texgyreheros-regular.otf}
\newfontfamily\greekfonttt[Script=Greek]{Latin Modern Mono}
%\usepackage[greek]{babel}
\setdefaultlanguage{greek}
\setotherlanguage{english}

%\usepackage[utf8]{inputenc}
%\usepackage[greek]{babel}


%\usepackage{tkz-euclide} % loads  TikZ and tkz-base
%\usetkzobj{angles} % important you want to use angles

\newlist{enumparen}{enumerate}{1}
\setlist[enumparen]{label=(\arabic*)}
\newlist{enumpar}{enumerate}{1}
\setlist[enumpar]{label=\arabic*)}

\newlist{enumgreek}{enumerate}{1}
\setlist[enumgreek]{label=\alph*.}
\newlist{enumgreekparen}{enumerate}{1}
\setlist[enumgreekparen]{label=(\alph*)}
\newlist{enumgreekpar}{enumerate}{1}
\setlist[enumgreekpar]{label=\alph*)}


\newlist{enumroman}{enumerate}{1}
\setlist[enumroman]{label=(\roman*)}

\newlist{enumlatin}{enumerate}{1}
\setlist[enumlatin]{label=(\alph*)}

\newlist{invitemize}{itemize}{1}
\setlist[invitemize]{noitemsep,label=}

\usepackage{letltxmacro}

\LetLtxMacro\OriginalLongrightarrow\Longrightarrow
\LetLtxMacro\OriginalLongleftarrow\Longleftarrow

% Implement new macros
% --------------------
\usepackage{trimclip}
\DeclareRobustCommand\Longrightarrow{\NewRelbar\joinrel\Rightarrow}
\DeclareRobustCommand\Longleftarrow{\Leftarrow\joinrel\NewRelbar}

\makeatletter
\DeclareRobustCommand\NewRelbar{%
  \mathrel{%
    \mathpalette\@NewRelbar{}%
  }%
}
\newcommand*\@NewRelbar[2]{%
  % #1: math style
  % #2: unused
  \sbox0{$#1=$}%
  \sbox2{$#1\Rightarrow\m@th$}%
  \sbox4{$#1\Leftarrow\m@th$}%
  \clipbox{0pt 0pt \dimexpr(\wd2-.6\wd0) 0pt}{\copy2}%
  \kern-.2\wd0 %
  \clipbox{\dimexpr(\wd4-.6\wd0) 0pt 0pt 0pt}{\copy4}%
}
\makeatother


\makeatletter
\pgfdeclareradialshading[tikz@ball]{ball}{\pgfqpoint{0bp}{0bp}}{%
	color(0bp)=(tikz@ball!50!white);
	color(10bp)=(tikz@ball!50!white);
	color(15bp)=(tikz@ball!70!black);
	color(20bp)=(black!70);
	color(30bp)=(black!70)}%
\makeatother


\makeatletter
\let\anw@true\anw@false

%\newcommand{\attnboxed}[1]{\textcolor{red}{\fbox{\normalcolor\m@th$\displaystyle#1$}}}
\makeatother
\tcbset{highlight math style={enhanced,colframe=red,colback=white,%
        arc=0pt,boxrule=1pt,shrink tight,boxsep=1.5mm,extrude by=0.5mm}}
\newcommand{\attnboxed}[1]{\tcbhighmath[colback=red!5!white,drop fuzzy shadow,arc=0mm]{#1}}
\newcommand{\infoboxed}[1]{%
	\tcbhighmath[colframe=blue!50!white,colback=blue!5!white,arc=0mm]{#1}}
\titleformat{\section}{\bf\Large}{Κεφάλαιο \thesection}{1em}{}
\newtcolorbox{attnbox}[1]{colback=red!5!white,%
    colframe=red!75!black,fonttitle=\bfseries,title=#1}
\newtcbox{quickattnbox}[1]{colback=red!5!white,%
	colframe=red!75!black,fonttitle=\bfseries,title=#1}
\newtcolorbox{infobox}[1]{colback=blue!5!white,%
    colframe=blue!75!black,fonttitle=\bfseries,title=#1}

\AtBeginDocument{%
\let\arg\relax
\let\Re\relax
\let\Im\relax
\DeclareMathOperator{\arg}{Arg}
\DeclareMathOperator{\Re}{Re}
\DeclareMathOperator{\Im}{Im}
}
\DeclareMathOperator{\sinc}{sinc}
\DeclareMathOperator{\sgn}{sgn}
\DeclareMathOperator{\erf}{erf}
\DeclareMathOperator{\cov}{cov}

\newif\ifhidetikz
\hidetikzfalse
%\hidetikztrue   % <---- comment/uncomment that line

\ifhidetikz

\let\oldtikzpicture\tikzpicture
\let\oldendtikzpicture\endtikzpicture

\renewenvironment{tikzpicture}{
    \tiny
    \tt
    \color{blue}
    \newcommand{\draw}{\textit{draw}}
    \newcommand{\filldraw}{\textit{filldraw}}
    %\newcommand{\x}{\textit{x}}
    %\newcommand{\p}{\textit{x}}
    \newcommand{\x1}{\textit{x1}}
    \newcommand{\y1}{\textit{y1}}
    \newcommand{\p1}{\textit{p1}}
}{
}
\newenvironment{axis}{
    \newcommand{\addplot}{\textit{addplot}}
}{
}
\fi

% Global amount of samples
% Set to a higher value (e.g. 200) for nicer graphs
% Set to a low value (e.g. 10) for performance
\newcommand*{\gsamples}{70}

% Equals command as a workaround for CircuiTikZ bug
% not allowing the = sign in labels
\newcommand*{\equals}{=}

\newcommand{\nesearrow}{%
	\,%
	\smash{\raisebox{-1.1ex}
		{$%
			\stackrel{\displaystyle\nearrow}{\displaystyle\searrow}%
			$}}%
}
\newcommand{\degree}{^{\circ}} % not great
\newcommand\numberthis{\addtocounter{equation}{1}\tag{\theequation}} % add an equation number to a number-less math environment

\newtcbtheorem[number within=section]{theorem}{Θεώρημα}%
{colback=green!5,colframe=green!35!black,colbacktitle=green!35!black,fonttitle=\bfseries,enhanced,attach boxed title to top left={yshift=-2mm,xshift=-7mm},width=.9\textwidth,arc=.7mm}{th}
\newtcbtheorem[number within=section]{defn}{Ορισμός}%
{colback=blue!5,colframe=cyan!35!black,colbacktitle=blue!35!black,fonttitle=\bfseries,enhanced,attach boxed title to top left={yshift=-2mm,xshift=-2mm}}{def}
\newtcbtheorem[number within=section]{exercise}{Άσκηση}%
{colback=gray!3,colframe=gray!35!black,colbacktitle=gray!35!black,fonttitle=\bfseries,enhanced,attach boxed title to top left={yshift=-2mm,xshift=-2mm}}{exc}




\title{Αριθμητική Ανάλυση
	\\
	{
	\normalsize Σημειώσεις από τις παραδόσεις
	}}
\date{2017
	\\
	{
	\small Τελευταία ενημέρωση: \today
	}
	}
\author{
	Για τον κώδικα σε \LaTeX, ενημερώσεις και προτάσεις:
\\
 \url{https://github.com/kongr45gpen/ece-notes}}

\setmainfont{Linux Libertine O}
\setsansfont{Ubuntu}
%\newfontfamily\greekfont[Script=Greek]{Linux Libertine O}
%\newfontfamily\greekfontsf[Script=Greek]{Linux Libertine O}
\usepackage{polyglossia}
\newfontfamily\greekfont[Script=Greek,Scale=0.95]{GFS Artemisia}


\begin{document}
	\maketitle

	\tableofcontents

	\vspace{50pt}

	\textbf{Αριθμητική ανάλυση - Numerical Analysis}
	
	Μάθημα 4 ώρες την εβδομάδα - δεν υπάρχει διάκριση μεταξύ θεωρίας και ασκήσεων.
	
	Και τα δύο βιβλία προτείνονται, το μάθημα γίνεται περισσότερο με βάση το βιβλίο του κ.
	Πιτσούλη, του κ. Δούγαλη είναι περισσότερο μαθηματικό.
	
	Στις εξετάσεις δεν θα υπάρχει τυπολόγιο/βιβλίο, αλλά θα δίνονται τύποι που χρειάζονται
	στα θέματα. Απαραίτητο το κομπιουτεράκι.
	
	\section{Εισαγωγή}
	Η αριθμητική ανάλυση μάς δίνει \textit{προσεγγιστικές} λύσεις σε μοντέλα και μαθηματικά
	προβλήματα.
	
	Σε δύσκολα προβλήματα, ζητάμε:
	\begin{itemize}
		\item \textbf{Ακρίβεια} αποτελέσματος
		\item \textbf{Ταχύτητα} υπολογισμού
	\end{itemize}
	
	Θα δούμε τα εξής προβλήματα:
	\begin{itemize}
		\item \textbf{Επίλυση εξισώσεων}
		\paragraph{Παράδειγμα}
		Ένας πελάτης θέλει να καταθέσει \( P \) € για \( N \) χρόνια στην τράπεζα, και εγώ
		του λέω ότι θα του επιστρέψω \( A \) € από την κατάθεση. Ο πελάτης όμως ενδιαφέρεται
		για το ετήσιο επιτόκιο \( R \).
		
		Προκύπτει μια εξίσωση της μορφής:
		\begin{gather*}
		A = P + P \left( 1 + \frac{R}{12} \right) + \dots \\
		f(R) = \frac{P}{\sfrac{R}{12}} \left[
		    \left(1+\frac{R}{12}\right)^N -1
		\right] = 0, \quad R = ?
		\end{gather*}
		
		Θυμόμαστε ότι μπορούμε να χρησιμοποιήσουμε το θεώρημα Bolzano για να βρούμε ότι
		υπάρχει μια τουλάχιστον λύση μέσα σε ένα διάστημα, οπότε μπορούμε να "φανταστούμε"
		έναν αριθμό κοντά στη λύση, και να κλείνουμε συνεχώς ένα διάστημα γύρω από αυτήν (το
		διάστημα στις εξετάσεις θα δίνεται, π.χ. \textit{βρείτε μία λύση στο διάστημα
			\( [2.5,\ 3.5] \) με ακρίβεια \( 10^{-5} \)}), αν και αυτό δεν θα γίνεται στον πραγματικό κόσμο.
		\item \textbf{Παρεμβολή}
		
		Σε έναν σταθμό διοδίων μετράω πόσα αυτοκίνητα περνάν το κάθε λεπτό (π.χ. το
		11\textsuperscript{ο} λεπτό περνάν 4, το 12\textsuperscript{ο} περνάν 7, κλπ.)
		
		%TODO Papalamprou Graph 1
		
		Θέλω να βρω ένα πολυώνυμο που να συνδέει όλα τα σημεία μεταξύ τους (θα αποδείξουμε
		ότι τέτοιο πολυώνυμο πάντα υπάρχει), ή ένα πολυώνυμο (αρκετά χαμηλού βαθμού, ώστε
		να μην γίνονται πολλές πράξεις) με αρκετά καλή προσέγγιση.
		
		Με αυτόν τον τρόπο θα μπορώ να προσεγγίσω τιμές που δεν γνωρίζω, π.χ. αν πήγα για
		καφέ στο 13\textsuperscript{ο} λεπτό
		\item \textbf{Προσέγγιση}
		\item \textbf{Αριθμητική Γραμμική Άλγεβρα}
		
		Exact και προσεγγιστικές λύσεις συστημάτων πολλών αγνώστων.
		\item \textbf{Ολοκλήρωση}
		\item \textbf{Υπολογισμός ιδιοτιμών \& ιδιοδιανυσμάτων}
		\item \textbf{Παραγοντοποίηση πινάκων σε γινόμενο πινάκων}
		
		Βολεύει κυρίως για την επίλυση συστημάτων.
		\item \textbf{Επίλυση κανονικών διαφορικών εξισώσεων}
		\item \textbf{Βελτιστοποίηση}
		
		Για παράδειγμα, να πρέπει να ελαχιστοποιήσω μια συνάρτηση τη στιγμή που πρέπει να
		τηρούνται κάποιες συνθήκες.
	\end{itemize}
	
	\subsection{Ακρίβεια vs Ταχύτητα}
	
	Απόλυτο Σφάλμα: \( \displaystyle
	\left| X_t - X_c \right|
	 \) \quad (απόσταση της λύσης που βρήκα από την πραγματική)
	 
	\vspace{5pt}
	
	Σχετικό Σφάλμα: \( \displaystyle
	\frac{\left|X_t - X_c\right|}{\left|X_t\right|}
	 \)
	
	Επειδή δεν θα γνωρίζουμε την πραγματική λύση \( X_t \), θα βρίσκουμε το μέγιστο σφάλμα.
	
	Για παράδειγμα, σε υπολογιστές έχουμε σφάλματα στρογγύλευσης και αποκοπής:
	\[
	\begin{array}{ll}
		0.66666 & \\
		0.66 & \leftarrow \text{αποκοπή} \\
		0.67 & \leftarrow \text{στρογγύλευση}
	\end{array}
	\]
	
	\section{Επίλυση Εξισώσεων}
	\begin{center}
		\( \displaystyle \mathlarger{
		f(x): \qquad \text{βρείτε $\bar x$ έτσι ώστε \( f(\bar x) = 0 \).}}
		\)
	\end{center}
	
	Η επίλυση είναι εύκολη όταν η \( f \) είναι πολυώνυμο μέχρι 2\textsuperscript{ου} βαθμού,
	όχι όμως όταν είναι μεγαλύτερου, ή όταν έχει κι άλλους όρους (π.χ. εκθετικούς) μέσα.
	
	\begin{itemize}
		\item Δημιουργούμε μια ακολουθία \( x_1, \dots, x_n \) προσέγγισης της \( \bar x \).
		\item Σε κάθε βήμα κάνουμε έλεγχο σύγκλισης για να δούμε πόσο κοντά είμαστε.
	\end{itemize}
	
	\subsection{Μέθοδος Διχοτόμησης}
	Θα χρησιμοποιήσω θεώρημα Bolzano (\( \mathsmaller{f(a) \cdot f(b) < 0} \))
	
	%TODO Papalamprou Graph 2
	
	Ξεκινάω από μία αρχική προσέγγιση/διάστημα (η μέθοδος διχοτόμησης δεν δίνει λύση
	συγκεκριμένη, αλλά διάστημα - τη λύση την παίρνω σαν το μέσο του διαστήματος).
	
	Συνέχεια κόβω το διάστημα στη μέση, έτσι ώστε \( f(\text{των άκρων}) \) να είναι
	ετερόσημα, και παίρνω συνεχώς ένα μικρότερο διάστημα.
	
	Σταματάω όταν είναι επιτυχής ο έλεγχος σύγκλισης, π.χ
	\( \mathlarger{\left|c_{k+1}-c_k\right| < \epsilon \rightarrow 10^{-5}} \), δηλαδή
	το διάστημα στο οποίο βρίσκεται η ρίζα είναι αρκετά μικρό.
	
	Θα διαπιστώσουμε ότι, αν και αυτή η μέθοδος λειτουργεί πάντα, είναι αρκετά αργή.
	
	\paragraph{Παράδειγμα}
	Να βρεθεί ρίζα της εξίσωσης
	\( \mathlarger{
	\mathlarger{f(x) = x^3+x-1} }
	 \) στο διάστημα \( [0,1] \) με ακρίβεια \( 10^{-3} \).
	
	\begin{gather*}
		f(0) \cdot f(1) < 0 \\
		k=1,\ a_1=0,\ b_1=1,\ m=\frac{a_1+b_1}{2} = 0.5 \\
		\quad f(m) = -0.375 < 0 \\
		\quad f(0.5) \cdot f(1) < 0
	\end{gather*}
	Άρα η λύση βρίσκεται μεταξύ \( 0.5 \) και \( 1 \).
	
	Για το επόμενο βήμα:
	\begin{gather*}
		k = 2,\ m_2 = \frac{0.5+1}{2} = 0.75 \\
		\quad f(m_2) = 0.172 > 0 \\
		\quad f(m_1)f(m_2) < 0 \\[3ex]
		k=3,\ m_3 = \frac{0.5+0.75}{2} = 0.625 \\
		\quad f(m_3) = -0.131 \\
		\quad f(m_2)f(m_3) < 0 \\[3ex]
		\mathlarger{\mathlarger{\vdots}}
	\end{gather*}
	
	\subsubsection{Σύγκλιση}
	Όπου \( r \) και \( m_n \) η πραγματική και προσεγγιστική λύση αντίστοιχα:
	\begin{align*}
		\left|r - m_n\right| \leq \left(\frac{1}{2}\right)^n (b-a)
	\end{align*}
	επειδή στη \( n- \)οστή επανάληψη έχω κόψει το διάστημα στα 2 σε \( n \) φορές.
	
	Για σφάλμα \( 10^{-5} \), θέλουμε \( |r-m_n| = 10^{-5} \):
	\begin{align*}
		&10^{-5} \leq \left(\frac{1}{2}\right)^n (1-0) \implies
		\\ \implies & n \simeq 16.667 \implies \mathlarger{n = 17}
	\end{align*}
	
\end{document}
