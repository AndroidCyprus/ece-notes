\usepackage{amsmath}
\usepackage{amsfonts}
\usepackage{amssymb}
\usepackage{commath}
\usepackage{xcolor}
\usepackage{hyperref}
\usepackage[skins,theorems]{tcolorbox}
\usepackage{titlesec}
\usepackage{circuitikz}
\usepackage{pgfplots}
\usepackage{mathtools}
\usepackage[makeroom]{cancel}
\usepackage{mathrsfs}
\usepackage{wrapfig}
\usepackage{subcaption}
\usepackage{floatrow}
\usepackage{esint}
\usepackage{enumitem}
\usepackage{bm}
\usepackage{relsize}
\usepackage{xfrac}
\usepackage{comment}


\usetikzlibrary{arrows.meta}
\usetikzlibrary{patterns}
\usetikzlibrary{decorations.pathmorphing,patterns}
\usetikzlibrary{decorations.markings}
\usetikzlibrary{backgrounds}
\usetikzlibrary{shapes.misc}
\usetikzlibrary{shapes.multipart}

\tikzset{cross/.style={cross out, draw,
        minimum size=2*(#1-\pgflinewidth),
        inner sep=0pt, outer sep=0pt}}
\tikzset{
    mark position/.style args={#1(#2)}{
        postaction={
            decorate,
            decoration={
                markings,
                mark=at position #1 with \coordinate (#2);
            }
        }
    }
}

\usepackage[left=2cm,right=2cm,top=2cm,bottom=2cm]{geometry}

\usepackage[no-math]{fontspec}
\setmainfont{Times New Roman}
\setsansfont{Arial}
%\newfontfamily\greekfont[Script=Greek]{Linux Libertine O}
%\newfontfamily\greekfontsf[Script=Greek]{Linux Libertine O}
\usepackage{polyglossia}
\newfontfamily\greekfont[Script=Greek]{Times New Roman}
\newfontfamily\greekfontsf[Script=Greek]{Arial}
\newfontfamily\greekfonttt[Script=Greek]{Latin Modern Mono}
%\usepackage[greek]{babel}
\setdefaultlanguage{greek}
\setotherlanguage{english}
\newcommand{\textlatin}[1]{#1}
%\newcommand{\mathlarger}{}

%\usepackage[utf8]{inputenc}
%\usepackage[greek]{babel}


\usetikzlibrary{arrows.meta}
\usetikzlibrary{calc}
%\usepackage{tkz-euclide} % loads  TikZ and tkz-base
%\usetkzobj{angles} % important you want to use angles

\newlist{enumparen}{enumerate}{1}
\setlist[enumparen]{label=(\arabic*)}
\newlist{enumpar}{enumerate}{1}
\setlist[enumpar]{label=\arabic*)}

\newlist{enumgreek}{enumerate}{1}
\setlist[enumgreek]{label=\alph*.}
\newlist{enumgreekparen}{enumerate}{1}
\setlist[enumgreekparen]{label=(\alph*)}
\newlist{enumgreekpar}{enumerate}{1}
\setlist[enumgreekpar]{label=\alph*)}


\newlist{enumroman}{enumerate}{1}
\setlist[enumroman]{label=(\@alph*)}

\newlist{enumlatin}{enumerate}{1}
\setlist[enumlatin]{label=(\alph*)}

\newlist{invitemize}{itemize}{1}
\setlist[invitemize]{noitemsep,label=}



\makeatletter
\let\anw@true\anw@false

%\newcommand{\attnboxed}[1]{\textcolor{red}{\fbox{\normalcolor\m@th$\displaystyle#1$}}}
\makeatother
\tcbset{highlight math style={enhanced,colframe=red,colback=white,%
        arc=0pt,boxrule=1pt,shrink tight,boxsep=1.5mm,extrude by=0.5mm}}
\newcommand{\attnboxed}[1]{\tcbhighmath[colback=red!5!white,drop fuzzy shadow,arc=0mm]{#1}}
\titleformat{\section}{\bf\Large}{Κεφάλαιο \thesection}{1em}{}
\newtcolorbox{attnbox}[1]{colback=red!5!white,%
    colframe=red!75!black,fonttitle=\bfseries,title=#1}
\newtcolorbox{infobox}[1]{colback=blue!5!white,%
    colframe=blue!75!black,fonttitle=\bfseries,title=#1}

\renewcommand{\arg}{\mathrm{Arg}\, }
\renewcommand{\Re}{\mathrm{Re}}
\renewcommand{\Im}{\mathrm{Im}}

\newif\ifhidetikz
\hidetikzfalse
%\hidetikztrue   % <---- comment/uncomment that line

\ifhidetikz

\let\oldtikzpicture\tikzpicture
\let\oldendtikzpicture\endtikzpicture

\renewenvironment{tikzpicture}{
    \tiny
    \tt
    \color{blue}
    \newcommand{\draw}{\textit{draw}}
    \newcommand{\filldraw}{\textit{filldraw}}
    \newcommand{\x}{\textit{x}}
}{
}
\newenvironment{axis}{
    \newcommand{\addplot}{\textit{addplot}}
}{
}
\fi

\newtcbtheorem[number within=section]{theorem}{Θ.}%
{colback=green!5,colframe=green!35!black,colbacktitle=green!35!black,fonttitle=\bfseries,enhanced,attach boxed title to top left={yshift=-8mm,xshift=-7mm},width=.6\textwidth,arc=.7mm}{th}
\newtcbtheorem[number within=section]{defn}{Ορισμός}%
{colback=blue!5,colframe=cyan!35!black,colbacktitle=blue!35!black,fonttitle=\bfseries,enhanced,attach boxed title to top left={yshift=-2mm,xshift=-2mm}}{def}
\newtcbtheorem[number within=section]{exercise}{Άσκηση}%
{colback=gray!3,colframe=gray!35!black,colbacktitle=gray!35!black,fonttitle=\bfseries,enhanced,attach boxed title to top left={yshift=-2mm,xshift=-2mm}}{exc}


