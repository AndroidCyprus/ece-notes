\documentclass[11pt,a4paper,notitlepage,fleqn,draft]{article}

\usepackage{amsmath}
\usepackage{amsfonts}
\usepackage{amssymb}
\usepackage{libs/commath2}
\usepackage[table]{xcolor}
\usepackage[hidelinks,draft=false]{hyperref}
\usepackage[skins,theorems]{tcolorbox}
\usepackage{titlesec}
\usepackage{tikz}
\usepackage{libs/circuitikz} % use our own recent version to make sure some bugs are fixed
\usepackage{pgfplots}
\usepackage{mathtools}
\usepackage[makeroom]{cancel}
\usepackage{mathrsfs}
\usepackage{wrapfig}
%\usepackage{subcaption}
%\usepackage{floatrow}
\usepackage{esint}
\usepackage{enumitem}
%\usepackage{bm}
\usepackage{relsize}
\usepackage{xfrac}
\usepackage{comment}
%\usepackage{siunitx}
%\usepackage{MnSymbol}
\usepackage[obeyDraft,disable]{todonotes}
%\usepackage[linesnumbered,lined]{algorithm2e}


\pgfplotsset{compat=1.13}
\usetikzlibrary{arrows.meta}
\usetikzlibrary{patterns}
\usetikzlibrary{decorations.pathmorphing,patterns}
\usetikzlibrary{decorations.markings}
\usetikzlibrary{backgrounds}
\usetikzlibrary{shapes.misc}
\usetikzlibrary{shapes.multipart}
\usetikzlibrary{shadows.blur}
\usetikzlibrary{fadings}
\usetikzlibrary{intersections}
\usetikzlibrary{arrows.meta}
\usetikzlibrary{calc}
\usetikzlibrary{matrix}
\usetikzlibrary{positioning}
\usetikzlibrary{shapes}
\usetikzlibrary{shadings}

\tcbuselibrary{breakable}

\tikzset{cross/.style={cross out, draw,
        minimum size=2*(#1-\pgflinewidth),
        inner sep=0pt, outer sep=0pt}}
\tikzset{
    mark position/.style args={#1(#2)}{
        postaction={
            decorate,
            decoration={
            	post length=1mm, % ??? Magic to fix "Dimension
            	pre length=1mm, % ???  too large" errors.
                markings,
                mark=at position #1 with \coordinate (#2);
            }
        }
    }
}
\makeatletter
\tikzset{
  use path for main/.code={%
    \tikz@addmode{%
      \expandafter\pgfsyssoftpath@setcurrentpath\csname tikz@intersect@path@name@#1\endcsname
    }%
  },
  use path for actions/.code={%
    \expandafter\def\expandafter\tikz@preactions\expandafter{\tikz@preactions\expandafter\let\expandafter\tikz@actions@path\csname tikz@intersect@path@name@#1\endcsname}%
  },
  use path/.style={%
    use path for main=#1,
    use path for actions=#1,
  }
}
\makeatother

\pgfmathdeclarefunction{sinc}{1}{%
	\pgfmathparse{abs(#1)<0.01 ? int(1) : int(0)}%
	\ifnum\pgfmathresult>0 \pgfmathparse{1}\else\pgfmathparse{sin(#1 r)/#1}\fi%
}
\pgfmathdeclarefunction{gauss}{2}{%
	\pgfmathparse{1/(#2*sqrt(2*pi))*exp(-((x-#1)^2)/(2*#2^2))}%
}

\usepackage[left=2cm,right=2cm,top=2cm,bottom=2cm]{geometry}

%\usepackage[no-math]{fontspec}
%\usepackage{fontspec}
\usepackage{mathspec}
%\usepackage{newtxtext,newtxmath}
%\usepackage{unicode-math}
%\setmainfont{texgyretermes-regular.otf}
%\setsansfont{texgyreheros-regular.otf}
%\newfontfamily\greekfont[Script=Greek]{Linux Libertine O}
%\newfontfamily\greekfontsf[Script=Greek]{Linux Libertine O}
\usepackage{polyglossia}
%\newfontfamily\greekfont[Script=Greek]{texgyretermes-regular.otf}
\newfontfamily\greekfontsf[Script=Greek]{texgyreheros-regular.otf}
\newfontfamily\greekfonttt[Script=Greek]{Latin Modern Mono}
%\usepackage[greek]{babel}
\setdefaultlanguage{greek}
\setotherlanguage{english}

%\usepackage[utf8]{inputenc}
%\usepackage[greek]{babel}


%\usepackage{tkz-euclide} % loads  TikZ and tkz-base
%\usetkzobj{angles} % important you want to use angles

\newlist{enumparen}{enumerate}{1}
\setlist[enumparen]{label=(\arabic*)}
\newlist{enumpar}{enumerate}{1}
\setlist[enumpar]{label=\arabic*)}

\newlist{enumgreek}{enumerate}{1}
\setlist[enumgreek]{label=\alph*.}
\newlist{enumgreekparen}{enumerate}{1}
\setlist[enumgreekparen]{label=(\alph*)}
\newlist{enumgreekpar}{enumerate}{1}
\setlist[enumgreekpar]{label=\alph*)}


\newlist{enumroman}{enumerate}{1}
\setlist[enumroman]{label=(\roman*)}

\newlist{enumlatin}{enumerate}{1}
\setlist[enumlatin]{label=(\alph*)}

\newlist{invitemize}{itemize}{1}
\setlist[invitemize]{noitemsep,label=}

\usepackage{letltxmacro}

\LetLtxMacro\OriginalLongrightarrow\Longrightarrow
\LetLtxMacro\OriginalLongleftarrow\Longleftarrow

% Implement new macros
% --------------------
\usepackage{trimclip}
\DeclareRobustCommand\Longrightarrow{\NewRelbar\joinrel\Rightarrow}
\DeclareRobustCommand\Longleftarrow{\Leftarrow\joinrel\NewRelbar}

\makeatletter
\DeclareRobustCommand\NewRelbar{%
  \mathrel{%
    \mathpalette\@NewRelbar{}%
  }%
}
\newcommand*\@NewRelbar[2]{%
  % #1: math style
  % #2: unused
  \sbox0{$#1=$}%
  \sbox2{$#1\Rightarrow\m@th$}%
  \sbox4{$#1\Leftarrow\m@th$}%
  \clipbox{0pt 0pt \dimexpr(\wd2-.6\wd0) 0pt}{\copy2}%
  \kern-.2\wd0 %
  \clipbox{\dimexpr(\wd4-.6\wd0) 0pt 0pt 0pt}{\copy4}%
}
\makeatother


\makeatletter
\pgfdeclareradialshading[tikz@ball]{ball}{\pgfqpoint{0bp}{0bp}}{%
	color(0bp)=(tikz@ball!50!white);
	color(10bp)=(tikz@ball!50!white);
	color(15bp)=(tikz@ball!70!black);
	color(20bp)=(black!70);
	color(30bp)=(black!70)}%
\makeatother


\makeatletter
\let\anw@true\anw@false

%\newcommand{\attnboxed}[1]{\textcolor{red}{\fbox{\normalcolor\m@th$\displaystyle#1$}}}
\makeatother
\tcbset{highlight math style={enhanced,colframe=red,colback=white,%
        arc=0pt,boxrule=1pt,shrink tight,boxsep=1.5mm,extrude by=0.5mm}}
\newcommand{\attnboxed}[1]{\tcbhighmath[colback=red!5!white,drop fuzzy shadow,arc=0mm]{#1}}
\newcommand{\infoboxed}[1]{%
	\tcbhighmath[colframe=blue!50!white,colback=blue!5!white,arc=0mm]{#1}}
\titleformat{\section}{\bf\Large}{Κεφάλαιο \thesection}{1em}{}
\newtcolorbox{attnbox}[1]{colback=red!5!white,%
    colframe=red!75!black,fonttitle=\bfseries,title=#1}
\newtcbox{quickattnbox}[1]{colback=red!5!white,%
	colframe=red!75!black,fonttitle=\bfseries,title=#1}
\newtcolorbox{infobox}[1]{colback=blue!5!white,%
    colframe=blue!75!black,fonttitle=\bfseries,title=#1}

\AtBeginDocument{%
\let\arg\relax
\let\Re\relax
\let\Im\relax
\DeclareMathOperator{\arg}{Arg}
\DeclareMathOperator{\Re}{Re}
\DeclareMathOperator{\Im}{Im}
}
\DeclareMathOperator{\sinc}{sinc}
\DeclareMathOperator{\sgn}{sgn}
\DeclareMathOperator{\erf}{erf}
\DeclareMathOperator{\cov}{cov}

\newif\ifhidetikz
\hidetikzfalse
%\hidetikztrue   % <---- comment/uncomment that line

\ifhidetikz

\let\oldtikzpicture\tikzpicture
\let\oldendtikzpicture\endtikzpicture

\renewenvironment{tikzpicture}{
    \tiny
    \tt
    \color{blue}
    \newcommand{\draw}{\textit{draw}}
    \newcommand{\filldraw}{\textit{filldraw}}
    %\newcommand{\x}{\textit{x}}
    %\newcommand{\p}{\textit{x}}
    \newcommand{\x1}{\textit{x1}}
    \newcommand{\y1}{\textit{y1}}
    \newcommand{\p1}{\textit{p1}}
}{
}
\newenvironment{axis}{
    \newcommand{\addplot}{\textit{addplot}}
}{
}
\fi

% Global amount of samples
% Set to a higher value (e.g. 200) for nicer graphs
% Set to a low value (e.g. 10) for performance
\newcommand*{\gsamples}{70}

% Equals command as a workaround for CircuiTikZ bug
% not allowing the = sign in labels
\newcommand*{\equals}{=}

\newcommand{\nesearrow}{%
	\,%
	\smash{\raisebox{-1.1ex}
		{$%
			\stackrel{\displaystyle\nearrow}{\displaystyle\searrow}%
			$}}%
}
\newcommand{\degree}{^{\circ}} % not great
\newcommand\numberthis{\addtocounter{equation}{1}\tag{\theequation}} % add an equation number to a number-less math environment

\newtcbtheorem[number within=section]{theorem}{Θεώρημα}%
{colback=green!5,colframe=green!35!black,colbacktitle=green!35!black,fonttitle=\bfseries,enhanced,attach boxed title to top left={yshift=-2mm,xshift=-7mm},width=.9\textwidth,arc=.7mm}{th}
\newtcbtheorem[number within=section]{defn}{Ορισμός}%
{colback=blue!5,colframe=cyan!35!black,colbacktitle=blue!35!black,fonttitle=\bfseries,enhanced,attach boxed title to top left={yshift=-2mm,xshift=-2mm}}{def}
\newtcbtheorem[number within=section]{exercise}{Άσκηση}%
{colback=gray!3,colframe=gray!35!black,colbacktitle=gray!35!black,fonttitle=\bfseries,enhanced,attach boxed title to top left={yshift=-2mm,xshift=-2mm}}{exc}




\title{ΣΑΕ 1
	\\
	{ 
		\normalsize Συστήματα Αυτομάτου Ελέγχου I
		\\
		\normalsize Σημειώσεις από τις παραδόσεις
	}}
\date{Οκτώβριος-Ιανουάριος 2017
	\\
	{ 
		\small Τελευταία ενημέρωση: \today
	}
}
\author{
	Για τον κώδικα σε \LaTeX, ενημερώσεις και προτάσεις:
	\\
	\url{https://github.com/kongr45gpen/ece-notes}}

\setallmainfonts(Digits,Latin,Greek){Asana Math}
\setmainfont{Noto Serif}
\setsansfont{Ubuntu}
\usepackage{polyglossia}
\newfontfamily\greekfont[Script=Greek,Scale=1.00]{Liberation Serif}

\hypersetup{pdftitle = {ΣΑΕ 1}}


\begin{document}
\maketitle

\hrule
\vspace{50pt}
	
Συστήματα Αυτομάτου Ελέγχου
	
Υπεύθυνη καθηγήτρια: Ζωή Δουλγέρη, ασκήσεις από τον Παπαγεωργίου Δημήτρη - δεν υπάρχει διαχωρισμός ασκήσεων και θεωρίας.

\section{Συστήματα}
Γνωρίζουμε από τα προηγούμενα μαθήματα τι είναι το σύστημα. Σκοπός του μαθήματος είναι να σχεδιάσουμε έναν "ελεγκτή" ώστε ένα σύστημα να έχει μια επιθυμητή έξοδο.

Για παράδειγμα, αν έχουμε έναν κινητήρα που επιθυμούμε να ελέγξουμε, μπορούμε να τον παραστήσουμε με το παρακάτω σχήμα:

\begin{tikzpicture}
\draw (0,0) rectangle (1,1) node[midway] {$\Sigma$};
\draw[->] (1,0.5) -- ++(1,0) node[right] {$y$};
\draw (1.5,0.5) -- (1.5,-1) -- (0,-1);
\draw (0,-0.75) rectangle (-0.5,-1.25) node[midway] {$H$};
\draw[->] (-0.5,-1) -- (-2,-1) -- (-2,0.25);
\draw (-2,0.5) circle (0.25);
\draw[->] (-3,0.5) -- (-2.25,0.5) node[above,midway] {$r$};
\draw (-1.35,0.25) rectangle (-0.65,0.85) node[midway] {$E$};
\draw (-1.75,0.5) -- (-1.35,0.5) ;
\draw (-0.65,0.5) -- (0,0.5) node[above left] {$u$};
\end{tikzpicture}

όπου:
\begin{itemize}
	\item \( \Sigma \) είναι ο κινητήρας
	\item \( u \) είναι η τάση εισόδου (που ρυθμίζουμε εμείς)
	\item \( y \) είναι η έξοδος του συστήματος, εδώ η ταχύτητα του κινητήρα
	\item Η ροπή του φορτίου εκφράζει την είσοδο της διαταραχής
	\item \( H \) είναι ένας μετρητής που μπορούμε να έχουμε για να ελέγχουμε την ταχύτητα
	του κινητήρα
	\item \( E \) είναι ο ελεγκτής που θέλουμε να υλοποιήσουμε, ώστε να ρυθμίζει την
	τάση \( u \) εισόδου του κινητήρα για να πετύχουμε την επιθυμητή ταχύτητα.
\end{itemize}

Έχουμε και μία \textbf{είσοδο αναφοράς} που καθορίζει την επιθυμητή έξοδο του συστήματος.

Στα πλαίσια των ΣΑΕ βρίσκουμε το μαθηματικό μοντέλο του συστήματος, καθώς και το μαθηματικό
μοντέλο του ελεγκτή, και τα υλοποιούμε με φυσικό τρόπο (για παράδειγμα μέσω κυκλωματικών
στοιχείων, μικροελεγκτών, arduino κ.ά).

Παραδείγματα συστημάτων αυτομάτου ελέγχου είναι: Τα αμορτισέρ του αυτοκινήτου, οι κινητήρες
των CD drives, οι κινητήρες των γραμμών παραγωγής (ώστε για παράδειγμα να είμαστε σίγουροι
ότι τα υλικά περνάν από έναν κλίβανο ακριβώς για 30 λεπτά, διατηρώντας σταθερή την
ταχύτητα μεταφοράς τους), κ.ά.

\subsection{Μοντελοποίηση συστημάτων}
\paragraph{}
Για το σύστημα ενός σώματος στο οποίο ασκείται δύναμη, έχουμε πολύ απλά:
\[
	F = m\ddot x
\]

Για μια δύναμη ελατηρίου, ισχύει \( F = \kappa \cdot \delta x \), και για μια δύναμη
απόσβεσης/ιξώδους: \( F = d\dot x \)

\paragraph{Ανάρτηση αυτοκινήτου}
Θεωρούμε ότι η ανάρτηση ενός αυτοκινήτου αποτελείται από ένα ελατήριο και έναν αποσβεστήρα:

\begin{tikzpicture}
\draw (0,0) rectangle (2,1) node[midway] {$m$};
\draw (0.5,0) -- ++(0,-0.4)
-- ++(-30:0.2)
-- ++(-150:0.4)
-- ++(-30:0.4)
-- ++(-150:0.4)
-- ++(-30:0.4)
-- ++(-150:0.4)
-- ++(-30:0.2)
-- (0.5,-2);
;
\draw (1.5,0) -- (1.5,-1);
\draw (1.25,-1) -- (1.75,-1);
\draw (1.1,-0.9) -- (1.1,-1.2) -- (1.9,-1.2) -- (1.9,-0.9);
\draw (1.5,-1.2) -- (1.5,-2);

\draw (-0.5,-2) -- (2.5,-2);
\draw[gray] (2.5,1.2) -- (2.5,-2.7);

\draw (1,-2) -- (1,-2.5);
\filldraw (1,-2.5) circle(1pt) node[below left] {$\rho$};

\draw[dashed] (0.5,0) -- (3,0) node[right] {$x_0$};
\draw[dashed] (0.5,-2) -- (3,-2) node[right] {$x_i$};
\end{tikzpicture}

Και όπως πριν προκύπτει η σχέση:
\[
m\ddot x_0 + b(\dot x_0 - \dot x_i) + \kappa (x_0-x_i) = 0
\]
η οποία μπορεί να μετασχηματιστεί κατά Laplace:
\begin{align*}
	m\ddot x_0 + b\dot x_0 + \kappa x_0 &= b\dot x_i + \kappa \dot x_i \\
	ms^2X_0(s) + bsX_0(s) + \kappa X_0(s) &= X_1(s)bs + \kappa X_1(s) \\
	\Aboxed{\frac{X_0(s)}{X_1(s)} &= \frac{bs+\kappa}{ms^2+bs+\kappa} }
\end{align*}



Αυτή είναι μία απλή μέθοδος μοντελοποίησης συστημάτων, αλλά η μοντελοποίηση δεν είναι
αντικείμενο αυτού του μαθήματος.

\subsection{Ορισμοί}

\begin{defn}{}{}
	\begin{itemize}
	\item
	\textbf{Συνάρτηση μεταφοράς:} \(
	\displaystyle G(s) = \frac{Y(s)\quad\text{\small (έξοδος)}}{U(s) \quad
		\text{\small (είσοδος)} }
	= \frac{N(s)\quad \text{\small (αριθμητής)}}{D(s)\quad \text{\small (παρονομαστής)} }
	\)
	\item \textbf{Χαρακτηριστικό πολυώνυμο:} \( D(s) \)
    \end{itemize}
\end{defn}
Θυμόμαστε ότι στα φυσικά συστήματα δεν γίνεται να έχουμε βαθμό του αριθμητή μεγαλύτερο από
το βαθμό του παρονομαστή.
\begin{defn}{Μορφές έκφρασης συνάρτησης μεταφοράς}{}
	\begin{align*}
		H(s) &= \frac{K(s+z_1)\cdots(s+z_m)}{(s+p_1)\cdots(s+p_n)} \\
		H(s) &= \frac{G(1+s\tau_{n+1})\cdots(1+s\tau_{n+m})}{(1+s\tau_1)\cdots(1+s\tau_n)}
		\quad \text{όπου } G=\frac{kz_1\cdots z_m}{p_1\cdots p_m}
	\end{align*}
\end{defn}
\begin{defn}{}{}
	\begin{itemize}
    \item \textbf{Πόλοι} ονομάζονται οι τιμές \( p \) για τις οποίες ισχύει:
    \[
    \mathlarger{\lim_{s\to p}  \left\lvert H(s) \right\rvert} = \infty
    \]
    \item \textbf{Μηδενικά} ονομάζονται οι τιμές \( z \) για τις οποίες ισχύει:
    \[
    \mathlarger{\lim_{s\to z}  \left\lvert H(s) \right\rvert} = 0
    \]
    \end{itemize}
\end{defn}
\begin{theorem}{Σύνδεση εν σειρά}{}
	Όταν συνδέουμε δύο απομονωμένα συστήματα εν σειρά, για τις συναρτήσεις μεταφοράς τους
	ισχύει:
	\[
	G(s) = G_1(s)G_2(s)
	\]
	
	\begin{center}
	\begin{tikzpicture}[scale=1.2]
	\draw[->] (0,0) -- (1,0) node[above,midway] {$X_1(s)$};
	\draw (1,-0.3) rectangle (2,0.3) node[midway] {$G_1(s)$};
	\draw (2,0) -- (3,0) node[above,midway] {$X_2(s)$};
	\draw (3,-0.3) rectangle (4,0.3) node[midway] {$G_2(s)$};
	\draw[->] (4,0) -- (5,0) node[above,midway] {$X_3(s)$};
	\end{tikzpicture}
    \end{center}
\end{theorem}
\paragraph{Παράδειγμα} \hspace{0pt}


\begin{circuitikz}[american,scale=1.4]
	\ctikzset{bipoles/thickness=3}
	\draw (0,2) to [V=$u(s)$] (0,0);
	\draw[color=green!50!black] (0,2)
	to[R=$R_1$] (2,2)
	to[C=$C_1$] (2,0)
	-- (0,0);
	
	\draw[color=green!50!cyan!50!black] (2,2)
	to[R=$R_2$] (4,2)
	to[C=$C_2$] (4,0)
	-- (2,0);
	
	\draw (4,2) to[short,-*] (5,2);
	\draw (4,0) to[short,-*] (5,0);
	\draw (5,2) to[open, v^=$y$] (5,0);
\end{circuitikz}

Για το παραπάνω κύκλωμα, αν και έχουμε δύο συστήματα ενωμένα σε σειρά, δεν μπορούμε να
εφαρμόσουμε το θεώρημα στο παραπάνω κύκλωμα, αφού τα επιμέρους κυκλώματα δεν είναι απομονωμένα
και παρουσιάζουν σύνθετες αντιστάσεις εισόδου και εξόδου. Πράγματι, αν επιλύσουμε το
κύκλωμα:
\begin{align*}
	G_1(s)G_2(s) &= \frac{1}{(R_1C_1s+1)}\frac{1}{(R_2C_2s+1)} \\
	\frac{y(s)}{u(s)} &= \frac{1}{R_1C_1R_2C_2s^2+(R_1C_1+R_2C_2+\underline{R_1C_2})s+1}
\end{align*}
Παρατηρούμε τον όρο \( R_1C_2s \) που δεν υπάρχει στον απλό πολλαπλασιασμό των δύο
συστημάτων.

\paragraph{Άσκηση}
Ποιές από τις παρακάτω συναρτήσεις μεταφοράς έχουν \textbf{ρυθμούς} (ρίζες του παρονομαστή)
που δεν είναι πόλοι;
\begin{enumroman}
	\item \( \displaystyle \frac{s+8}{(s+3)(s+10)} \)
	\item \( \displaystyle \frac{s+1}{(s+1)^2(s+2)} \)
	\item \( \displaystyle \frac{s+9}{(s+2)^2+9} \)
	\item \( \displaystyle \frac{s+1}{(s+1)(s+2)} \)
\end{enumroman}
\subparagraph{Απάντηση}
\begin{enumroman}
	\item Έχει μηδενικό στο \( -8 \) και πόλλους στα \( -3 \) και \( -10 \).
	\item Έχει μόνο έναν πόλο στο \( -1 \) και στο \( -2 \).
	\item Έχει μηδενικό στο \( -9 \) και πόλους στα \( -2+j3 \) και \( -2-j3 \).
	\item Έχει μόνο πόλο στο \( -2 \).
\end{enumroman}

Η λύση αυτή μπορεί να προκύψει από τους ορισμούς του πόλου και του μηδενικού.

\subsection{Σύστημα κλειστού βρόγχου}

\begin{tikzpicture}[scale=1.3]
\draw[->] (-0.25,0) -- (0.5,0) node[above,midway] {$r(s)$};
\draw (0.75,0) circle (0.25);
\draw (1,0) -- ++(0.5,0) node[above,midway,green!50!black,scale=0.8] {$w(s)$};
\draw (1.5,-0.5) rectangle (3,0.5) node[midway] {$G(s)$};
\draw (3,0) -- (3.75,0);
\draw (4,0) circle (0.25);
\draw[->] (4.25,0) -- ++(1,0) node[above right] {$y$};

\draw[<-] (4,0.25)  node[left] {$+$} --++ (0,0.5)
node[above,rectangle,align=center,scale=0.8]
{είσοδος\\διαταραχής\\$d(s)$};

\draw (4.5,0) -- ++(0,-1.5) -- (3,-1.5);
\draw (1.5,-2) rectangle (3,-1) node[midway] {$H(s)$};
\draw[->] (1.5,-1.5) -- (0.75,-1.5) node[above,midway,green!50!black,scale=0.8] {$f(s)$} 
-- ++(0,1.25) node[right,xshift=1mm,yshift=-1mm] {$-$};
\end{tikzpicture}

Ορίζουμε:
\begin{alignat*}{2}
	\text{συνάρτηση μεταφοράς κλειστού βρόγχου: } && T(s) &= \frac{y(s)}{r(s)} \\
	\text{συνάρτηση μεταφοράς εισόδου διαταραχής: } && T_d(s) &= \frac{y(s)}{d(s)}
\end{alignat*}

Για να υπολογίσουμε την έξοδο του συστήματος, αν δεν λάβουμε υπ' όψιν
την είσοδο διαταραχής:
\begin{align*}
	y(s) &= G(s)w(s)
	\\ &= G(s)\left( r(s)-f(s) \right) \\
	y(s) &= G(s)\left[ r(s)-H(s)y(s) \right] \\
	y(s)\left[1+G(s)H(s)\right] &= G(s)r(s) \\
	y(s) &= \frac{G(s)r(s)}{1+G(s)H(s)} \\
	T(s) &= \frac{G(s)}{1+G(s)H(s)}
\end{align*}

Αν συμπεριλάβουμε και την είσοδο διαταραχής, το ζητούμενο είναι η είσοδος αυτή να μην
επηρεάζει καθόλου (ή όσο το δυνατόν λιγότερο) την έξοδο. Έχουμε:
\begin{align*}
    y(s) &= d(s) + G(s)w(s)
    \\ &= d(s) - G(s)f(s) \\ &= d(s) - G(s)H(s)y(s) \implies \\
    y(s)\left[ 1+G(s)H(s) \right] &= d(s) \implies \\
	T_d(s) &= \frac{y(s)}{d(s)} = \frac{1}{1+G(s)H(s)}
\end{align*}

Συνοπτικά:
\begin{theorem}{Συναρτήσεις μεταφοράς σε σύστημα κλειστού βρόγχου}{}
	Για ένα σύστημα κλειστού βρόγχου με είσοδο \( r(s) \), είσοδο διαταραχής
	\( d(s) \), συνάρτηση \( G(s) \) στην ευθεία διαδρομή και \( H(s) \) στη
	διαδρομή ανάδρασης, οι συναρτήσεις μεταφοράς είναι:
	\begin{align*}
		T(s) &= \left. \frac{y(s)}{r(s)} \right\rvert_{d(s) = 0}
		= \frac{G(s)}{1+G(s)H(s)} \\
		T_{d}(s) &= \left. \frac{y(s)}{d(s)} \right\rvert_{r(s) = 0}
		=
		\frac{1}{1+G(s)H(s)}
	\end{align*}
	και η συνολική έξοδος του συστήματος είναι:
	\[
	y(s) = T(s)r(s) + T_d(s)d(s)
	\]
\end{theorem}

Παρατηρούμε ότι το χαρακτηριστικό πολυώνυμο είναι το ίδιο στις δύο συναρτήσεις μεταφοράς.

\paragraph{Παράδειγμα} \hspace{0pt}

\begin{tikzpicture}
\draw[->] (-0.25,0) -- (0.5,0) node[above,midway] {$r(s)$};
\draw (0.75,0) circle (0.25);
\draw (1,0) -- ++(0.5,0);
\draw (1.5,-0.5) rectangle (3,0.5) node[midway] {$\displaystyle\frac{\kappa}{s+a}$};
\draw (3,0) -- (4,0);
\draw[->] (4,0) -- ++(1,0) node[above right] {$y$};

\draw (4,0) -- ++(0,-1.5) -- (3,-1.5);
\draw[dashed] (3,-1.5) -- (1.5,-1.5);
\draw (1.5,-1.5) -- (0.75,-1.5) -- ++(0,1.25) node[below right] {$-$};
\end{tikzpicture}

Θα υπολογίσουμε την έξοδο του συστήματος \underline{χωρίς ανάδραση} και \underline{με ανάδραση} σε βηματική είσοδο \( r(s) \rightarrow \mathrm u(t) \).
\subparagraph{Χωρίς ανάδραση} \hspace{0pt}

Το σύστημα χωρίς ανάδραση είναι το παραπάνω χωρίς τον κάτω βρόχο:
\begin{tikzpicture}[baseline,scale=0.7]
\draw[->] (-0.25,0) -- (0.5,0) node[above,midway] {$r(s)$};
\draw (0.75,0) circle (0.25);
\draw (1,0) -- ++(0.5,0);
\draw (1.5,-0.5) rectangle (3,0.5) node[midway] {$\frac{\kappa}{s+a}$};
\draw (3,0) -- (4,0);
\draw[->] (4,0) -- ++(1,0) node[above right] {$y$};
\end{tikzpicture}

Και ισχύει:
\begin{align*}
	y(s) &= r(s)\frac{\kappa}{s+a} \\
	y(s) &= \kappa\frac{1}{s}\frac{1}{s+a} \\
	y(t) &= \kappa\left( 1-e^{\sfrac{-t}{\tau}} \right)
\end{align*}
(όπου η σταθερά χρόνου \( \tau = \frac{1}{a} \))

Για \( t\to \infty \) το αποτέλεσμα είναι \( y(t) = \kappa \).

\subparagraph{Με ανάδραση}
\begin{align*}
	y(s) &= \frac{G(s)r(s)}{1+G(s)} \implies \dots \implies
	\tau' = \frac{1}{a+\kappa}
\end{align*}
Παρατηρούμε πως το σύστημα αυτό φτάνει πολύ πιο γρήγορα στην τελική του τιμή. Αυτό φαίνεται
αν συγκρίνουμε τις σταθερές χρόνου μεταξύ τους, σκεπτόμενοι ότι λειτουργούν ως συντελεστές
στην εκθετική συνάρτηση:

\begin{tikzpicture}
\draw (-2,0) -- (2,0);
\draw (0,-2) -- (0,2);

\draw (-1.5,0) node[cross=4pt,thick,blue] {} node[below,yshift=-1mm] {$-a-\kappa$};
\draw (-0.5,0) node[cross=4pt,thick,yellow!50!brown!50!red] {} node[above,yshift=1mm] {$-a$};

\begin{scope}[xshift=4cm]
\draw (-1,0) -- (3,0);
\draw (0,-2) -- (0,2);

\draw [very thick, color=blue, domain=0:3,variable=\t, samples=\gsamples, smooth]
plot (\t,{ 1-exp(-\t) }) node[below] {$-a$};
\draw [very thick, color=yellow!50!brown!50!red, domain=0:3,variable=\t, samples=\gsamples, smooth]
plot (\t,{ 1-exp(-5*\t) }) node[above,xshift=-2cm] {$-a-k$};
\end{scope}
\end{tikzpicture}

\subsection{Ισοδύναμα λειτουργικά διαγράμματα}
Για τη διευκόλυνσή της εύρεσης της συνάρτησης μεταφοράς, μπορούμε αντί να βρούμε την έξοδο
αλγεβρικά χρησιμοποιώντας ενδιάμεσες συναρτήσεις, να εφαρμόσουμε κανόνες όπως τους
παρακάτω:
\begin{infobox}{}
	\begin{itemize}
		\item
		\begin{tikzpicture}[scale=1.3,baseline]
		\begin{scope}[yshift=2mm]
		\draw[->] (0,0) -- (0.75,0);
		\draw (0.75,-0.7/2) rectangle ++(1,0.7) node[midway] {$H$};
		\draw[->] (1.75,0) -- (2.5,0) node[right] {$y_1$};
		\draw[->] (2,0) -- (2,-0.5) -- (2.5,-0.5) node[right] {$y_2$};
		\end{scope}
		
		\draw[->,very thick,gray!20!black] (4,0) to[bend left=10] ++(1.5,0);
		
		\begin{scope}[xshift=6.5cm]
		\draw (0,0) -- (0.5,0);
		\draw (0.5,0) -- (0.5,0.3) -- (1,0.3);
		\draw (0.5,0) -- (0.5,-0.3) -- (1,-0.3);
		\draw (1,-0.5) rectangle (1.7,-0.1) node[midway] {$H$};
		\draw[->] (1.7,-0.3) -- ++(0.5,0) node[right] {$y_2$};
		\draw (1,0.5) rectangle (1.7,0.1) node[midway] {$H$};
		\draw[->] (1.7,0.3) -- ++(0.5,0) node[right] {$y_1$};
		\end{scope}
		\end{tikzpicture}
		\item
		\begin{tikzpicture}[scale=1.3,baseline]
		\begin{scope}[]
		\draw[->] (0,0) -- (0.75,0) node[above,midway] {$u$};
		\draw (0.75,0) -- (0.75,0.5) -- (1.25,0.5);
		\draw (1.25,0.5-0.7/2) rectangle ++(1,0.7) node[midway] {$H$};
		\draw[->] (2.25, 0.5) -- (2.75,0.5) node[right] {$y$};
		\draw[->] (0.75,0) -- (0.75,-0.5) -- (2.75, -0.5) node[right]{$y_1$};
		\end{scope}
		
		\draw[->,very thick,gray!20!black] (4,0) to[bend left=10] ++(1.5,0);
		
		\begin{scope}[xshift=6.5cm,yshift=0.3cm]
		\draw[->] (0,0) -- (0.75,0) node[above,midway] {$u$};
		\draw (0.75,-0.5/2) rectangle ++(0.75,0.5) node[midway] {$H$};
		\draw[->] (1.5,0) -- (2.25,0) node[right] {$y$};
		\draw[->] (1.75,0) -- (1.75,-0.5) -- (2.25,-0.5);
		\draw (2.25,-0.5-0.5/2) rectangle ++(0.75,0.5) node[midway] {$\frac{1}{H}$};
		\draw[->] (3,-0.5) -- ++(0.5,0) node[right] {$y_1$};
		\end{scope}
		\end{tikzpicture}
		\item
		\begin{tikzpicture}[scale=1.3,baseline]
		\begin{scope}[]
		\draw[->] (0,0.5) -- ++(0.75,0);
		\draw (0.75,0.5-0.5/2) rectangle ++(0.75,0.5) node[midway] {$H_1$};
		\draw[->] (1.5,0.5) -- (2.25,0.5) -- (2.25,0.2) node[right,yshift=1mm,xshift=1mm] {$+$};
		\draw[->] (0,-0.5) -- ++(0.75,0);
		\draw (0.75,-0.5-0.5/2) rectangle ++(0.75,0.5) node[midway] {$H_2$};
		\draw[->] (1.5,-0.5) -- (2.25,-0.5) -- (2.25,-0.2);
		\draw (2.25,0) circle (2mm);
		\draw[->] (2.45,0) -- ++(0.75,0) node[above,pos=.9] {$y$};
		\end{scope}
		
		\draw[->,very thick,gray!20!black] (4,0) to[bend left=10] ++(1.5,0);
		
		\begin{scope}[xshift=6.5cm]
		\draw[] (0,0.5) -- ++(0.6,0);
		\draw (0.6,0.5-0.5/2) rectangle ++(1,0.5) node[midway] {$\sfrac{H_1}{H_2}$};
		\draw[->] (1.6,0.5) -- (2.25,0.5) -- (2.25,0.2) node[right,yshift=1mm,xshift=1mm] {$+$};
		\draw[->] (0,-0.5) -- (2.25,-0.5) -- (2.25,-0.2);
		\draw (2.25,0) circle (2mm);
		\draw (3.05,-0.4/2) rectangle ++(0.8,0.4) node[midway] {$H_2$};
		\draw[] (2.45,0) -- ++(0.6,0);
		\draw[->] (3.85,0) -- ++(0.6,0) node[right] {$y$};
		\end{scope}
		\end{tikzpicture}
	\end{itemize}
\end{infobox}

\paragraph{Παράδειγμα} \hspace{0pt}

\begin{tikzpicture}
\def\h{0.6}
\def\l{1}
\def\ll{0.75}
\draw[->] (0,0) node[green!70!black,scale=.8,opacity=.8,left] {$r(s)$} -- (1.5-0.2,0);
\begin{scope}[xshift=1.5cm]
\draw (0,0) circle (2mm);
\draw (0.2,0) -- ++(0.75,0);
\draw (0.2+0.75, -\h/2) rectangle ++(\l, \h) node[midway] {$H_1$};
\draw[->] (\l+0.2+0.75,0) -- (2.8,0) node[green!70!black,scale=.8,opacity=.8,above,midway] {$w(s)$};
\draw (3,0) circle (2mm);
\end{scope}
\begin{scope}[xshift=4.5cm]
\draw (0.2,0) -- ++(0.75,0);
\draw (0.2+0.75, -\h/2) rectangle ++(\l, \h) node[midway] {$H_2$};
\draw[] (\l+0.2+0.75,0) -- (3.5,0) node[right] {$y(s)$};
\end{scope}

\draw (0.5,0) -- ++(0,1.5) -- (4.5,1.5) -- (4.5,1);
\draw (4.5-\ll/2,1) rectangle (4.5+\ll/2,0.6) node[midway] {$\kappa$};
\draw[->] (4.5,0.6) -- (4.5,0.2) node[right,xshift=1mm,yshift=1mm] {$+$};

\draw[->] (4.5+2.5,0) -- ++(0,-1.5) -- (1.5,-1.5) -- (1.5,-0.2) node[right,xshift=1mm,yshift=-1mm] {$-$};
\end{tikzpicture}

Χρησιμοποιώντας τους παραπάνω κανόνες, ή την προηγούμενη μέθοδο, μπορούμε να βρούμε:
\[
T(s) = \frac{\kappa H_2(s)+H_2(s)H_1(s)}{1+H_1(s)H_2(s)}
\]

Ενδεικτικά, με βοηθητικές συναρτήσεις, οι πράξεις γίνονται ως εξής:
\begin{align*}
	y(s) &= H_2(s)\cdot\left( \kappa r(s) + w(s) \right)
	\\ &= H_2 \cdot \left[ \kappa r + H_1 \left( r - y \right) \right]
	\\ &= \kappa H_2 r + H_2H_1r - H_1H_2y \implies
	\\ y \cdot (1+H_1H_2) &= \kappa H_2 r + H_2H_1 r \implies
	\\ y &= \frac{\kappa H_2 r + H_2H_1 r}{1+H_1H_2} \implies
	\\ T(s) &= \frac{\kappa H_2(s) + H_2(s)H_1(s) }{1+H_1(s)H_2(s)}
\end{align*}


\section{Προδιαγραφές}
Ορίζουμε κάποιες προδιαγραφές που επιθυμούμε να πληροί η έξοδος του συστήματος, όπως η
ακρίβεια θέσης, η ταχύτητα της απόκρισης, η ευστάθεια κλπ. Για να μετρήσουμε ποσοτικά αυτά
τα κριτήρια, ορίζουμε νέα μεγέθη και χρησιμοποιούμε διάφορες συναρτήσεις ως
"εισόδους αναφοράς", όπως την κρουστική \( \delta(t) \) (για μελέτη ευστάθειας),
τη βηματική \( \mathrm u(t) \), την ράμπα, την ημιτονοειδή (για μελέτη
απόκρισης συχνότητας και ταχύτητας) κλπ.

\subsection{Ακρίβεια}
Το ζητούμενο της ακρίβειας είναι η τελική έξοδος να είναι κοντά στην επιθυμητή είσοδο.

Για να υπολογίσουμε την τελική έξοδο, δεν χρειάζεται να υπολογίσουμε τον αντίστροφο Μ/Σ
Laplace της συνάρτησης για να πάμε στο πεδίο του χρόνου, αλλά αρκεί να χρησιμοποιηθεί το θεώρημα της τελικής τιμής:
\[
f(\infty) = \lim_{s\to 0} sF(s)
 \]

Για παράδειγμα, για βηματική είσοδο (\( \mathrm u(t) \rightarrow \frac{1}{s}\))
σε ένα σύστημα (ss = steady state):
\begin{align*}
	y_{\mathrm{ss}} &= \lim_{s\to 0} sF(s) = \lim_{s\to 0} sH(s)\frac{1}{s} =
	\lim_{s\to 0} H(s)
\end{align*}

\begin{comment}
\begin{tikzpicture}
\def\h{0.6}
\def\l{1.2}
\def\ll{0.75}
\draw[->] (0,0) node[left] {$r(s)$} -- (1-0.2,0);
\begin{scope}[xshift=1cm]
\draw (0,0) circle (2mm);
\draw (0.2,0) -- ++(0.75,0);
\draw (0.2+0.75, -\h/2) rectangle ++(\l, \h) node[midway] {$G_1(s)$};
\draw (\l+0.2+0.75,0) -- (3,0);
\draw  (3, -\h/2) rectangle ++(\l, \h) node[midway] {$G_2(s)$};
\draw (3+\l,0) -- (5-0.2,0);
\draw[->] (5,1) -- ++(0,-1+0.2) node[midway,right] {$d(s)$};
\draw (5,0) circle (2mm);
\draw[->] (5+0.2,0) -- ++(1,0) node[right] {$y$};
\end{scope}

\draw[->] (4.5+2.2,0) -- ++(0,-1.5) -- (4,-1.5);
\draw (4,-1.5-\h/2) rectangle ++(-\l,\h) node[midway] {$H(s)$};
\draw[->] (4-\l,-1.5) -- (1,-1.5) -- (1,-0.2);
\end{tikzpicture}
\end{comment}

Για να μελετήσουμε την ακρίβεια, ορίζουμε το σφάλμα:
\begin{defn}{Σφάλμα}{}
	\[
	e(s) = r(s) - y(s)
	\]
\end{defn}

Χρησιμοποιούμε διάφορες εισόδους για να βρούμε διάφορα είδη σφαλμάτων του συστήματος:

\begin{alignat*}{5}
\begin{tikzpicture}[scale=0.4,baseline]
\draw (-0.2,0) -- (2,0);
\draw (0,-0.2) -- (0,2);
\draw [very thick, color=blue] (0,1.8) -- (2,1.8);
\end{tikzpicture}\quad
y(t) &= \mathrm u(t) &&\quad y(s) = \frac{1}{s} &\qquad e_{\mathrm ssp} & \quad{\text{σφάλμα θέσης}} \\
\begin{tikzpicture}[scale=0.4,baseline]
\draw (-0.2,0) -- (2,0);
\draw (0,-0.2) -- (0,2);
\draw [very thick, color=blue] (0,0) -- (2,2);
\end{tikzpicture}\quad
y(t) &= t &&\quad y(s) = \frac{1}{s^2} &\qquad e_{\mathrm ssv} & \quad{\text{σφάλμα ταχύτητας}} \\
\begin{tikzpicture}[scale=0.4,baseline]
\draw (-0.2,0) -- (2,0);
\draw (0,-0.2) -- (0,2);
\draw [very thick, color=blue, domain=0:2,variable=\t, samples=\gsamples, smooth]
plot (\t,\t^2/2);
\end{tikzpicture}\quad
y(t) &= \sfrac{t^2}{2} &&\quad y(s) = \frac{1}{s^3} &\qquad e_{\mathrm ssa} & \quad{\text{σφάλμα επιτάχυνσης}}
\end{alignat*}

Για παράδειγμα, για το σύστημα κλειστού βρόγχου, θυμόμαστε ότι:

\begin{tikzpicture}[scale=1]
\draw[->] (-0.25,0) -- (0.5,0) node[above,midway] {$r(s)$};
\draw (0.75,0) circle (0.25);
\draw (1,0) -- ++(0.5,0);
\draw (1.5,-0.5) rectangle (3,0.5) node[midway] {$H(s)$};
\draw (3,0) -- (3.75,0);
\draw[->] (3.25,0) -- ++(1,0) node[above right] {$y$};

\draw (3.6,0) -- ++(0,-1.5) -- (3,-1.5) -- (1.5,-1.5);
\draw[dashed] (1.5,-2) rectangle (3,-1) node[midway,opacity=.5] {$G(s)$};
\draw[->] (1.5,-1.5) -- (0.75,-1.5) -- ++(0,1.25) node[right,xshift=1mm,yshift=-1mm] {$-$};
\end{tikzpicture}

Αν θεωρήσουμε ότι η \( G(s) \) είναι \( 1 \), τότε:
\begin{align*}
	\frac{y(s)}{r(s)} &= \frac{H(s)}{1+H(s)} \\
	\text{όπου } H(s) &=
	\frac{G(τ_{n+1}s+1)\cdots(τ_{n+m}s+1)}{s^N(1+sτ_1)\cdots(1+sτ_n)} \implies
	\\
	e(s) &= r(s)-y(s) = \frac{1}{1+H(s)}r(s) \\
	e_{\mathrm{ss}} &= \lim_{s\to 0}se(s)
\end{align*}
(δηλαδή \( N \) είναι η τάξη του τυχόν πόλου στο 0).

Εφαρμόζοντας τις διάφορες συναρτήσεις ως εισόδους, σύμφωνα με τα παραπάνω, έχουμε:
\begin{alignat*}{3}
	e_{\mathrm{ssp}} &= \frac{1}{\displaystyle 1+\underbrace{\lim_{s\to 0}H(s)}_{K_\mathrm{pos}}}
	= \frac{1}{1+{\color{green!50!black}K_\mathrm{pos}}}
	&&= \begin{cases}
	\frac{1}{1+{\color{green!50!black}G}} \qquad &\text{για } N = 0 \\
	\frac{1}{1+{\color{green!50!black}\infty}} = 0 \qquad &\text{για } N \geq 1
	\end{cases}
	\\
	e_{\mathrm{ssv}} &= \frac{1}{\displaystyle \underbrace{\lim_{s\to 0} sH(s)}_{K_v} }
	= \frac{1}{\color{cyan!50!black}K_v} &&=
	\begin{cases}
	\frac{1}{\color{cyan!50!black}0} = \infty \qquad &\text{για } N = 0
	\\
	\frac{1}{\color{cyan!50!black}G}\qquad &\text{για } N=1
	\\
	\frac{1}{\color{cyan!50!black}\infty} = 0\qquad &\text{για } N\geq 2
	\end{cases}
	\\
	e_{\mathrm{ssa}} &= \frac{1}{\displaystyle \underbrace{\lim_{s\to 0} s^2H(s)}_{K_a}}
	= \frac{1}{\color{orange!50!black}K_a}
	&&= \begin{cases}
	\infty \qquad &\text{για } N \leq 1 \\
	\frac{1}{\color{orange!50!black} G} \qquad &\text{για } N = 2 \\
	0 \qquad &\text{για } N > 2
	\end{cases}
\end{alignat*}

Δε συζητάμε για ακρίβειες πέραν της επιτάχυνσης, επειδή σπάνια τα συστήματα έχουν πάνω από
2 ολοκληρωτές.

\begin{defn}{Ολοκληρωτής}{}
	Ένας \textbf{πόλος στο 0} λειτουργεί σαν \textbf{ολοκληρωτής}, επειδή
	έχει την ιδιότητα να ολοκληρώνει το σήμα εισόδου.
\end{defn}
\begin{defn}{Τύπος συστήματος}{}
	Ο \textbf{τύπος του συστήματος} είναι ο αριθμός των ολοκληρωτών που έχει.
\end{defn}

\paragraph{Παράδειγμα}
Έστω το σύστημα:

\begin{tikzpicture}[scale=1]
\draw[->] (-0.25,0) -- (0.5,0) node[above,midway] {$r(s)$};
\draw (0.75,0) circle (0.25);
\draw (1,0) -- ++(0.5,0);
\draw (1.5,-0.5) rectangle (3,0.5) node[midway] {$H(s)$};
\draw (3,0) -- (3.75,0);
\draw (4,0) circle (0.25);
\draw[->] (4.25,0) -- ++(1,0) node[above right] {$y$};
\draw ({(3+1.5)/2},0.5) node[above] {$\frac{1}{s(s+a)}$};

\draw[<-] (4,0.25)  node[right,xshift=1mm,yshift=1mm] {$+$} --++ (0,0.5)
node[above,rectangle,align=center,scale=0.8]
{$d(s)=\frac{1}{s}$};

\draw[->] (4.5,0) -- ++(0,-1.5) -- (0.75,-1.5) -- ++(0,1.25)
node[right,xshift=1mm,yshift=-1mm] {$-$};
\end{tikzpicture}

Ποιά θα είναι η έξοδος του συστήματος στη μόνιμη κατάσταση αν προσθέσουμε την είσοδο
διαταραχής;

\subparagraph{}

Γνωρίζουμε για το σύστημα κλειστού βρόχου ότι:
\begin{align*}
	\frac{y(s)}{d(s)} &= \frac{1}{1+H(s)} \\
	\lim_{s\to 0}sy(s) &= \lim_{s\to 0} s\frac{d(s)}{1+H(s)}
	= \lim_{s\to 0} s\frac{\frac{1}{s}}{1+H(s)} = \lim_{s\to 0}\frac{1}{1+H(s)} = 0
%	\frac{y(s)}{r(s)} &= \frac{H(s)}{1+H(s)}\\
%	y(s) &= \frac{H(s)}{1+H(s)}r(s) + \frac{1}{1+H(s)}d(s)
\end{align*}
Δηλαδή \( y_{\mathrm{ss}} = 0 \), άρα το σύστημα έχει πάλι τέλεια ακρίβεια, και σφάλμα θέσης
0.

\paragraph{Παράδειγμα} \hspace{0pt}

Το παρακάτω διάγραμμα αντιστοιχεί σε έναν κινητήρα:

\begin{tikzpicture}[scale=1]
\draw[->] (-0.25,0) -- (0.5,0) node[above,midway] {$r(s)$};
\draw (0.75,0) circle (0.25);
\draw (1,0) -- ++(0.5,0);
\draw (1.5,-0.5) rectangle (3,0.5) node[midway] {$κ$};
\draw (3,0) -- (3.75,0);
\draw (4,0) circle (0.25);
\draw (5,-0.5) rectangle (6.5,0.5) node[midway] {$\frac{1}{s(Js+b)}$};
\draw (4.25,0) -- (5,0);
\draw[->] (6.5,0) -- ++(1,0) node[above right] {$y$};

\draw[<-] (4,0.25)  node[right,xshift=1mm,yshift=1mm] {$+$} --++ (0,0.5)
node[above] {$d$};

\draw[->] (7,0) -- ++(0,-1.5) -- (0.75,-1.5) -- ++(0,1.25)
node[right,xshift=1mm,yshift=-1mm] {$-$};
\end{tikzpicture}

Το μετασχηματίζουμε στο ισοδύναμό του, ώστε να εφαρμόσουμε τους τύπους κλειστού βρόγχου:

\begin{tikzpicture}[scale=1]
\draw[->] (-0.25,0) -- (0.5,0) node[above,midway] {$r(s)$};
\draw (0.75,0) circle (0.25);
\draw (1,0) -- ++(0.5,0);
\draw (1.5,-0.5) rectangle (3,0.5) node[midway] {$κ$};
\draw (3,0) -- (3.75,0);
\draw (4,0) circle (0.25);
\draw (6,-0.5) rectangle (7.5,0.5) node[midway] {$\frac{1}{s(Js+b)}$};
\draw (4.25,0) -- (6,0);
\draw[->] (7.5,0) -- ++(1,0) node[above right] {$y$};

\draw[<-] (4,0.25)  node[right,xshift=1mm,yshift=1mm] {$+$} --++ (0,0.5)
node[above] {$d$};


\draw[->] (5,0) -- ++(0,-1.5) -- (3.5,-1.5);
\draw[->]  (2,-1.5) -- (0.75,-1.5) -- ++(0,1.25)
node[right,xshift=1mm,yshift=-1mm] {$-$};

\draw (2,-1.5-0.5) rectangle (3.5,-1.5+0.5) node[midway] {$\frac{1}{s(Js+b)}$};
\end{tikzpicture}

\todo{πράξεις, + ή -?}

Τότε προκύπτει (για είσοδο \( \frac{1}{s} \), αφού αναζητούμε σφάλμα θέσης):
\begin{align*}
    T(s) &= \frac{1}{s(Js+b)}\frac{κ}{1+\frac{κ}{s(Js+b)}} 
    = \frac{κ}{s(Js+b)+κ}\\
    T_{d}(s) &= \frac{1}{s(Js+b)}\frac{1}{1+\frac{κ}{s(Js+b)}}
    = \frac{1}{s(Js+b)+κ}
     \\
    y(s) &= T(s)r(s) + T_d(s)d(s)
    \\ &= \frac{1}{s}\frac{κ}{s(Js+b)+κ} + \frac{1}{s}\frac{1}{s(Js+b)+κ}
    \\
    e(s) &= r(s) - y(s) \\
    e_{\mathrm{ssp}} &= \lim_{s\to 0} se(s)
    \\ &= \lim_{s \to 0} s\left[
    \frac{1}{s} - \frac{1}{s}\frac{κ}{s(Js+b)+κ} - \frac{1}{s}\frac{1}{s(Js+b)+κ}
    \right]
    \\
    &= \lim_{s\to 0} \left[
    1 - \frac{κ}{s(Js+b)+κ} - \frac{1}{s(Js+b)+κ}
    \right] \\
    &= 1-\frac{κ}{κ}-\frac{1}{κ} = -\frac{1}{κ}
\end{align*}
Δηλαδή βλέπουμε ότι η διαφορετική θέση της εισόδου διαταραχής επηρεάζει το σφάλμα θέσης του
συστήματος.

Αν, προσπαθώντας να μειώσουμε στο 0 το σφάλμα της εξόδου, προσθέσουμε έναν ολοκληρωτή
πριν από την είσοδο διαταραχής:

\begin{tikzpicture}[scale=1]
\draw[->] (-0.25,0) -- (0.5,0) node[above,midway] {$r(s)$};
\draw (0.75,0) circle (0.25);
\draw (1,0) -- ++(0.5,0);
\draw (1.5,-0.5) rectangle (3,0.5) node[midway] {$\sfrac{κ}{\color{cyan!70!black} s}$};
\draw (3,0) -- (3.75,0);
\draw (4,0) circle (0.25);
\draw (5,-0.5) rectangle (6.5,0.5) node[midway] {$\frac{1}{s(Js+b)}$};
\draw (4.25,0) -- (5,0);
\draw[->] (6.5,0) -- ++(1,0) node[above right] {$y$};

\draw[<-] (4,0.25)  node[right,xshift=1mm,yshift=1mm] {$+$} --++ (0,0.5)
node[above] {$d$};

\draw[->] (7,0) -- ++(0,-1.5) -- (0.75,-1.5) -- ++(0,1.25)
node[right,xshift=1mm,yshift=-1mm] {$-$};
\end{tikzpicture}

τότε το χαρακτηριστικό πολυώνυμο του συστήματος γίνεται:
\[
Js^3+bs^2+κ = 0
\]
που έχει ρίζες στο δεξί ημιεπίπεδο, άρα οδηγεί σε ασταθές σύστημα (θυμόμαστε από τα Κ3 ότι μόνο τα πολυώνυμα με θετικούς συντελεστές \textit{μπορεί} να οδηγήσουν σε ευστάθεια).

Για να διορθώσουμε αυτήν την ατέλεια, χρησιμοποιούμε έναν \textbf{ελεγκτή PI}
(Proportional \& Integral), δηλαδή πολλαπλασιάζουμε την είσοδό του \( e(t) \) 
με \( K_P e(t) \) και ολοκληρώνουμε με
\( K_I \int e(t) \).

\begin{infobox}{Ελεγκτής PI}
Σύμφωνα με την παραπάνω παράγραφο, η έξοδος \( u(t) \) ενός ελεγκτή PI είναι:
\[
u(t) = K_P\cdot e(t) + K_I \int e(t) \dif t
\]
(όπου \( e \) η είσοδος)
και, μετασχηματίζοντας κατά Laplace:
\[
u(s) = \left( K_P + \frac{K_I}{s} \right) =
\frac{K_P\left( s+\frac{K_I}{K_P} \right)}{s}
= \frac{K_P(s+z)}{s}
\]
όπου \( z = \frac{K_P}{K_I} \) μία σταθερή τιμή.
\end{infobox}

Στο συγκεκριμένο παράδειγμα, αντικαθιστούμε τη συνάρτηση μεταφοράς \( \sfrac{κ}{s}  \) με
τη συνάρτηση:
\[
K_p\left(\frac{s+z}{s}\right)
\]
όπου \( z = \frac{K_P}{K_I} \),
και το χαρακτηριστικό πολυώνυμο γίνεται:
\[
x(s) = Js^3 + bs^2 + K_p s + K_I = 0
\]
που μπορεί να είναι ευσταθές με κατάλληλη επιλογή των σταθερών.

\paragraph{Παράδειγμα} \hspace{0pt}

Στο σύστημα χωρίς βρόγχο:

\begin{tikzpicture}[scale=0.9]
\draw[->] (0,0) -- (1.5,0) node[above,midway] {$r(s)$};
\draw (1.5,-0.5) rectangle (3,0.5) node[midway] {$H_κ(s)$};
\draw[->] (3,0) -- ++(1.5,0) node[above,midway] {$y(s)$};
\end{tikzpicture}

\begin{align*}
	e_{\mathrm{ssp}} &= \lim_{s\to 0} se(s) = \lim_{s\to 0}\left(1-H_κ(s)\right) \\
	e_{\mathrm{ssv}} &= \lim_{s\to 0} \frac{1-H_κ(s)}{s} \\
	e_{\mathrm{ssa}} &= \lim_{s\to 0} \frac{1-H_κ(s)}{s^2}
\end{align*}

\paragraph{Παράδειγμα} \hspace{0pt}

\begin{tikzpicture}[scale=1]
\draw[->] (-0.25,0) -- (0.5,0) node[above,midway] {$r(s)$};
\draw (0.75,0) circle (0.25);
\draw (1,0) -- ++(0.5,0);
\draw (1.5,-0.5) rectangle (3,0.5) node[midway] {$H(s)$};
\draw (3,0) -- (3.75,0);
\draw[->] (3.25,0) -- ++(1,0) node[above right] {$y$};

\draw[->]  (3.6,0) -- ++(0,-1.5) -- (1.5,-1.5) -- (0.75,-1.5) -- ++(0,1.25) node[right,xshift=1mm,yshift=-1mm] {$-$};
\end{tikzpicture}

\[
H(s) = \frac{s+9}{s^2+7s+3}
\]

Τι σφάλμα θέσης έχει το παραπάνω σύστημα;

\begin{align*}
	e_{\mathrm{ssp}} &= \frac{1}{1+K_{\mathrm{pos}}} \\
	K_{\mathrm{pos}} &= \lim_{s\to 0} H(s) = \frac{9}{3} = 3 \implies \\
	e_{\mathrm{ssp}} &= \frac{1}{1+3} = \frac{1}{4}.
\end{align*}

\paragraph{Παράδειγμα} \hspace{0pt}

\begin{tikzpicture}[scale=1]
\draw[->] (-0.25,0) -- (0.5,0) node[above,midway] {$r(s)$};
\draw (0.75,0) circle (0.25);
\draw (1,0) -- ++(0.5,0);
\draw (1.5,-0.5) rectangle (3,0.5) node[midway] {$H(s)$};
\draw (3,0) -- (3.75,0);
\draw[->] (3.25,0) -- ++(1,0) node[above right] {$y$};

\draw[->]  (3.6,0) -- ++(0,-1) -- (1.5,-1) -- (0.75,-1) -- ++(0,0.75) node[right,xshift=1mm,yshift=-1mm] {$-$};
\end{tikzpicture}
Στο παραπάνω σχήμα, θέτουμε:

\[
H(s) = \frac{2(s+10)}{s(s+2)(s+5)}
\]

Τι σφάλματα έχει το παραπάνω σύστημα;
\subparagraph{}
\begin{alignat*}{2}
K_{\mathrm{pos}} &= \infty \\
K_{\mathrm v}	 &= \lim_{s\to 0} sH(s) &&= \frac{20}{10} = 2 \\
K_{\mathrm a}    &= 0 \\[2ex]
e_{\mathrm{ssp}} &= \frac{1}{1+\infty} &&= 0 \\
e_{\mathrm{ssv}} &= \frac{1}{K_{\mathrm v}} &&= \frac{1}{2} \\
e_{\mathrm{ssa}} &= \frac{1}{0} &&= \infty
\end{alignat*}

\paragraph{Παράδειγμα} \hspace{0pt}

\begin{tikzpicture}[scale=1]
\draw[->] (-0.25,0) -- (0.5,0) node[above,midway] {$r(s)$};
\draw (0.75,0) circle (0.25);
\draw (1,0) -- ++(0.5,0);
\draw (1.5,-0.5) rectangle (3,0.5) node[midway] {$H(s)$};
\draw (3,0) -- (3.75,0);
\draw[->] (3.25,0) -- ++(1,0) node[above right] {$y$};

\draw[->]  (3.6,0) -- ++(0,-1) -- (1.5,-1) -- (0.75,-1) -- ++(0,0.75) node[right,xshift=1mm,yshift=-1mm] {$-$};
\end{tikzpicture}

Στο παραπάνω σχήμα:
\[
H(s) = \frac{1.8κ}{s(s+3.3)}
\]

Ποιά πρέπει να είναι η σταθερά \( κ \) ώστε να ισχύει \( e_{\mathrm{ssv}} = 0.327 \);

\begin{align*}
	K_{\mathrm{v}} &= \lim_{s\to 0}
	sH(s) = \lim_{s \to 0} \frac{1.8κ}{s+3.3} = \frac{18κ}{33} \\
	e_{\mathrm{ssv}} &= \frac{1}{K_{\mathrm v}}
	= \frac{33}{18κ} = 0.327 \implies \\
	κ &= \frac{33}{18\cdot 0.327} \simeq 5.607
\end{align*}

\paragraph{Άσκηση}
Έστω ένα σύστημα με συνάρτηση μεταφοράς:
\[
H_{κ}(s) = \frac{κs+b}{s^2+as+b}
\]

Ποιά είναι τα σφάλματα του συστήματος, και ποιά είναι η συνάρτηση μεταφοράς, αν προέρχεται
από σύστημα κλειστού βρόγχου \textit{μοναδιαίας αρνητικής ανάδρασης};

\subparagraph{Σφάλματα}
\begin{alignat*}{2}
	e_{\mathrm{ssp}} &= \lim_{s\to 0}\left[1-H_κ(s)\right] = 1-\frac{b}{b} &&= 0 \\
	e_{\mathrm{ssv}} &= \lim_{s\to 0}\left[\frac{1-H_κ(s)}{s}\right] &&= \frac{a-κ}{b} \\
	e_{\mathrm{ssa}} &= \lim_{s\to 0}\left[\frac{1-Η_κ(s)}{s^2}\right] &&= \infty
\end{alignat*}
\subparagraph{Συνάρτηση μεταφοράς} \hspace{0pt}

\begin{tikzpicture}[scale=0.8,every node/.style={scale=.8}]
\draw[->] (-0.25,0) -- (0.5,0) node[above,midway] {$r(s)$};
\draw (0.75,0) circle (0.25);
\draw (1,0) -- ++(0.5,0);
\draw (1.5,-0.5) rectangle (3,0.5) node[midway] {$H(s)$};
\draw (3,0) -- (3.75,0);
\draw[->] (3.25,0) -- ++(1,0) node[above right] {$y$};

\draw[->]  (3.6,0) -- ++(0,-1.5) -- (1.5,-1.5) -- (0.75,-1.5) -- ++(0,1.25) node[right,xshift=1mm,yshift=-1mm] {$-$};
\end{tikzpicture}

Γνωρίζουμε ή βρίσκουμε ότι, για συστήματα κλειστού βρόγχου:
\[
H_κ(s) = \frac{H(s)}{1+H(s)}
\]
άρα:
\begin{align*}
	H_κ(1+H) &= H \implies \\
	H_κ + H_κH &= H \implies \\
	H(1-H_κ) &= H_κ \implies \\
	H &= \frac{H_κ}{1-H_κ} \implies \\
	H &= \frac{\frac{κs+b}{s^2+as+b}}{1-\frac{ks+b}{s^2+as+b}}
	= \frac{\frac{κs+b}{\cancel{s^2+as+b}}}{\frac{s^2+as+b-ks-b}{\cancel{s^2+as+b}}}
	\implies \\
	\Aboxed{H(s) &= \frac{κs+b}{s(s+a-κ)}}
\end{align*}

\paragraph{Παράδειγμα} \hspace{0pt}

\begin{tikzpicture}
\draw[->] (-0.1,0) -- (3,0) node[below right] {$t$};
\draw[->] (0,-0.1) -- (0,3);

\draw[dashed] (0,1.4) node[left] {$0.5$} -- ++(3,0);
\draw[dashed] (0,2.8) node[left] {$1$} -- ++(3,0);

\draw[very thick, blue!80!green]
plot[samples=10,variable=\x,domain=0:3,smooth]
({\x},{0.2/(0.01+2*\x)*sin(2*\x r)+1.4*(1-exp(-4*\x))})
node[right] {$y(t)$};

\foreach \i in {1,2,3}
\draw(\i*2.5/3,0.05) -- ++(0,-0.1) node[below] {$\i$};
\end{tikzpicture}

Έστω το σύστημα με την παραπάνω απόκριση στη βηματική συνάρτηση. Ποιός είναι ο τύπος του
συστήματος;

\subparagraph{} \hspace{0pt}

\begin{tikzpicture}
\draw[dashed] (0,1.4) node[left] {$0.5$} -- ++(3,0);
\draw[dashed] (0,2.8) node[left] {$1$} -- ++(3,0);

\draw[very thick, blue!80!green]
plot[samples=10,variable=\x,domain=0:3,smooth]
({\x},{0.2/(0.01+2*\x)*sin(2*\x r)+1.4*(1-exp(-4*\x))})
node[right] {$y(t)$};

\draw[very thick, blue!40!cyan]
(0,2.8) -- ++(3,0) node[right] {$r(s)$};

\draw[thin,<->,blue!40!cyan]
(1.5,2.7) -- ++(0,-1.2) node[midway,right,scale=.8] {$0.5$};

\foreach \i in {1,2,3}
\draw(\i*2.5/3,0.05) -- ++(0,-0.1) node[below] {$\i$};

\draw[->] (-0.1,0) -- (3,0) node[below right] {$t$};
\draw[->] (0,-0.1) -- (0,3);
\end{tikzpicture}

Η επιθυμητή έξοδος του συστήματος είναι 1, αλλά η έξοδός του στη μόνιμη κατάσταση είναι 0.5,
επομένως υπάρχει σταθερό σφάλμα \( 0.5 \). Άρα το σύστημα έχει τύπο \( 0 \).

Γνωρίζουμε ότι, αφού είναι τύπου \( 0 \), θα έχει άπειρο σφάλμα ταχύτητας. Πράγματι, αν
βάλουμε ως είσοδο τη συνάρτηση ράμπας, το σφάλμα όσο \( t \to \infty \) θα αυξάνεται όλο
και περισσότερο:

\begin{tikzpicture}
\draw[dashed] (0,2.6/3) node[left] {$1$} -- ++(3,0);
\draw[dashed] (0,2.6) node[left] {$3$} -- ++(3,0);

\begin{scope}
\pgfmathsetseed{22}
\clip  (0,0) --
plot[samples=10,variable=\x,domain=0.1:3,smooth]
({\x},{0.2*\x+0.1*rand}) -- (3,3) -- (0,0);
\fill[draw=yellow!50!black,opacity=.5,fill=yellow!50!white,postaction={pattern=vertical lines}] (0,0) -- (3,3) -- (3,0) -- cycle;
\end{scope}

\pgfmathsetseed{22}
\draw[very thick, blue!80!green] (0,0) --
plot[samples=10,variable=\x,domain=0.1:3,smooth]
({\x},{0.2*\x+0.1*rand})
node[right] {$y(t)$};

\draw[very thick, blue!40!cyan]
(0,0) -- (3,3) node[right] {$r(s)$};

\foreach \i in {1,2,3}
\draw(\i*2.5/3,0.05) -- ++(0,-0.1) node[below] {$\i$};

\draw[->] (-0.1,0) -- (3,0) node[below right] {$t$};
\draw[->] (0,-0.1) -- (0,3);
\end{tikzpicture}

\paragraph{Άσκηση} \hspace{0pt}

Ποιά είναι τα σφάλματα του παρακάτω συστήματος;

\begin{tikzpicture}[scale=1]
\draw[->] (-0.25,0) -- (0.5,0) node[above,midway] {$r(s)$};
\draw (0.75,0) circle (0.25);
\draw (1,0) -- ++(0.5,0);
\draw (1.5,-0.5) rectangle (3,0.5) node[midway] {$\displaystyle \frac{κ}{Js^2}$};
\draw (3,0) -- (3.75,0);
\draw[->] (3.25,0) -- ++(1,0) node[above right] {$y$};

\draw (3.6,0) -- ++(0,-1.5) -- (3,-1.5);
\draw (1.5,-2) rectangle (3,-1) node[midway] {$1+bs$};
\draw[->] (1.5,-1.5) -- (0.75,-1.5) -- ++(0,1.25) node[right,xshift=1mm,yshift=-1mm] {$-$};
\end{tikzpicture}

Επειδή δεν έχουμε μοναδιαία αρνητική ανάδραση, δεν μπορούμε να εφαρμόσουμε τους τύπους
σφάλματος για συστήματα μοναδιαίας αρνητικής ανάδρασης.

Βρίσκουμε τη συνάρτηση μεταφοράς κλειστού βρόγχου:
\begin{align*}
	y &= H(s)\left(r-G(s)y\right) = \dots = \frac{H(s)}{1+H(s)G(s)}
	= \frac{\sfrac{κ}{Js^2} }{1+bs}
	= \frac{κ}{Js^2 + κbs + κ}
\end{align*}

Επομένως μπορούμε να εφαρμόσουμε τους τύπους σφάλματος χωρίς βρόγχο:
\[
e_{\mathrm{ssp}} = 0 \hspace{100pt}
e_{\mathrm{ssv}} = b \hspace{100pt}
e_{\mathrm{ssa}} = \infty
\]

\paragraph{Άσκηση} \hspace{0pt}

\begin{tikzpicture}[scale=1]
\draw[->] (-0.25,0) -- (0.5,0) node[above,midway] {$r(s)$};
\draw (0.75,0) circle (0.25);
\draw (1,0) -- ++(0.5,0);
\draw (1.5,-0.5) rectangle (3,0.5) node[midway] {$H_1(s)$};
\draw[->] (3,0) -- (3.75,0);
\draw (4,0) circle (0.25);
\draw (5,-0.5) rectangle (6.5,0.5) node[midway] {$H_2(s)$};
\draw[->] (4.25,0) -- (5,0);
\draw[->] (6.5,0) -- ++(1,0) node[above right] {$y$};

\draw[<-] (4,0.25)  node[right,xshift=1mm,yshift=1mm] {$+$} --++ (0,0.5)
node[above] {$d(s)$};

\draw[->] (7,0) -- ++(0,-1.5) -- (0.75,-1.5) -- ++(0,1.25)
node[left,xshift=-1mm,yshift=-1mm] {$-$};
\end{tikzpicture}

\begin{enumgreekparen}
	\item
	\( \displaystyle
	H_1(s) = \frac{s+2}{s+1} \qquad
	H_2(s) = \frac{10}{s(s+6)}
	 \)
	\item
	\( \displaystyle
	H_1(s) = \frac{s+2}{s(s+1)} \qquad
	H_2(s) = \frac{10}{s+6}
	 \)
\end{enumgreekparen}
Να βρεθούν και στις δύο περιπτώσεις τα σφάλματα θέσης και ταχύτητας με διάφορες
περιπτώσεις διαταραχών.

\subparagraph{Λύση, αν δεν υπάρχει διαταραχή d(s)}
\begin{enumgreekparen}
\item Έχουμε μοναδιαία αρνητική ανάδραση και το σύστημα έχει έναν ολοκληρωτή, άρα είναι τύπου 1.
Το σφάλμα θέσης είναι \( e_{\mathrm{ssp}} = 0 \), και το σφάλμα ταχύτητας μία σταθερά
\(
\displaystyle e_{\mathrm{ssv}}=\frac{1}{K_{\mathrm v}} = \frac{1}{\lim_{s\to 0} sH(s)}
= \frac{1}{\lim_{s\to 0}s\frac{s+2}{s+1}\frac{10}{s(s+6)}}
=\frac{1}{\sfrac{20}{6} }=\frac{3}{10}
 \).
 
\item
Αντίστοιχα με το προηγούμενο ερώτημα, έχουμε:
\begin{align*}
K_{\mathrm{pos}} &=
\lim_{s\to 0} H(s) = \lim_{s\to 0}
\frac{s+2}{s(s+1)}\frac{10}{s+6} = \infty \\
K_{\mathrm{v}} &=
\lim_{s\to 0}sH(s) = \lim_{s\to 0}
\frac{s+2}{s+1}\frac{10}{s+6} = 2\cdot\frac{10}{6} = \frac{10}{3}
\intertext{Άρα:}
e_{\mathrm{ssp}} &= \frac{1}{1+K_{\mathrm{pos}}} = 0 \\
e_{\mathrm{ssv}} &= \frac{1}{K_{\mathrm v}} = \frac{3}{10}
\end{align*}
\end{enumgreekparen}


\subparagraph{Λύση με διαταραχή \( d(s) = \frac{A}{s} \)}
Υπολογίζουμε:
\begin{align*}
	y &= H_2\left( d+H_1(r-y) \right) \implies \dots \\
	\implies y &= \underbrace{\frac{H_1H_2}{1+H_1H_2}}_{y_1} r +
	\underbrace{\frac{H_2}{1+H_1H_2}d}_{y_2}
\end{align*}

Φαίνεται η επαλληλία στη απόκριση του συστήματος, την οποία θα μπορούσαμε να εκμεταλλευτούμε.

Θυμόμαστε ότι:
\[
e(s) = r(s)-y(s) \qquad \text{και} \qquad
e_{\mathrm{ss}} = \lim_{s\to 0} se(s)
\]

Για την εύρεση του αποτελέσματος, θα χρησιμοποιήσουμε τον ορισμό του σφάλματος.

\subparagraph{(α) σφάλμα θέσης}
Για το σφάλμα θέσης, θεωρούμε \( r(s) = \frac{1}{s} \),  και έχουμε:
\begin{align*}
	e_{\mathrm{ssp}} &= \lim_{s\to 0}s\left[
	\frac{1}{s} - \frac{H_1H_2}{1+H_1H_2} \frac{1}{s}
	- \frac{H_2}{1+H_1H_2}\frac{A}{s}
	\right]
	\\ &=
	\lim_{s\to 0} \left[
	1 - \frac{\frac{s+2}{s+1}\frac{10}{s(s+6)}}{1+\frac{s+2}{s+1}\frac{10}{s(s+6)}}
	- \frac{\frac{10}{s(s+6)}}{1+\frac{s+2}{s+1}\frac{10}{s(s+6)}}
	\right]
	\\ &= \lim_{s\to 0} \left[
	1-\frac{\frac{10(s+2)}{\cancel{s(s+1)(s+6)}}}{\frac{s(s+1)(s+6)+10(s+2)}{\cancel{s(s+1)(s+6)}}} - A
	\frac{\frac{10}{\cancel{s(s+6)}}}{\frac{s(s+1)(s+6)+10(s+2)}{\cancel{s(s+6)}(s+1)}}
	\right]
	\\ &= \lim_{s\to 0} \left[
	1- \frac{10(s+2)}{s(s+1)(s+6)+10(s+2)}
	-A\frac{10(s+1)}{s(s+1)(s+6)+10(s+2)}
	\right]
	\\ &= 1 - \frac{20}{20} - A\cdot\frac{10}{20}
	\\ &= -\frac{A}{2}.
\end{align*}
\subparagraph{(β) σφάλμα θέσης}
Αντίστοιχα με παραπάνω, βρίσκουμε:
\begin{align*}
    e_{\mathrm{ssp}} &= \lim_{s\to 0}s\left[
    \frac{1}{s} - \frac{H_1H_2}{1+H_1H_2} \frac{1}{s}
    - \frac{H_2}{1+H_1H_2}\frac{A}{s}
    \right]
    \\ &=
    \lim_{s\to 0} \left[
    1 - \frac{\frac{s+2}{s(s+1)}\frac{10}{s+6}}{1+\frac{s+2}{s(s+1)}\frac{10}{s+6}}
    - A\frac{\frac{10}{s+6}}{1+\frac{s+2}{s(s+1)}\frac{10}{s+6}}
    \right]
    \\ &=
    \lim_{s\to 0} \left[
    1-
    \frac{\frac{10(s+2)}{\cancel{s(s+1)(s+6)}}}{\frac{s(s+1)(s+6)+10(s+2)}{\cancel{s(s+1)(s+6)}}}
    - A\frac{\frac{10}{\cancel{s+6}}}{
    	\frac{s(s+1)(s+6)+10(s+2)}{s(s+1)\cancel{(s+6)}}
    	}
    \right]
    \\ &=
    \lim_{s\to 0} \left[
    1 - \frac{10(s+2)}{s(s+1)(s+6)+10(s+2)}
    - A \frac{10s(s+1)}{s(s+1)(s+6)+10(s+2)}
    \right]
    \\ &= 1 - \frac{20}{20} - A\cdot \frac{0}{20}
    \\ &= 0.
\end{align*}

Παρατηρούμε ότι η αλλαγή της θέσης του ολοκληρωτή \( \displaystyle \frac{1}{s} \) επηρεάζει
και το σφάλμα του συστήματος.

\subsubsection{Ασκήσεις (Δημήτρης)}
\paragraph{Αντιστοιχίες ηλεκτρικού συστήματος σε μηχανικό σύστημα}

\[
\begin{array}{rcl}
	\text{τάση } V & \rightarrow & \text{δύναμη } F \\
	\text{ρεύμα } I & \rightarrow & \text{ταχύτητα } u \\
	\text{αντίσταση } R=\frac{V}{I} & \rightarrow & \text{αποσβεστήρας } B = \frac{F}{u}
	\quad
	\begin{circuitikz}[scale=0.7]
	\ctikzset{bipoles/length=.9cm}
	\draw (0,0) to[damper] (1,0);
	\end{circuitikz} \\
	\text{πυκνωτής } i_c=C\od{u_c}{t} & \rightarrow & \text{ελατήριο } \frac{1}{κ} \od{F}{t}
	\rightsquigarrow F=kx \\
	\text{πηνίο } u_L = L\od{i_L}{t} & \rightarrow & \text{αδράνεια } F = ma = m\od{u}{t}
\end{array}
\]

Ένα παράδειγμα μηχανικού συστήματος είναι τα \textbf{αμορτισέρ}, τα οποία ουσιαστικά
"κόβουν" τις υψηλές συχνότητες που μπορεί να οφείλονται σε ανομοιομορφίες του δρόμου, ώστε
να νιώθουμε άνετα μέσα σε ένα αυτοκίνητο.

\paragraph{Άσκηση 3.8} \hspace{0pt}

\begin{infobox}{}
	Οι αριθμοί των ασκήσεων συνήθως δίνονται από το βιβλίο του \textbf{Πετρίδη}.
\end{infobox}

\begin{circuitikz}
	\draw (2.5+0.2,-0.7) circle (0.2);
	\draw (3.5-0.2,-0.7) circle (0.2);
	
	\draw (5+0.2,-0.7) circle (0.2);
	\draw (6-0.2,-0.7) circle (0.2);
	
	\draw (1,-0.9) -- (8,-0.9);
	\fill[pattern=north east lines] (1,-0.9) rectangle (8,-1.5);
	
	\draw (8,-0.9) -- (8,1);
	\fill[pattern=north east lines] (8,-1.5) rectangle (9,1);
	
	\draw (0,0)
	to[spring={$K_1$}] (2,0)
	to[short] (2.5,0);
	;
	\draw (0,0.5) -- ++(0,-1);
	\draw (2.5,-0.5) rectangle ++(1,1) node[midway] {$m_1$};
	\draw (3.5,0.35) to[spring={$K_2$}] (5,0.35);
	\draw (3.5,-0.35) to[damper] (5,-0.35);
	\draw (5,-0.5) rectangle ++(1,1) node[midway] {$m_2$};
	\draw (6,0) to[damper={$B_2$}] (8,0);
	
	%\draw (3.5/2+5/2) node [label={[inner sep=1pt, fill=white,text=black, fill opacity=0.75, text opacity=1]above left:$(65, 35)$}] {};
	\draw ({(3.5+5)/2},-0.75) node[inner sep=1pt, fill=white,text=black, fill opacity=0.75, text opacity=1,below] {$B_1$};
	
	\draw[thick,orange!50!green,->] (-0.7,0) -- (0,0) node[midway,above] {$F$};
	\draw[thick,green!80!blue,->] (0,0.8) -- ++(0,-0.2) ++(0,0.1) -- ++(0.7,0)
	node[above,midway] {$R_0$};
	\draw[thick,green!80!blue,->] (2.5,0.8) -- ++(0,-0.2) ++(0,0.1) -- ++(0.7,0)
	node[above,midway] {$l_1$};
	\draw[thick,green!80!blue,->] (5,0.8) -- ++(0,-0.2) ++(0,0.1) -- ++(0.7,0)
	node[above,midway] {$l_2$};
\end{circuitikz}

Να υπολογιστεί η συνάρτηση μεταφοράς του παραπάνω συστήματος.
Σαν έξοδος θεωρείται η μετατόπιση της \( m_1 \).

\subparagraph{Λύση}
Βρίσκουμε το άθροισμα των δυνάμεων που ασκούνται σε κάθε μάζα, και εφαρμόζουμε το
νόμο του Νεύτωνα:
\begin{align*}
	\text{Για τη μάζα 2: } F = m_2\od[2]{l_2}{t}
	&=
	\underbrace{-K_2(l_2-l_1)}_{\text{ελατήριο}}
	-\underbrace{B_2\od[2]{l_2}{t}}_{\text{αποσβ.}}
	-\underbrace{B_1\left(
	\od{l_2}{t}-\od{l_1}{t}
	\right)}_{\text{αποσβεστήρας}}
	\\
	\text{Για τη μάζα 1: } F = m_1\od[2]{l_1}{t}
	&=
	-K_1(l_1-l_0)-B_1\left(\od{l_1}{t}-\od{l_2}{t}\right)
	-K_2(l_1-l_2)
\end{align*}

Μετασχηματίζουμε τις δύο εξισώσεις κατά Laplace:
\begin{align}
	m_2s^2L_2 &= -K_2(L_2-L_1) - B_2sL_2 - B_1(sL_2-sL_1) \label{3.8.lap2} \\
	m_1s^2L_1 &= -K_1(L_1-L_0) - B_1(sL_1-sL_2) - K_2(L_1-L_2) \label{3.8.lap1}
\end{align}
Θυμόμαστε ότι δεν λαμβάνουμε υπ' όψιν τις αρχικές συνθήκες, αφού αναζητούμε τη συνάρτηση
μεταφοράς.

Αναζητούμε μια σχέση της μορφής \( H(s) = \frac{L_0(s)}{L_1(s)} \), άρα πρέπει να απαλείψουμε
το \( L_2 \) από τις δύο παραπάνω σχέσεις:
\begin{align*}
	\eqref{3.8.lap2} \implies L_2 &=
	L_1 \frac{K_2+B_1s}{m_2s^2+K_2+(B_1+B_2)s} \xRightarrow{\eqref{3.8.lap1}} \\
	\frac{L_1}{L_0}
	&= \frac{K_1m_2s^2+K_1K_2+K_1(B_1+B_2)s}{
		(m_1s+K_1+K_2+B_1s)\left(m_2s^2+K_2+(B_1+B_2)s\right)-(K_2+B_1s)^2}
\end{align*}

\paragraph{Άσκηση με brushed DC κινητήρα} \hspace{0pt}

Έστω ένας brushed DC κινητήρας:

\begin{tikzpicture}[scale=1.1]
\draw (0,0.7) -- (2,0.7);
\draw (0,0) -- (2,0);
\draw (2,0.35) ellipse (0.2 and 0.35);

\draw (2,0.35) ellipse (0.2/2 and 0.35/2);
\draw (2,0.35+0.35/2) -- ++ (-0.17,0);
\draw (2,0.35-0.35/2) -- ++ (-0.17,0);

\draw (0,0.7) arc (120:242:0.4);

\draw[fill=gray!50!white]
(2,0.35+0.07) to[bend right] (2,0.35-0.07)
-- ++(1.4,0)
-- ++(0,2*0.07)
-- cycle;
%(2,0.35+0.07) arc (120:242:0.082)

\begin{scope}[xshift=3.5cm]
%,fill=gray!10!white,path fading=north
\draw[top color=gray!5!white,bottom color=gray!10!white!90!blue] (0,0.7) arc (120:242:0.4) -- (2,0) -- (2,0.7)
-- (0,0.7);
\draw (0,0) -- (2,0);
\draw[fill=white] (2,0.35) ellipse (0.2 and 0.35);
\draw (0,0.7) arc (120:242:0.4);

\draw[->,black!80!gray] (2.3,0.9) to [bend left] node[midway,right] {$θ$} (2.3,-0.2);
\end{scope}

\draw[->,gray!30!black] (1,1.5) node[right] {στάτης}  to[bend left=5] (0.7,0.7);
\draw[->,gray!30!black] (3.5,1.5) node[right] {δρομέας}  to[bend right=10] (3,0.45);
\draw[->,gray!70!black!80!green] (2.4,1) node[above right] {$J_m$}  to[bend right=10] (2,0.45);
\end{tikzpicture}

με ισοδύναμο κύκλωμα δρομέα:

\begin{circuitikz}[american]
\draw
(0,0) to[V=$V_t$] (0,-2)
;
\draw
(0,0) to [R=$R_A$,i^={$i_A$}]
(2,0) to[L=$L_A$]
(4,0) to[V_=$V$,label={\small ηλεκτρεγερτική δύναμη λόγω περιστροφής}]
(4,-2) to[short] (0,-2)
;
\end{circuitikz}

Το ισοδύναμο κύκλωμα περιγράφεται από την εξίσωση:
\begin{equation}
V_t = R_Ai_A +L_a\od{i_A}{t}
+ K_v \od{\theta}{t} \label{ex.brusheddc.elec}
\end{equation}
και το μηχανικό ισοδύναμο:
\begin{equation}
\underbrace{(J_L+J_m)}_{\mathrm{\frac{Nms^2}{rad}}}
\underbrace{\od[2]{θ}{t}}_{\mathrm{\frac{rad}{s^2}}}
= \underbrace{K_T i_A}_{\mathrm{\frac{Nm}{A}}} \label{ex.brusheddc.mech}
\end{equation}
όπου \( J_L \) και \( J_m \) είναι οι ροπές αδράνειας του κάθε μέρους.

Στη συγκεκριμένη άσκηση θεωρούμε σαν έξοδο τη \textit{γωνία του κινητήρα}, και σαν είσοδο
την \textit{τάση του δρομέα} \( V_t \).

Λύνουμε το σύστημα:
\begin{align*}
	\eqref{ex.brusheddc.mech} \implies
	i_A &= \frac{(J_L+J_m)}{K_T} \od[2]{θ}{t}
	\implies \\
	I_A &= \frac{J_L+J_m}{K_T}s^2θ
	\xRightarrow{\eqref{ex.brusheddc.elec}}
	\\
	V_T &= \left[(R_A+\cancelto{0\mathclap{\raisebox{3ex}{\text{\tiny(συνήθως αγνοούμε το $L_A$)}}}}{L_A}s)
	\frac{J}{K_T}s^2+K_vs
	\right]θ \implies \\
	\frac{θ}{V} &= \frac{\sfrac{1}{K_v} }{\frac{R_AJ}{K_TK_v}s^2+s}
\end{align*}

Βέβαια σε πολλές περιπτώσεις μπορεί να τοποθετηθεί ένας \textbf{\textit{μειωτήρας}}
μεταξύ του \( J_m \) και του \( J_L \):

\begin{tikzpicture}
\draw (0,0.7) -- (2,0.7);
\draw (0,0) -- (2,0);
\draw (2,0.35) ellipse (0.2 and 0.35);

\draw (2,0.35) ellipse (0.2/2 and 0.35/2);
\draw (2,0.35+0.35/2) -- ++ (-0.17,0);
\draw (2,0.35-0.35/2) -- ++ (-0.17,0);

\draw (0,0.7) arc (120:242:0.4);

\draw[fill=gray!50!white]
(2,0.35+0.07) to[bend right] (2,0.35-0.07)
-- ++(1.4,0)
-- ++(0,2*0.07)
-- cycle;
%(2,0.35+0.07) arc (120:242:0.082)

\begin{scope}[xshift=3.5cm]
%,fill=gray!10!white,path fading=north
\draw[top color=gray!5!white,bottom color=gray!10!white!90!blue] (0,0.7) arc (120:242:0.4) -- (2,0) -- (2,0.7)
-- (0,0.7);
\draw (0,0) -- (2,0);
\draw[fill=white] (2,0.35) ellipse (0.2 and 0.35);
\draw (0,0.7) arc (120:242:0.4);

\draw[->,black!80!gray] (2.3,0.9) to [bend left] node[midway,right] {$θ$} (2.3,-0.2);
\end{scope}

\draw[fill=white] (2.45,0.35-0.25) rectangle ++(0.5,0.5);


\draw[->,gray!70!black!80!green] (2.4,1) node[above] {$J_m$}  to[bend right=10] (2,0.45);
\draw[->,gray!70!black!80!green] (3.2,0.9) node[above] {$J_L$}  to[bend right=10] (3.3,0.45);
\end{tikzpicture}

Ο μειωτήρας είναι το μηχανικό ισοδύναμο ενός ενισχυτή, που κατασκευάζεται ουσιαστικά με δύο
γρανάζια, ώστε να αυξηθεί η δύναμη μειώνοντας την ταχύτητα ή αντίστροφα:

\begin{tikzpicture}[scale=0.7]
% Source: https://tex.stackexchange.com/a/58735

% #1 number of teeths
% #2 radius intern
% #3 radius extern
% #4 angle from start to end of the first arc
% #5 angle to decale thesecond arc from the first
\newcommand{\gear}[5]{%
	\foreach \i in {1,...,#1} {%
		[rotate=(\i-1)*360/#1] (0:#2) arc (0:#4:#2) {[rounded corners=1.5pt]
			-- (#4+#5:#3) arc (#4+#5:360/#1-#5:#3)} -- (360/#1:#2)
	}}
	
	\draw[thick]
	\foreach \i in {1,2,...,10} {%
		[rotate=(\i-1)*36]
		(0:0.7) arc (0:18:0.7) {[rounded corners=2pt] -- ++(18: 0.3) arc (18:36:0.7+0.3) } -- ++(36: -0.3)
	};
	\draw[thick,xshift=3cm] \gear{22}{2}{2.3}{7}{2.6};
	
	\draw (-1,-1.3) node {$u_1F_1$};
	\draw (5.7,-1.3) node {$u_2F_2$};
\end{tikzpicture}


Τότε θα έχουμε:
\[
J = J_m + \left(
\frac{N_1}{N_2}
\right)^2 J_L
\]

Βέβαια ένα πραγματικό σύστημα θα περιλαμβάνει και δυναμική τριβή με τη μορφή αποσβεστήρα.

\paragraph{Παράδειγμα 3.5.2} \hspace{0pt}

\begin{circuitikz}
	\draw
	(0,0) to [R={$R_{\gamma f}$},o-] (2,0)
	to [L={$L_{\gamma f}$}] (2,-2)
	to [short,-o,i^={$i_{\gamma f}$}] (0,-2)
	to [open,v={$e_{\gamma f}$}] (0,0)
	;
\end{circuitikz}

\begin{circuitikz}[scale=1.4]
	\ctikzset{bipoles/length=1.2cm}
	
	\draw (0,1) node[elmech](motor1){};
	\draw (4,1) node[elmech](motor2){};
	
	\draw
	(motor1.north)
	-- (0,2)
	to[R] (1,2)
	to[L] (2,2)
	to[R] (3,2)
	to[L] (4,2)
	-- (motor2.north)
	;
	
	\draw
	(motor2.south)
	-- (4,0)
	to[short,i=$i_a$] (0,0)
	-- (motor1.south)
	;
	
	\draw
	([xshift=4mm]motor1.south)
	to[open,v=$e_\gamma$] ([xshift=4mm]motor1.north);
	
	\draw
	(6,0) to[L,l_=$L_F$]
	(6,2) to[R=$R_F$]
	(8,2) to[battery1=$V_F$]
	(8,0) -- (6,0);
	
	\draw[<-] (motor1) -- ++(-1,-1) node[below left]
	{
		$\underset{\mathclap{\substack{\downarrow\\ \text{σταθερό}}}}{n}
		=\od{\theta}{t}$
	};
	
	\def\l{0.4}
	\draw[xshift=5cm,rotate=45] (0,0) ellipse (0.22 and 0.14);
	\draw (5-0.17,0-0.14) -- ++(\l,-\l) node(C) {};
	\draw (5+0.17,0+0.14) -- ++(\l,-\l) node(D) {};
	\draw (C.center) to[bend right=65] node[midway,below] {$J,B,T_L$} (D.center);
	
	\def\d{0.04}
	\path (4,1) ++(\d,\d) -- ++(1,-1) node[] (A) {};
	\path (4,1) ++(-\d,-\d) -- ++(1,-1) node[] (B) {};
	
	\fill[white] (A.center) to[bend left] (B.center) -- ++(90+45:0.2) -- ++(2*\d,2*\d) -- cycle;
	
	\draw[<-] (A) ++ (45+70:0.6) node[above ] {$\omega$} to[bend left=60] ($(B) + (45+110:0.6)$);
	
	\draw (4,1) ++(\d,\d) -- ++(1,-1);
	\draw (4,1) ++(-\d,-\d) -- ++(1,-1);
	\draw (A.center) to[bend left] (B.center);
	
	\draw[->] (-0.5,0.5) ++ (45+90:0.2) to[bend left=60] ($(-0.5,0.5) + (45-90:0.2)$);
\end{circuitikz}

\begin{align*}
	e_{\gamma f} &=
	R_{\gamma f} \cdot i_{\gamma f}
	+ L_{γf} \od{i_{γf}}{t} \\
	e_γ &= K\cdot n \cdot i_{γf}\quad
	\left(
	i_{γf} = \frac{e_γ}{γ}
	\right) \\
	E_{γf} &= \left(
	\frac{R_{γf}}{γ}
	+ \frac{L_{γf}}{γ}s
	\right)E_γ \\
	J\od[2]{θ}{t} + B\od{θ}{t} + T_L &= K_m i_a
	\rightarrow (Js^2+Bs)Θ + T_L &= K_mI_a \\
	L\od{i_a}{t} + Ri_a + K\od{θ}{t} &= e_γ
	\rightarrow L_sI_a + R\cdot I_A + R\cdot I_A + K_S Θ = E_γ \\
	(Js^2+Bs)Θ + T_L &= K_m \left(
	\frac{E_γ - K_sΘ}{Ls + R}
	\right) \\[2ex]
	\left( Js^2+Bs+\frac{K_mK_s}{Ls+R} \right)Θ
	+T_L &= \frac{K_m}{Ls+R}\left(
	\frac{1}{\frac{R_γf}{γ}}
	+\frac{L_γ f}{γ}s
	\right)E_{γf} \\
	\left(
	Js+B+\frac{K_mK}{Ls+R}
	\right)
	ω + T_L &= \frac{K_m}{Ls+R} \left(
	\frac{1}{\frac{R_γ f}{γ} + \frac{L_γ f}{γ}s}
	\right)E_{γf}
\end{align*}

Για \( T_L = 0 \):
\[
\frac{Ω}{E_{γf}} =
\frac{γK_m}{\left(
	(Js+B)(Ls+R)+KK_m
	\right)(Lγ_f s + R_{γf})}
\]

\paragraph{Τελεστικός ενισχυτής} \hspace{0pt}

\begin{center}
	\begin{circuitikz}[scale=2.5] \draw
		(0,0) node[op amp] (opamp) {}
		(opamp.+) to[short,-o] ++(-0.5,0)
		(opamp.-) to[short,-o] ++(-0.5,0)
		(opamp.out) to[short,-o] ++(0.5,0)
	;\end{circuitikz}
\end{center}

Ο τελεστικός ενισχυτής είναι ιδανικά ένας διαφορικός ενισχυτής άπειρου κέρδους.
Αν οι δύο είσοδοι είναι 0, τότε η έξοδος είναι 0, ενώ αν διαφέρουν, η είσοδος είναι άπειρη
(ή \( V_{cc} \) σε πραγματικό ενισχυτή).

Για τον ιδανικό τελεστικό ενισχυτή κάνουμε τις παραδοχές:
\begin{itemize}
	\item \( R_m = \infty \)
	\item \( V_{+} = V_{-} \) (για να έχουμε πεπερασμένη έξοδο)
\end{itemize}

\paragraph{Άσκηση Α35} \hspace{0pt}

\begin{circuitikz}[scale=1,american]
	\draw (0,0) node[op amp] (opamp) {};
	
	\draw (opamp.+) to[short,-*] ++(-0.5,0) node[below right] {$B$}
	to[short] ++(0,-1.2) node[ground] {};
	
	\draw (opamp.-) to[short,-*] ++(-0.5,0) node[above right] {$A$}
	to[short] ++(0,1.5)
	to[C] ++(3,0)
	to[short,-*] (1.3,0) node[above right] {$\Gamma$};
	
	\draw (opamp.-) ++ (-0.5,0) to[short] ++(0,2.5)
	to[R,i=$i_R$] ++(3,0) -- ++(0,-1);
	
	\draw (opamp.out) to[short,-o] ++ (3,0)
	to[open,v=$e_0$] ++(0,-1.5) to[short,o-] ++(0,-0.2) node[ground] {};
	
	\draw (opamp.-) to[R=$R$,i<=$I_i$,-o] ++(-4,0)
	to[open,v=$e_i$] ++(0,-2) to[short,o-] ++(0,-0.2) node[ground] {};
\end{circuitikz}

Μας ζητείται να βρούμε τον λόγο \( \displaystyle \frac{E_o}{E_i} \).

Χρησιμοποιούμε τη δεύτερη παραδοχή του ιδανικού τελεστικού, και παρατηρούμε ότι:
\[
V_B = V_A = 0
\]
άρα:
\begin{align*}
	V_{AΓ} + V_{ΓB} &= 0 \\
	\implies V_C + e_o &= 0 \\
	\implies V_C &= - e_0
\end{align*}

Υπολογίζουμε τα ρεύματα μέσω κυκλωματικών νόμων, και λαμβάνοντας υπ' όψιν ότι (λόγω της
άπειρης αντίστασης) δεν πηγαίνει ρεύμα στις εισόδους A και B του ενισχυτή:
\begin{align}
	I_i &= i_R + i_2 \label{ex.A35.cur} \\
	I_i &=\frac{e_i}{R} \label{ex.A35.ohm}
\end{align}
άρα:
\begin{align*}
	\eqref{ex.A35.cur} + \eqref{ex.A35.ohm}
	\implies \Aboxed{\frac{e_i}{R}&= -c\od{e_o}{t}-\frac{e_o}{R}}
	\intertext{και}
	i_R &= \frac{V_C}{R} = -\frac{e_o}{R}
\end{align*}

Μετασχηματίζοντας την παραπάνω εξίσωση κατά Laplace και λύνοντας, έχουμε:
\[
\boxed{\frac{E_o}{E_i} = \frac{1}{RCs+1}}
\]

\paragraph{Άσκηση 2.19} \hspace{0pt}

\begin{circuitikz}[scale=1]
	\draw (0,0) node[mixer] (m1) {};
	\draw (7,0) node[mixer] (m2) {};
	\draw (12,0) node[mixer] (m3) {};
	
	\draw (m1.west) node[inputarrow] {} node[above left] {$+$} -- ++(-1.5,0) node[above,pos=.7] {$u$};
	\draw (m1.east) to[twoport,t=$\frac{1}{s+6}$,i>_=$e$] (4,0)
	to[twoport,t=$\frac{s+3}{s+7}$,>] (5.5,0) 
	to[short,i^>=$w_1$] (m2.west) node[above,xshift=-1mm] {$+$};
	\draw (3.5,0) node[above left] {$A$} to[short,*-] ++(0,1.5) node(V) {};
	\draw (m2.north) node[inputarrow,rotate=-90] {} node[above right] {$+$} -- (m2.north |- 0,1.5) node[circ] {} -- (V.center);
	\draw (m2.east) to[twoport,i>^=$w$,t=$\frac{2}{s+5}$,-*] (10.5,0) node (B) {} node[above] {$B$};
	\draw (B.center) -- ++(0,-1.5) to[twoport,t=$\frac{1}{s}$] (m2 |- 0,-1.5)
	to[short] (m2.south) node[inputarrow,rotate=90] {} node[below right] {$-$};
	\draw (B.center) -- (m3.west) node[inputarrow] {};
	\draw (V) -- (m3 |- 0,1.5) -- (m3.north) node[inputarrow,rotate=-90] {} node[above right] {$+$};
	\draw (m3.east) -- ++(1,0) node[inputarrow] {} node[midway,above] {$y$};
	
	\draw (m3.east) ++(0.5,0) to[short,*-] ++(0,-3)
	to[twoport,t=$8$] (m1 |- 0,-3)
	-- (m1.south) node[inputarrow,rotate=90] {} node[below right] {$-$};
\end{circuitikz}

Θα μετασχηματίσουμε το παραπάνω σύστημα σε ένα ισοδύναμό του.

Αρχικά, θυμόμαστε την ισοδυναμία:

\begin{circuitikz}[scale=1]
	\draw (0,0) node[mixer] (m1) {};
	
	\draw (m1.west) node[inputarrow] {} node[above left] {$+$} -- ++(-1,0);
	\draw (m1.east) to[twoport,t=$H_1$] (3,0) to[short,*-] (4,0) node[inputarrow] {};
	\draw (3,0) -- (3,-2) to[twoport,t=$H_2$] (m1.east |- 0,-2) -- (m1 |- 0,-2) -- (m1.south)
	node[inputarrow,rotate=90] {} node[below right] {$\pm$};
	
	\draw (current bounding box.east) node[right,scale=2] {$\equiv$};
	\begin{scope}[xshift=5.2cm,yshift=-1cm]
		\ctikzset{bipoles/length=2.4cm}
		\draw (0,0) to[twoport,t=$\frac{H_1}{1\mp H_1H_2}$,>] (5,0) node[inputarrow] {};
	\end{scope}
\end{circuitikz}

Μετασχηματίζουμε το κομμάτι από \( w_1 \) ως \( B \):

\begin{circuitikz}[scale=1]
	\draw (0,0) node[mixer] (m1) {};
	
	\draw (m1.west) node[inputarrow] {} node[above left] {$+$} -- ++(-1,0) node[pos=.8,above] {$w_1$};
	\draw (m1.east) to[twoport,t=$\frac{2}{s+5}$] (3,0) node[above] {$B$}  to[short,*-] (4,0) node[inputarrow] {};
	\draw (3,0) -- (3,-2) to[twoport,t=$\frac{1}{s}$] (m1.east |- 0,-2) -- (m1 |- 0,-2) -- (m1.south)
	node[inputarrow,rotate=90] {} node[below right] {$-$};
	\draw (m1.north) node[inputarrow,rotate=-90] {} -- ++(0,1);
	
	\draw (current bounding box.east |- 0,-1) node[right,scale=2] {$\equiv$};
	\begin{scope}[xshift=7cm,yshift=-1cm]
		\draw (0,0) node[mixer] (m2) {};
		\draw (m2.west) node[inputarrow] {} node[above left] {$+$} -- ++(-1,0) node[pos=.8,above] {$w_1$};
		\draw (m2.north) node[inputarrow,rotate=-90] {} -- ++(0,1);
		\ctikzset{bipoles/length=2.4cm}
		\draw (m2.east) to[twoport,t=$\frac{2s}{s(s+5)+2}$,>] (4,0) node[inputarrow] {};
	\end{scope}
\end{circuitikz}

και το κομμάτι από \( A \) ως \( w_1 \) (αφού κάνουμε τις πράξεις και βρούμε \( \frac{w_1}{A} = 1+\frac{s+3}{s+7} \)):

\begin{circuitikz}[scale=1]
	\draw (0,0) node[mixer] (m1) {};
	
	\draw (m1.west) node[inputarrow] {} node[above left] {$+$};
	\draw (-4,0) to[twoport,t=$\frac{s+3}{s+7}$,>] (m1.west);
	\draw (-3.5,0) node[above left] {$A$} to[short,*-] ++(0,1.2) -- (m1 |- 0,1.2) -- (m1.north)
	node[inputarrow,rotate=-90] {} node[above right] {$+$};
	\draw (m1.east) -- ++(1,0) node[inputarrow] {} node[above] {$w_1$};
	
	\draw (2,0) node[right,scale=2] {$\equiv$};
	\begin{scope}[xshift=3.5cm]
		\ctikzset{bipoles/length=1.8cm}
		\draw (0,0) to[twoport,t=$\frac{2s+10}{s+7}$,>] (4,0) node[inputarrow] {};
	\end{scope}
\end{circuitikz}

Τα δύο παραπάνω μετασχηματισμένα διαγράμματα είναι συνδεδεμένα εν σειρά, άρα:

\begin{circuitikz}[scale=1]
	\draw (0,0) node[above] {$A$}
	to[twoport,t=$\frac{2s+10}{s+7}$,>] (2,0);
	\ctikzset{bipoles/length=2.2cm}
	\draw (2,0)
	to[twoport,t=$\frac{2s}{s(s+5)+2}$,>] (4,0);
	\ctikzset{bipoles/length=1.4cm}
	\draw (4,0) -- ++(0.7,0)
	node[above] {$B$} node[inputarrow] {};
	
	\draw (5,0) node[right,scale=2] {$\equiv$};
	\begin{scope}[xshift=6cm]
		\ctikzset{bipoles/length=3.4cm}
		\draw (0,0) to[twoport,t=$\frac{(2s+10)2s}{(s+7)\left(s(s+5)+2\right)}$,>] (5,0) node[inputarrow] {};
	\end{scope}
\end{circuitikz}

(όπου θέτουμε για ευκολία \( G(s) = \frac{(2s+10)2s}{(s+7)\left(s(s+5)+2\right)} \))
και το τμήμα του συστήματος εντός του βρόγχου ανάδρασης με το
\begin{circuitikz}[baseline,scale=.7]
	\ctikzset{bipoles/length=0.8cm}
	\draw (0,0) to[twoport,t=$8$,scale=.7] (2,0);
\end{circuitikz}
γίνεται:

\begin{circuitikz}[scale=1]
	\draw (0,0) node[mixer] (m1) {};
	
	\draw (m1.west) node[inputarrow] {} node[above left] {$+$};
	\draw (-3.5,0) to[twoport,t=$G(s)$,>] (m1.west);
	\draw (-3.5,0) -- (-4,0);
	\draw (-3.5,0) node[above left] {$A$} to[short,*-] ++(0,-1.2) -- (m1 |- 0,-1.2) -- (m1.south)
	node[inputarrow,rotate=90] {} node[below right] {$+$};
	\draw (m1.east) -- ++(1,0) node[inputarrow] {} node[above] {$w_1$};
	
	\draw (2,-0.5) node[right,scale=2] {$\equiv$};
	\begin{scope}[xshift=3.5cm,yshift=-0.5cm]
		\ctikzset{bipoles/length=2.2cm}
		\draw (0,0) to[twoport,t=$1+G(s)$,>] (4,0) node[inputarrow] {};
	\end{scope}
\end{circuitikz}

με \( 1+G(s) = \frac{\text{αριθμητής }N_3}{\text{παρονομαστής }D_3}
= \frac{
	(2s+10)2s+(s+7)\left[s(s+5)+2\right]}{
	(s+7)\left[s(s+5)+2\right]} \).

Άρα το τελικό σύστημα είναι:

\begin{circuitikz}[scale=1]
	\draw (0,0) node[mixer] (m1) {};
	
	\draw (m1.west) node[inputarrow] {} node[above left] {$+$} -- ++(-1,0) node[above] {$u$};
	\draw (m1.east) to[twoport,t=$H_1$] (2.5,0)
	to[twoport,t=$\frac{N_3}{D_3}$] (4,0) -- (4.5,0) node[above] {$y$} to[short,*-] (5.5,0) node[inputarrow] {};
	\draw (4.5,0) -- (4.5,-2) to[twoport,t=$8$] (m1.east |- 0,-2) -- (m1 |- 0,-1.5) -- (m1.south)
	node[inputarrow,rotate=90] {} node[below right] {$-$};
\end{circuitikz}

και επομένως προκύπτει από πράξεις:
\begin{align*}
	\frac{y}{u} &=
	\frac{\frac{N_3}{D_3(s+6)}}{1+8\frac{N_3}{D_3}\frac{1}{s+6}}
	\\ &= \frac{N_3}{D_3(s+6)+8N_3}
	\\ &= \frac{N_3}{D_3(s+6)+8N_3}
	\\ &= \frac{
		(2s+10)2s+(s+7)(s^2+5s+2)
		}{
		(s+14)(s+7)(s^2+5s+2)+16(2s+10)s
		}
\end{align*}

Η άσκηση αυτή μπορεί βεβαίως να λυθεί και αλγεβρικά. Συνοπτικά:
\begin{align}
	y &= A + B \label{ex.2.19.1} \\
	B &= \frac{2}{s+5}w \label{ex.2.19.2} \\
	A &= \frac{1}{s+6}e \label{ex.2.19.3} \\
	e &= u-8y \label{3x.2.19.4} \\
	w &= \left(1+\frac{s+3}{s+7}\right)A - \frac{1}{s}B \label{ex.2.19.5}
\end{align}
και λύνοντας το σύστημα μπορεί να βρεθεί η συνάρτηση μεταφοράς.

\paragraph{Ασκήσεις στα σφάλματα}
\begin{enumgreekparen}
	\item Τι σφάλμα θέσης έχει το σύστημα
	\( \displaystyle H(s) = \frac{s+9}{s^2+7s+3} \) αν συνδεθεί σε μοναδιαία αρνητική
	ανάδραση;
	
	\begin{circuitikz}[scale=1]
		\draw (0,0) node[mixer] (m1) {};
		
		\draw (m1.west) node[inputarrow] {} node[above left] {$+$} -- ++(-1,0);
		\draw (m1.east) to[twoport,t=$H(s)$] (3,0)  to[short,*-] (4.5,0) node[inputarrow] {};
		\draw (3,0) -- (3,-1.5) -- (m1 |- 0,-1.5) -- (m1.south)
		node[inputarrow,rotate=90] {} node[below right] {$-$};
	\end{circuitikz}
	
	\subparagraph{}
	Η συνάρτηση μεταφοράς είναι \( H_c = \frac{H}{1+H}
	= \frac{s+9}{s^2+7s+3+s+9} \). Το σφάλμα θέσης
	στη μόνιμη κατάσταση είναι το \( \lim_{s\to 0} s
	(1-H_C)\frac{1}{s} \), ή θυμόμαστε ότι \( e_{\mathrm{ss}} 
	= \frac{1}{1+K_P}
	\) όπου \( K_P = \lim_{s\to 0} t \).
	
	Ισχύει:
	\begin{align*}
		K_p &= \lim_{s\to 0} \cancel{8}
		\frac{s+9}{s^2+7s+3} = 3\\
		e_{\mathrm{ss}} &= \frac{1}{4} = 0.25\%
	\end{align*}
	\item
	Έστω συνάρτηση μεταφοράς που συνδέεται σε μοναδιαία αρνητική ανάδραση:
	\[
	H(s) = \frac{2(s+10)}{s(s^2+2)(s+5)}
	\]
	
	\subparagraph{}
	Τότε τα σφάλματα είναι:
	\begin{align*}
		e_{\mathrm{sv}} &= \lim_{s\to 0} \cancel{s}(1-H_c)\frac{1}{s^{\cancel{2}}} \\
		K_v &= \lim_{s\to 0} Hs \\
		e_{\mathrm{sv}} &= \frac{1}{K_v} \\
		e_{\mathrm{sa}} &= \lim_{s\to 0} \frac{1}{\cancel{2}}s(1-H_c)\frac{\cancel{2}}{s^3} \\
		K_a &= \lim_{s\to 0} H(s) \\
		e_{\mathrm{sa}} &= \frac{1}{K_a}
	\end{align*}
\end{enumgreekparen}

\subsection{Ταχύτητα}
Η δεύτερη προδιαγραφή που θα μελετήσουμε για τα συστήματα είναι η ταχύτητα, δηλαδή το πόσο
γρήγορα φτάνει ένα σύστημα στην επιθυμητή κατάσταση. Για να μελετήσουμε αυτήν την
προδιαγραφή θα χρησιμοποιήσουμε ως είσοδο τη \textbf{βηματική συνάρτηση}.

Ως παράδειγμα, ας δούμε το παρακάτω σύστημα:

\begin{tikzpicture}
\draw (0,0) node[rectangle,draw,inner sep=10pt] (DM) {$H(s)$};
\draw (DM.west) -- ++(-1,0) node[above right] {$r(s)$};
\draw (DM.east) -- ++(1,0) node[above left] {$y(s)$};
\end{tikzpicture}

με \( 
\displaystyle
H(s) = \frac{a}{s+a}
\).

Θυμόμαστε ότι το σφάλμα \( r(s)-y(s) \) στη μόνιμη κατάσταση (\( t\to \infty \))
είναι ορισμένο, και κάνοντας υπολογισμούς, η έξοδος στη μόνιμη κατάσταση προκύπτει:
\todo{πράξεις}
\[
\frac{k}{a}
\]

και για να μηδενιστεί το σφάλμα, πρέπει \( \frac{k}{a} = 1 \implies k=a \).

Σε μορφή σταθεράς χρόνου, η \( H \) γράφεται:
\[
H(s) = \frac{1}{τs+1}
\]
όπου
\[
τ = \frac{1}{a}
\]

Έχουμε θεωρήσει ότι \( k=a \), επομένως το σφάλμα είναι μηδενικό.

Μάλιστα, αν πάρουμε αντίστροφο μετασχηματισμό Laplace, για είσοδο τη βηματική
συνάρτηση, η έξοδος θα είναι:
\todo{πράξεις}
\[
y(t) = y_{\mathrm{ss}} \left( 1-e^{-\sfrac{t}{τ} } \right)
\]
(όπου \( y_{\mathrm{ss}} \) η έξοδος στη μόνιμη κατάσταση) ή, στο πεδίο της συχνότητας:
\[
y(s) = \frac{1}{τs+1}\frac{1}{s} = \frac{1}{s}-\frac{τ}{τs+1}
\]

\todo{Graph 35}

Έχουμε μια αποσβενύμενη απόκριση που μοιάζει:

\todo{Graph 36}

Θα ψάξουμε μετά από πόσο χρόνο έχει φτάσει η έξοδος στο 10\% και στο 90\% της επιθυμητής
τιμής:
\begin{gather*}
	1-e^{\sfrac{-t_1}{τ} }
	= 0.1 \implies t_1 = τ(\ln 10 - \ln 9)
	\\
	1-e^{\sfrac{-t_2}{τ} }
	= 0.9 \implies t_1 = τ(\ln 9)
\end{gather*}

\begin{defn}{Χρόνος ανόδου σε πρωτοβάθμιο σύστημα}{}
Υπολογίζουμε τον \textbf{χρόνο ανόδου} \( t_r \) (rise), δηλαδή τον χρόνο μεταξύ
της στιγμής που η έξοδος είναι στο 10\% και στο 90\% της επιθυμητής:
\[
t_r = t_2-t_1 \implies \boxed{t_r = τ\ln 9 \approx 2.2τ}
\]
\end{defn}

\begin{defn}{Χρόνος αποκατάστασης σε πρωτοβάθμιο σύστημα}{}
Ορίζουμε το \textbf{χρόνο αποκατάστασης} \( t_s \) (steady), ως εξής:
\[
1-e^{\sfrac{-t_s}{τ} } = 0.98
\implies \boxed{t_s \approx 4τ}
\]
\end{defn}

Επιβεβαιώνουμε δηλαδή ότι η σταθερά χρόνου \( τ \) σχετίζεται με την ταχύτητα απόκρισης
του συστήματος. Όσο μεγαλύτερη είναι η σταθερά χρόνου, τόσο μικρότερη είναι η ταχύτητά του.

Μάλιστα, αν βρούμε το σφάλμα ταχύτητας του συστήματος αυτού, θα δούμε ότι γίνεται μικρότερο,
όσο μεγαλώνει η ταχύτητά του.

\paragraph{Συνδυασμός συστημάτων}
Έστω δύο εν σειρά συστήματα:
\[
Y(s) = H(s)U(s)
\]
και πιο συγκεκριμένα:
\[
H(s) = \frac{N(s)}{D(s)} \qquad
U(s) = \frac{N_u(s)}{D_u(s)}
\]
επομένως, αν σπάσουμε σε κλάσματα όπως γνωρίζουμε από τα μαθηματικά:
\[
Y(s) = \frac{N}{D}\frac{N_u}{D_u}
= \frac{N_1}{D} + \frac{N_2}{D}
\]
που μπορεί να δώσει ένα αποτέλεσμα της μορφής:
\begin{gather*}
\frac{A}{s+p} + \cdots + \frac{B}{(s+p)^k} + \frac{C}{(s+a)^2+β}
\intertext{ή, αντίστοιχα}
(A_1+A_2t+\cdots)e^{-pt} + \dots
\end{gather*}

\paragraph{Παράδειγμα}
Έστω το σύστημα:
\[
H_z(s) = \frac{k(s+b)}{s+a}
\]
με τελική τιμή (είσοδος = \( \frac{1}{s} \)):
\[
y_{ss} = \lim_{s\to 0} sy(s) = \lim_{s\to 0}s\frac{k(s+b)}{s+a}\frac{1}{s} = \frac{kb}{a}
\]
και αρχική τιμή (για \( t=0 \)):
\[
y(0) = \lim_{t\to0} y(t) = \lim_{s\to \infty} sy(s) = k
\]
δηλαδή αυτό το σύστημα δεν ξεκινάει από μηδενική αρχική τιμή.

\todo{Graph 37}

\paragraph{Μελέτη μηδενικών}
Θα εξετάσουμε την επίδραση των μηδενικών στην έξοδο του συστήματος.

Έστω ένα σύστημα με έξοδο στο χρόνο:
\begin{align*}
	y_z(t) &= \frac{kb}{a} s(t) + \frac{k(a-b)}{a}e^{-at}
	\intertext{Το οποίο αναλύουμε περαιτέρω:}
	&= \overbrace{\frac{kb}{a}}^{y_{\mathrm{ss}}}\left(1-e^{-at}\right)
	+ ke^{-at}
	\\ &=
	y_{\mathrm{ss}}\left( 1-e^{-at} \right) + \frac{\dot{y}(t)}{b}
\end{align*}

(Τοποθετήσαμε την παράγωγο στην τελευταία σχέση, επειδή μαθηματικά φαίνεται ότι
\( y(t) = \frac{kb}{a} (1-e^{-at}) \implies \dot y = kbe^{-at} \)).

Γραφικά:

\todo{Graph 38}

Παρατηρούμε ότι αν υπάρχει μηδενικό, και όσο πιο κοντά πλησιάζει στον φανταστικό άξονα, τόσο
μεγαλύτερη επίδραση έχει στην έξοδο του συστήματος, αυξάνοντας την ταχύτητά του.

\paragraph{Ένα απλό πρόβλημα πρωτοβάθμιου συστήματος}
Έστω ένα σύστημα:
\[
G(s) = \frac{1}{s+1}
\]

Αυτό έχει πόλο στο \( -1 \), και σταθερά χρόνου \( τ = 1 \), άρα χρόνο αποκατάστασης \( t_s = 4\ \mathrm{sec} \),
κάτι που είναι πολύ μεγάλο και δεν μας αρέσει καθόλου. Επομένως, προσθέτουμε έναν βρόγχο ανάδρασης μαζί με τον
απλούστερο ελεγκτή, που είναι το αναλογικό κέρδος \( \boxed{k} \):

\todo{Graph 39}

Απαιτούμε χρόνο αποκατάστασης \( t_s \leq 1\ \mathrm{sec} \), και το ζητούμενο της άσκησης είναι η περιοχή τιμών
του \( k \) για τις οποίες ικανοποιείται αυτή η προδιαγραφή.

\subparagraph{Λύση}
Αρχικά βρίσκουμε τη συνάρτηση μεταφοράς του συστήματος:
\[
H(s) = \frac{\frac{k}{s+1}}{1+\frac{k}{s+1}}
= \frac{k}{s+(k+1)}
\]
Αυτό έχει σταθερά χρόνου \( τ = \frac{1}{k+1} \), επομένως:
\[
t_s \approx 4τ = \frac{4}{k+1}
\]

Πρέπει:
\[
t_s \leq 1 \implies \frac{4}{k+1} \leq 1 \implies k \geq 3.
\]

\paragraph{Ένα παράδειγμα δευτεροβάθμιου συστήματος}
Έστω το σύστημα ενός περιστρεφόμενου κυλίνδρου με απόσβεση:

\todo{Graph 40}
\[
u = J \ddot y + B\dot y
\]

Παραπάνω δίνεται η διαφορική εξίσωση του συστήματος, που θα μετασχηματίσουμε κατά Laplace:
\begin{gather*}
Js^2y(s) +Bsy(s) = u(s) \\
\frac{y(s)}{u(s)} = \frac{1}{s(Js+b)}
\end{gather*}

Αντίστοιχα με το προηγούμενο παράδειγμα, προσθέτουμε έναν ελεγκτή και έναν βρόγχο ανάδρασης:
\todo{Graph 41}

με έξοδο:
\todo{πράξεις}
\[
\frac{y(s)}{r(s)}
= \frac{k}{Js^2+Bs+k}
= \frac{\sfrac{k}{J} }{s^2 +\sfrac{B}{J}s + \sfrac{k}{J}   }
= \frac{ω_n^2}{s^2+2ζω_n^2s+ω_n^2}
\boxed{
= \frac{ω_n^2}{(s+σ)^2 + ω_d^2}
}
\]
όπου θέσαμε:
\begin{align*}
	ω_n^2 &= \sfrac{k}{J}  \\
	2ζω_ν &= \sfrac{B}{J}, \quad ζ \leq 1 \\
	s_{1,2}  &= -σ \pm jω_d^2
\end{align*}

Μετά από πράξεις που δεν παρουσιάζονται, η έξοδος του συστήματος είναι:
\begin{itemize}
	\item Αν έχουμε \textbf{δύο πραγματικούς πόλους} \( p1,\ p2 \):\[
	y(t) = y_{\mathrm{ss}}
	\left[
	1+\frac{p_1p_2}{p_1-p_2}
	\left(
	\frac{e^{p_1t}}{p_1}
	-
	\frac{e^{p_2t}}{p_2}
	\right)
	\right]
	\]
	
	\todo{Graph 42}
	
	\todo{check spelling αποσβενύμενων}
	Δηλαδή η απόκριση του συστήματος αποτελείται από το άθροισμα δύο εκθετικά αποσβενύμενων
	αποσβέσεων. Πρακτικά, παίζει ρόλο ο πόλος που προκαλεί την πιο αργή απόσβεση (ειδικά αν
	οι πόλοι απέχουν αρκετά μεγάλη απόσταση μεταξύ τους), καθώς η άλλη απόσβεση εξαλείφεται
	γρήγορα. Ο αργός πόλος επιβάλλει το ρυθμό του.
	\item Αν έχουμε \textbf{δύο μιγαδικές συζυγείς ρίζες} \( -σ \pm jω_δ^2 \):
	\[
	y(t) = y_{\mathrm{ss}}\left[
	1-e^{-ζ\omega_nt} -
	\left(
	\cos \omega_nt + \frac{ζ}{\sqrt{1-ζ^2}} \sin \omega_n t
	\right)
	\right]
	\]
	
	\todo{Graph 43}
	\todo{Replace all ω with \omega}
	
	Σε αυτήν την περίπτωση, αποδεικνύεται μάλιστα ότι:
	\[
	t_s \approx \frac{4}{ζ\omega_n}
	\]
	
	Η έξοδος ενός τέτοιου συστήματος μοιάζει κάπως έτσι:
	\todo{Graph 44}
	
	Και ο χρόνος ανόδου ορίζεται διαφορετικά:
	\begin{defn}{Χρόνος ανόδου σε δευτεροβάθμιο σύστημα με συζυγείς πόλους}{}
		Ως \textbf{χρόνος ανόδου} ορίζεται ο χρόνος μέχρι η έξοδος να φτάσει την πρώτη
		φορά στην επιθυμητή τιμή.
		
		Αποδεικνύεται ότι είναι ίσος με:
		\[
		\boxed{t_s = \frac{π-θ}{ω_d}}
		\]
		
		όπου \( θ \) η γωνία του ενός πόλου.
	\end{defn}
	
	Ο χρόνος \( t_p \) (peak) μέχρι η έξοδος να φτάσει στην πρώτη κορυφή είναι:
	\[
	\boxed{t_p = \frac{π}{\omega_d}}
	\]
	
	Βλέπουμε ότι αρκετές στιγμές η έξοδος έχει τιμή μεγαλύτερη της εξόδου. Η υπερύψωση αυτή
	μπορεί σε κάποιες περιπτώσεις να μην είναι επιθυμητή, επειδή για παράδειγμα δεν θέλουμε
	το ρεύμα σε ένα κύκλωμα να ξεπεράσει μια μέγιστη τιμή, ή ο βραχίονας ενός ρομπότ να
	φτάσει εκτός από κάποια όρια.
	
	Αποδεικνύεται ότι η πρώτη κορυφή φτάνει στην τιμή:
	\todo{check}
	\[
	M_p =
	e^{\left(
		\frac{ζπ}{\sqrt{1-ζ^2}}
		\right)}
		= e^{-σ t_p}
	\]
	
	Τέλος, για τη γωνία του πόλου, αποδεικνύεται ότι:
	\[
	\boxed{\cos θ = ζ}
	\]
	
	\paragraph{Παράδειγμα συστήματος δεύτερης τάξης} \hspace{0pt}
	
	\todo{Graph 45}
	
	\subparagraph{(α)}
	Η συνάρτηση μεταφοράς του συστήματος, μετά από πράξεις, είναι:
	
	\todo{πράξεις}
	\[
	H(s) = \frac{\sfrac{k}{J} }{s^2
		+\left(\frac{B+kK_v}{J}\right)s + \sfrac{k}{J} 
		}
	\]
	
	\subparagraph{(β)}
	Η συνάρτηση μεταφοράς είναι:
	\[
	H(s) =
	\frac{\sfrac{K}{J}(1+K_vs) }{s^2 + \left(\frac{B+kK_v}{J}s\right)
		+ \sfrac{K}{J} }
	\]
	
	\subparagraph{}
	Παρατηρούμε ότι τα δύο συστήματα έχουν παρόμοιες συναρτήσεις μεταφοράς, αλλά το
	δεύτερο έχει ένα μηδενικό.
\end{itemize}


\end{document}
