\documentclass[12pt,a4paper,titlepage,fleqn]{article}

\usepackage{amsmath}
\usepackage{amsfonts}
\usepackage{amssymb}
\usepackage{libs/commath2}
\usepackage[table]{xcolor}
\usepackage[hidelinks,draft=false]{hyperref}
\usepackage[skins,theorems]{tcolorbox}
\usepackage{titlesec}
\usepackage{tikz}
\usepackage{libs/circuitikz} % use our own recent version to make sure some bugs are fixed
\usepackage{pgfplots}
\usepackage{mathtools}
\usepackage[makeroom]{cancel}
\usepackage{mathrsfs}
\usepackage{wrapfig}
%\usepackage{subcaption}
%\usepackage{floatrow}
\usepackage{esint}
\usepackage{enumitem}
%\usepackage{bm}
\usepackage{relsize}
\usepackage{xfrac}
\usepackage{comment}
%\usepackage{siunitx}
%\usepackage{MnSymbol}
\usepackage[obeyDraft,disable]{todonotes}
%\usepackage[linesnumbered,lined]{algorithm2e}


\pgfplotsset{compat=1.13}
\usetikzlibrary{arrows.meta}
\usetikzlibrary{patterns}
\usetikzlibrary{decorations.pathmorphing,patterns}
\usetikzlibrary{decorations.markings}
\usetikzlibrary{backgrounds}
\usetikzlibrary{shapes.misc}
\usetikzlibrary{shapes.multipart}
\usetikzlibrary{shadows.blur}
\usetikzlibrary{fadings}
\usetikzlibrary{intersections}
\usetikzlibrary{arrows.meta}
\usetikzlibrary{calc}
\usetikzlibrary{matrix}
\usetikzlibrary{positioning}
\usetikzlibrary{shapes}
\usetikzlibrary{shadings}

\tcbuselibrary{breakable}

\tikzset{cross/.style={cross out, draw,
        minimum size=2*(#1-\pgflinewidth),
        inner sep=0pt, outer sep=0pt}}
\tikzset{
    mark position/.style args={#1(#2)}{
        postaction={
            decorate,
            decoration={
            	post length=1mm, % ??? Magic to fix "Dimension
            	pre length=1mm, % ???  too large" errors.
                markings,
                mark=at position #1 with \coordinate (#2);
            }
        }
    }
}
\makeatletter
\tikzset{
  use path for main/.code={%
    \tikz@addmode{%
      \expandafter\pgfsyssoftpath@setcurrentpath\csname tikz@intersect@path@name@#1\endcsname
    }%
  },
  use path for actions/.code={%
    \expandafter\def\expandafter\tikz@preactions\expandafter{\tikz@preactions\expandafter\let\expandafter\tikz@actions@path\csname tikz@intersect@path@name@#1\endcsname}%
  },
  use path/.style={%
    use path for main=#1,
    use path for actions=#1,
  }
}
\makeatother

\pgfmathdeclarefunction{sinc}{1}{%
	\pgfmathparse{abs(#1)<0.01 ? int(1) : int(0)}%
	\ifnum\pgfmathresult>0 \pgfmathparse{1}\else\pgfmathparse{sin(#1 r)/#1}\fi%
}
\pgfmathdeclarefunction{gauss}{2}{%
	\pgfmathparse{1/(#2*sqrt(2*pi))*exp(-((x-#1)^2)/(2*#2^2))}%
}

\usepackage[left=2cm,right=2cm,top=2cm,bottom=2cm]{geometry}

%\usepackage[no-math]{fontspec}
%\usepackage{fontspec}
\usepackage{mathspec}
%\usepackage{newtxtext,newtxmath}
%\usepackage{unicode-math}
%\setmainfont{texgyretermes-regular.otf}
%\setsansfont{texgyreheros-regular.otf}
%\newfontfamily\greekfont[Script=Greek]{Linux Libertine O}
%\newfontfamily\greekfontsf[Script=Greek]{Linux Libertine O}
\usepackage{polyglossia}
%\newfontfamily\greekfont[Script=Greek]{texgyretermes-regular.otf}
\newfontfamily\greekfontsf[Script=Greek]{texgyreheros-regular.otf}
\newfontfamily\greekfonttt[Script=Greek]{Latin Modern Mono}
%\usepackage[greek]{babel}
\setdefaultlanguage{greek}
\setotherlanguage{english}

%\usepackage[utf8]{inputenc}
%\usepackage[greek]{babel}


%\usepackage{tkz-euclide} % loads  TikZ and tkz-base
%\usetkzobj{angles} % important you want to use angles

\newlist{enumparen}{enumerate}{1}
\setlist[enumparen]{label=(\arabic*)}
\newlist{enumpar}{enumerate}{1}
\setlist[enumpar]{label=\arabic*)}

\newlist{enumgreek}{enumerate}{1}
\setlist[enumgreek]{label=\alph*.}
\newlist{enumgreekparen}{enumerate}{1}
\setlist[enumgreekparen]{label=(\alph*)}
\newlist{enumgreekpar}{enumerate}{1}
\setlist[enumgreekpar]{label=\alph*)}


\newlist{enumroman}{enumerate}{1}
\setlist[enumroman]{label=(\roman*)}

\newlist{enumlatin}{enumerate}{1}
\setlist[enumlatin]{label=(\alph*)}

\newlist{invitemize}{itemize}{1}
\setlist[invitemize]{noitemsep,label=}

\usepackage{letltxmacro}

\LetLtxMacro\OriginalLongrightarrow\Longrightarrow
\LetLtxMacro\OriginalLongleftarrow\Longleftarrow

% Implement new macros
% --------------------
\usepackage{trimclip}
\DeclareRobustCommand\Longrightarrow{\NewRelbar\joinrel\Rightarrow}
\DeclareRobustCommand\Longleftarrow{\Leftarrow\joinrel\NewRelbar}

\makeatletter
\DeclareRobustCommand\NewRelbar{%
  \mathrel{%
    \mathpalette\@NewRelbar{}%
  }%
}
\newcommand*\@NewRelbar[2]{%
  % #1: math style
  % #2: unused
  \sbox0{$#1=$}%
  \sbox2{$#1\Rightarrow\m@th$}%
  \sbox4{$#1\Leftarrow\m@th$}%
  \clipbox{0pt 0pt \dimexpr(\wd2-.6\wd0) 0pt}{\copy2}%
  \kern-.2\wd0 %
  \clipbox{\dimexpr(\wd4-.6\wd0) 0pt 0pt 0pt}{\copy4}%
}
\makeatother


\makeatletter
\pgfdeclareradialshading[tikz@ball]{ball}{\pgfqpoint{0bp}{0bp}}{%
	color(0bp)=(tikz@ball!50!white);
	color(10bp)=(tikz@ball!50!white);
	color(15bp)=(tikz@ball!70!black);
	color(20bp)=(black!70);
	color(30bp)=(black!70)}%
\makeatother


\makeatletter
\let\anw@true\anw@false

%\newcommand{\attnboxed}[1]{\textcolor{red}{\fbox{\normalcolor\m@th$\displaystyle#1$}}}
\makeatother
\tcbset{highlight math style={enhanced,colframe=red,colback=white,%
        arc=0pt,boxrule=1pt,shrink tight,boxsep=1.5mm,extrude by=0.5mm}}
\newcommand{\attnboxed}[1]{\tcbhighmath[colback=red!5!white,drop fuzzy shadow,arc=0mm]{#1}}
\newcommand{\infoboxed}[1]{%
	\tcbhighmath[colframe=blue!50!white,colback=blue!5!white,arc=0mm]{#1}}
\titleformat{\section}{\bf\Large}{Κεφάλαιο \thesection}{1em}{}
\newtcolorbox{attnbox}[1]{colback=red!5!white,%
    colframe=red!75!black,fonttitle=\bfseries,title=#1}
\newtcbox{quickattnbox}[1]{colback=red!5!white,%
	colframe=red!75!black,fonttitle=\bfseries,title=#1}
\newtcolorbox{infobox}[1]{colback=blue!5!white,%
    colframe=blue!75!black,fonttitle=\bfseries,title=#1}

\AtBeginDocument{%
\let\arg\relax
\let\Re\relax
\let\Im\relax
\DeclareMathOperator{\arg}{Arg}
\DeclareMathOperator{\Re}{Re}
\DeclareMathOperator{\Im}{Im}
}
\DeclareMathOperator{\sinc}{sinc}
\DeclareMathOperator{\sgn}{sgn}
\DeclareMathOperator{\erf}{erf}
\DeclareMathOperator{\cov}{cov}

\newif\ifhidetikz
\hidetikzfalse
%\hidetikztrue   % <---- comment/uncomment that line

\ifhidetikz

\let\oldtikzpicture\tikzpicture
\let\oldendtikzpicture\endtikzpicture

\renewenvironment{tikzpicture}{
    \tiny
    \tt
    \color{blue}
    \newcommand{\draw}{\textit{draw}}
    \newcommand{\filldraw}{\textit{filldraw}}
    %\newcommand{\x}{\textit{x}}
    %\newcommand{\p}{\textit{x}}
    \newcommand{\x1}{\textit{x1}}
    \newcommand{\y1}{\textit{y1}}
    \newcommand{\p1}{\textit{p1}}
}{
}
\newenvironment{axis}{
    \newcommand{\addplot}{\textit{addplot}}
}{
}
\fi

% Global amount of samples
% Set to a higher value (e.g. 200) for nicer graphs
% Set to a low value (e.g. 10) for performance
\newcommand*{\gsamples}{70}

% Equals command as a workaround for CircuiTikZ bug
% not allowing the = sign in labels
\newcommand*{\equals}{=}

\newcommand{\nesearrow}{%
	\,%
	\smash{\raisebox{-1.1ex}
		{$%
			\stackrel{\displaystyle\nearrow}{\displaystyle\searrow}%
			$}}%
}
\newcommand{\degree}{^{\circ}} % not great
\newcommand\numberthis{\addtocounter{equation}{1}\tag{\theequation}} % add an equation number to a number-less math environment

\newtcbtheorem[number within=section]{theorem}{Θεώρημα}%
{colback=green!5,colframe=green!35!black,colbacktitle=green!35!black,fonttitle=\bfseries,enhanced,attach boxed title to top left={yshift=-2mm,xshift=-7mm},width=.9\textwidth,arc=.7mm}{th}
\newtcbtheorem[number within=section]{defn}{Ορισμός}%
{colback=blue!5,colframe=cyan!35!black,colbacktitle=blue!35!black,fonttitle=\bfseries,enhanced,attach boxed title to top left={yshift=-2mm,xshift=-2mm}}{def}
\newtcbtheorem[number within=section]{exercise}{Άσκηση}%
{colback=gray!3,colframe=gray!35!black,colbacktitle=gray!35!black,fonttitle=\bfseries,enhanced,attach boxed title to top left={yshift=-2mm,xshift=-2mm}}{exc}




\setmainfont{Ubuntu Light}
\setsansfont{Arial}
%\newfontfamily\greekfont[Script=Greek]{Linux Libertine O}
%\newfontfamily\greekfontsf[Script=Greek]{Linux Libertine O}
\usepackage{polyglossia}
\newfontfamily\greekfont[Script=Greek,Scale=0.9]{Ubuntu Light}

\title{Εφαρμοσμένα Μαθηματικά - Σημειώσεις}
\date{2016}
\author{\textlatin{\csuse{no\greek @numbers}\selectlanguage{english} \url{https://github.com/kongr45gpen/ece-notes}}}

\begin{document}
	\url{http://users.auth.gr/natreas} \\
	Σημειώσεις: Εγώ Κεφ. 3-4-5 \\
	Κεχαγιάς Κεφ. 1-2-6

	Βιβλία:
	\begin{itemize}
		\item Churchill - Brown (για μηχανικούς)
		\item Marsden (πιο μαθηματικό)
	\end{itemize}

	\part{Ατρέας}
	\section{Μιγαδικοί Αριθμοί}
	\textbf{Έστω} \( \mathbb C = \left\lbrace\qquad z = \overbrace{(x,y)}^{\mathclap{\text{γεωμετρική παράσταση μιγαδικού}}};\ x,y\in\mathbb R  \right\rbrace \)

	Είναι σύνολο εφοδιασμένο με τις πράξεις:
	\begin{enumgreekparen}
		\item Πρόσθεση μιγαδικών

		Αν \( z_1=(x_1,y_1) \) και \( x_2=(x_2,y_2) \), τότε:\[
		z_1+z_2 = (x_1+x_2,\ y_1+y_2)
		\]

		\item Γινόμενο \( \lambda \in \mathbb R  \) με μιγαδικό \( z \)

		Αν \( z=(x,y) \), τότε ορίζω:
		\[
		\lambda z = (\lambda x,\lambda y)
		\]

		\item \attnboxed{\text{Πολλαπλασιασμό μιγαδικών αριθμών}}

		Αν \( z_1=(x_1,y_1),\ z_2=(x_2,y_2) \), τότε ορίζω:
		\[
		z_1z_2 = \left(x_1x_2-y_1y_2,\ x_1y_2+x_2y_1\right)
		\]
	\end{enumgreekparen}

	Καλείται σύνολο των μιγαδικών αριθμών.

	\begin{itemize}
		\item Δεν μπορώ να συγκρίνω μιγαδικούς
		\item Οι γνωστές ιδιότητες των πράξεων ισχύουν στους μιγαδικούς
	\end{itemize}

	Η γεωμετρική παράσταση του \( \mathbb C \) είναι το λεγόμενο μιγαδικό επίπεδο.

	\begin{center}
	\begin{tikzpicture}[scale=2.5]
		\draw[->] (0,-1.3) -- (0,1.5);
		\draw[->] (-1.5,0) -- (1.7,0);

		\draw[dashed] (0,1) -- (1,1) -- (1,-1);
		\filldraw (1,1) circle(0.8pt) node[above right] {$z=(x,y)$} ;
		\filldraw (0,1) circle(0.6pt) node[below right] {$(0,1)=i$};
		\filldraw (1,0) circle(0.6pt);

		\draw[->] (1.4,-0.8) -- (1.4,-0.2) node[midway,right] {πραγματικός άξονας \( \Re(z) \)};

		\draw[->] (-1,0.7) -- (-0.2,0.7) node[pos=.1,below] {φανταστικός άξονας \( \Im(z) \)};

		\draw[gray,->] (0,0) -- (1,1);
		\draw[->] (.3,0) arc (0:45:.3) node[midway,right] {$\theta$};

		\draw[gray,->] (0,0) -- (1,-1);
		\draw[->] (.3,0) arc (0:-45:.3);
		\filldraw (1,-1) circle(0.8pt) node[right] {$\bar z=(x,-y)$} ;
	\end{tikzpicture}
	\end{center}

	\[
	x \in \mathbb R \xleftrightarrow{\text{1-1}} A = \left\lbrace (x,0): x \in \mathbb R  \right\rbrace
	\]

	\begin{itemize}
		\item \(
		    (x,0),(y,0) \in A \implies (x,0)+(y,0)=(x+y,0) \in A
		\)
		\item \(
		    (x,0)(y,0) = (xy,0) \in A
		\)
	\end{itemize}

	Στο εξής γράφω: \begin{align*}
	    1 &= (1,0) \\
	    x &= (x,0)
	\end{align*}

\textbf{Ορίζω}:
	\[
	\mathlarger{\mathlarger{\mathlarger{i = (0,1)}}}
	\]
	και καλείται φανταστική μονάδα του μιγαδικού επιπέδου.

	\begin{gather*}
	i^2 = (0,1)(0,1) = (0\cdot0-1\cdot1,\ 0\cdot1+1\cdot0) = (-1,0) = -1 \\
	\boxed{i^2=-1}
	\end{gather*}

	\textbf{Έτσι}:
	\begin{gather*}
	    z=(x,y) = x(1,0) + y(0,1) \\
	    \overset{x=(x,0)}{\underset{i=(0,1)}{=}} x \cdot 1 + yi \\
	    \implies \boxed{z=x+iy}
	\end{gather*}

	\[
	\mathlarger{\mathlarger{\underbrace{z=x+iy}_{\mathclap{\text{άλγεβρα}}}
			\iff \underbrace{z=(x,y)}_{\mathclap{\text{γεωμετρία}}}
			}}
	\]

	\paragraph{}
	Έστω \( z=x+iy \)
	\begin{gather}
		\overset{\text{πολικές}}{\underset{\text{του } (x,y)}{=}}
		\rho\cos\theta+i\rho\sin\theta = \nonumber
		\\ = \mathlarger{\rho(\cos\theta+i\sin\theta)} \label{eq:1}
	\end{gather}

	Έτσι, η (\ref{eq:1}) γράφεται ως:
	\begin{align*}
	z &= |z| \underbrace{(\cos\theta+i\sin\theta)} \\
	  &= |z| \cdot \mathlarger{\mathlarger{e^{i\theta}}}
	\end{align*}
	όπου στο εξής:
	\begin{align*}
	\Aboxed{e^{i\theta} = \cos\theta+i\sin\theta} \\
	\Aboxed{\text{τύπος του Euler}}
	\end{align*}

	Τελικά: \[
	\boxed{\mathlarger{\mathlarger{\mathlarger{\mathlarger{z=|z|e^{i\theta}}}}}}
	\text{ (πολική μορφή μιγαδικών)}
	\]

	\subparagraph{Σημείωση:} \( \cos\theta + i\sin\theta \)
	\begin{gather*}
	\overset{\text{σειρές}}{\underset{\text{McLaurin}}{=}} \left(
	1-\frac{\theta^2}{2!} + \frac{\theta^4}{4!} + \dots
	\right) + i \left(\theta-\frac{\theta^3}{3!}+\frac{\theta^5}{5!}-\dots\right)
	\\
	\overset{i^2=-1}{=} \left(
	1+\frac{(i\theta)^2}{2!}+\frac{(i\theta)^4}{4!}+\dots
	\right) + \left(
	i\theta+\frac{(i\theta)^3}{3!}+\frac{(i\theta)^5}{5!}+\dots
	\right)
	\\ =
	1 + (i\theta) + \frac{(i\theta)^2}{2!} + \frac{(i\theta)^3}{3!}
	+ \dots + \frac{(i\theta)^n}{n!} + \dots = \mathlarger{e^{i\theta}}
	\end{gather*}

	\begin{itemize}
		\item Ορίζω {\large πρωτεύον όρισμα} \( \mathlarger{\mathlarger{\mathrm{Arg} z}} \) (μη μηδενικού) μιγαδικού \( z \) να είναι η γωνία \( \theta \)
		που σχηματίζει ο θετικός πραγματικός ημιάξονας του \( \mathbb C \) με την
		ημιευθεία \( OA \), όπου \( A \) το σημείο της γεωμετρικής παράστασης του
		\( z=x+iy \).
	\end{itemize}

	\subparagraph{Έτσι:}
	\[
	z = |z|e^{i\arg z} \quad \text{πολική μορφή του } z
	\]

	\begin{align*}
	z_1z_2 &= |z_1|e^{i\arg z_1}|z_2|e^{i\arg z_2} \\
	\Aboxed{z_1z_2 &= |z_1||z_2|e^{i(\arg z_1 + \arg z_2)}
	}
	\end{align*}
	\begin{align*}
	\frac{z_1}{z_2} &= \frac{|z_1|}{|z_2|} \frac{e^{i\theta_1}}{e^{i\theta_2}}
	\\ &= \left| \frac{z_1}{z_2} \right| e^{i(\theta_1-\theta_2)}
	\end{align*}

	\begin{tikzpicture}[scale=2.5]
	\draw[gray,->] (0,-0.7) -- (0,2);
	\draw[gray,->] (-1.5,0) -- (1.7,0);

	\filldraw (0,0) -- ++(35:1.2) circle(0.6pt) node[above right] {$z_1$};
	\draw[->] (.3,0) arc (0:35:.3) node[midway,right] {$\theta_1$};

	\filldraw (0,0) -- ++(75:1.7) circle(0.6pt) node[above right] {$z_2$};
	\draw[->] (.6,0) arc (0:75:.6) node[pos=.8,above right] {$\theta_2$};

	\filldraw (0,0) -- ++(110:{1.2*1.7}) circle(0.6pt) node[above right] {$z_1z_2$};
	\draw[->] (1,0) arc (0:110:1) node[pos=.6,above right] {$\theta_1+\theta_2$};
	\end{tikzpicture}

	\textbf{Ιδιότητα:} \( z\bar{z} = |z|^2 \)

	\section{Μιγαδικές συναρτήσεις}
	Κάθε συνάρτηση \( f: A \subseteq \mathbb C \to \mathbb C \) καλείται μιγαδική
	συνάρτηση μιγαδικής μεταβλητής.

	\[
	f = \underbrace{f(\underbrace{z}_{\text{η μεταβλητή μιγαδικός}})}_{\text{μιγαδική συνάρτηση διότι έχει τιμή μιγαδική}}
	\]

	\paragraph{π.χ.}
	\begin{gather*}
	f(z) = z^2 \implies
	f(x+iy) = (x+iy)^2 = x^2 + (iy)^2+2x\cdot \underbrace{x^2-y^2}_{\Re(f)}+i\underbrace{(2xy)}_{\Im(f)}
	\\
	\overset{\text{γεωμετρική}}{\underset{\text{μορφή}}{=}} (x^2-y^2,\ 2xy)
	\end{gather*}
	\subparagraph{Τελικά:} \(\boxed{f(x,y)=(x^2-y^2,\ 2xy)} \quad \mathbb R^2 \to \mathbb R^2 \)

	\paragraph{π.χ.}
	\begin{gather*}
	f(z) = \frac{1}{|z|\bar{z}} \overset{z=x+iy}{=}
	\frac{1}{\sqrt{x^2+y^2}}\cdot \frac{z}{\bar{z}z} \\
	\overset{z\bar{z}=|z|^2}{=} \frac{1}{\sqrt{x^2+y^2}} \cdot \frac{z}{|z|^2}
	= \frac{x+iy}{(x^2+y^2)^{\sfrac{3}{2}}}
	\\ \overset{\text{γεωμ}}{=}
	\frac{(x,y)}{(x^2+y^2)^{\sfrac{3}{2}}}
	\overset{\vec{r} = (x,y)}{=} \boxed{\frac{\vec{r}}{|\vec{r}|^3}}
	\end{gather*}

	Κεντρικό διαν. πεδίο που θυμίζει το πεδίο Coulomb.

	\[
	\underbrace{f=f(z)}_{\mathclap{\text{μιγαδική μιγ. μεταβλ.}}} \xleftrightarrow{\quad\text{1-1}\quad}
	\begin{array}{l}
	\text{διανυσμ. πεδίο του } \mathbb R^2 \\
	F(x,y) = \left( u(x,y),\ v(x,y) \right)
	\end{array}
	\]
	όπου \( u,v \) πραγματ. συναρτ. 2 μεταβλητών

	\paragraph{Υπάρχουν} \( f:A \subseteq \mathbb R \to \mathbb C \),
	μιγαδικές πραγματικής μεταβλητής

	π.χ \begin{align*}
	f(t) &= e^{it},\ t \in (0,\pi] \\
	&= \cos t + i \sin t
	\end{align*}
	\[
	t \to (\cos t, \sin t) \quad \text{καμπύλη } x^2+y^2=\cos^2 t +\sin^2 t = 1
	\]

	\begin{tikzpicture}[scale=1.5]
	\draw[->] (0,-1.5) -- (0,1.5);
	\draw[->] (-1.5,0) -- (1.5,0);

	\draw[thick,
		decoration={markings, mark=at position 0.125 with {\arrow{>}}},
		postaction={decorate}
	] (0,0) circle (1);

	\draw (0,1) node[above right] {$f(t)=e^{it}$};
	\end{tikzpicture}

	Η γραφ. παράσταση της \( f(t)=e^{it},\ t \in (-\pi,\pi) \) είναι ο μοναδιαίος κύκλος
	κέντρου \( (0,0) \) με αντιωρολογιακή φορά.

	\[
	g(t) = 1+it, t\in \mathbb R,\ =(1,t) = (1,0)+t(0,1)
	\]

	\paragraph{}
	Το πεδίο ορισμού μιγαδικών συναρτήσεων μιγαδ. μεταβλητών
	υπολογίζεται ως συνήθως (με τις πραγματικές συναρτήσεις)
	ΜΕ ΚΑΠΟΙΕΣ Διαφοροποιήσεις

	\[
	f(z)=\frac{1}{z}
	\]
	Πρέπει ο παρον. να είναι διάφορος του μηδενός: Έτσι
	\( z \neq 0 \) Άρα Π.Ο \( = \mathbb C - \left\lbrace (0,0) \right\rbrace \)

    \[
    g(z) = \frac{z}{z^2+2}
    \]
    \subparagraph{Σημείωση} Η \( g \) είναι \textbf{ρητή} συνάρτηση
    (δηλ. πηλίκο δύο (μιγαδικών) πολυωνύμων).

    Κάθε συνάρτηση της μορφής \(
    a_0+a_1z+\dots+a_nz^n,\ a_0,\dots,a_n \in \mathbb Z
     \) καλείται (μιγαδικό) πολυώνυμο.

    Πρέπει παρον. \( \neq0 \) δηλ:
    \begin{gather*}
    z^2+2=0\
        \left(
        \begin{array}{l}
        \text{\textbf{ΠΡΟΣΟΧΗ!!} Κάθε μιγαδικό} \\
        \text{πολυώνυμο βαθμού $N$ έχει} \\
        \text{ΑΚΡΙΒΩΣ $N$ ρίζες στο $\mathbb C$}
        \end{array}
        \right)
     \\
    z^2+2 = 0 \xRightarrow{i^2=-1} z^2-2i^2=0 \\
    \implies \left( z-\sqrt{2}i \right)\left(z+\sqrt{2}i \right)=0
    \\ \implies \boxed{z = \pm \sqrt{2}i}
    \end{gather*}

    \subparagraph{Τελικά} Π.Ο = \( \mathbb C -
    \left\lbrace \pm \sqrt{2}i \right\rbrace
     \)

    \paragraph{}
    \[ \boxed{
    	h(z) = \arg z,\ \text{Π.Ο} = \mathbb C - \left\lbrace 0 \right\rbrace
    } \]

    Για \( z=0 \) ΔΕΝ ορίζεται όρισμα, επειδή \( 0 = |0|\cdot e^{i\theta}
    \ \forall \theta
     \)

  \paragraph{Σημείωση}
  \( az^2+bz+c = 0 \) \\ \(\qquad a,b,c \in \mathbb C \)

  Λύνεται με διακρίνουσα κατά τα γνωστά.

  Επίσης μπορείτε να χρησιμοποιήσετε και σχήμα Horner για πολυώνυμα
  (με πραγματικούς συντελεστές) βαθμού \( N \geq 3 \).

 \paragraph{}
 \begin{align*}
 a(z) &=e^z = e^{x+iy} = e^x\cdot e^{iy} \\
 &= e^x (\cos y + i \sin y) \\
 &= \left( e^x\cos y,\ e^x\sin y \right),\quad x,y\in\mathbb R
 \end{align*}
 Ως διανυσματικό πεδίο προφανώς Π.Ο = \( \mathbb R ^2 \)

 Έτσι Π.Ο = \( \mathbb C \).

   \paragraph{}
   \begin{gather*}
   l(z) = \mathrm{Log}\. z \text{ (αντίστροφη της } e^z \text{)} \\
   \underbrace{\mathrm{Log}\. z}_{\mathclap{\text{μιγαδικός λογάριθμος}}}
   \overset{\text{ορισμός}}{:=} \ln|z| +i\arg z \\
   \text{Π.Ο} = \mathbb C - \left\lbrace 0 \right\rbrace
   \end{gather*}

   \begin{tikzpicture}[scale=0.5 ]
   	\draw (-2,0) -- (2,0);
   	\draw (0,-1) -- (0,2);

   	\draw (-1,0) arc (180:0:1);
   	\filldraw[fill=white] (-1,0) circle(3pt) node[below] {$-3$};
   	\filldraw[fill=white] (1,0) circle(3pt);
   \end{tikzpicture}

   \begin{align*}
   \mathrm{Log}(3) &= \ln|-3| = i\arg(-3) \\ &= \ln3+i\pi
   \end{align*}

    \paragraph{}
    \[
    \lambda(z) = \sin z \overset{\text{ορισμός}}{:=} \frac{e^{iz}-e^{-iz}}{2i}
    \]
    \[
    \left(
    \begin{array}{ll}
    e^{i\theta} &=\cos\theta+i\sin\theta  \quad \theta\in (-\pi,\pi] \\
    e^{-i\theta} &= \cos\theta -i\sin\theta \\[0.3pt] \hline
    \sin\theta &= \frac{e^{i\theta}-e^{-i\theta}}{2i}
    \end{array}
    \right)
    \]
    Π.Ο = \( \mathbb C \)

    \begin{gather*}
    m(z) = \cos z \overset{\text{ορισμός}}{:=} \frac{e^{iz}+e^{-iz}}{2} \\
    \text{Π.Ο} = \mathbb C
    \end{gather*}

    Όλες οι γνωστές τριγωνομετρικές ταυτότητες ισχύουν στο \( \mathbb C \)
    όπως στο \( \mathbb R  \).

    \paragraph{}
    \begin{align*}
    h(z) = \sqrt[n]{z} :=
    \sqrt[n]{|z|} e^{i\frac{2k\pi+\arg z}{n}} \quad (k=0,1,\dots,n-1)
    \end{align*}
    (Η \( \sqrt[n]{a} \) ορίζεται ως το \textbf{σύνολο} όλων των λύσεων
    της εξίσωσης \( z^n=a,\quad a\in\mathbb C \) )
    \[
    \text{Π.Ο} = \mathbb C - \left\lbrace 0 \right\rbrace
    \]

    \subsection{Όριο/Συνέχεια\\μιγαδικών συναρτήσεων μιγαδικής μεταβλητής}
    \begin{defn*}{}
    	Έστω \( f(z)=f(x+iy)=u(x,y)+iv(x,y) \)
    	μιγ. συνάρτηση ορισμένη σε σύνολο \( A \subset \mathbb C,
    	\ z_0=x_0+iy_0 \) είναι σ.συσσ. του \( A \) και έστω \( a=a_0+ib_0 \).
    	Τότε

    	\begin{gather*}
    	\lim_{z\to z_0}f(z) = a \in \mathbb C \\
    	\qquad \Updownarrow \\
    	\begin{cases}
    	\lim\limits_{(x,y)\to(x_0,y_0)} u(x,y) = a_0 \\ \qquad \text{\textbf{ΚΑΙ}} \\
    	\lim\limits_{(x,y)\to(x_0,y_0)} v(x,y) = b_0
    	\end{cases}
    	\end{gather*}
    \end{defn*}
    	\textbf{Επίσης,} αν \( z_0\in A \), τότε

    	\( f \) συνεχής στο σημείο \( z_0 \)
    	\[ \Updownarrow \]

    	οι συναρτήσεις \( u,v:A \subset \mathbb R^2\to\mathbb R  \)
    	είναι ΣΥΝΕΧΕΙΣ στο σημείο \( (x_0,y_0 \) (ως πραγματικές συναρτήσεις
    	δύο μεταβλητών)

    \subparagraph{Έτσι:}
    \begin{align*}
    \left.
    \begin{array}{l}
    \text{οι πολυωνυμικές}\\
    \text{η εκθετική}\\
    \text{οι τριγωνομετρικές }(\sin z,\cos z)\\
    \text{οι υπερβολικές }(\mathrm{ch}\.z,\mathrm{sh}\.z)
    \end{array}
    \right\rbrace &\ \text{συνεχείς στο } \mathbb C
    \\
    \left.
    \begin{array}{l}
    \text{οι ρητές}\\
    \text{οι τριγωνομετρικές }(\tan z,\cot z)\\
    \end{array}
    \right\rbrace &\ \text{συνεχείς στο \textbf{πεδίο ορισμού τους}}
    \end{align*}

    Ορίζω το \( \infty \) του μιγαδικού επιπέδου να είναι το σύνολο
    σημείων που απέχουν "άπειρη" απόσταση από την αρχή των αξόνων.

    Το επεκτεταμένο μιγαδικό επίπεδο ορίζεται ως:
    \[
    \overline{\mathbb C} = \mathbb C \cup \left\lbrace \infty \right\rbrace,\text{ όπου:}
    \]
    \begin{align*}
    \infty+z &= \infty \quad \forall z \in \mathbb C \\
    \infty\cdot z &= \infty \quad \forall z \neq 0 \\
    \frac{z}{\infty} &= 0 \quad \forall z \neq \infty
    \end{align*}

    Όλες οι πράξεις του ορίου που ξέρετε ισχύουν και στους μιγαδικούς
    (αρκεί να μην εμφανίζονται οι γνωστές απροσδιόριστες μορφές):
    \[
    0\cdot\infty,\frac{\infty}{\infty},0^0,1^{\infty},\infty^0
    \]

    Ο κανόνας De l' Hospital ισχύει στους μιγαδικούς.

    \paragraph{Σημείωση:}
    \begin{gather*}
    \lim_{z\to \infty}f(z) = a \in \mathbb C \iff
    \lim_{z\to0}f\left(\frac{1}{z}\right) = a\in\mathbb C  \\
    \lim_{z\to z_0}f(z) = \infty \iff \lim_{z\to z_0}\frac{1}{f(z)} = 0\\
    \lim_{z\to z_0}f(z) = 0 \iff \lim_{z\to z_0} \left|f(z)\right|=0
    \end{gather*}

    \begin{theorem*}[sidebyside,width=\textwidth]{}
    	Έστω \( \arg z:\mathbb C - \left\lbrace 0 \right\rbrace
    	\to (-\pi,\pi]
    	 \)

    	Τότε η \( \arg z \) \textbf{είναι συνεχής} στο σύνολο:
    	\[
    	\mathbb C^* = \mathbb C -
    	\left\lbrace
    	    x+iy: x \leq 0 \text{ ΚΑΙ } y = 0
    	 \right\rbrace
    	\]
    	\tcblower
    	\begin{tikzpicture}
	    	\fill[inner color=green!50!black,outer color=green!5] (-3,-2) rectangle (3,2);

    		\draw[->] (-3,0) -- (3,0) node[right] {$x$};
    		\draw[->] (0,-2) -- (0,2) node[above] {$y$};

    		\draw[line width=1mm, red!80!green] (-3.1,0) -- (0,0);
    		\filldraw[red!80!green,fill=white] (0,0) circle (4pt);
    	\end{tikzpicture}
    \end{theorem*}

    Έστω \( z = x+iy \)

    \begin{tikzpicture}
    	\draw (-2,0) -- (2,0);
    	\draw (0,-2) -- (0,2);

    	\draw(0,0) -- (1.5,1.5);
    	\draw (0.4,0) arc (0:45:.4) node[midway,right] {$\theta$};

    	\draw[dashed] (0,1.5) node[left] {$y$} -- (1.5,1.5) -- (1.5,0) node[below] {$x$};

    	\draw (current bounding box.north) node[above left] {(α) {$x>0,\ y>0$}};
    \end{tikzpicture}

    \begin{tikzpicture}
    \draw (-2,0) -- (2,0);
    \draw (0,-2) -- (0,2);

    \draw(0,0) -- (1.5,1.5) node[above right] {$(-x,y)$};
    \draw(0,0) -- (-1.5,1.5);
    \draw[dashed] (0,1.5) -- (-1.5,1.5) -- (-1.5,0);
    \draw (0.4,0) arc (0:135:.4);
    \draw (0.7,0) arc (0:45:.7) node[midway,right] {$\phi$};

    \draw[dashed] (0,1.5) -- (1.5,1.5) -- (1.5,0);

    \draw (current bounding box.north) node[above left] {(β) {$x<0,\ y>0$}};
    \end{tikzpicture}

    \begin{tikzpicture}
    \draw (-2,0) -- (2,0);
    \draw (0,-2) -- (0,2);

    \draw(0,0) -- (1.5,1.5) node[above right] {$(-x,-y)$};
    \draw[dashed] (0,0) -- (-1.5,0) -- (-1.5,-1.5) -- (0,-1.5);
    \draw (0,0) -- (-1.5,-1.5);
    \draw (-0.4,0) arc (180:225:.4) node[midway,left,yshift=-1mm] {$\mathsmaller{\theta_a}$};
    \draw (0.7,0) arc (0:45:.7) node[midway,right] {$\theta_a$};
    \draw (0.5,0) arc(360:225:.5) node[midway,below right] {$-\pi+\theta_a$};

    \draw[dashed] (0,1.5) -- (1.5,1.5) -- (1.5,0);

    \draw (current bounding box.north) node[above left] {(γ) {$x<0,\ y<0$}};
    \end{tikzpicture}

        \begin{tikzpicture}
        \draw (-2,0) -- (2,0);
        \draw (0,-2) -- (0,2);

        \draw(0,0) --(1.5,-1.5);
        \draw[dashed] (1.5,0) -- (1.5,-1.5) -- (0,-1.5);
        \draw[->] (0.7,0) arc (360:315:.7);

        \draw (current bounding box.north) node[above left] {(δ) {$x>0,\ y<0$}};
        \end{tikzpicture}


    \[
    \arg z = \begin{cases}
    \arctan\left|\frac{y}{x}\right|, \qquad & x,y>0 \\
    \pi - \arctan\left|\frac{y}{x}\right|, \qquad & x<0,\ y>0 \\
    -\pi + \arctan\left|\frac{y}{x}\right|, \qquad & x<0,\ y<0 \\
    -\arctan\left|\frac{y}{x}\right|, \qquad & x>0,\ y<0
    \end{cases}
    \]

    Για \(
    \begin{array}{ll}
    x=0,\ & \text{τότε } \arg := \frac{\pi}{2} \text{ ή } -\frac{\pi}{2}\\
    y=0,\ & \text{τότε } \arg := 0\text{ ή }\pi
    \end{array}
     \)

    Έστω \( z_0 = x_0 < 0 \)
    \begin{itemize}
    	\item Έστω \( z = x_0+it \quad (t>0) \)

    	Για \( t\to0^+,\ z\to z_0=x_0 \), αλλά:
    	\[
    	\lim_{z\to z_0}\arg z \overset{z=x_0+it}{=}
    	\lim_{t\to0^+} \arg(x_0+it) \overset{\text{2ο τετ.}}{=}
    	\lim_{t\to0^+}\left(\pi-\arctan\left|\frac{t}{x_0}\right|\right)
    	=\pi-\arctan0=\pi
    	\]
    	\item Για \( z=x_0+it \quad (t<0) \), τότε:
    	\[
    	t\to0^-,\quad z\to z_0,\text{ και}
    	\]
    	\[
    	\lim_{z\to z_0}\arg z = \lim_{t\to0^-} \arg(x_0+it)
    	\overset{\text{3ο τετ.}}{=} -\pi+\arctan0 = -\pi
    	\]
    \end{itemize}
    Άρα το όριο στο \( z_0=x_0 \) ΔΕΝ υπάρχει, και έτσι η \( \arg z \)
    ασυνεχής στα \( z=x_0 \) με \( x_0\leq 0 \).

    Αν \( \arg z \in [0,2\pi) \) πού είναι ασυνεχής;
    
    \subsection{Μιγαδική παράγωγος}
    Την εβδομάδα της 28\textsuperscript{ης} θα γίνουν κανονικά τα μαθήματα του
    Ατρέα.
    \begin{defn*}{}
       	Έστω \( f:A \subset \mathbb C \to \mathbb C  \), \( A \) ανοικτό,
       	\( z_0 \in A \). Λέμε ότι η \( f \) είναι μιγαδικά παραγωγίσιμη στο σημείο
       	\( z_0 \), αν υπάρχει το ΟΡΙΟ:
       	\[
       	\lim_{z\to z_0}
       	\frac{f(z)-f(z_0)}{z-z_0} = a \in \mathbb C 
       	\]
       	(ή ισοδύναμα \( \lim_{h\to0}\frac{f(z_0+h)-f(z_0)}{h}=a\in\mathbb C \))
       	
       	Στο εξής το όριο αυτό συμβολίζουμε με \( f'(z_0) \) ή
       	\( \od{f(z_0)}{z} \)
    \end{defn*}
    \begin{defn*}{}
       	Αν \( f:A\in\mathbb C\to\mathbb C \), \( A \) ανοικτό, \( z_0\in A \),
       	θα λέμε στο εξής ότι η \( f \) είναι ΟΛΟΜΟΡΦΗ (ή ΑΝΑΛΥΤΙΚΗ -
       	holomorphic/analytic)
       	\textbf{στο σημείο \( \mathbf{z_0} \)}, εάν η \( f \) είναι μιγαδικά
       	παραγωγίσιμη \textbf{ΣΕ ΚΑΘΕ ΣΗΜΕΙΟ} του ανοικτού δίσκου
       	\begin{tikzpicture}[scale=0.4]
       	\filldraw[dashed,fill=green!30] (0,0) circle(1);
       	\draw (0,0) -- ++(135:1) node[midway,above,sloped] {$\epsilon$};
       	\filldraw (0,0) circle(1pt) node[above right] {$z_0$};
       	\end{tikzpicture}
       	\[
       	D_\epsilon(z_0) = \left\lbrace 
       	z\in\mathbb C: |z-z_0|<\epsilon
       	\right\rbrace
       	\]
       	για κάποιο \( \epsilon>0 \)
    \end{defn*}
    
    Αν \( f \) ολόμορφη σε ΚΑΘΕ σημείο του \( A \) λέμε ότι η \( f \) ολόμορφη στο
    \( A \).
    
    \begin{defn*}{}
       	Αν \( A \) μη ανοικτό, λέμε ότι η \( f \) ολόμορφη στο \( A \), αν
       	υπάρχει \( B \supset A \), \( B \) ανοικτό ώστε η \( f \) στο \( B \).
    \end{defn*}
    
    \paragraph{}
    Όλες οι γνωστές ιδιότητες της παραγώγου που γνωρίζετε ισχύουν και για τη
    μιγαδική παράγωγο
    \subparagraph{π.χ.}
    Έστω \( f,g \) \textbf{μιγαδικά} παραγωγίσιμες σε σημείο \( z_0 \). Τότε:
    \begin{itemize}
       	\item \( f \) παραγ. στο \( z_0 \implies f \) συνεχής στο \( z_0 \)
       	\item \( \big(af\pm by\big)'(z_0)=
       	af'(z_0)+bg'(z_0)\ \forall a,b\in\mathbb C  \)
       	\item \( \big(fg\big)'(z_0) = f'(z_0)g(z_0)+f(z_0)g'(z_0) \)
       	\item \(  \left(
       	\frac{f}{g} \right)(z_0)= \frac{f'(z_0)g(z_0)-f(z_0)g'(z_0)}{g^2(z_0)}
       	\quad \left(g(z_0)\neq0\right)
       	\)
       	\item Ο κανόνας αλυσίδας ισχύει στις μιγαδικές συναρτήσεις:
       	\[
       	\big( h\circ g \big)'(z_0)=h'\left( g(z_0) \right)g'(z_0)
       	\]
       	υπό την προϋπόθεση ότι η σύνθεση καλά ορισμένη
    \end{itemize}
    
    \paragraph{Παραγώγιση αντίστροφης συνάρτησης} %TODO toc?
    Έστω \( f \) ολόμορφη σε σημείο \( z_0 \) με \( f'(z_0)\neq 0 \).
    
    Αν \( w_0=f(z_0) \), τότε υπάρχουν \( \epsilon,\epsilon' >0 \) ώστε η
    αντίστροφη συνάρτηση \( f^{-1}:\mathrm D_\epsilon(w_0)
    \to\mathrm D_{\epsilon'}(z_0)
    \) καλά ορισμένη, ολόμορφη στο \( w_0 \) και
    \[
    \mathlarger{
       	\left( f^{-1} \right)'(w_0) = \frac{1}{f'(z_0)}
    }
    \]
    
    \paragraph{}
    \begin{theorem*}[width=.7\textwidth]{Εξισώσεις Cauchy-Riemann}
       	\vspace{15pt}
       	Έστω \( f:A\subseteq\mathbb C \to\mathbb C:f(z)=f(x+iy)  
       	= u(x+y)+iv(x,y)
       	\). Θεωρώ \( z=x+iy,\ z_0=x_0+iy_0 \) και \( A \) ανοικτό.
       	
       	Τότε:
       	
       	\( f \) μιγαδικά παραγωγίσιμη στο \( z_0 \)
       	\[
       	\hfill \Updownarrow \hfill
       	\]
       	\begin{enumlatin}
       		\item Η \( \mathbf F(x,y) = \left(
       		u(x,y),\ v(x,y)
       		\right) \) είναι \textbf{διαφορίσιμο} διανυσμ. πεδίο στο σημείο
       		\( (x_0,y_0) \)
       		\\
       		\[
       		\hfill \boxed{\text{ΚΑΙ}} \hfill
       		\]
       		\item \[\begin{cases}
       		u_x(x_0,y_0) = v_y(x_0,y_0) \\
       		u_y(x_0,y_0) = -v_x(x_0,y_0)
       		\end{cases} \xleftarrow{ \displaystyle \text{εξισώσεις C-R}}
       		\]
       	\end{enumlatin}
       	
    \end{theorem*}
    
    \paragraph{Πόρισμα (ΠΡΑΚΤΙΚΟΤΑΤΟ)}
    Αν \( f(z)=f(x+iy)=u(x,y)+iv(x,y) \) είναι έτσι ώστε:
    \begin{enumgreekparen}
       	\item \( u,v \) έχουν συνεχείς μερικές παραγώγους στο \( (x_0,y_0) \)
       	και "κοντά" στο \( (x_0,y_0) \)
       	\item \( \begin{cases}
       	u_x(x_0,y_0) = v_y(x_0,y_0) \\
       	u_y(x_0,y_0) = -v_x(x_0,y_0)
       	\end{cases} \xleftarrow{\displaystyle\text{C-R}} \)
    \end{enumgreekparen}
    
    Τότε \( (\implies) \) η \( f \) είναι μιγαδικά παραγωγίσιμη στο \( z_0=x_0+iy_0 \)
    
    \subparagraph{Παρ.}
    \begin{align*}
    z^2 &= (x+iy)^2=x^2+2ixy-y^2 =
    \\  &= x^2-y^2+i(2xy),\ \text{άρα}\\
    f &= (x^2-y^2,2xy) \quad \left| \begin{array}{l}
    u_x=v_y\\ u_y = -v_x
    \end{array} \right.
    \end{align*}
    
    \paragraph{Παρατηρήσεις}
    \begin{enumgreekparen}
       	\item
       	Έστω \( f \) μιγαδικά παραγ. συνάρτηση σε σημείο \( z_0=x_0+iy_0 \). Τότε
       	εξ' ορισμού υπάρχει το όριο
       	\[
       	f'(z_0)=\lim_{z\to z_0}\frac{f(z)-f(z_0)}{z-z_0}
       	\]
       	%TODO Atreas Graph 06
       	
       	\begin{itemize}
       		\item Έστω \( z=x+iy_0\quad (x\in\mathbb R ) \) είναι τυχαίο σημείο της
       		"οριζόντιας" ευθείας που διέρχεται από το \( z_0 \)
       		\item Για \( x\to x_0 \), τότε \( z=x+iy_0 \to x_0+iy_0=z_0 \)
       		(δηλ. \( z\to z_0 \) όταν \( x\to x_0 \) πάνω στην οριζόντια ευθεία)
       	\end{itemize}
       	
       	
       	Τότε για \( z=x+iy_0 \) έχω:
       	\begin{align*}
       	f'(z_0) &= \lim_{x\to x_0}
       	\frac{u(x,y_0)+iv(x,y_0)-\left(
       		u(x_0,y_0)+iv(x_0,y_0)
       		\right)}{x+iy_0-(x_0+iy_0)}
       	\\ &= \lim_{x\to x_0}\frac{
       		u(x,y_0)-u(x_0,y_0)
       	}{x-x_0}+i\lim_{x\to x_0}\frac{v(x,y_0)-v(x_0,y_0)}{x-x_0}
       	\\ &= \mathlarger{u_x(x_0,y_0)+iv_x(x_0,y_0)}
       	\\ &\implies \boxed{
       		\mathlarger{
       			f'(z_0) = u_x(x_0,y_0)+iv_x(x_0,y_0)
       		}
       	} := \pd{f(x_0,y_0)}{x}
       	\end{align*}
       	
       	Με όμοιο τρόπο, αν εργαστούμε κατά μήκος της "κάθετης" ευθείας που διέρχεται
       	από το \( z_0 \), έχουμε:
       	\[
       	\boxed{\mathlarger{
       			f'(z_0) = v_y(x_0,y_0)-iu_y(x_0,y_0)
       		}} := -i\pd{f(x_0,y_0)}{y}
       		\]
       		
       		\item Γεωμετρική ερμηνεία της παραγώγου
       		\begin{align*}
       		& f'(z_0)=\frac{\dif f(z_0)}{\dif z}
       		\\ \implies & \boxed{\dif f(z_0)=f'(z_0)\dif z}
       		\end{align*}
       		%TODO Atreas Graph 07
       		\[
       		\dif z := \begin{array}{l}
       		\text{στοιχειώδης όγκος} \\
       		\text{στο επίπεδο } xy
       		\end{array}
       		\]
       		\[
       		\dif f(z_0):= \begin{array}{l}
       		\text{στοιχειώδες χωρίο στο επίπεδο } uv \\
       		\text{στο οποίο μετασχηματίζεται το } \dif z\\
       		\text{μέσω της απεικόνισης} f
       		\end{array}
       		\]
       		
       		\begin{gather*}
       			\dif f(z_0) = \left|
       			f'(z_0)\right|e^{i\arg f'(z_0)}\dif z\quad
       			\mathsmaller{\left( f'(z_0)\neq 0 \right)}
       		\end{gather*}
       		
      	\end{enumgreekparen}


	\paragraph{}
	Για τις παραγώγους στοιχειωδών συναρτήσεων ισχύουν τα συνήθη από την πραγματική
	ανάλυση.
	\subparagraph{π.χ}
	Αν \( f(z)=e^z \), τότε \( (e^z)'=e^z\ \forall z\in\mathbb C  \)
	
	\begin{align*}
		f(z)= e^z &=e^{x+iy} = e^xe^{iy} = e^x(\cos y+\sin y)
		\\ &= \underbrace{e^x\cos y}_{\mathclap{u(x,y)}}
		+ i \underbrace{(e^x\sin y)}_{\mathclap{v(x,y)}}
	\end{align*}
	
	Ορίζω \( \begin{cases}
	u(x,y) = \Re(e^z) = e^x\cos y \\
	v(x,y) = \Im(e^z) = e^x\sin y
	\end{cases} \)
	
	\begin{itemize}
		\item \( u,v \) καλά ορισμένες \( \forall (x,y)\in\mathbb R^2 \), και επιπλέον
		\( u,v \) είναι \textbf{ΣΥΝΕΧΕΙΣ} \( \forall (x,y)\in\mathbb R ^2 \)
		\item \(\begin{matrix}
			u_x=e^x\cos y& u_y=-e^x\sin y\\
			v_x=e^x\sin y& v_y=e^x\cos y
		\end{matrix}\), έτσι παρατηρώ ότι \( 
		\begin{cases}
		& u_x = v_y \\ \text{ΚΑΙ } & u_y=-v_x
		\end{cases} \forall (x,y)\in\mathbb R ^2
		 \)
	\end{itemize}
	\( \xRightarrow{\text{πόρισμα}} f(z)=e^z \) μιγαδικά παραγωγίσιμη \( \forall z\in\mathbb C  \)
	\begin{itemize}
		\item Γνωρίζω ότι αν η \( f=u+iv \) είναι μιγ. παραγ., τότε \( f'(z)=u_x+iv_x \).
		
		\textbf{Έτσι} στην προκειμένη περίπτωση:
		\begin{align*}
		f'(z)=\left( e^z \right)'=u_x+iv_x=e^x\cos y+ie^x\sin y =
		e^x(\cos y+i\sin y)=e^xe^{iy}=e^z
		\end{align*}
	\end{itemize}
	
	\subparagraph{π.χ}
	\( \mathrm{Log}z=\frac{1}{z}\ \forall z \in \mathbb C^* = \mathbb C -
	\left\lbrace x+iy: x\leq0 \text{ και } y=0 \right\rbrace
	\Big(
	\text{υπό την προϋπόθεση ότι } \arg z \in (-\pi,\pi]\
	\Big)
	 \) \\ διότι
	 \( \mathrm{Log} z = w \xLeftrightarrow{\text{ορ.}} z=e^w \),
	 άρα \( \forall z \in\mathbb C ^* \), από το θεώρ. παραγώγισης αντίστροφης
	 συνάρτησης έχουμε: \( (\mathrm{Log}z)' = \frac{1}{e^w}=\frac{1}{z} \)
	 
	 Με την ίδια λογική (και με χρήση των ιδιοτήτων παραγώγου) αποδεικνύεται ότι
	 \begin{itemize}
	 	\item \( (z^n)' =nz^{n-1}\quad \forall n\in\mathbb N \quad \forall z\in\mathbb C \)
	 	\item \( 
	 	(z^{-n})' = -nz^{-n-1}\quad\forall n\in\mathbb N \quad\forall z\in
	 	\mathbb C - \left\lbrace 0 \right\rbrace
	 	 \)
	    \item \( 
	    (z^a)'=az^{a-1}\quad\forall a\in\mathbb Q \text{
	    	ή $a$ άρρητος ή $a$ έχει μη μηδενικό φανταστικό μέρος
	    	}\quad\forall z\in\mathbb C^* (\mathbb C^* \text{ 
	    	όπως στο λογάριθμο πριν
	    	})
	     \)
	    \item \( (\sin z)'=\cos z\quad\forall z\in\mathbb C  \)
	    \item \( (\cos z)'=-\sin z \quad\forall z\in\mathbb C  \)
	    \item \( (\sinh z)' = \cosh z\quad\forall z\in\mathbb C \)
	    \item \( (\cosh z)' = \sinh z\quad\forall z\in\mathbb C \)
	    \item \( (a^z)'=a^z\mathrm{Log}a\quad\forall z\int\mathbb C \)
	 \end{itemize}
	 κλπ.
	 
	 \subsection{Ασκήσεις}
	 \paragraph{} ΝΔΟ
	 η \( f(z)=\bar z \) ΔΕΝ είναι μιγαδικά παραγωγίσιμη \textbf{σε κανένα}
	 σημείο του \( \mathbb C \).
	 
	 \begin{itemize}
	 	\item \( \bar z = \overline{x+iy}=x-iy \), ορίζω
	 	\( \left| \begin{array}{l}
	 	u(x,y) = x \\ v(x,y) = -y
	 	\end{array} \right. \)
	 	\item Προφανώς \( u \) και \( v \) καλά ορισμένες και συνεχείς
	 	\( \forall (x,y)\in\mathbb R^2 \), αλλά:
	 	\[
	 	u_x=1\neq -1 = v_y
	 	\]
	 	\( \forall(x,y)\in \mathbb R \), άρα αφού η μία από τις δύο εξισ.
	 	C-R δεν ισχύει \underline{\( \forall(x,y)\in\mathbb R^2 \)},
	 	η \( f(z)=\bar z \) \textbf{ΔΕΝ} είναι μιγαδικά παραγ.
	 	\( \forall z\in\mathbb C \).
	 \end{itemize}
	 
	 %TODO Atreas Graph 08
	 %TODO Atreas Graph 09
	 
	 \paragraph{}
	 \begin{gather*}
	 f(z)=e^z=e^x\cos y+ie^x\sin y \\
	 \left|
	 \begin{array}{l}
	 u = e^x\cos y_0 \\ v=e^x\sin y_0
	 \end{array}
	 \right.
	 \end{gather*}
	 
	 %TODO Atreas Graph 10
	 
	 \paragraph{Άσκ. 2}
	 Η συνάρτηση \( f(z)=|z| \) \textbf{ΔΕΝ} είναι μιγαδικά παραγωγίσιμη σε
	 \textbf{ΚΑΝΕΝΑ} σημείο του \( \mathbb C  \).
	 
	 \begin{infobox}{}
	 	Οι εξισώσεις C-R σε πολικές συντ/νες είναι οι εξής:
	 	\[
	 	\begin{cases}
	 	u_\rho = \frac{1}{\rho}v_\theta \quad \forall \rho>0,
	 	         \theta\in(-\pi,\pi]\\
	 	u_\theta=-\rho v_\rho
	 	\end{cases}
	 	\]
	 	\tcblower
	 	\begin{align*}
	 	f(z) &= f(x+iy) \\
	 	     &= f\left( |z|e^{i\arg z} \right) = f\left(\rho e^{i\theta}\right)
	 	     = u(\rho,\theta)+iv(\rho,\theta)
	 	\end{align*}
	 \end{infobox}
	 
	 \( f(z)=|z|=\rho \), άρα \( \begin{cases}
	 u(\rho,\theta)=\rho \\ v(\rho,\theta)=0
	 \end{cases} \)
	 
	 Οι \( u,v \) καλά ορισμένες και συνεχείς \( \forall \rho>0,
	 \theta\in (-\pi,\pi]
	  \) αλλά \[
	  u_\rho = 1 \neq \frac{1}{\rho}\cdot 0 =\frac{1}{\rho}v_\theta
	  \quad \forall \rho>0,\theta\in(-\pi,\pi]
	  \]
	  και αφού μία από τις εξισώσεις C-R δεν ισχύει \( \forall \rho>0,
	  \theta\in(-\pi,\pi]
	   \) αναγκαστικά η \( f(z)=|z| \) δεν είναι μιγαδικά παραγ. σε κανένα
	   σημείο του \( \mathbb C  \).
	\subparagraph{π.χ}
	\begin{align*}
	f(z) &= \frac{\bar z}{|z|^2} \quad z\neq0
	\\ &\overset{|z|^2=z\bar z}{=} \frac{\bar z}{z\bar z}=\frac{1}{z}
	\end{align*}
	άρα η \( f \) είναι παραγωγίσιμη.
	
	\paragraph{Άσκ. 3} Υπολογίστε τα όρια:
	\begin{enumgreekparen}
		\item \( 
		\displaystyle \lim_{z\to0} \frac{e^{z^2}-1}{z^2}
		 \)
		\item \( 
		\displaystyle \lim_{z\to1} \frac{z^2-1}{\bar z^2-1}
		 \)
		\item \( 
		\displaystyle \lim_{z\to \infty} e^z
		 \)
	\end{enumgreekparen}
	\begin{infobox}{}
		Στα όρια ισχύει ο De L' Hospital
	\end{infobox}
	\subparagraph{}\begin{enumgreekparen}
		\item
		\begin{align*}
		\lim_{z\to0} \frac{e^{z^2}-1}{z^2} 
		&\underbrace{\overset{\left(\frac{0}{0}\right)}
			{\underset{\text{L'Hospital}}{=}}}_{
			\mathclap{\text{διότι $e^{z^2}-1$ και $z^2$ μιγ. παραγ.}}}
		\lim_{z\to 0} \frac{2ze^{z^2}-0}{2z} = \lim_{z\to0}e^{z^2}=e^0=1
		\end{align*}
		
        \item
        Θα προσπαθήσω να αποδείξω ότι το όριο δεν υπάρχει, κάτι που φαντάζομαι
        επειδή μέσα στο όριο υπάρχει ο \( \bar z \).
        %TODO Atreas Graph 11
        \begin{itemize}
        	\item Θεωρώ την "κίνηση κατά μήκος του οριζόντιου άξονα" που διέρχεται
        	      από το \( z_0 =1 \).
        	      \\
        	      \textbf{Δηλ. } θεωρώ σημεία \( z \) της μορφής \[ z=x+i0 \quad
        	      (x\in\mathbb R )
        	      \]
        	      
        	      Προφανώς για \( x\to 1 \), έχω: \( z\to z_0=1 \).\\
        	      Τότε \( \forall z=x \) έχω:
        	      \begin{align*}
        	      \lim_{z\to 1}\frac{z^2-1}{\bar z^2-1}
        	      \overset{\text{κατα μήκος}}{\underset{\text{του οριζ. άξονα}}{=}}
        	      \lim_{x\to 1}\frac{x^2-1}{x^2-1}=1
        	      \end{align*}
            \item Θεωρώ την "κίνηση κατά μήκος του κάθετου άξονα" που διέρχεται
                  από το \( z_0=1 \), δηλαδή σημεία:
                  \[
                  z=1+ix\quad (x\in\mathbb R )
                  \]
                  
                  Προφανώς για \( x\to0 \), έχω \( z\to z_0=1 \), και
                  \begin{align*}
                  \lim_{z\to1} \frac{z^2-1}{\bar z^2-1}
                  &\overset{\text{κατα μήκος}}{%
                  	\underset{\text{του κατακόρυφου άξονα}}{=}}
                  \lim_{x\to 0}\frac{(1+ix)^2-1}{(1-ix)^2-1}
                  =\lim_{x\to0}
                  \frac{\cancel{1}+2ix-x^2-\cancel{1}}{\cancel{1}-2ix-x^2-\cancel{1}}
                  \\ &= \lim_{x\to0} \frac{2ix-x^2}{-2ix-x^2}
                  =\lim_{x\to 0}\frac{2i-x}{-2i-x}=\frac{2i}{-2i}=-1
                  \end{align*}
        \end{itemize}
        Εφόσον \( 1\neq -1 \) το όριο ΔΕΝ υπάρχει.
        
        \item \( \lim\limits_{x\to\infty}e^x=? \)
        \begin{itemize}
        	\item Έστω \( z=x \quad (x<0) \), για \( x\to -\infty \), τότε
        	\( z\to \infty \) και \( \lim\limits_{z\to \infty} e^z
        	=\lim\limits_{x\to-\infty}e^x=0
        	 \)
        	\item Έστω \( z=x \quad (x>0) \), για \( x\to +\infty \), τότε
        	\( z\to \infty \), αλλά: \( \lim\limits_{z\to \infty} e^z
        	=\lim\limits_{x\to+\infty}e^x=+\infty
        	\), συνεπώς το \( \lim\limits_{z\to \infty}e^z \) ΔΕΝ υπάρχει.
        \end{itemize}
	\end{enumgreekparen}
	
	\paragraph{Άσκ. 4}
	Αν \( f(z)=u+iv \) είναι ακεραία (ολόμορφη στο \( \mathbb C \)) και αν
	\[
	au+bv =c
	\] όπου \( a,b,c\in\mathbb R \) σταθερές όχι όλες ίσες με μηδέν, ΝΔΟ
	\( f(z)=A, \ A\in\mathbb C \) σταθερά.
	
	\begin{itemize}
		\item Έστω \underline{\( c=0 \)}, εξ' υποθέσεως \( a^2+b^2\neq 0 \)
		\item Έστω \( c\neq 0 \), πάλι πρέπει \( a^2+b^2\neq 0 \)
		      (διότι αλλιώς \( 0=c \), άτοπο)
		\item Τελικά \( a^2+b^2\neq 0 \) σε κάθε περίπτωση.
	\end{itemize}


    \begin{align*}
    \begin{cases}
    au_x+bv_x = 0 \\
    au_y+bv_y = 0
    \end{cases} &\implies \left[
    \begin{matrix}
    u_x & v_x \\ u_y & v_y
    \end{matrix}
    \right]\left[\begin{matrix}
    a \\ b
    \end{matrix}\right] = \left[\begin{matrix}
    0 \\ 0
    \end{matrix}\right]
    \\ &\xRightarrow[u_y=-v_x\text{ αφού $f$ ακεραία}]{u_x=v_y}
    \left[\begin{matrix}
    u_x&-u_y\\ u_y& u_x
    \end{matrix}\right]\left[
    \begin{matrix}
    a\\b
    \end{matrix}
    \right]=\left[\begin{matrix}
    0 \\ 0
    \end{matrix}\right]
    \end{align*}
    \begin{gather*}
    \left|\begin{matrix}
    u_x & -u_y \\ u_y & u_y
    \end{matrix}\right| = u_x^2+u_y^2
    \end{gather*}
    και επειδή \( a^+b^2\neq 0 \), πρέπει \( u_x^2+u_y^2=0 \) για να έχει λύση
    το σύστημα \( \implies u_x=0 \) και \( u_y=0 
    \xRightarrow{\text{C-R}} u_x=u_y=v_x=v_y=0 \ \forall (x,y)\in\mathbb R ^2
    \implies f(z) = A\in\mathbb C
    \) σταθερά.
    
	\paragraph{Άσκ.}    
	Βρείτε τα σημεία ολομορφίας των συναρτήσεων:
	\begin{enumgreekparen}
		\item \( f(z)=\mathrm{Log}(z-i) \)
		\item \( g(z)=\tan z \)
	\end{enumgreekparen}
	\begin{enumgreekparen}
		\item
		Έστω ότι \( \arg{z}\in(-\pi,\pi] \). Τότε είναι γνωστό ότι η \( \mathrm{Log} z \)
		είναι μιγαδικά παραγ. στο \( \mathbb C ^*=\mathbb C-
		\left\lbrace x+iy\quad x\leq 0 \text{ και } y = 0 \right\rbrace
		 \).
		 
	    Έτσι η \( \mathrm{Log}(z-i) \) είναι μιγ. παραγ. στο σύνολο
	    \begin{align*}
	    &\mathbb C-\left\lbrace x+iy: \Re(z-i)\leq 0\text{ και }\Im(z-i)=1\right\rbrace
	    \\ &\overset{\mathclap{z=x+iy}}{=}
	    \mathbb C -\left\lbrace x+iy: x\leq 0 \text{ και }y-1=0 \right\rbrace
	    \\ &= \mathbb C - \left\lbrace 
	    x+iy: x\leq 0 \text{ και } y = 1
	     \right\rbrace
	    \end{align*}
	    %TODO Atreas Graph 12
	    
	    \item \( \tan z = \frac{\sin z}{\cos z} \), η \( g \) είναι ολόμορφη στο
	    \( \mathbb C  \) εκτός των σημείων που μηδενίζουν τον παρονομαστή.
	    
	    \begin{itemize}
	    	\item \( 
	    	\cos z = 0 \iff \cos(x+iy) = 0 \iff \cos x\cos(iy)-\sin x\sin(iy)=0
	    	\xLeftrightarrow{\text{ορ. } \sin \text{ \& } \cos}
	    	\cos x \cdot \frac{e^{-y}+e^y}{2}-\sin x\cdot\frac{e^{-y}-e^y}{2i}=0
	    	\iff \cos x\cdot\cosh y -i\sin x\cdot\sinh y = 0
	    	\iff \left|
	    	\begin{array}{l}
	    	\cos x \cdot \cosh y = 0 \\ \qquad \text{και} \\
	    	\sin x \cdot \sinh y = 0
	    	\end{array}\right. \iff \left|
	    	\begin{array}{l}
	    	\cos x = 0 \\ \qquad \text{και} \\ \sin x = 0 \\ \ \text{(Αδύνατο)}
	    	\end{array}\right. \text{ ή } \left|
	    	\begin{array}{l}
	    	\cos x = 0 \\ \qquad \text{και} \\ \sinh y = 0
	    	\end{array}
	    	\right. \iff \left|
	    	\begin{array}{l}
	    	x=k\pi + \frac{\pi}{2} \\ y=0
	    	\end{array}
	    	\right., k\in\mathbb Z
	    	 \).
	    	 
            %TODO Atreas Graph 13
	    	 
	    	\textbf{Τελικά} \( \cos z \iff \boxed{z=k\pi+\frac{\pi}{2},\ k
	    		\int\mathbb Z
	    		}  \) και έτσι \( g \) είναι ολόμορφη στο
	    		\[
         		\mathbb C - \left\lbrace k\pi+\frac{\pi}{2}:k\in\mathbb Z \right\rbrace
	    		\]
	    \end{itemize}

	    
	\end{enumgreekparen}
	
	\paragraph{Άσκ.}
	Έστω \( f(x+iy) = (x^2+2y)+i(x^2+y^2) \)
	\begin{enumroman}
		\item Να γραφεί η \( f \) συναρτήσει του \( z=x+iy \)
		\item Να βρείτε όλα τα σημεία, όπου η \( f \) είναι μιγαδικά παραγωγίσιμη
		\item Να βρείτε όλα τα σημεία στα οποία η \( f \) είναι ολόμορφη
	\end{enumroman}
	\begin{enumroman}
		\item \( x=\frac{z+\bar z}{2},\ y=\frac{z-\bar z }{2i} \) \\
		\( \big(z=x+iy\big) \)
		
		\begin{align*}
			f(z) &= \left(\frac{z+\bar z}{2}\right)^2+2\left(\frac{z-\bar z}{2i}\right)
			+i\left(\left(\frac{z+\bar z}{2}\right)^2+\left(\frac{z-\bar z}{2i}\right)^2\right)
			\\ &= \frac{z^2+2z\bar z+\bar z^2}{4}-
			i\left(z-\bar z\right)+i\left(\frac{z^2+2z\bar z+\bar z^2}{4}
			-\frac{z^2-2z\bar z+z^2}{4}\right)
			\\ &= \frac{z^2+2|z|^2+\bar z^2}{4}-i\left(
			z-\bar z-|z|^2
			\right)
		\end{align*}
		\item Προφανώς
		\( 
		\left|
		\begin{array}{l}
		\Re(f) := u(x,y) = x^2+2y \\
		\Im(f) := v(x,y) = x^2+y^2
		\end{array}
		 \right.
		 \)
		 \begin{itemize}
		 	\item Οι \( u \) και \( v \) είναι συνεχείς (ως πολυωνυμικές) \( \forall
		 	(x,y)\in\mathbb R^2
		 	 \)
		 	\item
		 	\( 
		 	\begin{cases}
		 	u_x=v_y \\ \quad\text{ΚΑΙ} \\ u_y=-v_x
		 	\end{cases} \implies \begin{cases}
		 	2x=2y\\ \quad\text{ΚΑΙ} \\ 2=-2x
		 	\end{cases} \implies \begin{cases}
		 	x=y \\ \quad \text{ΚΑΙ} \\ x = -1
		 	\end{cases} \iff \begin{cases}
		 	x=-1 \\ \quad\text{ΚΑΙ} \\ y=-1
		 	\end{cases}
		 	 \)
		 \end{itemize}
		 Άρα η \( f \) είναι μιγαδ. παραγ. \textbf{μόνον} στο \( \boxed{z=-1-i} \),
		 και μάλιστα εφ' όσον \( f(z)=f'(x+iy)=u_x+iv_x \):
		 \[
		 f'(-1-i) = 2(-1)+i2(-1) = \underline{-2-i2}
		 \]
		 \item ΔΕΝ υπάρχουν σημεία όπου η \( f \) είναι ολόμορφη.
	\end{enumroman}
	
	\section{Μιγαδική ολοκλήρωση}
	\paragraph{Εισαγωγή} \hspace{0pt}\\
	
	\begin{defn*}{}
        Καλούμε \textbf{καμπύλη} στο μιγαδικό επίπεδο κάθε \underline{συνεχή} συνάρτηση
        \[
        \gamma:[a,b]\to\mathbb C :\gamma(t)=x(t)=iy(t)
        \]
        όπου \( x,y:[a,b]\to\mathbb R  \) συνεχείς πραγματικές συναρτήσεις.
	\end{defn*}
	
	\subparagraph{Έτσι:} \( \gamma(t) \) καλείται
	\begin{invitemize}
		\item \textbf{ΑΠΛΗ} αν είναι 1-1 (δεν αυτοτέμνεται)
		\item \textbf{ΚΛΕΙΣΤΗ} αν έχει ίδια αρχή και πέρας
		\item \textbf{ΛΕΙΑ} αν είναι παραγωγίσιμη στο \( [a,b] \) με συνεχή παράγωγο
		\[
		\gamma'(t)=x'(t)+iy'(t)
		\]
		και μη μηδενική παράγωγο \( \forall t \)
	\end{invitemize}
	\begin{itemize}
		\item Κάθε τέτοια καμπύλη έχει ΠΡΟΣΑΝΑΤΟΛΙΣΜΟ (φορά διαγραφής) προς την
		κατεύθυνση αύξησης του \( t \)
		\subparagraph{π.χ.} \( \gamma(t)=e^{it},\ t\in(-\pi,\pi] \)
		%TODO Atreas Graph 14
		
		\( \gamma(t)=e^{-it},\ t\in(-\pi,\pi] \) %TODO Atreas Graph 15
		
		\item Αν \( \gamma \) κλειστή λέω ότι είναι \underline{θετικά}
		προσανατολισμένη αν η φορά διαγραφής είναι η αντιωρολογιακή
		
		\item \( -\gamma \): ίδιο ίχνος με τη \( \gamma \), αλλά
		αντίθετη φορά διαγραφής
		
		\item \( \gamma_1+\gamma_2 \): %TODO Atreas Graph 16
	\end{itemize}
	
	\begin{defn*}{}
		Έστω \( f=f(z) \) ΣΥΝΕΧΗΣ μιγαδική συνάρτηση μιγαδικής μεταβλητής και
		\( \gamma:[a,b]\to\mathbb C  \) λεία καμπύλη. Καλώ επικαμπύλιο ολοκλήρωμα
		της \( f \) ΠΑΝΩ στη \( \gamma \) να είναι ο ΜΙΓΑΔΙΚΟΣ ΑΡΙΘΜΟΣ
		\[
		\int_\gamma f(z)\dif z = \int_a^b f\left(\gamma(t)\right)
		\underbrace{\gamma'(t)\dif t}_{\dif\gamma(t)}
		\]
	\end{defn*}
	\paragraph{ΣΗΜΕΙΩΣΗ}
	\begin{align*}
		\dif\gamma(t) &=\dif\left(x(t)+iy(t)\right) = \\
		&=\dif x(t)+i\dif y(t) = \left( x'(t)+iy'(t) \right)\dif t \\
		\Aboxed{\dif\gamma(t) &= \gamma'(t)\dif t}
	\end{align*}
	
	Οι κλασικές ιδιότητες των επικαμπυλίων ολοκληρωμάτων έργου ισχύουν στους μιγαδικούς.
	
	Ενδεικτικά:
	\begin{itemize}
		\item \( \displaystyle\int_{-\gamma} f(z)\dif z = - \int_{-\gamma} f(z)\dif z \)
		\item \( \displaystyle\int_{\gamma} (af+by)(z)\dif z =
		a\int_{\gamma} f(z)\dif z + b\int_{\gamma} g(z)\dif z \ \forall a,b\in
		\mathbb C
		 \)
	    \item \( \displaystyle
	    \int_{\gamma_1+\gamma_2}f(z)\dif z = \int_{\gamma_1}f(z)\dif z
	    +\int_{\gamma_2} f(z)\dif z
	     \)
	    \item \( \displaystyle \left|
	    \int_\gamma f(z)\dif z
	    \right| \leq \int_\gamma \left|f(z)\right|\dif z \leq
	    M \cdot (\text{μήκος της } \gamma)
	      \) όπου \( M \) μέγιστο της \( |f| \) επί της \( \gamma \)
	    \item \( \displaystyle \int_\gamma |\dif z| = 
	    \int_a^b \sqrt{\left(x'(t)\right)^2+\left(y'(t)\right)^2}\dif t
	    := \text{ μήκος της καμπ. } \gamma
	     \)
	\end{itemize}
    
    \paragraph{Πρόταση:}
    Έστω \( f(z)=f(x+iy)=u(x,y)+iv(x,y) \) συνεχής επί καμπύλης λείας
    \( \gamma(t)=x(t)+iy(t) \).
    
    \textbf{Τότε:}
    \[
    \int_\gamma f(z)\dif z = 
    \underbrace{\left(\int_\gamma u\dif x-v\dif y\right)}%
    _{\mathclap{\begin{array}{l}
    	\text{επικαμπύλιο ολοκλ.}\\
    	\text{διαν. πεδίου στον } \mathbb R^2
    	\end{array}}}
    +i
    \underbrace{\left(\int_\gamma u\dif y+v\dif x\right)}%
    _{\mathclap{\begin{array}{l}
    	\text{επικαμπύλιο ολοκλ.}\\
    	\text{διαν. πεδίου στον } \mathbb R^2
    	\end{array}}}
    \]
    
    \subparagraph{Απόδ.}
    \begin{align*}
    &\int_\gamma (u+iv)\dif(x+iy)\\
    =& \int_a^b \left[ u\left(x(t),y(t)\right)+iv\left(x(t),y(t)\right)\right]
    \left( x'(t)+iy'(t) \right)\dif t
    \\ =& \int_a^b \left(
    u\left(x(t),y(t)\right)x'(t)-v\left(x(t),y(t)\right)y'(t)
    \right)\dif t+i
    \int_a^b \left(
    u\left(x(t),y(t)\right)y'(t)+v\left(x(t),y(t)\right)x'(t)
    \right)\dif t
    \\ \overset{\text{ορ.}}{=}&
    \left(\int_{\gamma} u\dif x-v\dif y \right)
    +i\left( \int_\gamma u\dif y+v\dif x\right)
    \end{align*}
    
    Ορίζω \( \bar f(z) = u(x,y)-iv(x,y) \)
    
    \textbf{Τότε}
    \begin{align*}
    \int_\gamma u\dif x-v\dif y &\overset{\text{Λογ. II}}{:=}
    \text{έργο του πεδίου $\bar f$ επί της καμπύλης } \gamma
    \\
    \int_\gamma u\dif y+v\dif x &\overset{\text{Λογ. II}}{:=}
    \text{\underline{ροή} του $\bar f$ διά μέσου της } \gamma
    \\
     \end{align*}

%TODO A LOT OF STUFF MISSING
\subsection{Αντιπαράγωγος και ανεξαρτησία δρόμου}
\begin{defn*}{}
	Έστω \( f=f(z) \) είναι μια συνεχής μιγαδική συνάρτηση (μιγαδικής μεταβλητής)
	σε τόπο GCC (τόπος := ανοικτό και συνεκτικό σύνολο). Αν υπάρχει
	\underline{ολόμορφη} συνάρτηση \( F=F(z) \), έτσι ώστε:
	\[
	F'(z) = f(z) \ \forall z\in \mathbf G, \text{
		τότε η $F$ καλείται αντιπαράγωγος της $f$.
	}
	\]
\end{defn*}
\begin{theorem*}[width=.9\textwidth]{}
	Έστω \( f = f(z) \) είναι συνεχής μιγαδική συνάρτηση σε τόπο \( \mathbf G \).
	Οι ακόλουθες συνθήκες είναι ισοδύναμες:
	\begin{itemize}
		\item Η \( f \) είναι ΜΟΝΑΔΙΚΗ αντιπαράγωγο \( F \) (με προσέγγιση σταθεράς)
		\item \( 
		\displaystyle \oint_\gamma f(z)\dif z = 0, \text{ για ΚΑΘΕ κλειστή λεία
			καμπύλη εντός του $G$
		}
		\)
		\item \( 
		\displaystyle \oint \int_\gamma f(z)\dif z \text{
			είναι ανεξάρτητο του δρόμου (δηλαδή εξαρτάται μόνον από το αρχικό
			και τελικό σημείο της $\gamma$ και όχι από τον τύπο της $\gamma$
			)
		}
		\)
	\end{itemize}
\end{theorem*}

Οι συνήθεις αντιπαράγωγοι εξακολουθούν να ισχύουν, π.χ.:
\begin{gather*}
\int z^n\dif z = \frac{z^{n+1}}{n+1}+c, \forall z\in\mathbb C,n\in\mathbb N\\
\int \frac{1}{z}\dif z= \mathrm{Log} z +c, \forall z \in \mathbb C^{*} \\
\int z^{-n}\dif z = \frac{z^{-n+1}}{-n+1}+c,\forall n\in\mathbb N-
\left\lbrace 1 \right\rbrace,c\in\mathbb C \text{ στάθερα} \\
\int \sin z \dif z = -\cos z+c \\
\int \cos z \dif z = \sin z +c \\
\qquad \text{ κλπ. }
\end{gather*}

%TODO here

\subsection{Θεώρημα Caychy}
Έστω \( f=f(z) \) είναι \textbf{ολόμορφη} συνάρτηση \textbf{πάνω} και στο
\textbf{εσωτερικό} \textbf{απλής}, κλειστής και λείας καμπύλης \( \gamma \).

Τότε: \[
\oint_\gamma f(z)\dif z = 0
\]

\paragraph{Απόδ.}
Έστω \( f = u+iv \), όπου \( u=u(x,y) \) και \( v=v(x,y) \) έχουν συνεχείς
μερικές παραγώγους πάνω και στο εσωτερικό της \( \gamma \). Τότε:
\begin{align*}
\oint_\gamma f(z)\dif z &= \left(
\oint_\gamma u\dif x-v\dif y
\right) + i\left( \oint u\dif y+v\dif x \right)
\\ &\overset{\mathclap{\text{Θεώρ.}}}{\underset{\mathclap{\text{Green}}}{=}}
\iint_R (-v_x-u_y)\dif x \dif y + i\iint_R (u_x-v_y)\dif x\dif y
\end{align*}
και επειδή η \( f \) ολόμορφη ικανοποιούνται οι συνθήκες Cauchy-Riemann
\( \forall (x,y) \) στο εσωτερικό της \( \gamma \), δηλαδή το \( R \), άρα:
\[
\oint_\gamma f(z)\dif z
\overset{u_x=v_y}{\underset{u_y=v_x}{=}}
\iint_R 0\dif x\dif y+i\iint_\gamma 0\dif x \dif y = 0
\]

\paragraph{}
\[
\int_\gamma f(z)\dif z =
\underbrace{a}_{\mathclap{\text{έργο του πεδίου $f$ κατά μήκος $ \gamma $}}} 
+ i\overbrace{b}^{\mathclap{\text{ροή του πεδίου $\bar f$ διά μέσου της $\gamma$}}}
\]

\paragraph{ΣΗΜΕΙΩΣΗ:}
Αν υπάρχει έστω και ένα σημείο όπου η \( f \) δεν είναι μιγαδικά
παραγωγίσιμη στο εσωτερικό της \( \gamma \), τότε το \textbf{θεώρ. Cauchy
	δεν ισχύει εν γένει}.

\subparagraph{π.χ} \( \displaystyle \oint_{|z|=1\text{ με θετική φορά}}
\frac{\dif z}{z} \)
%TODO Atreas Graph 17

    \begin{align*}
    \oint_{|z|=1} \frac{\dif z}{z} & \overset{\gamma(t)=e^{it}}{\underset{t\in[0,2\pi)}{=}}
    \\ &\overset{\text{ορ.}}{=}\int_{0}^{2\pi}\frac{\dif\, \left(e^{it}\right)}{e^{it}}
    =\int_{0}^{2\pi} \frac{(e^{it})^2}{e^{it}}\dif t
    \\ &= \int_0^{2\pi} \frac{ie^{it}}{e^{it}} = 2\pi i
    \end{align*}
    
    \begin{theorem*}[width=\textwidth]{Παραμόρφωση δρόμων}
    	\vspace{25pt}
    	Έστω \( f=f(z) \) είναι ολόμορφη σε τόπο \( G \) με σύνολο
    	\( \partial G = \gamma_1 \cup \gamma_2 \) όπου \( \gamma_1,\gamma_2 \)
    	απλές λειστές καμπύλες, λείες, με κοινό προσανατολισμό π.χ. όπως στο σχήμα
    	%TODO Atreas Graph 18
    	Τότε \( \displaystyle \oint_{\gamma_1}f(z)\dif z
    	= \oint_{\gamma_2} f(z)\dif z
    	 \)
    \end{theorem*}
    
    \subparagraph{Απόδ.}
    Φέρνω δύο ευθ. τμήματα \( L_1 \) και \( L_2 \) που διαμερίζουν το \( G \) σε
    δύο χωρία έστω \( G_1,G_2 \). Τότε το θ. Cauchy ισχύει και στο \( G_1 \)
    και στο \( G_2 \).
    
    \begin{itemize}
    \item \( \displaystyle 
    \int_{\gamma_1^+ + L_1 + \gamma_2^+ + L_2} f(z)\dif z = 0 \)
    \quad (θ. Cauchy για το χωρίο \( G_1 \))
    \item \( \displaystyle
    \int_{\gamma_2^- - L_1 - \gamma_2^- - L_2} f(z)\dif z = 0
     \) \quad (θ. Cauchy για το χωρίο \( G_2\))
    \end{itemize}
    \[
    \implies \left| \begin{array}{l}
    \left( \int_{\gamma_1^+} + \int_{L_1} - \int_{\gamma_2^+} + \int_{L_2} \right)
    f(z)\dif z = 0 \\
    \left( \int_{\gamma_1^-} - \int_{L_1} - \int_{\gamma_2^-} - \int_{L_2} \right)
    f(z)\dif z = 0
    \end{array}
    \right. \implies \oint_{\gamma_1}f(z)\dif z - \oint_{\gamma_2}f(z)\dif z = 0
    \]
    
    \paragraph{Πόρισμα (Γενικευμένο θεώρ. Cauchy)} 
    Έστω \( f=f(z) \) ολόμορφη σε τόπο \( G \) με σύνορο \( \partial G =
    \Gamma \cup \left( \gamma_1 \cup \cdots \cup \gamma_2 \right)
    \), όπου:
    \begin{itemize}
       	\item \( \Gamma, \gamma_1,\gamma_2,\dots,\gamma_n \) απλές, κλειστές, λείες
       	και ΘΕΤΙΚΑ ΠΡΟΣΑΝΑΤΟΛΙΣΜΕΝΕΣ καμπύλες
       	\item Οι \( \gamma_1,\gamma_2,\dots,\gamma_n \) βρίσκονται εντός της
       	\( \Gamma \) και
       	\item Κάθε καμπύλη \( \gamma_j \quad j=1,\dots,n \) βρίσκεται εκτός των
       	υπόλοιπων
       	\( \gamma_1,\gamma_2,\dots,\gamma_{i-1},\gamma_{i+1},\dots,\gamma_n \)
       	%TODO Atreas Graph 19
       	
       	Τότε: \( 
       	\displaystyle \oint_\Gamma f(z)\dif z= \sum_{j=1}^k \oint_{\gamma_1}f(z)
       	\dif z
       	\)
    \end{itemize}
    
    \begin{theorem*}[width=.8\textwidth]{Ολοκληρωτικός τύπος Cauchy}
    	\vspace{25pt}
    	Έστω \( f=f(z) \) είναι ολόμορφη πάνω και στο εσωτερικό απλής, κλειστής,
    	τμημ. λείας και { \large ΘΕΤΙΚΑ ΠΡΟΣΑΝΑΤΟΛΙΣΜΕΝΗΣ  } καμπύλης \( \gamma \).
    	
    	Τότε { \large ΓΙΑ ΚΑΘΕ } σημείο \( z_0 \) ΣΤΟ ΕΣΩΤΕΡΙΚΟ της \( \gamma \)
    	ισχύει:
    	
    	\[
    	\boxed{
    		f(z_0) = \frac{1}{2\pi i}\oint_\gamma
    		\frac{f(z)}{z-z_0}\dif z
    		}
    	\]
    \end{theorem*}
    \subparagraph{Απόδειξη}
    %TODO Atreas Graph 20
    
    Έστω \( \left|z-z_0\right| =r \) κύκλος ακτίνας \( r \) κατάλληλης ώστε ο δίσκος
    \( \left|z-z_0\right| \leq r \) να βρίσκεται εξ' ολοκλήρου στο εσωτερικό της
    \( \gamma \).
    
    Τότε από το θεώρημα παραμόρφ. δρόμων, εφ' όσον \( \frac{f(z)}{z-z_0} \) ολόμορφη
    στο γραμμοσκιασμένο χωρίο, έχουμε:
    \[
    \oint_\gamma \frac{f(z)}{z-z_0} \dif z =
    \oint_{\left|z-z_0\right|=r} \frac{f(z)}{z-z_0}\dif z
    = \underbrace{\oint_{\left|z-z_0\right|=r} \frac{f(z)-f(z_0)}{z-z_0}\dif z}_{I_2}
    +\underbrace{\oint_{\left|z-z_0\right|=r} \frac{f(z_0)}{z-z_0}\dif z}_{I_1}
    \]
    
    Για το \( I_2 \) έχω:
    \begin{align*}
    I_2 &= \oint_{\left|z-z_0\right|=r} \frac{f(z_0)}{z-z_0}\dif z
    \overset{z=z_0+re^{i\theta}}{\underset{\theta\in[0,2\pi]}{=}}
    f(z_0)\int_{0}^{2\pi}\frac{1}{re^{i\theta}}rie^{i\theta}\dif\theta
    \\ &= 2\pi i f(z_0)
    \end{align*}

    \begin{infobox}{}
    \vspace{-10pt}\[
    \left(
    \begin{array}{lcl}
    l = l' & \iff & \left| l-l' \right|<\epsilon\ \forall \epsilon>0 \\
    & "\implies" & \text{προφ. ισχύει} \\
    & "\impliedby" & \text{Έστω } l \neq l' \implies \left| l-l' \right| \geq
    \epsilon_0 > 0 \text{ άτοπο } \implies l=l'
    \end{array}
    \right)
    \]
    \end{infobox}

    \textbf{Έτσι:} \[
    \boxed{
    	\oint_\gamma \frac{f(z)}{z-z_0}\dif z -2\pi i f(z_0) = I_1
    	}
    \]
    
    \begin{align*}
    \left|I_1\right| &\leq
    \oint_{\left|z-z_0\right|=r} \frac{\left|f(z)-f(z_0)\right|}{\left|z-z_0\right|}
    \dif z \leq 
    M \cdot \oint_{\left|z-z_0\right|=r} \frac{1}{\left|z-z_0\right|}\dif z,
    \\ \intertext{όπου $
    	M=\max\left\lbrace \left|f(z)-f(z_0)\right|\ \forall
    	z:\left|z-z_0\right|=1
    	 \right\rbrace
    	$}
    &= M\oint_{\left|z-z_0\right|=r} \frac{1}{r}|\dif z|
    \\ &= \frac{M}{r}
    \underbrace{\oint_{\left|z-z_0\right|=r} |\dif z|}_{%
    	\mathclap{\text{μήκος καμπύλης}}}
    =\frac{2\pi M r}{r} = \underline{2\pi M}
    \end{align*}
    
    Αλλά \( f \) ολόμορφη στο \( z_0 \), άρα \( f \) συνεχής στο \( z_0 \).
    
    Εξ' ορισμού λοιπόν: \( \forall \epsilon>0\ \exists r_1>r>0:\
    \forall z: 0<\left|z-z_0\right|<r<r_1 \implies \left|f(z)-f(z_0)\right|<\epsilon
     \)
     
    Έτσι \( \forall \epsilon > 0 \) μπορώ να βρω ακτίνα 
    \( r: \left|f(z)-f(z_0)\right|\ \forall z:\left|z-z_0\right|=r \), δηλ.
    \( M \leq \epsilon \) και τελικά \( \left|I_1\right| \leq 2\pi M \leq
    2\pi\epsilon\ \forall \epsilon>0 \implies I_1 = 0 \)
    
    \begin{theorem*}{Ολοκληρ. τύπος Cauchy για παραγώγους}
    	\vspace{20pt}
    	Έστω \( f \) είναι ολόμορφη πάνω και στο εσωτερικό απλής, κλειστής, λείας
    	και θετικά προσανατολισμένης καμπύλης \( \gamma \).
    	
    	Αν \( z_0 \) σημείο στο ΕΣΩΤΕΡΙΚΟ της \( \gamma \), τότε η \( f \)
    	ΕΧΕΙ ΠΑΡΑΓΩΓΟΥΣ \underline{ΚΑΘΕ ΤΑΞΗΣ} στο σημείο \( z_0 \) και μάλιστα:
    	\[
    	f^{(n)}(z_0) = \frac{n!}{2\pi i}\oint_\gamma
    	\frac{f(z)}{\left(z-z_0\right)^{n+1}}\dif z
    	\]
    \end{theorem*}
    
    \begin{attnbox}{}
    	Ο Ατρέας θα δίνει τύπους σε τυπολόγιο:
    	\url{http://users.auth.gr/natreas/Efarmosmena/ΤΥΠΟΛΟΓΙΟ.pdf}
    \end{attnbox}
    
    \subsection{Εφαρμογές}
    \begin{enumparen}
    	\item \textbf{Θεώρ. μέσης τιμής Gauss}
    	
    	Αν \( f \) ολόμορφη πάνω και στο εσωτερικό θετικά προσανατολισμένου κύκλου
    	\( \left|z-z_0\right| = R \), τότε:
    	\[
    	f(z_0) = \frac{1}{2\pi} \int_{0}^{2\pi}
    	\mathlarger{f}\!\left( z_0+Re^{i\theta} \right)\dif \theta
    	\]
    	\subparagraph{Απόδειξη}
    	Εφαρμόζω τον ολοκλ. τύπο του Cauchy με τα δεδομένα μου και έχω:
    	\begin{align*}
    	f(z_0) &= \frac{1}{2\pi i}
    	\oint_{\left|z-z_0\right|=R} \frac{f(z)}{z-z_0}\dif z \\
    	&\overset{z=z_0+Re^{i\theta}}{\underset{\text{ορισμός}}{=}}
        \frac{1}{2\pi i}\int_{0}^{2\pi}
    	\frac{f\left(z_0+Re^{i\theta}\right)}{Re^{i\theta}}\dif\left(
    	z_0+Re^{i\theta}
    	\right) \\ &=
    	\frac{1}{2\pi i}\int_{0}^{2\pi}
    	\frac{f\left(z_0+Re^{i\theta}\right)}{\cancel{Re^{i\theta}}}
    	i\cancel{Re^{i\theta}}\dif\theta
    	\\ &= \text{ ζητούμενο}
    	\end{align*}
    	
    	\item \textbf{Ανισότητα Cauchy}
    	Έστω \( f \) ολόμορφη πάνω και στο εσωτερικό θετικά προσανατολισμένου κύκλου
    	\( \left|z-z_0\right| = R \) και \( M_R = \max \left\lbrace 
    	\left|f(z)\right|,\ \forall z:\left|z-z_0\right|=R
    	 \right\rbrace \)
    	 
    	\textbf{Τότε:} \[
    	\left| f^{(n)}(z_0) \right| \leq \frac{n!M_R}{R^n},\ 
    	n=1,2,3,\dots
    	\]
    	\subparagraph{Απόδ.}
    	Εφαρμόζουμε τον ολοκλ. τύπο Cauchy για παραγώγους προσαρμοσμένο στα δεδομένα:
    	\begin{align*}
    	\left|f^{(n)}(z_0)\right| &= \left|
    	\frac{n!}{2\pi i}\oint_{|z-z_0|=R} \frac{f(z)}{(z-z_0)^{n+1}}\dif z
    	\right|
    	\\ &\leq \frac{n!}{2\pi} 
    	\oint_{|z-z_0|=R} \frac{\left|f(z)\right|}{\left|z-z_0\right|^{n+1}}|\dif z|
    	\\ &\leq \frac{n!}{2\pi} M_R \oint_{|z-z_0|=R}
    	\frac{1}{|z-z_0|^{n+1}}|\dif z|
    	\\ &= \frac{n!}{2\pi} M_R \oint_{|z-z_0|=R} \frac{1}{R^{n+1}}|\dif z|
    	\\ &= \frac{n!}{2\pi} M_R \frac{1}{R^{n+1}}
    	\underbrace{\oint_{|z-z_0|=R} |\dif z|}_{\mathclap{\text{μήκος κύκλου } z-z_0=R}}
    	\\ &= \frac{n!}{2\pi} M_R \frac{1}{R^{n+1}}\cdot 2\pi R
    	= \frac{n!M_R}{R^n}
    	\end{align*}
    	
    	\item \textbf{Θεώρ. Liouville}
    	
    	Κάθε \textbf{ακεραία} συνάρτηση (δηλ. ολόμορφη στο \( \mathbb C  \)) και φραγμένη
    	\( \boxed{\text{στο } \mathbb C}  \) είναι η σταθερή συνάρτηση.
    	\subparagraph{Απόδ.}
    	Έστω \( z \in \mathbb C  \) τυχαίο. Χρησιμοποιώ ανισότητα Cauchy για \( n=1 \):
    	\[
    	\left|f'(z)\right| \leq \frac{1!\ M_R}{R},\quad M_R = \max
    	\left\lbrace \left|f(z)\right|: |z-z_0|=R \right\rbrace
    	\]
    	
    	Αφού \( f \) εξ' υποθέσεως είναι φραγμένη, άρα \( \exists \underline{M > 0}:
    	\left|f(z)\right| \leq M \quad \forall z \in \mathbb C 
    	 \)
    	 
    	\textbf{Δηλ.} \(  \displaystyle
    	\left|f'(z)\right| \leq \frac{M_R}{R} \leq \frac{M}{R}
    	\xrightarrow[R\to \infty] 0
    	 \)
    	 
        Τότε \( \left|f'(z)\right| = 0 \iff f'(z)=0 \forall z\in
        \iff f(z) = c\in\mathbb C 
        \)
        
        \item \textbf{Αρχή μεγίστου/ελαχίστου}
        
        Έστω \( f \) ολόμορφη σε ανοικτό και συνεκτικό σύνολο \( G \) και μη σταθερή στο
        \( G \). Τότε η \( |f| \) \textbf{ΔΕΝ έχει μέγιστη τιμή} στο \( G \).
        
        Αν μάλιστα \( f(z) \neq 0 \quad \forall z \in G \), τότε η \( |f| \)
        \textbf{ΔΕΝ έχει ελάχιστη τιμή στο \( \mathbf G \)}.
        
        Ειδικά αν \( G \) είναι και \textbf{ΦΡΑΓΜΕΝΟ} και η \( f \) είναι συνεχής στο
        σύνορο του \( G \) (το οποίο είναι απλή, λεία καμπύλη), τότε η \( |f| \)
        \textbf{παίρνει ΜΕΓΙΣΤΗ ΤΙΜΗ ΠΑΝΩ στο σύνορο του \( \mathbf G \)}.
        Ομοίως αν \( f(z) \neq 0 \quad \forall z \in G \), τότε η \( |f| \)
        παίρνει ελάχιστη τιμή ΠΑΝΩ στο σύνορο του \( G \).
    	
    \end{enumparen}
    
    \paragraph{Άσκ.} Υπολογίστε το \( \int_\gamma \left( i\bar z - z \right)\dif z \)
    \\ όπου \( \gamma \) είναι η παραβολή \( y=2t^2+1 \) με αρχή το σημείο \( (1,3) \)
    και πέρας το σημείο \( 2,9 \).
    \subparagraph{}
    Γενικά, μπορώ να κινηθώ μέσω ορισμού, αντιπαραγώγου ή θεωρημάτων. Η \( \bar z \)
    δεν έχει παράγωγο, άρα δεν έχει αντιπαράγωγο (διαφορετικά από προηγούμενη εφαρμογή
    θα είχε άπειρες παραγώγους).
    
    \subparagraph{Έχουμε:} 
    \begin{align*}
    \int_\gamma (i\bar z - z)\dif z &=
    i\int_\gamma \bar z\dif z - \int_\gamma z\dif z = I_1+I_2
    \end{align*}
    \begin{itemize}
    	\item όσον αφορά το \( I_2 \), εφ' όσον η \( f(z)=z \) είναι ολόμορφη στο
    	\( \mathbb C  \) ως πολυώνυμο, έχει μοναδική αντιπαράγωγο (με προσέγγιση σταθεράς),
    	άρα:
    	\begin{align*}
    	\int_\gamma z\dif z &= \left. \frac{z^2}{2} \right|_{z_0=1+3i}^{z_1=2+9i}
    	\intertext{(αντιπαράγωγος \( \xRightarrow{\text{θεωρία}} \) ανεξαρτησία δρόμου)}
    	\\ &= \frac{(2+9i)^2}{2} - \frac{(1+3i)^2}{2}
    	\\ &= \frac{69}{2} - 15i \\ &= B
    	\end{align*}
    	\item Για το \( I_1 \):
    	\begin{align*}
    	I_1 &= i\int_\gamma \bar z \dif z
    	\overset{\text{ορισμός}}{\underset{\text{διότι η $\bar z$ ΔΕΝ
    				είναι παραγωγίσιμη σε κανένα σημείο}}{=}}
    	\\ &
    	\overset{(t,2t^2+1)}{\underset{t+i(2t^2+1)=\gamma(t)}{=}}
    	\int_1^2 \overline{\gamma(t)} \dif\left(\gamma(t)\right)
    	\\ &= \int_1^2 \left[
    	t-i\left(2t^2+1\right)
    	\right]\underbrace{\left[
    	1+4ti
    	\right]\dif t}_{\mathclap{
    		\gamma'(t)\dif t := \dif\gamma(t)
    		}}
        \\ &= i\int_1^2 \left[
        t+4t\left(2t^2+1\right)
        \right]+i\left[ 1+2t^2+4t^2 \right]\dif t
        \\ &= i\int_1^2 \left( 5t+8t^3 \right) + i\left( 6t^2+1 \right)\dif t
        \\ &= i \left[ \frac{5t^2}{2} + 2t^4 \right]_1^2 - \left(2t^3+t\right)_1^2
        \\ &= A + B
    	\end{align*}
    	
    	\textbf{Τελικά}
    	
    \end{itemize}
    
    Από εδώ και στο εξής, μέχρι νεωτέρας, όλοι μαζί, Τρίτη και Πέμπτη.
    
    \paragraph{Άσκ.}
    Υπολογίστε τα επικαμπύλια ολοκληρώματα
    \begin{enumgreekparen}
    	\item \( \displaystyle
    	\oint_{\left|z-\frac{1}{z}\right|=\frac{3}{2}}
    	\frac{z\cos z}{2z+1}\dif z
    	 \)
    	\item \( \displaystyle
    	\oint_{|z|=3} \frac{z^3+2}{(z-2)^3}\dif z
    	 \)
    	\item \( \displaystyle
    	\oint_{|z|=2} \frac{\rho^z}{z^2-1} \dif z
    	 \)
    \end{enumgreekparen}
    Όλες οι καμπύλες θεωρούνται προσανατολισμένες με τη θετική φορά.
    
    \paragraph{}
    
    \begin{enumgreekparen}
    	\item 
    	%TODO Atreas Graph 21
    	Θα χρησιμοποιήσω ολοκλ. τύπο Cauchy.
    	
    	Έστω \( f(z)=z\cos z \), ολόμορφη στο \( \mathbb C  \) άρα και πάνω και στο
    	εσωτερικό του κύκλου \[ \left| z-\frac{1}{2} \right| =\frac{3}{2} \]
    	
    	\textbf{Προφανώς:} \begin{align*}
    		&\ \oint_{\left| z-\frac{1}{2} \right|=\frac{3}{2}}
    		\frac{z\cos z}{2z+1}\dif z \\
    		=&\ \frac{1}{2} \oint_{\left| z-\frac{1}{2} \right|=\frac{3}{2}}
    		\frac{z\cos z}{z-\left(-\frac{1}{2}\right)} \dif z,
    	\end{align*}
    	όπου \( z_0 = -\frac{1}{2} \ \in \) \underline{εσωτερικό} του κύκλου
    	\( \left|z-\frac{1}{2}\right| = \frac{3}{2} \), ο οποίος είναι
    	\underline{θετικά προσανατολισμένος}.
    	
    	Τότε ικανοποιούνται όλες οι συνθήκες ώστε να έχω
    	\begin{align*}
    		&\ \frac{1}{2\pi i}\oint_{\left| z-\frac{1}{2} \right|=\frac{3}{2}}
    		\frac{z\cos z}{z-\left(-\frac{1}{z}\right)}\dif z
    		\\ =&\ \big|z\cos z\big|_{z_0=-\frac{1}{2}} \implies
    		\oint_{\left| z-\frac{1}{2} \right|=\frac{3}{2}} \frac{z\cos z}{z+\frac{1}{2}}
    		\dif z = 2\pi \left(-\frac{1}{2}\right) \cos \left(-\frac{1}{2}\right)
    	\end{align*}
    	
    	\textbf{Τελικά:} \( 
    	\displaystyle \boxed{
    		\oint_\gamma \frac{z\cos z}{2z+1}\dif z = \frac{-\pi i}{2}
    		\cos\left(-\frac{1}{2}\right)
    		}
    	 \)
    	 
    	 \item 
    	 %TODO Atreas Graph 22
    	 Θα χρησιμοποιήσω τύπο Cauchy για παραγώγους με \( \mathbf{n=2} \).
    	 
    	 Έστω \( g(z) = z^3+2 \), προφανώς ακεραία (ολόμορφη σε όλο το \( \mathbb C  \)),
    	 άρα ολόμορφη πάνω και στο εσωτερικό του κύκλου μας.
    	 
    	 Επίσης, \( z_0 = 2 \ \in \ \) εσωτερικό του θετικά προσανατολισμένου κύκλου
    	 μας, άρα από τύπο Cauchy για παραγώγους έχουμε:
    	 \begin{align*}
    	 g''(2) &= \frac{2!}{2\pi i} \oint_{|z|=3} \frac{g(z)}{(z-2)^3} \dif z
    	 \qquad \left(g(z) = z^3+2 \right) \\
    	 \implies \oint_{|z|=3} \frac{z^3+2}{(z-2)^3} \dif z &= \pi i \cdot g''(2)
    	 \end{align*}
    	 \begin{align*}
    	 	g'(z) &= 3z^2 \\
    	 	g''(z) &= 6z \\
    	 	g''(2) &= 12
    	 \end{align*}
    	 
    	 \textbf{Τελικά} \( 
    	 \displaystyle \oint_{|z|=3} \frac{z^3+2}{(z-2)^3} \dif z = 12\pi i
    	  \)
        \item
        %TODO Atreas Graph 23
        Χρησιμοποιώ κατ' αρχήν γενικευμένο θεώρημα Cauchy, και έχω:
        \begin{align*}
        I_{\text{ζητούμενο}} &= \oint_{\gamma_1} \frac{e^z \dif z}{(z-1)(z+1)}
        + \oint_{\gamma_2} \frac{e^z\dif z}{(z-1)(z+1)}
        \\ &= \oint_{\gamma_1} \frac{\sfrac{e^z}{(z+1)} }{z-1} \dif z
        + \oint_{\gamma_2} \frac{\sfrac{e^z}{(z-1)} }{z+1}\dif z
        \\ &\overset{\text{τύπος}}{\underset{\text{Cauchy}}{=}}
        \left. 2\pi i \frac{e^z}{z+1} \right|_{z=1}
        + \left. 2\pi i \frac{e^z}{z-1} \right|_{z=-1}
        \\ &= \pi i e - \pi i e^{-1}
        \end{align*}
        (διότι οι αριθμητές \( 
        \left|
        \begin{array}{l}
        a(z) = \frac{e^z}{z+1} \\ b(z)=\frac{e^z}{z-1}
        \end{array}
        \right.
         \) είναι ολόμορφες συναρτήσεις πάνω και στο εσωτερικό των καμπύλων \( \gamma_1 \)
         και \( \gamma_2 \) αντιστοίχως και \( z_0 = 1 \in \) εσωτερικό της \( \gamma_1 \)
         ενώ \( z_1 = -1 \in \) εσωτερικό \( \gamma_2 \))
    \end{enumgreekparen}
    
    \paragraph{Θέμα:}
    Υπολογίστε το \( \displaystyle \oint_{|z|=R} \frac{1}{(z-i)^2} \dif z \)
    για όλες τις τιμές του \( R \), όπου \( R > 0 \) και \( R \neq 1 \)
    \subparagraph{}
    \begin{enumgreekparen}
    	\item \( \underline{R < 1} \)
    	
    	Τότε \( I = 0 \) από θεώρ Cauchy αφού ανωμαλία \( z_0 = i \) εκτός κύκλου
    	\( |z| = R \)
    	
    	\item \( \underline{R>1} \)
    	
    	Τότε \( z_0 = i \ \in \ \) εσωτερικό κύκλου \( |z| = R \) οπότε χρησιμοποιώ τύπο
    	Cauchy για παραγώγους και βρίσκω \[ I = 0 \].
    \end{enumgreekparen}
    
    \paragraph{Άσκ. }    
    Έστω \( f \) ολόμορφη πάνω και στο εσωτερικό κύκλου \( |z| = R \), με
    \( f(z) \neq 0\ \forall z \) στο εσωτερικό του κύκλου και \( f(z) = c \) για κάθε
    \( z \) πάνω στον κύκλο \( |z| = R \).
    
    ΝΔΟ \( \left|f(z)\right| = A \geq 0 \ \forall z \) στο εσωτερικό του κύκλου.
    
    \subparagraph{}
    Θα χρησιμοποιήσω αρχή μεγίστου/ελαχίστου, η οποία λέει ότι η \( \left|f(z)\right| \)
    παίρνει τόσο τη μέγιστη, όσο και την ελάχιστη τιμή της ΠΑΝΩ στον κύκλο \( |z|=R \).
    
    Εφ' όσον όμως \( f(z) = c\ \forall z: |z| = R \) τότε \( \left|f(z)\right| = |c| = \)
    σταθερό \( \forall z \) πάνω στον κύκλο, όπου όμως η \( |f| \) παίρνει και μέγιστη και
    ελάχιστη τιμή. Άρα η \( \max |f| = \min |f| \ \forall z:|z| = R \), συνεπώς
    \( |f| =  \) σταθερά \( \forall z \) στο εσωτερικό του κύκλου.
    
    \paragraph{Άσκ.}
    Έστω \( f \) ακεραία και \( \left|f(z)\right| \leq A|z| \forall z\in\mathbb C  \).
    ΝΔΟ \( f(z) = cz \), όπου \( c \in \mathbb C  \) σταθερά.
    \subparagraph{}
    Θα χρησιμοποιήσω ανισότητα Cauchy για \( n=2 \), προσπαθώντας να δείξω ότι:
    \[
    \left|f''(z)\right| = 0 \quad \forall z \in \mathbb C 
    \]
    τότε \( f'(z) = c \implies f(z) = cz+d \quad c,d\in\mathbb C  \). Από υπόθεση,
    για \( z=0 \) έχω \( \left|f(0)\right| \leq A \cdot 0 \implies f(0) = 0 \) άρα \(d = 0\).
    
    Ανισότητα Cauchy
    \[
    \left|f^{(n)} (z_0)\right| \leq \frac{n! \cdot M_R}{R^n}
    \qquad n=1,2,3,\dots
    \]
    όπου \( M_R = \max\left\lbrace \left|f(z)\right| : |z-z_0|=R \right\rbrace \)
    
    Έτσι για \( n=2 \) έχω για \( z_0 \in \mathbb C  \)
    \[
    f''(z_0) \leq \frac{2! \cdot M_R}{R^2} = \frac{2MR}{R^2}
    \]
    
    Για \( |z-z_0| = R \) δηλ. για \( z = z_0 + Re^{i\theta} \) έχω
    \( \left|f(z)\right|
    \overset{\mathclap{\text{εξ' υποθέσεως}}}{\leq}
    A|z| = A \left|z_0+Re^{i\theta}\right| \leq A|z_0| + AR
     \)
     
    \textbf{Τότε:}
    \[
      \left| f''(z_0) \right| \leq
      \frac{2\left(|z_0+R|\right)}{R^2}
      \xrightarrow[R\to \infty] 0      
    \]
    
    άρα \( \left|f''(z_0)\right| = 0\ \forall z_0\in\mathbb C  \), άρα
    \( f''(z_0) = 0\ \forall z_0 \in \mathbb C  \).


	\newpage

	\part{Κεχαγιάς}
	Σπιτεργασίες λιγότερες από πέρσι, για 1 βαθμό, αφορούν μόνο το μέρος του Κεχ.
	\begin{enumerate}
		\item ΜΙΓΑΔΙΚΟΙ ΑΡΙΘΜΟΙ
		\item ΒΑΣΙΚΕΣ ΜΙΓΑΔΙΚΕΣ ΣΥΝΑΡΤΗΣΕΙΣ
		\item ΑΚΟΛΟΥΘΙΕΣ, ΣΕΙΡΕΣ
		\item ΔΥΝΑΜΟΣΕΙΡΕΣ
		\item ΑΡΜΟΝΙΚΕΣ ΣΥΝΑΡΤΗΣΕΙΣ
		\item ΔΙΑΦΟΡΙΚΕΣ ΕΞΙΣΩΣΕΙΣ με μερικές παραγώγους
	\end{enumerate}

	\setcounter{section}{0}

	\section{Μιγαδικοί αριθμοί}
	\begin{align*}
	    z = & x+iy \in \mathbb C \\
	    & x,y \in \mathbb R \qquad i^2=-1
	\end{align*}
	\begin{align*}
	z_1 &= x_1 +iy_1 \\
	z_2 &= x_2 +iy_2 \\
	z_1+z_2 &= (x_1+x_2)+i(y_1+y_2) \\
	z_1\cdot z_2 &= (x_1+iy_1)\cdot(x_2+iy_2) \\
	&= x_1x_2+iy_1y_2+ix_1y_2+ix_2y_1 \\
	&= (x_1x_2-y_1y_2)+i(x_1y_2+x_2y_1) \\
	\frac{z_1}{z_2} &= \frac{x_1+iy_1}{x_2+iy_2}
	= \frac{(x_1+iy_1)(x_2-iy_2)}{(x_2+iy^2)(x_2-iy_2)}
	\\ &= \frac{x_1x_2+y_1y_2}{x_2^2+y_2^2} + i \frac{-x_1y_2+x_2y_1}{x_2^2+y_2^2}
	\\ z &= x+iy
	\\ \bar{z} &= x-iy
	\\ \Re(z) &= x \in \mathbb R
	\\ \Im(z) &= y \in \mathbb R
	\end{align*}

	\begin{center}
		\begin{tikzpicture}[scale=2.5]
		\draw[->] (0,-0.5) -- (0,1.5) node[above right]{$y$};
		\draw[->] (-0.5,0) -- (1.7,0) node[below right]{$x$};

		\draw[gray,dashed]
			(0,1) node[above right,black] {$y$}
			-- (1,1) --
			(1,0) node[below,black] {$x$};
		\filldraw (1,1) circle(0.8pt) node[above right] {$z=x+iy$} ;

		\draw(0,0) -- (1,1) node[midway,above,sloped] {$r$};
		\draw[->] (.3,0) arc (0:45:.3) node[midway,right] {$\theta$};

		\draw (current bounding box.south) node {$r=\sqrt{x^2+y^2}$};
		\draw (current bounding box.south) node[below] {$\theta=\arctan\frac{y}{x}$};
		\end{tikzpicture}
	\end{center}
	\[
	r = \sqrt{x^2+y^2} = \sqrt{z\bar{z}} = |z| \leftarrow \text{μέτρο του } z
	\]
	γενίκευση της απόλυτης τιμής (δηλ. \( z=x \in \mathbb R,\ |z|=\sqrt{x^2}=|x| \))

	\begin{align*}
	z=x+iy &= r\cdot\cos\theta + ir\sin\theta \\
	&= r(\cos\theta+i\sin\theta) \\
	&= r\cdot e^{i\theta} \quad \text{(Euler)}
	\end{align*}
	\begin{align*}
	e^{i\theta} &= \cos\theta+i\sin\theta \text{ διότι}\\
	e^{i\theta} &= 1 + i\theta + \frac{(i\theta)^2}{2!} + \frac{(i\theta)^3}{3!}
	+ \frac{(i\theta)^4}{4!} + \dots \\
	&= \left(
	    1 - \frac{\theta^2}{2!} + \frac{\theta^4}{4!} - \dots
	\right) + i \left(
	    \theta-\frac{\theta^3}{3!} + \frac{\theta^5}{5!} - \dots
	\right)
	\\ &= \cos\theta+i\sin\theta
	\end{align*}

	Επίσης:
	\begin{align*}
	z &= x+iy \\
	&= \sqrt{x^2+y^2} \left( \frac{x}{\sqrt{x^2+y^2}}+i\frac{y}{\sqrt{x^2+y^2}} \right)
	\\ &= r \cdot (\cos\theta + i\sin\theta)
	\\ &= r \cos\theta + ir\sin\theta
	\end{align*}

	\begin{tikzpicture}[scale=2]
	\draw[->] (0,-1.5) -- (0,1.5);
	\draw[->] (-1.5,0) -- (1.5,0);

	\draw[gray,dashed] (0,1) -- (1,1) -- (1,0);
	\draw[gray,dashed] (0,-1) -- (-1,-1) -- (-1,0);

	\filldraw (1,1) circle(0.8pt) node[above right] {$z_1=1+i$} ;
	\filldraw (-1,-1) circle(0.8pt) node[above left] {$z_2=-1-i$} ;

	\draw[->,thick] (0,0) -- (1,1);
	\draw[->,thick] (0,0) -- (-1,-1);
	\draw[->] (.7,0) arc (0:45:.7);
	\draw[->] (.2,0) arc (0:225:.2);

	\end{tikzpicture}

	\begin{gather*}
		z_1=1+i = \sqrt{2}\cdot e^{i\sfrac{\pi}{4}} \\
		r_1 = \sqrt{1^2+1^2} = \sqrt{2} \\
		\theta_1 = \arctan\frac{1}{1} = \frac{\pi}{4}
	\end{gather*}
	\begin{gather*}
		z_2=-1-i=\sqrt{2}e^{i\sfrac{5\pi}{4}} = \sqrt{2}e^{i\cdot\left(
			-\sfrac{3\pi}{4} = \sqrt{2}e^{i\sfrac{13\pi}{4}}
			\right)} \\
		r_2=\sqrt{(-1)^2+(-1)^2} = \sqrt{2} \\
		\theta_2 = \arctan\frac{-1}{-1} = \frac{\pi}{4}
	\end{gather*}

	Γενικά: \( \mathlarger{-1-i=\sqrt{2}e^{i\left(
			\frac{5\pi}{4}+2k \pi
			\right)}},\quad k \in \mathbb Z \)

    \subsection{Συναρτήσεις}
    \[
    \mathlarger{\mathlarger{\mathbb C \to \mathbb R}}
    \]
    \begin{gather*}
    z=x+iy \\
    \mathrm{mod}(z)= \sqrt{x^2+y^2} \\
    \arg(z) = \begin{cases}
    \theta_0 \quad & \text{αν } z \in \text{1\textsuperscript{ο} τεταρτημόριο} \\
    \pi - \theta_0 \quad & \text{αν } z \in \text{2\textsuperscript{ο} τεταρτημόριο} \\
    \pi + \theta_0 \quad & \text{αν } z \in \text{3\textsuperscript{ο} τεταρτημόριο} \\
    2\pi - \theta_0 \quad & \text{αν } z \in \text{4\textsuperscript{ο} τεταρτημόριο}
    \end{cases} \qquad \theta_0 = \arctan\left(\left|\frac{y}{x}\right|\right) \\
    \forall z \in \mathbb C - \left\lbrace 0 \right\rbrace\ \arg(z) \in [0,2\pi)
    \end{gather*}

    Ορίζω και την πλειότιμη συνάρτηση \( \mathrm{arg}(z) = \left\lbrace
    \arg(z)+2k\pi,\ k \in \mathbb Z
     \right\rbrace \)

    \begin{align*}
    z = x+iy &= \mathrm{mod}(z) \cdot e^{i\arg(z)}
    \\ &= \mathrm{mod}(z) \cdot e^{i\left(\arg(z)+2k\pi\right)}
    \end{align*}
    \begin{align*}
    z_1 = \mathrm{mod}(z_1)e^{i\arg(z_1)}\\
    z_2 = \mathrm{mod}(z_2)e^{i\arg(z_2)}\\
    z_1z_2 = \mathrm{mod}(z_1)\mathrm{mod}(z_2)e^{i\cdot\left(
    	\arg(z_1)+\arg(z_2)
    	\right)} \\
    \arg(z_1z_2) \neq \arg(z_1)+\arg(z_2) \text{ επειδή} \\
    \arg\left(
        e^{i\frac{7\pi}{4}}e^{i\frac{7\pi}{4}}    \right) =
         \frac{7\pi}{4} + \frac{7\pi}{4} -2\pi
    \end{align*}

    Γενικά, αν \( A+B = \left\lbrace a+b: a \in A, b \in B \right\rbrace \), τότε:
    \[
    \mathrm{arg}(z_1z_2) = \mathrm{arg}(z_1) + \mathrm{arg}(z_2)
    \]

    Όμως:
    \begin{align*}
    \mathrm{arg}(z^z) &= \mathrm{arg}(z)+\mathrm{arg}(z) \\
    &\neq 2\mathrm{arg}(z)
    \end{align*}
    διότι:
    \begin{gather*}
    	A = \left\lbrace a_1,a_2,\dots \right\rbrace \\
    	B = \left\lbrace b_1,b_2,\dots \right\rbrace \\
    	A+B = \left\lbrace a+b: a\in A, b\in B \right\rbrace \\
    	A+A = \left\lbrace a_1+a_2:a_1,a_2\in A \right\rbrace \\
    	2A = \left\lbrace 2a:a\in A \right\rbrace \\[.3pt]
    	A = \left\lbrace 1,2,3 \right\rbrace \\
    	B = \left\lbrace 4,5 \right\rbrace \\
    	A+B = \left\lbrace a+b:a\in A, b\in B \right\rbrace
    	= \left\lbrace 1+4,1+5,2+4,2+5,3+4,3+5 \right\rbrace
    	= \left\lbrace 5,6,7,8 \right\rbrace\\
    	A+A = \left\lbrace 2,3,4,5,6 \right\rbrace \\
    	2A = \left\lbrace 2,4,6 \right\rbrace
    \end{gather*}


   	\subsection{n-οστές ρίζες}
   	\[
   	z = a^{\sfrac{1}{n}} \iff z^n=a
   	\]
   	Δηλ. ποιο \( z \) ικανοποιεί αυτή
   	\begin{align*}
   	a &= |a|e^{i\theta} \\
   	z &= re^{i\phi}
   	\end{align*}
   	\begin{align*}
   	& \left( re^{i\phi} \right)^n = |a|e^{i\theta} \\
   	\implies& r^n\cdot e ^{in\phi} = |a|e^{i\theta} \\
   	\implies& r^n\cdot(\cos n\phi+i\sin n\phi) = |a|\cdot(\cos\theta+i\sin\theta)\\
   	\implies& \begin{cases}
   	r^n=|a|\implies r =\sqrt[n]{|a|} \\
   	\left.\begin{array}{l}
   	\cos(n\phi)=\cos\theta \\
   	\sin(n\phi)=\sin\theta \implies
   	\end{array}\right| \implies
   	n\phi = \theta+2k\pi \ \in\mathbb Z \implies \phi=\frac{\theta+2k\pi}{n}
   	\end{cases} \\
   	\implies& z=a^{\sfrac{1}{n}} = \sqrt[n]{|a|}\cdot
   	e^{\sfrac{i(\theta+2k\pi)}{n}} \quad k\in\mathbb Z
   	\end{align*}

   	(Όμως αρκεί να πάρω \( k\in\left\lbrace 0,1,\dots,N-1 \right\rbrace \))

   	\[
   	a^{\sfrac{1}{n}} = \left\lbrace
   	\quad \sqrt[n]{|a|}e^{\sfrac{i\theta}{n}},\
   	\sqrt[n]{|a|}e^{\frac{i\theta+2\pi}{n}},\dots
   	\right.
   	\]
   	
   	\begin{tikzpicture}
   		\draw (-4.5,0) -- (4.5,0);
   		\draw (0,-4.5) -- (0,4.5);
   		
   		\draw (0,0) circle(4);
   		
   		\draw[very thick,->,blue!50!black] (0,0) -- ++(25:4)
   		node[right,black,xshift=2mm] {$u_1= \sqrt[n]{|a|}e^\frac{i\theta+2\pi}{n} $};
   		
   		\draw[very thick,->,blue!50!black] (0,0) -- ++(50:4)
   		node[right,black,xshift=2mm] {$u_0=\sqrt[n]{|a|}\cdot e^{\sfrac{i\theta}{n}}=
   			u_n=\sqrt[n]{|a|}e^\frac{i\theta+2k\pi}{n} $};
   		
   		\draw (1,0) arc (0:25:1) node[midway,right] {$\sfrac{\theta}{n}$};
   		\draw[<->] ++(25:3) arc (25:50:3) node[midway,right] {$\sfrac{2\pi}{n}$};
   		
   		\filldraw ++(-25:4) circle (.7mm) node[right,xshift=2mm]
   		{$u_{n-1}= \sqrt[n]{|a|}e^\frac{i\theta+2(n-1)\pi}{n} $};
   	\end{tikzpicture}

   	\paragraph{Παρ.} \( a^{\sfrac{1}{2}} = 1^{\sfrac{1}{2}} \)
   	\begin{gather*}
   	a=1=1\cdot e^{i0} \quad |a|=1,\theta=0\\
   	u_0 = \sqrt[2]{1}\cdot e^{i\left(\frac{0+2\cdot0\cdot\pi}{2}\right)}=e^{i0}=1 \\
   	u_1 = \sqrt[2]{1}\cdot e^{i\left(\frac{0+2\cdot\pi}{2}\right)}=e^{i\pi}=-1
   	\end{gather*}
   	\begin{tikzpicture}
   		\draw (-2,0) -- (2,0);
   		\draw (0,-2) -- (0,2);
   		
   		\draw (0,0) circle (1.7);
   		\filldraw (1.7,0) circle (2pt) node[below right] {$u_0=1$};
   		\filldraw (-1.7,0) circle (2pt) node[below left] {$u_1$};
   	\end{tikzpicture}

   	\paragraph{Παρ.} \( a^{\sfrac{1}{3}}=1^{\sfrac{1}{3}}=z \)
   	\begin{gather*}
   	a=1=e^{i0},|a|=1,\theta=0 \\
   	u_0 = 1\\
   	u_1 =e^{\sfrac{i2\pi}{3}} = \frac{-1+i\sqrt{3}}{2} \\
   	u_2 =e^{\sfrac{i4\pi}{3}} = \frac{-1-i\sqrt{3}}{2}
   	\end{gather*}
   	\begin{tikzpicture}
   	\draw[->] (-2.3,0) -- (2.3,0);
   	\draw[->] (0,-2.3) -- (0,2.3);
   	
   	\draw (0,0) circle (2);
   	\filldraw (2,0) circle (2pt) node[below right] {$u_0=1$};
   	\filldraw ++(120:2) circle (2pt) node[above left] {$u_1$};
   	\draw (0,0) -- ++(120:2);
   	\filldraw ++(240:2) circle (2pt) node[below left] {$u_2$};
   	\draw (0,0) -- ++(240:2);
   	
   	\draw[->] (0.5,0)  arc(0:120:0.5) node[midway,right] {$\sfrac{2\pi}{3}$};
   	\draw[->] ++(120:0.4)  arc(120:240:0.4) node[midway,below left] {$\sfrac{2\pi}{3}$};
   	\draw[->] ++(240:0.7)  arc(240:360:0.7) node[midway,below right] {$\sfrac{2\pi}{3}$};
   	\end{tikzpicture}
   	\subparagraph{Διαφορετικά}
   	\begin{gather*}
   	1^{\sfrac{1}{3}}=z \iff 1=z^3 \\
   	\iff z^3-1 = 0 \\
   	\iff (z-1)(z^2+z+1)=0 \\
   	\iff (z-1)\left(z+\frac{1-i\sqrt{3}}{2}\right)\left(
   	z+\frac{1+i\sqrt{3}}{2}\right) = 0
   	\end{gather*}

   	\paragraph{Παρ.} \( 1^{\sfrac{1}{11}}=z \iff 1=z^{11} \)
   	\begin{gather*}
   	    \iff z^{11}-1 = 0 \\
   		\iff (z-1)(z^{11}+z^{10}+\cdots+z^1+1) = 0
   	\end{gather*}
   	\[
   	\left\lbrace u_09,u_1,\dots,u_{10} \right\rbrace
   	\]
   	
   	\begin{tikzpicture}
	   	\draw[->] (-2.3,0) -- (2.3,0);
	   	\draw[->] (0,-2.3) -- (0,2.3);
   	
   		\draw[very thick] (0:2) \foreach \x in {32.727,65.454,...,360} {
   			-- (\x:2)
   		} -- cycle (90:2) ;
   	\end{tikzpicture}



    \section{Βασικές μιγαδικές συναρτήσεις}
    \( e^z,\ \log(z) \)
    \[
    e^z \overset{\text{ορισμός}}{=} e^xe^{iy}
    = e^x(\cos y+i\sin y)
    \]

    Ήξερα \( \begin{array}{l}
    e^x: \mathbb R \to \mathbb R  \\
    e^{iy}: \mathbb R \to \mathbb C
    \end{array} \).

    Τώρα η νέα συνάρτηση \( e^z: \mathbb C \to \mathbb C \) και \textbf{γενικεύει}
    τις δύο προηγούμενες συναρτήσεις.

    \paragraph{Παρ.}
    \begin{align*}
    e^{1+i}= ee^i &= e\cdot(\cos 1+i\sin 1) \\ &= e\cdot\cos1+i\cdot e\cdot \sin 1
    \\ \Re\left(e^{1+i}\right) &= e\cos1
    \\ \Im\left(e^{1+i}\right) &= e\sin1
    \end{align*}

    \paragraph{}
    \begin{align*}
    \log(e) &= 1 \\
    \log(-1) &= \log\left(e^{i(\pi+2k\pi)} \right) = i(\pi+2k\pi)
    \end{align*}
    Δηλ. η λογαριθμική συνάρτηση είναι \textbf{πλειότιμη}.
    \begin{align*}
    z &= |z|e^{i\theta} \\
    \log(z) &= \ln\left(|z|\right)+i\theta
    \end{align*}

    \paragraph{Ορίζω}
    \subparagraph{Πλειότιμη} \( \log(z) = \ln\left(|z|\right)+i\mathrm{arg}(z) \)
    \subparagraph{Μονότιμη} \( \mathrm{Log}(z) = \ln\left(|z|\right)+i\arg(z)\)
    είναι ο πρωτεύων κλάδος της πλειότιμης

    \begin{align*}
    \log(1+i) &= \log\left(
        \sqrt{2}e^{i\left(\sfrac{\pi}{4}+2k\pi \right)}
    \right) \\ &=
    \log\left(\sqrt{2}\right) + i\left( \frac{\pi}{4} +2k\pi \right)
    \end{align*}
    \begin{align*}
    \left\lbrace
    \frac{1}{2}\ln(2) + i\left( \frac{\pi}{4} +2 k \pi \right)
     \right\rbrace
    \end{align*}

    \subsection{}
    Από σήμερα: \( \arg(z) \in (-\pi,\pi] \)

    Πριν 7 ημέρες: \( \cancel{e^z=e^{x+iy}=e^x{\cos y+i\sin y}} \)

    \paragraph{Σήμερα:} \( \exp(z) \overset{\text{ορ}}{=}
    1+z+\frac{z^2}{2!}+\dots=\sum_{n=0}^\infty \frac{z^n}{n!} \)

    \begin{theorem*}{}
    	Η \( \exp(z) \) είναι παραγωγίσιμη σε κάθε \( z\in\mathbb C  \)
    	και ικανοποιεί:
    	\begin{enumparen}
    		\item \( \forall z: \left(\exp(z)\right)' = \exp(z) \)
    		\item \( \forall z_1,z_2:\exp(z_1+z_2)=\exp(z_1)\exp(z_2) \)
    		\item \( \forall \theta\in\mathbb R
    		:\exp(i\theta)=\cos\theta+i\sin\theta \)
    	\end{enumparen}
    \end{theorem*}
    \subparagraph{Απόδ.}
    \begin{enumparen}{}
    	\item \begin{align*}
    	\left(\exp(z)\right)'
    	&= \left(1+z+\frac{z^2}{2!}+\frac{z^3}{3!}+\dots\right)'
    	\\ &= 0+1+z+\frac{z^2}{2!}+\dots = \exp(z)
    	\end{align*}
    	\item \( g(z)=\exp(z)\exp(\zeta-z) \)
    	\begin{align*}
    	\od{g}{z} &=
    	\exp(z)\exp(\zeta-z)+\exp(z)\exp(\zeta-z)(-1) = 0
    	\\ \implies g(z) &= c \implies c=g(0)=\exp(\zeta) \\
    	\implies \exp(\zeta)&=g(z)=\exp(z)\exp(\zeta-z)
    	\end{align*}

    	\textbf{Θέτω:} \( z=z_1,\ \zeta=z_1+z_2 \)

    	\textbf{Οπότε:}
    	\[
    	\exp(z_1+z_2) = \exp(z_1)\cdot\exp(z_2)
    	\]

    	\item \begin{align*}
    	\exp(i\theta) &= 1+i\theta + \frac{(i\theta)^2}{2!} + \frac{(i\theta)^3}{3!}
    	+\frac{(i\theta)^4}{4!} + \dots
    	\\ &= \left(1-\frac{\theta^2}{2!}+\frac{\theta^4}{4!}-\dots \right)
    	+ i\cdot\left(
    	\theta-\frac{\theta^3}{3!}+\dots
    	\right)
    	\\ &= \cos\theta+i\sin\theta
    	\end{align*}

    	\paragraph{\( \exp(z) \quad e^z \)}
    	\begin{align*}
    	\exp(1+i)&=1+(1+i)+\frac{(1+i)^2}{2!}+\dots\\[.4pt]
    	e^{1+i} &= 1+(i+1)+\dots\\
    	\text{ή}\quad &\text{ο αρ. $e=2.718$ υψωμένος στη μιγαδική δύναμη } 1+i
    	\end{align*}

    	\paragraph{}
    	\begin{theorem*}{}
    		Η \( \exp(z) \) είναι περιοδική με περίοδο \( 2\pi i \)
    	\end{theorem*}
    	\subparagraph{Απόδ.}
    	\[
    	\exp(z+2\pi i) = \exp(z)\exp(2\pi i) = \exp(z)
    	\]

    	Η εικόνα του συνόλου \( A \subseteq \mathbb C  \) υπό την συνάρτηση \( f(z) \)
    	Δηλ.
    	\[ f(A)=\left\lbrace w=f(z),\ z\in A \right\rbrace \]
    	\paragraph{Παρ.} Να δειχθεί ότι \( \exp(\mathbb C ) =
    	\mathbb C -\left\lbrace 0 \right\rbrace \)

    	Διότι: έστω \( w=re^{i\phi}\in\mathbb C -\left\lbrace 0 \right\rbrace \).

    	Θα βρω \( z=\rho e^{i\theta}=x+iy \) τ.ώ: \( \exp(z)=w \).
    	\begin{gather*}
    	\exp(z)=\exp(x+iy) = \exp(x)\exp(iy) \\
    	w=re^{i\phi} \\
    	\exp(x) = \left|\exp(z)\right|=|w|=r \implies \boxed{x = \ln(r)} \\[.5pt]
    	\arg\left( \exp(z) \right) = \arg(w) \\
    	\arg\left(\exp(z)\right) = \arg\left(\exp(x)\exp(iy)\right) = y \\
    	\arg(w)=\arg(re^{i\phi}) = \phi \\
        \arg\left(\exp(z)\right)=\arg(w) \implies \boxed{y=\phi}
    	\end{gather*}

    	Τελικά \( z=x+iy = \ln(r)+i\phi \) ικανοποιεί \( \exp(z)=re^{i\phi}=w \).
    	Άρα \( \exp(\mathbb C)=\mathbb C-\left\lbrace 0 \right\rbrace \)

    	Στην πραγματικότητα, δεν χρειάζομαι όλο το \( \mathbb C \) διότι:
    	\[
    	\exp(U) = \mathbb C-\left\lbrace 0 \right\rbrace,\quad
    	\text{όπου } U = \left\lbrace
    	    x+iy:x\in\mathbb R,\ y\in(-\pi,\pi]
    	 \right\rbrace
    	\]

    	\begin{tikzpicture}
 	    	\fill[fill=green!20] (-1.8,-1) rectangle (1.8,1);

    		\draw[->,gray] (-2,0) -- (2.2,0);
    		\draw[gray] (0,-2) -- (0,2);

    		\draw[very thick] (-1.8,1) -- (1.8,1);
    		\draw[very thick] (-1.8,-1) -- (1.8,-1);

    		\draw (-1.4,1) node[above] {$\pi$};
    		\draw (-1.4,-1) node[below] {$-\pi$};

    		\draw[ultra thick,->] (3,0) -- (5,0) node[midway,above] {$\exp(z)$};

    		\fill[inner color=green!70!black] (6,-2) rectangle (10,2);
    		\draw[->] (6,0) -- (10,0);
    		\draw (8,-2) -- (8,2);

    		\filldraw[thick,red!80!blue,fill=white] (8,0) circle(4pt);
    	\end{tikzpicture}
    \end{enumparen}
    
    \subsection{Λογαριθμική Συν.}
    \[
    w = \overbrace{\log(z)}^{\mathclap{\text{πλειότιμη}}}
    \iff z=\exp(w)
    \]
    
    \begin{align*}
    	w &= \log(1+i) \\
    	1+i &= \sqrt{2}e^{i\left( \frac{\pi}{4}+2k\pi \right)}\\
    	\log(1+i) &= \log\left( \sqrt{2}e^{i\left( \frac{\pi}{4}+2k \pi\right)} \right) \\
    	&= \ln(\sqrt{2})+\log\left( e^{i\left[ \frac{\pi}{4}+2k\pi \right]} \right) \\
    	&= \ln(\sqrt{2}) + i\left( \frac{\pi}{4}+2k\pi \right)\quad k \int\mathbb Z\\
    	&= \left\lbrace
    	\dots,\ln(\sqrt{2})-i\frac{7\pi}{4},
    	\ln(\sqrt{2})+i\frac{\pi}{4},
    	\ln(\sqrt{2})+i\frac{3\pi}{4},
    	\ln(\sqrt{2})-i\frac{17\pi}{4},\dots
    	\right\rbrace
    \end{align*}
    
    \begin{align*}
    	\boxed{
    		\log(z)=\ln(r)+i\mathrm {arg}(z)
    		} \leftarrow& \text{ πλειότιμη}\\
    	\boxed{
    		\mathrm {Log}(z)=\ln(r)+i\arg(z)
    	} \leftarrow\ & \begin{array}{l}\text{μονότιμη}
    	\\ \text{ασυνεχής για } x\in(-\infty,0]\end{array}
    \end{align*}
    
    %TODO Kehagias Graph 08
    
    
    \subsection{Μιγαδικές δυνάμεις}
    \begin{align*}
    z^c &= e^{\mathrm {log}(z^c)}
    =e^{c\mathrm{log} z} = e^{c\left(
    	\ln\left(|z|\right)+i\mathrm{arg}(z)
    	\right)} \intertext{ή} \\
    z^c &= e^{\mathrm{Log}(z^c)} = e^{c\left(
    	\ln\left(|z|\right)+i\mathrm{Arg}(z)
    	\right)}
    \end{align*}
    
    \begin{align*}
    \underbrace{(1+i)^{2-i}}_{\mathclap{\begin{array}{l}
    		z=1+i\\ c=2-i
    		\end{array}}} &= e^{(2-i)\log(1+i)}
    = e^{(2-i)\left( \ln\left(\sqrt{2}\right)+i\left(
    	\frac{\pi}{4}+2k\pi
    	\right) \right)}
    \\ &= e^{
       	\left(
       	2\ln\left(\sqrt{2}\right)+\left(\frac{\pi}{4}+2k\pi\right)
       	\right)+i\left(
       	-\ln\left(\sqrt{2}\right)+\frac{\pi}{4}+4k\pi
       	\right)
    }
    \\ &= e^{2\ln\left(\sqrt{2}\right)+\frac{\pi}{4}+2k\pi} \cdot
    e^{i\left(
       	-\ln\sqrt{2}+\frac{\pi}{2}+4k\pi
       	\right)}
    \\ &= 2e^{\sfrac{\pi}{4}+2k\pi}\cdot\left[
    \cos\left(
    -\ln\left(\sqrt{2}\right)+\frac{\pi}{2}+\cancel{4k\pi}
    \right)+i\sin\left(
    -\ln\left(\sqrt{2}\right)+\frac{\pi}{2}+\cancel{4k\pi}
    \right)
    \right]
    \end{align*}
    
    \begin{align*}
    \sqrt{1+i} = (1+i)^{\sfrac{1}{2}} &= e^{\sfrac{1}{2}\cdot\log(1+i)} \\
    &= e^{\sfrac{1}{2}\left( \ln\left(\sqrt{2}\right)
    	+i\left( \frac{\pi}{4}+2k\pi \right)
    	 \right)}
    \\ &= e^{\sfrac{1}{2}\ln\left(\sqrt{2}\right)}\cdot e^{
    	\frac{1}{2}\left( \sfrac{\pi}{4}+2k\pi \right)
    	}
    \\ &= \sqrt[4]{2}e^{i\left( \frac{\pi}{8}+k\pi \right)}
    \\ &= \sqrt[4]{2}\left( \cos\left( \frac{\pi}{8}+k\pi \right)
    +i\sin\left( \frac{\pi}{8}+k\pi \right)\right)
    \\ &= \begin{cases}
    \sqrt[4]{2} \left( \cos\left( \frac{\pi}{8} \right)
    +i\sin\left( \frac{\pi}{8} \right)\right) \\
    \sqrt[4]{2} \left( \cos\left( \frac{9\pi}{8} \right)
    +i\sin\left( \frac{9\pi}{8} \right)\right)
    \end{cases}
    \end{align*}
    \begin{align*}
    (-1)^i &= e^{\log\left( (-1)^i \right)}
    =e^{i\log(-1)} = e^{
    	i\left( i(2k+1)\pi \right)
    	} = e^{-(2k+1)\pi}
    \end{align*}
    
    \begin{align*}
    (1+i)^{\sqrt{2}} &= e^{\sqrt{2}\log(1+i)} = e^{\sqrt{2}\left(
    	\ln\left(\sqrt{2}\right)+i\left(\frac{\pi}{4}+2k\pi\right)
    	\right)} \\ &=
    \sqrt{2}^{\sqrt{2}}\cdot e^{i\sqrt{2}\left(\frac{\pi}{4}\right)+2k\pi}
    \\ &=
    \sqrt{2}^{\sqrt{2}}
    \left[
    \cos\left(\frac{\sqrt{2}\pi}{4}+2k\sqrt{2}\pi\right)
    +i\sin\left(\frac{\sqrt{2}\pi}{4}+2k\sqrt{2}\pi\right)
    \right]
    \\ & \cos\left( \frac{\sqrt{2}\pi}{4} \right) = \cos\left(
    \frac{\sqrt{2}\pi}{4}+2k\sqrt{2}\pi
    \right)\\ & \implies \frac{\sqrt{2}\pi}{4}=
    \frac{\sqrt{2}\pi}{4}+2k\sqrt{2}\pi+2m\pi \implies
    2k\sqrt{2}\pi=2m\pi\implies \sqrt{2}=\sfrac{m}{k}
    \end{align*}
    
    \begin{gather*}
    (1+i)^{\sfrac{p}{q}}=\dots\\ m=\lambda q
    \end{gather*}
    
    \paragraph{Παρ.}
    Να βρεθούν οι τιμές του \( n \) τ.ώ:
    \[
    c_n = \sum_{k=0}^n i^k \in \mathbb I
    \]
    
    \subparagraph{}
    \[
    \begin{array}{r|l}
    n & \\ \hline
    0 & 1 \\ 1 & 1+i \\ 2 & 1+i+i^2=1 \\ 3 & 1+i+i^2+i^3=0 \\ 4 & 1
    \\ \hline
    5 & 1+i \\ 6 & i \\ \vdots & \vdots
    \end{array}
    \]
    
    Αρα \( \forall_{n,m}: c_n=c_{n+4m} \)
    
    Οι φανταστικές τιμές του \( c_n \) προκύπτουν για
    \begin{align*}
    	n =& 2,3, \\& 6,7, \\ & 10,11,
    \end{align*}
    
    %TODO Kehagias Graph 09
    \subparagraph{Απ.} \( 
    n\in\left\lbrace m+4l: m\in\left\lbrace 2,3 \right\rbrace
    ,l\in \mathbb N_0
    \right\rbrace
     \)
     
    \paragraph{Παρ.}
    Να λυθεί η \( (1+z)^{2n}=-(1-z)^{2n}\quad n\in\mathbb N \)
    \subparagraph{Λύση}
    Φαίνεται άμεσα ότι \( z\neq 1 \)
    \begin{align*}
    \left(
    \frac{1+z}{1-z}
    \right)^{2n}&=-1 \implies \\
    \frac{1+z}{1-z} &= (-1)^{\sfrac{1}{2n}}
    = \left(e^{i(2k+1)\pi}\right)^{\sfrac{1}{2n}}
    \\ z&=\frac{
    	e^{i(2k+1)\sfrac{\pi}{2n}}-1
    	}{e^{i(2k+1)\sfrac{\pi}{2n}}+1}
    \\ &= \frac{
    	\cos\left((2k+1)\frac{\pi}{2n}\right)+i\sin\left((2k+1)\frac{\pi}{2n}\right)-1
    	}{
    	\cos\left((2k+1)\frac{\pi}{2n}\right)+i\sin\left((2k+1)\frac{\pi}{2n}\right)+1
    	}
    \\ & \underset{\textstyle \begin{cases}
    	\mathsmaller{\cos\phi-1=-2\sin^2\frac{\phi}{2}}\\
    	\mathsmaller{\cos\phi+1=2\cos^2\frac{\phi}{2}}
    	\end{cases}}{=} -\frac{
    	2\sin^2\left( (2k+1)\frac{\pi}{4n} \right)+i2\sin\left((2k+1)\frac{\pi}{4n}\right)\cos\left(
    	(2k+1)\frac{\pi}{4}
    	\right)
    	}{2\cos^2\left( (2k+1)\frac{\pi}{4n} \right)+i2\sin\left((2k+1)\frac{\pi}{4n}\right)\cos\left(
    	(2k+1)\frac{\pi}{4}
    	\right)}
    \\ &= \frac{
    	\sin\left((2k+1)\frac{\pi}{4n}\right)\left[
    	-\sin\left(
    	(2k+1)\frac{\pi}{4n}
    	\right)+i\cos\left(
    	(2k+1)\frac{\pi}{4n}
    	\right)
    	\right]
    	}{\cos\left(
    	(2k+1)\frac{\pi}{4n}
    	\right)\left[
    	\cos\left((2k+1)\frac{\pi}{4n}\right)+i\sin\left(
    	(2k+1)\frac{\pi}{4n}
    	\right)
    	\right]}
    \\ z_k &= i\tan\left((2k+1)\frac{\pi}{4n}\right) \qquad
    k=0,1,\dots,2n-1
    \end{align*}
    
    \subsection{}
    \subsubsection{Η γραμμική απεικόνιση}
    \paragraph{Τύπος:} \( f(z) = az+b,\quad a,b\in\mathbb C  \)
    
    Προφανώς η \( w = f(z)=az+b \) είναι 1-1 συνάρτηση και εύκολα βρίσκουμε την
    αντίστροφή της λύνοντας την \( w=az+b \) ως προς \( z \).
    
    Έτσι: \( f: \mathbb C \to \mathbb C : f(z)=az+b \) και μπορώ να την επεκτείνω
    στο \( \bar{\mathbb C}  \) με 1-1 τρόπο θέτοντας \[
    f(\infty) = \infty
    \]
    
    \subparagraph{Γεωμετρική ερμηνεία}
    %TODO Kehagias Graph 10
    %TODO Kehagias Graph 11
    
    \begin{itemize}
    	\item Προφανώς, από τη δομή της, η γραμμική απεικόνιση απεικονίζει
    	ευθείες σε ευθείες και κύκλους σε κύκλους.
    \end{itemize}
    \paragraph{Ερώτηση}
    Έστω \( f(z) = az+b \).
    
    Αν \( Ax+By+\Gamma = 0 \) ευθεία τυχαία, βρείτε πού αυτή απεικονίζεται
    μέσω της \( f(z) \).
    
    \begin{align*}
    w &= u+iv = az+b \\
    &= u(x,y) + iv(x,y) = a(x+iy) + b
    \end{align*}
    \begin{itemize}
    	\item \( w = az+b \iff z = \frac{w-b}{a} \iff
    	x+ i y = \frac{u+iv-(b_0+ib_1)}{a_0+ia_1}
    	= \frac{\left[(u-b_0)+i(v-b_1)\middle]\middle[a_0-a_1i \right]}{|a|^2}
    	= \frac{(u-b_0)a_0+a_1(v+b_1)}{|a|^2}+i
    	\frac{-a_1(u-b_0)+a_0(v-b_1)}{|a|^2}
    	 \)
    \end{itemize}
    Άρα:
    \begin{align*}
    x &= \frac{a_0u+a_1v-(a_0b_0+a_1b_1)}{|a|^2} \\
    y &= \frac{-a_1u+a_0v-(a_1b_0+a_0b_1)}{|a|^2}
    \end{align*}
    (μπορεί να είναι λάθος)
    
    \begin{align*}
    & Ax+By+\Gamma = 0 \\ \iff &
    A\left( a_0u+a_1v - (a_0b_0)-(a_0b_0+a_1b_1 \right)
    + B\left( -a_1u+a_0v-(a_1b_0+a_0b_1) \right) +\Gamma|a|^2 = 0
    \end{align*}
    
    \subsubsection{Αντιστροφή \( f(z)=\frac{1}{z} \quad (z \neq 0) \)}
    Προφανώς: \(  \mathbb C - \left\lbrace 0 \right\rbrace \to
    \mathbb C  : f(z) = \frac{1}{z}
     \) \\ και μορεί να επεκταθεί στο \( \bar{\mathbb C} \) θέτοντας
     \( f(0) = \infty \) και \( f(\infty) = 0 \).
     
     Εννοείται ότι είναι 1-1 με \( w = \frac{1}{z} \iff \frac{1}{w} \)
     δηλ. για \( \begin{cases}
     w=u+iv \\ z = x+iy
     \end{cases} \) έχουμε: \begin{equation} \label{reveq}
     x+iy = \frac{u-iv}{u^2+v^2} \iff \left|\begin{array}{l}
     x = \frac{u}{u^2+v^2} \\ y = \frac{-v}{u^2+v^2} 
     \end{array} \right.
     \end{equation}
     
     Έτσι φαίνεται ότι η συνάρτηση αυτή απεικονίζει ευθείες σε ευθείες ή κύκλους,
     και κύκλους σε ευθείες ή κύκλους.
     
     Πράγματι, αν
     \( Ax+By+\Gamma = 0 \) τυχαία ευθεία στο επίπεδο του \( z \), τότε
     από \eqref{reveq}:
     \begin{align*}
     & A\frac{u}{u^2+v^2} - B \frac{v}{u^2+v^2} + \Gamma = 0
     \\ \iff & \boxed{
     	Au-Bv + \Gamma\left( u^2+v^2 \right) = 0
     	}
     \end{align*}
     
     \begin{itemize}
     	\item \( \underline{\Gamma = 0} \) τότε \( Au-By = 0 \) 
     	άρα ευθεία απεικον. σε ευθεία, ενώ:
     	\item \( \underline{\Gamma \neq 0} \) τότε \( 
     	u^2+v^2+\frac{A}{\Gamma}u-\frac{B}{\Gamma}v = 0
     	 \) \textbf{δηλ.} κύκλος
     	 \[
     	 x^2+y^2+Ax+By+\Gamma = 0
     	 \]
     \end{itemize}
     
     Η\( f(z) = \frac{1}{z} \) απεικονίζει το εσωτερικό μοναδιαίου κύκλου
     \( |z| = 1 \) με κέντρο το \( z=0 \), στο εξωτερικό του (με 1-1 τρόπο, και
     αντιστρόφως).
     
     \textbf{Πράγματι:}
     \begin{align*}
     & z: |z| < 1, \text{ τότε } f(z) = \frac{1}{z} \text{ με } \left|f(z)\right|
     = \frac{1}{|z|}>1 \implies \left|f(z)\right|> 1 \\
     & z: |z| > 1, \text{ τότε } f(z) = \frac{1}{z} \text{ με } \left|f(z)\right|
      = \frac{1}{|z|}<1 \implies \left|f(z)\right|< 1 \\
     & z: |z| = 1, \text{ τότε } f(z) = \frac{1}{z} = \frac{\bar z}{z\bar z}
     = \frac{\bar z}{|z|^2} = \bar z
     \end{align*}
     
     \subsubsection{Μετασχ. M\"obius}
     Καλούμε ρητογραμμικό μετασχηματισμό (M\"obius) κάθε συνάρτηση
     \[
     f(z) = \frac{az+b}{cz+d}\qquad (\text{με } ad-bc \neq 0)
     \]
     
     Προφανώς: \( f: \mathbb C - \left\lbrace - \sfrac{d}{c}  \right\rbrace 
     \to \mathbb C  - \left\lbrace \frac{a}{c} \right\rbrace
     \), αν \( c \neq 0 \)
     \\ ή \( f: \mathbb C \to \mathbb C \) \quad αν \( c = 0 \)
    
     H \( f \) είναι 1-1 (εύκολο).
     
     \begin{itemize}
     	\item
     	Αποδεικνύεται ότι ο μετασχ. M\"obius είναι σύνθεση \underline{διαστολής},
     	\underline{περιστροφής}, \underline{μετάθεσης} και \underline{αντιστροφής},
     	άρα απεικονίζει ευθείες σε ευθείες
     	ή κύκλους, και κύκλους σε ευθείες ή κύκλους.
     	
     	\item
     	Ο μετασχ. M\"obius (στην περίπτωση ευθείας ή κύκλου) απεικονίζει
     	συμπληρωματικούς τόπους σε συμπληρωματικούς τόπους.
     	\subparagraph{π.χ.}
     	%TODO Kehagias Graph 12
     	
     	\item
     	Αποδεικνύεται ότι \underline{υπάρχει} ΜΟΝΑΔΙΚΟΣ μετασχ. M\"obius που
     	απεικονίζει ΤΡΙΑ σημεία \( z_1, z_2, z_3 \) σε ΤΡΙΑ ΑΛΛΑ σημεία
     	\( w_1 = f(z_1),\ w_2 = f(z_2),\ w_3 = f(z_3) \), και έχει τη μορφή:
     	\[
     	\frac{(w-w_1)(w_2-w_3)}{(w-w_3)(w_2-w_1)}=
     	\frac{(z-z_1)(z_2-z_3)}{(z-z_3)(z_2-z_1)}.
     	\]
     \end{itemize}
     
     \subsubsection{Τριγωνομετρικές και αντίστροφές τους}
     
     \begin{itemize}
     	\item Η συνάρτηση
     	\[
     	\sin: \mathbb C \to \mathbb C : \sin z = \frac{e^{iz}-e^{-iz}}{2}
     	\] είναι 2\( \pi \)-περιοδική, άρα μη αντιστρέψιμη στο \( \mathbb C  \)
     	
     	Έστω \( 
     	E_k = \left\lbrace 
     	x+iy: \kappa \pi -\frac{\pi}{2}<x<\kappa\pi+\frac{\pi}{2},y\in\mathbb R
     	 \right\rbrace, k \in \mathbb Z
     	 \) είναι "κατακόρυφες λωρίδες". Για \underline{\( k=0 \)}, έχω:
     	 %TODO Kehagias Graph 13
     	 
     	 Τότε η \( \sin z \) γίνεται 1-1 με πεδίο τιμών το σύνολο
     	 \[
     	 A = \mathbb C - \left\lbrace u+iv: |u| \geq 1 \text{ και }
     	 v = 0
     	  \right\rbrace
     	 \]
     	 
     	 Έτσι η \( \sin: E_k \to A \) είναι 1-1 (για κάθε συγκεκριμένο \( k\in
     	 \mathbb Z
     	  \)), άρα αντιστρέψιμη.
     	  
     	  \begin{itemize}
     	  	\item Από \( w=\sin z \) παίρνω: \[
     	  	\left| \begin{array}{l}
     	  	u(x,y) = \sin x \cosh y \\ v(x,y) = \cos x\sinh y
     	  	\end{array} \right.
     	  	\]
     	  	\item Για \( x=\frac{\pi}{2},\ y\in\mathbb R 
     	  	\xrightarrow[\text{απεικονίζεται}]{\sin z} (\cosh y,0 )
     	  	 \)
     	  	 
     	  	 Αλλά \( \cosh y \) δεν είναι 1-1 \( \forall y \), επομένως η
     	  	 \( \sin z \) ΔΕΝ μπορεί να είναι 1-1 πάνω στην \( x=\frac{\pi}{2} \), η
     	  	 οποία εξαιρείται από το πεδίο ορισμού. Έτσι, από το πεδίο τιμών,
     	  	 εξαιρείται η ημιευθεία \[ \left\lbrace
     	  	 u+iv: u \geq 1,\ v = 0 \right\rbrace
     	  	 \]
     	  	 \item Για \( x=-\frac{\pi}{2} \), ομοίως εξαιρείται η ημιευθεία \[
     	  	 \left\lbrace
     	  	 u+iv: u \leq 1,\ v = 0 \right\rbrace
     	  	 \]
     	  	 \item Για \underline{\( x = 0 \)}, \( y\in\mathbb R
     	  	 \xrightarrow[\text{απεικον.}]{\sin z}
     	  	 (0, \sinh y)
     	  	  \) και επειδή \( \sinh y \) 1-1 γν. αύξ. με πεδίο τιμών το
     	  	  \( \mathbb R \ \forall y \) η \( x = 0 \) απειον. στην \( u=0 \).
     	  	 \item Έστω \( x=a \quad (a \neq \pm \frac{\pi}{2}),\ a\neq 0 \).
     	  	 
     	  	 \textbf{Τότε: } \( 
     	  	 \left|
     	  	 \begin{array}{l}
     	  	 u = \sin a \cosh y \\ v = \cos a \sinh y
     	  	 \end{array}
     	  	  \right. \implies \boxed{ \frac{u^2}{\sin^2 a} - \frac{v^2}{\cos^2 a} = 1 }
     	  	  \). Αν \( a \in \left( 0,\frac{\pi}{2} \right) \), τότε \( 
     	  	  \left| \begin{array}{l}
     	  	  u > 0 \\ v \in \mathbb R
     	  	  \end{array}  \right.
     	  	   \), και αντίστοιχα για \( a \in \left( -\frac{\pi}{2},0 \right) \).
     	  	   
     	  	   Έτσι ορίζουμε:
     	  	   \[
     	  	   \mathrm{τοξημ } = \arcsin: E' \to E_k:
     	  	   \]
     	  	   \begin{gather*}
     	  	   \arcsin z = w \iff z = \sin w = \frac{e^{iw}-e^{-iw}}{2} \\
     	  	   \iff e^{2iw}-2ize^{iw}-1 = 0 \iff e^{iw} = \frac{2iz+\sqrt{4-4z^2}}{2}
     	  	   \\ \implies iw = \log\left( z+\frac{2}{\sqrt{1-z^2}} \right) \\
     	  	   \implies \boxed{
     	  	   	\arcsin z = \log\left(
     	  	   	z+\frac{2}{\sqrt{1-z^2}}
     	  	   	\right)
     	  	   	}
     	  	   \end{gather*}
     	  \end{itemize}
     	  
     \end{itemize}
     
     \begin{infobox}{}
     	\begin{itemize}
     		\item \textbf{Τρίτη 22/11} Ατρέας, 2 τμήματα
     		\item \textbf{Πέμπτη 24/11} Κεχαγιάς, 2 τμήματα
     		\item \textbf{Παρασκευή 25/11} Κεχαγιάς, 2 τμήματα
     	\end{itemize}
     \end{infobox}
     
     \section{Ακολουθίες \& Σειρές\\(Μιγαδικών αριθμών/συναρτήσεων)}
     \begin{defn*}{}
     	Ακολουθία \(\displaystyle \left( u_n(z) \right)_{n=1}^\infty \)
     \end{defn*}
     \begin{defn*}{}
     	Λέμε ότι η \( u_n(z) \) τείνει σε \( u(z) \).
     	
     	Γράφουμε \( \displaystyle \lim_{n\to \infty} u_n(z)=u(z) \)
     	\[
     	\forall z,\forall\epsilon > 0 \ \exists N_{\epsilon,z}:
     	n \geq N{\epsilon,z} \implies \left| u_n(z)-u(z) \right|<\epsilon
     	\]
     \end{defn*}
     \paragraph{Παρ.}
     \( u_n(z) = 1 +\frac{z}{n} \)
     \[ \lim_{n\to \infty} = 1 \text{ διότι } \]
     \begin{align*}
     \forall z\in\mathbb C , \forall \epsilon >0:
     n \geq \frac{|z|}{\epsilon}+1 &\implies \left|
     \underbrace{1+\frac{z}{n}-1}
     \right| < \epsilon \\ &\iff \left| \frac{z}{n} \right| < \epsilon
     \\ &\iff n> \frac{|z|}{\epsilon}
     \end{align*}
     \begin{defn*}{}
     	Έστω ακολουθία \( \left(u_n(z)\right)_{n=1}^\infty \)
     	
     	Ορίζω νέα ακολουθία \( \left( S_1(z) \right)_{n=1}^\infty \)
     	ως εξής: \( \begin{array}{l}
     		S_1(z) = u_1(z) \\
     		S_2(z) = u_1(z)+u_2(z) \\
     		\qquad \cdots \qquad \\
     		S_n(z) = u_1(z)+u_2(z)+\dots+u_n(z)
     	\end{array} \)
     	
     	Εάν \(\displaystyle \exists \lim_{n\to \infty} S_n(z) = S(z) \)
     	γράφω \( \displaystyle \sum_{n=1}^\infty u_n(z)=S(z) \) και το
     	ονομάζω \textbf{σειρά}.
     \end{defn*}
     
     \paragraph{Παρ.}
     για \( n\in\mathbb N \) ορίζω \( u_n(z)=z^n\cdot (1-z) \). Τότε
     \begin{align*}
     \sum_{n=1}^\infty u_n(z) &= \sum_{n=1}^\infty z^n(1-z)
     \\ &= \lim_{N\to \infty} \sum_{n=1}^N \left( z^n - z^{n+1} \right)
     \\ &= \lim_{N\to\infty}\left(z-z^2+z^2-z^3+z^3-z^4+\dots-z^{N+1}\right)
     \\ &= \lim_{N\to\infty}\left(z-z^{N+1}\right) = z-\lim_{N\to \infty} z^{N=1}
     \\ &\overset{\text{Θέτω } z=re^{i\theta}}{=} z-\lim_{N\to \infty} r^Ne^{iN\theta}
     \\ &= z \text{ όταν } |z|<1
     \end{align*}
     
     Τελικά \[
     \sum_{n=1}^\infty z^n\cdot(1-z) = \begin{cases}
     z &\text{όταν } |z| < 1 \\
     \text{δεν ορίζεται } &\text{όταν } |z| \geq 1
     \end{cases}
     \]
     
     (Για \( z=0 \),  \( \sum_{n=0}^\infty = 0^n(1-z) = 0 = z \))
     
     Ισχύει \(\displaystyle |z|<1 \implies \lim_{N\to \infty} z^N = 0 \),
     διότι
     \begin{align*}
     \forall z \text{ με } |z|<1, \epsilon>0
     \quad n \geq \frac{\ln e}{\ln|z|}+1 &\implies \left|z^n\right| < \epsilon
     \\ &\iff |z|^n < \epsilon
     \\ &\iff n\cdot\ln|z| < \ln e
     \\ &\iff n < \frac{\ln e}{\ln|z|}
     \end{align*}
     
     \begin{defn*}{}
     	Λέω ότι η \( \displaystyle \sum_{n=1}^\infty u_n(z) \)
     	\textbf{συγκλίνει απολύτως} ανν \(\displaystyle 
     	\sum_{n=1}^\infty \left|u_n(z)\right| \) συγκλίνει.
     \end{defn*}
     
     \begin{defn*}{}
     	Λέμε ότι η \( \displaystyle \left( u_n(z) \right)_{n=1}^\infty \)
     	\textbf{συγκλίνει ομοιόμορφα} στην \( u(z) \) ανν
     	\[
     	\forall z,\forall \epsilon >0 \exists
     	\underbrace{N_\epsilon}_{\mathclap{\text{Το $N_e$ δεν εξαρτάται από το $z$}}}
     	:\quad n \geq N_\epsilon \implies \left|
     	u_n(z)-u(z)
     	\right| < \epsilon
     	\]
     	
     	Παρόμοια πράγματα λέμε και για την \( \displaystyle \boxed{
     		\sum_{n=1}^\infty u_n(z)
     		} \)
     \end{defn*}
     \paragraph{Παρ.}
     Η \( \sum_{n=1}^\infty z^n\cdot(1-z) \) συγκλίνει ομοιόμορφα για κάθε \( z \)
     με \( |z| \leq \frac{1}{z} \)
     
     \textbf{Διότι}
     \[
     \lim_{N\to \infty}\sum_{n=1}^N z^n(1-z)=z-\lim_{N\to \infty}z^{N+1}
     \]
     \(\rightarrow\) Αρκεί να δείξω ότι \( z^{N+1}\to 0 \) ομοιόμορφα.
     \begin{align*}
     \forall z,|z|\leq\frac{1}{2}, \forall \epsilon>0, \forall n \geq
     \frac{\ln \epsilon}{\ln|z|}+1\implies\left|z^{N+1}\right|<\epsilon
     \end{align*}
     
     Ισχυρίζομαι ότι
     \begin{align*}
     \forall z,|z|\leq\frac{1}{z}, \forall \epsilon>0, \forall n \geq
     \frac{\ln \epsilon}{\ln\frac{1}{2}}+1\implies\left|z^{N+1}\right|<\epsilon
     \end{align*}
     διότι \( \displaystyle
     |z| \leq \frac{1}{2}\implies
     \frac{\ln\epsilon}{\ln\frac{1}{2}} \geq \frac{\ln\epsilon}{\ln|z|} \)
     
     \begin{theorem*}[width=.7\textwidth]{}
     	Έστω \( \left(u_n(z)\right)_{n=1}^\infty \) ακολουθία συνεχών συναρτήσεων
     	και \( \sum_{n=1}^\infty u_n(z)=u(z) \) ομοιόμορφα στο χωρίο \( \Delta \).
     	
     	Τότε \(\displaystyle
     	\int_c u(z)\dif z = \int_c \sum_{n=1}^\infty u_n(z)\dif z 
     	= \sum_{n=1}^\infty \int_c u_n(z)\dif z
     	\)
     \end{theorem*}
     \begin{theorem*}[width=.7\textwidth]{}
     	Θέτω \( \left( u_n(z) \right)_{n=1}^\infty \) ακολουθία αναλυτικών (ολόμορφων)
     	συναρτήσεων και \( \sum_{n=1}^\infty u_n(z) = u(z)\) συγκλίνει ομοιόμορφα
     	στο χωρίο \( D \). Τότε \[
     	\od{u}{z} = \od{}{z}\left(\sum u_n(z) \right) = \sum_{u=1}^\infty
     	\od{u}{z}
     	\]
     \end{theorem*}
     
     \begin{theorem*}{}
     	Αν συγκλίνει η \( \displaystyle \sum_{n=1}^\infty \left|u_n(z)\right| \),
     	τότε συγκλίνει και η \( \displaystyle_{n=1}^\infty u_n(z)\)
     	
     	Το αντίστροφο δεν ισχύει πάντα.
     \end{theorem*}
     
     \begin{theorem*}{}
     	Αν συγκλίνει η \( \displaystyle \sum_{n=1}^\infty \left|v_n(z)\right| \)
     	και \( \forall n,z:\left|u_n(z)\right| \leq \left|v_n(z)\right|\),
        τότε συγκλίνει και η \( \displaystyle \sum_{n=1}^\infty \left|
        u_n(z)
        \right| \)
     \end{theorem*}
     
     \begin{theorem*}{Κριτήριο του λόγου}
     	Έστω \( \displaystyle L(z) = \lim_{n\to \infty} \left|
     	\frac{u_{n+1}(z)}{u_n(z)}
     	\right| \). Τότε
     	\[
     	\begin{array}{ll}
     	L(z) < 1 & \text{η $\displaystyle\sum_{n=1}^\infty$ συγκλίνει} \\
     	L(z) > 1 & \text{η $\displaystyle\sum_{n=1}^\infty$ δεν συγκλίνει} \\
     	L(z) = 1 & \text{δεν μπορούμε να αποφανθούμε} \\
     	\end{array}
     	\]
     \end{theorem*}
 
      \begin{theorem*}{Κριτήριο της ρίζας}
       	Έστω \( \displaystyle 
       	L(z) = \lim_{n\to \infty} \sqrt[n]{\left|u_n(z)\right|}. \). Τότε
       	\[
       	\begin{array}{ll}
       	L(z) < 1 & \text{η $\displaystyle\sum_{n=1}^\infty$ συγκλίνει} \\
       	L(z) > 1 & \text{η $\displaystyle\sum_{n=1}^\infty$ δεν συγκλίνει} \\
       	L(z) = 1 & \text{δεν μπορούμε να αποφανθούμε} \\
       	\end{array}
       	\]
      \end{theorem*}
      
     \paragraph{Παρ.}
     Η \( \sum_{n=1}^\infty \frac{z^n}{n\cdot(n+1)} \) συγκλίνει όταν
     \( |z|\leq 1 \), δεν συγκλίνει όταν \( |z|>1 \)
     
     \begin{align*}
     \text{Θέτω } L(z) &= \lim_{n\to \infty} \left|
     \frac{\sfrac{z^{n+1}}{(n+1)(n+2)} }{\sfrac{z^n}{n(n+1)} }
     \right| = \lim_{n\to \infty} |z|\frac{n}{n+2}=|z|
     \end{align*}
     
     Όταν \( |z|=1 \), τότε
     \begin{align*}
     \sum_{n=1}^\infty \left| \frac{z^n}{n(n+1)} \right|
     = \sum_{n=1}^\infty \frac{|z|^n}{n(n+1)} = \sum_{n=1}^\infty \frac{1}{n(n+1)}
     = \sum_{n=1}^\infty \left( \frac{1}{n}-\frac{1}{n+1} \right)=1
     \end{align*}
     
     Αφού συγκλίνει απολύτως, συγκλίνει (για κάθε \( z:|z|=1 \))
     
     \paragraph{Παρ.} \( \sum_{n=1}^\infty z^n \) συγκλίνει όταν \( |z|<1 \)
     \begin{align*}
     \sum_{n=1}^N z^n &= z+z^2+z^3+\dots+z^N \\
     &= z\cdot(1+z+z^2+\dots+z^{N-1}) \\
     &= z\cdot\frac{1-z^N}{1-z} = \frac{z}{1-z}\left(1-z^N\right)
     \\ &= \frac{z}{1-z} \text{ για } N\to \infty \text{ όταν } |z| < 1
     \end{align*}
     
     Για \( |z|>1 \) δεν συγκλίνει.
     
     Για \( |z|=1 \) δεν συγκλίνει (τουλάιστον για κάποιες τιμές).
     
     Εναλλακτικά, με κριτήριο ρίζας: \( L(z)=\lim_{n\to \infty}
     \sqrt[n]{\left|z^n\right|} = |z|
      \)
      
     και κριτήριο λόγου: \( L(z) = \lim_{n\to\infty}\left|
     \frac{z^{n+1}}{z^n}\right| =|z| \)
    
     
     \paragraph{Παρ.}
     \( \sum_{n=1}^\infty \frac{(z+2)^n}{(n+1)^3 4^n} \)
     
     \begin{align*}
     L(z) &= \lim_{n\to \infty}\frac{\frac{(z+2)^{n+1}}{(n+1)^3 4^n}}{
     	\frac{(z+2)^n}{(n+1)^3 4^n}
     	} = \lim_{n\to\infty} \frac{|z+2|}{4}\left(
     	\frac{n+1}{n+2}
     	\right)^3= \frac{|z+2|}{4} \to \begin{cases}
     	|z+2| < 4 & \text{συγκλίνει} \\
     	|z+2| = 4 & * \text{} \\
     	|z+2| > 4 & \text{δεν συγκλίνει}
     	\end{cases}
     \end{align*}
     
     Αν \( |z+2|= 4 \), ελέγχουμε αν συγκλίνει απολύτως: \[ 
     \sum_{n=1}^\infty \left| \frac{(z+2)^n}{(n+1)^3 4^n} \right|
     = \sum_{n=1}^\infty \frac{4^n}{(n+1) 4^n} = \sum_{n=1}^\infty
     \frac{1}{(n+1)^2} < \sum_{n=1}^\infty \frac{1}{n^3} < \infty
      \]
   
   \paragraph{Παρ.} \( \sum_{n=1}^\infty n!z^n \)
   
   \[
   L(z) = \lim_{n\to \infty} \frac{(n+1)!|z|^{n+1}}{n!|z|^n}
   = \lim_{n\to \infty}(n+1)|z|=\begin{cases}
   \infty \quad & |z| \neq 0 \\
   0 \quad & |z| = 0
   \end{cases}
   \]
   
  \begin{infobox}{}
     	\begin{itemize}
     		\item
     		\( \displaystyle \sum\frac{z^n}{n(n+1)} \) συγκλίνει για \( |z|\leq 1 \)
     		\item
     		\( \displaystyle \sum z^n \) συγκλίνει για \( |z|\leq 1 \)
     		\item
     		\( \displaystyle \sum \frac{(z+2)^n}{(n+1)^3 4^n} \)
     		συγκλίνει για \( |z+2| \leq 4 \)
     		\item
     		\( \displaystyle \sum n!z^n \) συγκλίνει για \( |z| = 0 \)
     	\end{itemize}
  \end{infobox}
  
  \begin{defn*}{}
  	Δυναμοσειρά: \( \displaystyle \sum_{n=0}^\infty a_n(z-z_0)^n \)
  \end{defn*}
  
  \begin{theorem*}{}
  	Για κάθε δυναμοσειρά \( \displaystyle \sum_{n=0}^\infty a_n(z-z_0)^n \) υπάρχει
  	\( R \geq 0 \), τ.ώ:
  	\[
  	\begin{array}{ll}
  	|z-z_0|<R & \text{η δυναμοσειρά συγκλίνει ομοιόμορφα} \\
  	|z-z_0|>R & \text{η δυναμοσειρά δεν συγκλίνει} \\
  	\end{array}
  	\]
  	Στο \( |z-z_0|=R \) η ΔΣ μπορεί να συγκλίνει σε κάποια σημεία και να μην
  	συγκλίνει σε άλλα. Αυτό παρατηρούμε και στα 4 παραπάνω παραδείγματα.
  \end{theorem*}
  
  Όταν \( \displaystyle \sum_{n=0}^\infty a_n(z-z_0)^n \rightarrow \) σειρά Taylor.
  
  Όταν \( \displaystyle \sum_{n=-\infty}^\infty a_n(z-z_0)^n \rightarrow \)
  σειρά Laurent.
  
  Η σειρά Laurent περιλαμβάνει την Taylor ως ειδική περίπτωση.
  
  Αν είναι "γνήσια" σειρά Laurent (\( a_n \neq 0 \) για κάποια αρνητικά \( n \)),
  τότε το \( z_0 \) λέγεται ανώμαλο σημείο της σειράς Laurent.
  
  \begin{infobox}{}
  	Από την ΑΛΛΗ ΕΒΔΟΜΑΔΑ \\
  	ΤΡΙ \textbf{και} ΠΕ στον ΑΤΡΕΙΑ
  	\textbf{ΚΑΙ ΤΑ 2} τμήματα
  \end{infobox}
  
  \begin{theorem*}{}
  	Για κάθε ΔΣ \( \displaystyle f(z) = \boxed{ \sum_{n=0}^\infty a_n (z-z_0)^n } \)
  	υπάρχει αριθμός \( R \geq 0 \) (ακτίνα σύγκλισης) \textbf{τ.ώ}
  	\begin{enumgreekparen}
  		\item Η ΔΣ συγκλίνει ομοιόμορφα και απόλυτα στο
  		\( \underbrace{D_R}_{\mathclap{\text{δίσκος σύγκλισης}}}(z_0) =
  		\left\lbrace z: |z-z_0| < R \right\rbrace
  		\)
  		\item Η ΔΣ αποκλίνει στο
  		\( \left\lbrace z:|z-z_0| > R \right\rbrace \)
  		\item Σε κάθε σημείο του συνόρου του δίσκου σύγκλισης
  		\( \left\lbrace z:|z-z_0| = R \right\rbrace \)
  		η ΔΣ μπορεί να συγκλίνει ή να αποκλίνει
  	\end{enumgreekparen}
  \end{theorem*}
  
  \begin{theorem*}{}
    Για κάθε ΔΣ \( f(z) = \sum_{n=0}  a_n(z-z_0)^n \), \(\forall z \in D_R(z_0) \)
    ισχύουν:
    \begin{enumgreekparen}
    	\item 
    	\[
    	\frac{\dif\; f}{\dif z} = 
    	\sum_{n=0}^\infty n\cdot a_n \cdot (z-z_0)^{n-1}
    	\]
    	
    	\item Για κάθε \( C \subseteq D_R(z_0) \)
    	\[
    	\int_C f(z) \dif z = \sum_{n=0}^\infty a_n \int_C (z-z_0)^n \dif z
    	\]
    \end{enumgreekparen}
  \end{theorem*}
  
  \paragraph{}
  \begin{theorem*}{}
  	Έστω \( f(z) \) αναλυτική στο εσωτερικό κλειστής καμπύλης \( C \).
  	Έστω \( z_0,\ z \) σημεία στο εσωτερικό της \( C \). Τότε
  	\begin{align*}
  	\Aboxed{f(z) &= \sum_{n=0}^\infty \frac{f^{(n)}}{n!} (z-z_0)^n}
  	\\ \Aboxed{
  		\qquad &= \sum_{n=0}^\infty \frac{1}{2\pi i} \left(
  		    \oint_{C'} \frac{f(z)}{(z-z_0)^{n-1}} \dif z
  		\right)(z-z_0)^n
  		}
  	\end{align*}
  	%TODO Kehagias Graph 13
  \end{theorem*}
  
  \paragraph{ΠΑΡ.}
  Να βρεθεί η σειρά Taylor της \( f(z) = \sin z \), γύρω από το \( z_0 = 0 \).
  \subparagraph{ΛΥΣΗ}
  \[
  \begin{array}{rlrl}
  f(z) &= \sin(z)  \qquad & f(0)&=0                    \\
  f'(z) &= \cos(z) & f'(0) &= 1                     \\
  f''(z) &= -\sin(z) & f''(0) &= 0                  \\
  f'''(z) &= -\cos(z) & f'''(0) &= -1
  \end{array}
  \]
  \begin{align*}
  \sin(z) &= 0(z-\cancelto{0}{z_0})^0 + \frac{1}{1!}(z-\cancelto{0}{z_0})^1
  + \frac{0}{2!}(z-\cancelto{0}{z_0})^2 + \frac{1}{3!}(z-\cancelto{0}{z_0})^3+\dots
  \\ \sin(z) &= z - \frac{z^3}{3!} + \frac{z^5}{5!} + \dots
  \end{align*}
  
  \paragraph{ΠΑΡ.}
  Να βρεθεί η σειρά Taylor
  \subparagraph{Α' τρόπος}
  \begin{align*}
  f\left(\sfrac{\pi}{3} \right) = \sfrac{\sqrt{3}}{2} \qquad &,
  f''\left(\sfrac{\pi}{3} \right) = -\sfrac{\sqrt{3}}{2} \\
  f'\left(\sfrac{\pi}{3} \right) = \sfrac{1}{2} \quad &,
  f'''\left(\sfrac{\pi}{3} \right) = -\sfrac{1}{2},\dots\end{align*}\[
  \\ \sin(z)=\frac{\sqrt{3}}{2}+\frac{1}{2}\left(z-\sfrac{\pi}{3} \right)
  -\frac{\sqrt{3}}{2\cdot 2!} \left(z-\sfrac{\pi}{3} \right)^2
  -\frac{1}{2\cdot 3!}\left( z-\sfrac{\pi}{3}  \right)^3
  \]
  \subparagraph{Β' τρόπος}
  \( u = z-\sfrac{\pi}{3} \implies z=u+\sfrac{\pi}{3}   \)
  
  \begin{align*}
  \sin(z) = \sin(u+\sfrac{\pi}{3} ) &=
  \sinh \cancelto{\sfrac{1}{2} }{\cos\frac{\pi}{3}}
  +\cosh \cancelto{\sfrac{\sqrt{3}}{2} }{\sin\sfrac{\pi}{3}}
  \\ &= \frac{1}{2}\left(
  u-\sfrac{u^3}{3!} +\sfrac{u^5}{5!} -\dots
  \right) + \frac{\sqrt{3}}{2} \cdot \left(
  1-\frac{u^2}{2!} + \frac{u^4}{4!} - \dots
  \right)
  \\ &= \frac{\sqrt{3}}{2} + \frac{1}{2}u-\frac{\sqrt{3}}{2\cdot 2!}u^2
  -\frac{1}{2\cdot 3!}u^3+\dots
  \\ &= \frac{\sqrt{3}}{2}+\frac{1}{2}\left( z-\frac{\pi}{3} \right)
  -\frac{\sqrt{3}}{2\cdot 2!} \left( z-\frac{\pi}{3} \right)^2 - \dots
  \end{align*}
  
  \paragraph{ΠΑΡ.}
  Να βρεθεί η σειρά Taylor της \( f(z) = \frac{1}{1-z} \) γύρω από το \( z_0 = 2 \)
  \subparagraph{ΛΥΣΗ}
  \begin{align*}
  z-2 &= u \\
  z-1 &= u+ 1 \\
  1-z &= -(1+u) \\
  & \forall u:|u|<1:\frac{1}{1+u} = 1-u+u^2-u^3+\dots
  \end{align*}
  \begin{align*}
  \frac{1}{1-z} &= - \frac{1}{1+u} = -1+u-u^2+u^3-\dots \\
  |z-2| < 1 : \frac{1}{1-z} &= -1 + (z-2) - (z-2)^2 + (z-2)^3
  \end{align*}
  (για \( z=3,\ f(z)=-1+1-1+1-\dots \))
  
  %TODO Kehagias Graph 14
  
  \paragraph{ΠΑΡ.}
  Να βρεθεί η σειρά Taylor της \( f(z) = \frac{z}{z^2-2z-3} \) γύρω από το \(z_0 = 0\)
  \begin{align*}
  \frac{z}{z^2-2z-3} &= \frac{A}{z-3}+\frac{B}{z-1} = \frac{1}{4}\cdot\frac{1}{z+1}
  + \frac{3}{4} \cdot \frac{1}{z-3} \\
  &= \frac{1}{4} \cdot (1-z+z^2-z^3+\dots) - \frac{3}{4}\cdot\frac{1}{3}\cdot
  \frac{1}{1-\frac{z}{3}} \\ &=
  \frac{1}{4}\cdot\left(1-z+z^3-z^3+\dots \right)
  -\frac{1}{4}\left( 1+\frac{z}{3} + \frac{z^2}{9}+\dots \right)
  \\ &= -\frac{1}{3}z+\frac{8}{36}z^2+\dots
  \end{align*}
  \( |z|<1 \quad \left|\frac{z}{3}\right|<1 \implies |z| < 3 \)
  
  \subparagraph{}
  \begin{align*}
  \frac{1}{z^2-2z-3} &= \frac{1}{(z-1)^2 - 4} = -\frac{1}{4}\cdot
  \frac{1}{1-\left(\frac{z-1}{z}\right)^2} \\ &=
  -\frac{1}{4}\cdot\left(
  1+\left(\frac{z-1}{2}\right)^2+\left(\frac{z-1}{2}\right)
  -\left(\frac{z-1}{2}\right)^6+\dots
  \right) \\ |z-1|<2 : \quad \frac{1}{z^2-2z-3}
  &= -\frac{1}{4} -\frac{\left(\frac{z-1}{2}\right)^2}{4} 
  - \frac{\left( \frac{z-1}{z} \right)}{4}
  \end{align*}
  
  \paragraph{}
  
  \begin{theorem*}{}
  	Έστω \( R_2 < R_1,\ \begin{array}{l}
  	C_1 = \left\lbrace z:|z-z_0| = R_1 \right\rbrace \\
  	C_2 = \left\lbrace z:|z-z_0| = R_2 \right\rbrace 
  	\end{array} \)
  	
  	\begin{align*}
  	A_{R_2,R_1}(z_0) &= \left\lbrace 
  	z:\ R_1 < |z-z_0| < R_2
  	 \right\rbrace \qquad \text{δακτύλιος χωρίς σύνορο} \\
  	\bar A_{R_2,R_1}(z_0) &= \left\lbrace 
  	z:\ R_1 \leq | z - z_0 | \leq R_2
  	 \right\rbrace \qquad \text{δακτύλιος με σύνορο}
  	\end{align*}
  	
  	%TODO Kehagias Graph 15
  	
  	Έστω \( f(z) \) αναλυτική στο \( \bar A_{R_2,R_1}(z_0) \). Τότε \( \forall z
  	\in A_{R_1,R_2}(z_0) \) ισχύει \[
  	f(z) = \sum_{n=-\infty}^\infty a_0 (z-z_0)^n
  	\]
  	
  	Αυτή λέγεται σειρά Laurent (Λοράντ) της \( f(z) \) γύρω από το \( z_0 \).
  	
  	\[
  	\forall n \in \mathbb Z: \quad a_n=\frac{1}{2\pi i}\oint_C
  	\frac{f(z)}{(z-z_0)^{n+1}}\dif z
  	\] (η \( C \) εντός του \( A_{R_2,R_1}(z_0) \))
  \end{theorem*}
  
  \paragraph{ΠΑΡ.}
  Βρείτε την σειρά Laurent της \( f(z) = \frac{\sin z}{z^2} \) γύρω από το
  \( z_0 = 0 \).
  \subparagraph{Λύση}
  \begin{align*}
  \frac{\sin z}{z^2}&=\frac{1}{z^2}\cdot \sin z 
  \\ &= \frac{1}{z^2}\left( z-\frac{z^3}{3!}+\frac{z^5}{5!}-\dots \right)
  \\ \forall z: 0<|z|<\infty: \quad \frac{\sin z}{z^2} &= \frac{1}{z}-\frac{z}{3!}
  +\frac{z}{5!}-\dots
  \end{align*}
  Το \( z_0 = 0 \) είναι \textbf{πόλος} πρώτης τάξης.
  
  \paragraph{ΠΑΡ} \( \frac{\sin z}{z} \)
  \[
  0<|z|<\infty:\quad \frac{\sin z}{z} = 1-\frac{z^2}{3!}+\frac{z^4}{5!}-\dots
  \]
  Το \( z_0 = 0 \) είναι απαλείψιμο ανώμαλο σημείο.
  
  \paragraph{ΠΑΡ}
  Να βρεθεί η σειρά Laurent της \( f(z) = \frac{e^{2z}}{(z+1)^2} \) γύρω
  από το \( z_0 = -1 \)
  
  \subparagraph{Λύση}
  \begin{align*}
  \frac{e^{2z}}{(z+1)^2} &= \frac{e^{-2}}{(z+1)^2} \cdot e^{2(z+1)}
  \\ &= e^{-2} \cdot \frac{1}{(z+1)^2}\cdot\left(
  1+2(z+1)+\frac{4\cdot(z+1)^2}{2!}+\frac{8\cdot(z+1)^3}{3!}+\dots
  \right) \\ 0<|z+1|<\infty :\quad \frac{e^{2z}}{(z+1)^2} &=
  \frac{e^{-2}}{(z+1)^2}+\frac{2e^{-2}}{z+1}+2e^{-2}+\frac{4e^{-2}}{3}
  \cdot (z+1)+\dots
  \end{align*}
  Παρατηρώ ότι \( z_0=-1 \) είναι πόλος 2\textsuperscript{ης} τάξης
  
  \begin{align*}
  \oint \frac{e^{2z}}{(z+1)^2} \dif z &=
  \cancelto{0}{\oint_C \frac{e^{-2}}{(z+1)^2}}\dif z
  +\cancelto{2e^{-2}2\pi i}{\oint_C \frac{2e^{-2}}{z+1}\dif z}
  +\cancelto{0}{\oint_C \frac{4e^{-2}}{3}(z+1)\dif z} + \dots \\
  \Aboxed{\oint \frac{e^{2z}}{(z+1)^2} &= 2e^{-2}2\pi i}
  \end{align*}
  
  \begin{defn*}{}
  	Έστω \( \displaystyle f(z) = \sum_{n=-\infty}^\infty a_n(z-z_0)^n \).
  	
  	Έστω \( n_1 \) ο ελάχιστος \( n \) τ.ώ. \( a_n \neq 0 \). \\
  	(αν δεν υπάρχει το ελάχιστο, θέτω \( n_1=-\infty \))
  	
  	\begin{enumgreek}
  		\item Αν \( -n_1 = 0 \), τότε το \( z_0 \) είναι απαλείψιμο ανώμαλο
  		σημείο (πόλος μηδενικής τάξης)
  		\item Αν \( \infty > -n_1 > 0 \), τότε το \( z_0 \) είναι πόλος τάξης
  		\( -n_1 \)
  		\item Αν \( n_1 = -\infty \), τότε το \( z_0 \) είναι ουσιώδες ανώμαλο
  		σημείο (πόλος \( \infty \) τάξης)
  	\end{enumgreek}
  \end{defn*}
  
  Η \( f(z) = e^{\sfrac{1}{z} } \) έχει το \( z_0 = 0 \) πόλο άπειρης τάξης.
\end{document}
